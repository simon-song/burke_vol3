%%%%%%%%%%%%%%%%%%%%%%%%%%%%%%%%%%%%%%%%%%%%%%%%%%%%%%%%%%%%%%%%%%%%%%%
\chapter*[Speech on the Nabob of Arcot's Debts]{
Speech
\\{\small On the}
\\Motion Made for Papers
\\{\small Relative to the}
\\Directions for Charging the Nabob of Arcot's Private Debts to Europeans on the Revenues of the Carnatic,
\\{\small February 28, 1785.}
\\With an Appendix,
\\{\small Containing Several Documents}
}
\addcontentsline{toc}{chapter}{
SPEECH ON THE NABOB OF ARCOT'S DEBTS, February 28, 1785; with an Appendix}

\begin{quotation}
{\g Ἐνταῦθα τί πράττειν ἐχρῆν ἄνδρα τῶν Πλάτωνος καὶ Ἀριστοτέλους ζηλωτὴν δογμάτων; ἆρα περιορᾶν ἀνθρώπους ἀθλίους τοῖς κλέπταις ἐκδιδομένους, ἢ κατὰ δύναμιν αὐτοῖς ἀμύνειν, οἶμαι ὡς ἤδη τὸ κύκνειον; ἐξᾴδουσι διὰ τὸ θεμισές ἐργαστήριον τῶν τοιούτων; Ἐμοὶ μὲν οὖν αἰσχρὸν εἶναι δοκεῖ τοὺς μὲν χιλιάρχους, ὅταν λείπωσι τὴν τάξιν, καταδικάζειν' ... τὴν δὲ ὑπέρ ἀθλίων ἀνθρώπων ἀπολείπειν τάξιν, ὅταν δὲῃ πρὸς κλέπτας ἀγωνίζεσθαι τοιούτους, καὶ ταῦτα τοῦ θεοῦ συμμαχοῦντος ἡμῖν, ὅσπερ οὖν ἔταξεν.}
\sourceatright{JULIANI Epist. 17.}
\end{quotation}

\vspace{0.3cm}
%%%%%%%%%%%%%%%%%%%%%%%%%%%%%%%%%%%%%%%%%
\begin{center}
  \textbf{\large ADVERTISEMENT} \par 
\end{center}
\addcontentsline{toc}{section}{ADVERTISEMENT}

That the least informed reader of this speech may be enabled to enter fully into the spirit of the transaction on occasion of which it was delivered, it may be proper to acquaint him, that, among the princes dependent on this nation in the southern part of India, the most considerable at present is commonly known by the title of the Nabob of Arcot.

This prince owed the establishment of his government, against the claims of his elder brother, as well as those of other competitors, to the arms and influence of the British East India Company. Being thus established in a considerable part of the dominions he now possesses, he began, about the year 1765, to form, at the instigation (as he asserts) of the servants of the East India Company, a variety of designs for the further extension of his territories. Some years after, he carried his views to certain objects of interior arrangement, of a very pernicious nature. None of these designs could be compassed without the aid of the Company's arms; nor could those arms be employed consistently with an obedience to the Company's orders. He was therefore advised to form a more secret, but an equally powerful, interest among the servants of that Company, and among others both at home and abroad. By engaging them in his interests, the use of the Company's power might be obtained without their ostensible authority; the power might even be employed in defiance of the authority, if the case should require, as in truth it often did require, a proceeding of that degree of boldness.

The Company had put him into possession of several great cities and magnificent castles. The good order of his affairs, his sense of personal dignity, his ideas of Oriental splendor, and the habits of an Asiatic life, (to which, being a native of India, and a Mahometan, he had from his infancy been inured,) would naturally have led him to fix the seat of his government within his own dominions. Instead of this, he totally sequestered himself from his country, and, abandoning all appearance of state, he took up his residence in an ordinary house, which he purchased in the suburbs of the Company's factory at Madras. In that place he has lived, without removing one day from thence, for several years past. He has there continued a constant cabal with the Company's servants, from the highest to the lowest,—creating, out of the ruins of the country, brilliant fortunes for those who will, and entirely destroying those who will not, be subservient to his purposes.

An opinion prevailed, strongly confirmed by several passages in his own letters, as well as by a combination of circumstances forming a body of evidence which cannot be resisted, that very great sums have been by him distributed, through a long course of years, to some of the Company's servants. Besides these presumed payments in ready money, (of which, from the nature of the thing, the direct proof is very difficult,) debts have at several periods been acknowledged to those gentlemen, to an immense amount,—that is, to some millions of sterling money. There is strong reason to suspect that the body of these debts is wholly fictitious, and was never created by money bonâ fide lent. But even on a supposition that this vast sum was really advanced, it was impossible that the very reality of such an astonishing transaction should not cause some degree of alarm and incite to some sort of inquiry.

It was not at all seemly, at a moment when the Company itself was so distressed as to require a suspension, by act of Parliament, of the payment of bills drawn on them from India,—and also a direct tax upon every house in England, in order to facilitate the vent of their goods, and to avoid instant insolvency,—at that very moment, that their servants should appear in so flourishing a condition, as, besides ten millions of other demands on their masters, to be entitled to claim a debt of three or four millions more from the territorial revenue of one of their dependent princes.

The ostensible pecuniary transactions of the Nabob of Arcot with very private persons are so enormous, that they evidently set aside every pretence of policy which might induce a prudent government in some instances to wink at ordinary loose practice in ill-managed departments. No caution could be too great in handling this matter, no scrutiny too exact. It was evidently the interest, and as evidently at least in the power, of the creditors, by admitting secret participation in this dark and undefined concern, to spread corruption to the greatest and the most alarming extent.

These facts relative to the debts were so notorious, the opinion of their being a principal source of the disorders of the British government in India was so undisputed and universal, that there was no party, no description of men in Parliament, who did not think themselves bound, if not in honor and conscience, at least in common decency, to institute a vigorous inquiry into the very bottom of the business, before they admitted any part of that vast and suspicious charge to be laid upon an exhausted country. Every plan concurred in directing such an inquiry, in order that whatever was discovered to be corrupt, fraudulent, or oppressive should lead to a due animadversion on the offenders, and, if anything fair and equitable in its origin should be found, (nobody suspected that much, comparatively speaking, would be so found,) it might be provided for,—in due subordination, however, to the ease of the subject and the service of the state.

These were the alleged grounds for an inquiry, settled in all the bills brought into Parliament relative to India,—and there were, I think, no less than four of them. By the bill commonly called Mr. Pitt's bill, the inquiry was specially, and by express words, committed to the Court of Directors, without any reserve for the interference of any other person or persons whatsoever. It was ordered that they should make the inquiry into the origin and justice of these debts, as far as the materials in their possession enabled them to proceed; and where they found those materials deficient, they should order the Presidency of Fort St. George (Madras) to complete the inquiry.

The Court of Directors applied themselves to the execution of the trust reposed in them. They first examined into the amount of the debt, which they computed, at compound interest, to be 2,945,600l. sterling. Whether their mode of computation, either of the original sums or the amount on compound interest, was exact, that is, whether they took the interest too high or the several capitals too low, is not material. On whatever principle any of the calculations were made up, none of them found the debt to differ from the recital of the act, which asserted that the sums claimed were "very large." The last head of these debts the Directors compute at 2,465,680l. sterling. Of the existence of this debt the Directors heard nothing until 1776, and they say, that, "although they had repeatedly written to the Nabob of Arcot, and to their servants, respecting the debt, yet they had never been able to trace the origin thereof, or to obtain any satisfactory information on the subject."

The Court of Directors, after stating the circumstances under which the debts appeared to them to have been contracted, add as follows:—"For these reasons we should have thought it our duty to inquire very minutely into those debts, even if the act of Parliament had been silent on the subject, before we concurred in any measure for their payment. But with the positive injunctions of the act before us to examine into their nature and origin, we are indispensably bound to direct such an inquiry to be instituted." They then order the President and Council of Madras to enter into a full examination, \&c., \&c.

The Directors, having drawn up their order to the Presidency on these principles, communicated the draught of the general letter in which those orders were contained to the board of his Majesty's ministers, and other servants lately constituted by Mr. Pitt's East India Act. These ministers, who had just carried through Parliament the bill ordering a specific inquiry, immediately drew up another letter, on a principle directly opposite to that which was prescribed by the act of Parliament and followed by the Directors. In these second orders, all idea of an inquiry into the justice and origin of the pretended debts, particularly of the last, the greatest, and the most obnoxious to suspicion, is abandoned. They are all admitted and established without any investigation whatsoever, (except some private conference with the agents of the claimants is to pass for an investigation,) and a fund for their discharge is assigned and set apart out of the revenues of the Carnatic. To this arrangement in favor of their servants, servants suspected of corruption and convicted of disobedience, the Directors of the East India Company were ordered to set their hands, asserting it to arise from their own conviction and opinion, in flat contradiction to their recorded sentiments, their strong remonstrance, and their declared sense of their duty, as well under their general trust and their oath as Directors, as under the express injunctions of an act of Parliament.

The principles upon which this summary proceeding was adopted by the ministerial board are stated by themselves in a number in the appendix to this speech.

By another section of the same act, the same Court of Directors were ordered to take into consideration and to decide on the indeterminate rights of the Rajah of Tanjore and the Nabob of Arcot; and in this, as in the former case, no power of appeal, revision, or alteration was reserved to any other. It was a jurisdiction, in a cause between party and party, given to the Court of Directors specifically. It was known that the territories of the former of these princes had been twice invaded and pillaged, and the prince deposed and imprisoned, by the Company's servants, influenced by the intrigues of the latter, and for the purpose of paying his pretended debts. The Company had, in the year 1775, ordered a restoration of the Rajah to his government, under certain conditions. The Rajah complained, that his territories had not been completely restored to him, and that no part of his goods, money, revenues, or records, unjustly taken and withheld from him, were ever returned. The Nabob, on the other hand, never ceased to claim the country itself, and carried on a continued train of negotiation, that it should again be given up to him, in violation of the Company's public faith.

The Directors, in obedience to this part of the act, ordered an inquiry, and came to a determination to restore certain of his territories to the Rajah. The ministers, proceeding as in the former case, without hearing any party, rescinded the decision of the Directors, refused the restitution of the territory, and, without regard to the condition of the country of Tanjore, which had been within a few years four times plundered, (twice by the Nabob of Arcot, and twice by enemies brought upon it solely by the politics of the same Nabob, the declared enemy of that people,) and without discounting a shilling for their sufferings, they accumulate an arrear of about four hundred thousand pounds of pretended tribute to this enemy; and then they order the Directors to put their hands to a new adjudication, directly contrary to a judgment in a judicial character and trust solemnly given by them and entered on their records.

These proceedings naturally called for some inquiry. On the 28th of February, 1785, Mr. Fox made the following motion in the House of Commons, after moving that the clauses of the act should be read:—"That the proper officer do lay before this House copies or extracts of all letters and orders of the Court of Directors of the United East India Company, in pursuance of the injunctions contained in the 37th and 38th clauses of the said act"; and the question being put, it passed in the negative by a very great majority.

The last speech in the debate was the following; which is given to the public, not as being more worthy of its attention than others, (some of which were of consummate ability,) but as entering more into the detail of the subject.

\vspace{0.3cm}
%%%%%%%%%%%%%%%%%%%%%%%%%%%%%%%%%%%%%%%%%
\begin{center}
  \textbf{\large SPEECH} \par 
\end{center}
\addcontentsline{toc}{section}{SPEECH}

The times we live in, Mr. Speaker, have been distinguished by extraordinary events. Habituated, however, as we are, to uncommon combinations of men and of affairs, I believe nobody recollects anything more surprising than the spectacle of this day. The right honorable gentleman
%[1]
\footnote{ Right Honorable Henry Dundas.}
 whose conduct is now in question formerly stood forth in this House, the prosecutor of the worthy baronet
%[2]
\footnote{ Sir Thomas Rumbold, late Governor of Madras.}
 who spoke after him. He charged him with several grievous acts of malversation in office, with abuses of a public trust of a great and heinous nature. In less than two years we see the situation of the parties reversed; and a singular revolution puts the worthy baronet in a fair way of returning the prosecution in a recriminatory bill of pains and penalties, grounded on a breach of public trust relative to the government of the very same part of India. If he should undertake a bill of that kind, he will find no difficulty in conducting it with a degree of skill and vigor fully equal to all that have been exerted against him.

But the change of relation between these two gentlemen is not so striking as the total difference of their deportment under the same unhappy circumstances. Whatever the merits of the worthy baronet's defence might have been, he did not shrink from the charge. He met it with manliness of spirit and decency of behavior. What would have been thought of him, if he had held the present language of his old accuser? When articles were exhibited against him by that right honorable gentleman, he did not think proper to tell the House that we ought to institute no inquiry, to inspect no paper, to examine no witness. He did not tell us (what at that time he might have told us with some show of reason) that our concerns in India were matters of delicacy, that to divulge anything relative to them would be mischievous to the state. He did not tell us that those who would inquire into his proceedings were disposed to dismember the empire. He had not the presumption to say, that, for his part, having obtained, in his Indian presidency, the ultimate object of his ambition, his honor was concerned in executing with integrity the trust which had been legally committed to his charge: that others, not having been so fortunate, could not be so disinterested; and therefore their accusations could spring from no other source than faction, and envy to his fortune.

Had he been frontless enough to hold such vain, vaporing language in the face of a grave, a detailed, a specified matter of accusation, whilst he violently resisted everything which could bring the merits of his cause to the test,—had he been wild enough to anticipate the absurdities of this day,—that is, had he inferred, as his late accuser has thought proper to do, that he could not have been guilty of malversation in office, for this sole and curious reason, that he had been in office,—had he argued the impossibility of his abusing his power on this sole principle, that he had power to abuse,—he would have left but one impression on the mind of every man who heard him, and who believed him in his senses: that in the utmost extent he was guilty of the charge.

But, Sir, leaving these two gentlemen to alternate as criminal and accuser upon what principles they think expedient, it is for us to consider whether the Chancellor of the Exchequer and the Treasurer of the Navy, acting as a Board of Control, are justified by law or policy in suspending the legal arrangements made by the Court of Directors, in order to transfer the public revenues to the private emolument of certain servants of the East India Company, without the inquiry into the origin and justice of their claims prescribed by an act of Parliament.

It is not contended that the act of Parliament did not expressly ordain an inquiry. It is not asserted that this inquiry was not, with equal precision of terms, specially committed, under particular regulations, to the Court of Directors. I conceive, therefore, the Board of Control had no right whatsoever to intermeddle in that business. There is nothing certain in the principles of jurisprudence, if this be not undeniably true, that when, a special authority is given to any persons by name to do some particular act, that no others, by virtue of general powers, can obtain a legal title to intrude themselves into that trust, and to exercise those special functions in their place. I therefore consider the intermeddling of ministers in this affair as a downright usurpation. But if the strained construction by which they have forced themselves into a suspicious office (which every man delicate with regard to character would rather have sought constructions to avoid) were perfectly sound and perfectly legal, of this I am certain, that they cannot be justified in declining the inquiry which had been prescribed to the Court of Directors. If the Board of Control did lawfully possess the right of executing the special trust given to that court, they must take it as they found it, subject to the very same regulations which bound the Court of Directors. It will be allowed that the Court of Directors had no authority to dispense with either the substance or the mode of inquiry prescribed by the act of Parliament. If they had not, where in the act did the Board of Control acquire that capacity? Indeed, it was impossible they should acquire it. What must we think of the fabric and texture of an act of Parliament which should find it necessary to prescribe a strict inquisition, that should descend into minute regulations for the conduct of that inquisition, that should commit this trust to a particular description of men, and in the very same breath should enable another body, at their own pleasure, to supersede all the provisions the legislature had made, and to defeat the whole purpose, end, and object of the law? This cannot be supposed even of an act of Parliament conceived by the ministers themselves, and brought forth during the delirium of the last session.

My honorable friend has told you in the speech which introduced his motion, that fortunately this question is not a great deal involved in the labyrinths of Indian detail. Certainly not. But if it were, I beg leave to assure you that there is nothing in the Indian detail which is more difficult than in the detail of any other business. I admit, because I have some experience of the fact, that for the interior regulation of India a minute knowledge of India is requisite. But on any specific matter of delinquency in its government you are as capable of judging as if the same thing were done at your door. Fraud, injustice, oppression, peculation, engendered in India, are crimes of the same blood, family, and cast with those that are born and bred in England. To go no farther than the case before us: you are just as competent to judge whether the sum of four millions sterling ought or ought not to be passed from the public treasury into a private pocket without any title except the claim of the parties, when the issue of fact is laid in Madras, as when it is laid in Westminster. Terms of art, indeed, are different in different places; but they are generally understood in none. The technical style of an Indian treasury is not one jot more remote than the jargon of our own Exchequer from the train of our ordinary ideas or the idiom of our common language. The difference, therefore, in the two cases is not in the comparative difficulty or facility of the two subjects, but in our attention to the one and our total neglect of the other. Had this attention and neglect been regulated by the value of the several objects, there would be nothing to complain of. But the reverse of that supposition is true. The scene of the Indian abuse is distant, indeed; but we must not infer that the value of our interest in it is decreased in proportion as it recedes from our view. In our politics, as in our common conduct, we shall be worse than infants, if we do not put our senses under the tuition of our judgment, and effectually cure ourselves of that optical illusion which makes a brier at our nose of greater magnitude than an oak at five hundred yards' distance.

I think I can trace all the calamities of this country to the single source of our not having had steadily before our eyes a general, comprehensive, well-connected, and well-proportioned view of the whole of our dominions, and a just sense of their true bearings and relations. After all its reductions, the British empire is still vast and various. After all the reductions of the House of Commons, (stripped as we are of our brightest ornaments and of our most important privileges,) enough are yet left to furnish us, if we please, with means of showing to the world that we deserve the superintendence of as large an empire as this kingdom ever held, and the continuance of as ample privileges as the House of Commons, in the plenitude of its power, had been habituated to assert. But if we make ourselves too little for the sphere of our duty, if, on the contrary, we do not stretch and expand our minds to the compass of their object, be well assured that everything about us will dwindle by degrees, until at length our concerns are shrunk to the dimensions of our minds. It is not a predilection to mean, sordid, home-bred cares that will avert the consequences of a false estimation of our interest, or prevent the shameful dilapidation into which a great empire must fall by mean reparations upon mighty ruins.

I confess I feel a degree of disgust, almost leading to despair, at the manner in which we are acting in the great exigencies of our country. There is now a bill in this House appointing a rigid inquisition into the minutest detail of our offices at home. The collection of sixteen millions annually, a collection on which the public greatness, safety, and credit have their reliance, the whole order of criminal jurisprudence, which holds together society itself, have at no time obliged us to call forth such powers,—no, nor anything like them. There is not a principle of the law and Constitution of this country that is not subverted to favor the execution of that project.
%[3]
\footnote{ Appendix, No. 1.}
 And for what is all this apparatus of bustle and terror? Is it because anything substantial is expected from it? No. The stir and bustle itself is the end proposed. The eye-servants of a short-sighted master will employ themselves, not on what is most essential to his affairs, but on what is nearest to his ken. Great difficulties have given a just value to economy; and our minister of the day must be an economist, whatever it may cost us. But where is he to exert his talents? At home, to be sure; for where else can he obtain a profitable credit for their exertion? It is nothing to him, whether the object on which he works under our eye be promising or not. If he does not obtain any public benefit, he may make regulations without end. Those are sure to pay in present expectation, whilst the effect is at a distance, and may be the concern of other times and other men. On these principles, he chooses to suppose (for he does not pretend more than to suppose) a naked possibility that he shall draw some resource out of crumbs dropped from the trenchers of penury; that something shall be laid in store from the short allowance of revenue-officers overloaded with duty and famished for want of bread,—by a reduction from officers who are at this very hour ready to batter the Treasury with what breaks through stone walls for an increase of their appointments. From the marrowless bones of these skeleton establishments, by the use of every sort of cutting and of every sort of fretting tool, he flatters himself that he may chip and rasp an empirical alimentary powder, to diet into some similitude of health and substance the languishing chimeras of fraudulent reformation.

Whilst he is thus employed according to his policy and to his taste, he has not leisure to inquire into those abuses in India that are drawing off money by millions from the treasures of this country, which are exhausting the vital juices from members of the state, where the public inanition is far more sorely felt than in the local exchequer of England. Not content with winking at these abuses, whilst he attempts to squeeze the laborious, ill-paid drudges of English revenue, he lavishes, in one act of corrupt prodigality, upon those who never served the public in any honest occupation at all, an annual income equal to two thirds of the whole collection of the revenues of this kingdom.

Actuated by the same principle of choice, he has now on the anvil another scheme, full of difficulty and desperate hazard, which totally alters the commercial relation of two kingdoms, and, what end soever it shall have, may bequeath a legacy of heartburning and discontent to one of the countries, perhaps to both, to be perpetuated to the latest posterity. This project is also undertaken on the hope of profit. It is provided, that, out of some (I know not what) remains of the Irish hereditary revenue, a fund, at some time, and of some sort, should be applied to the protection of the Irish trade. Here we are commanded again to task our faith, and to persuade ourselves, that, out of the surplus of deficiency, out of the savings of habitual and systematic prodigality, the minister of wonders will provide support for this nation, sinking under the mountainous load of two hundred and thirty millions of debt. But whilst we look with pain at his desperate and laborious trifling, whilst we are apprehensive that he will break his back in stooping to pick up chaff and straws, he recovers himself at an elastic bound, and with a broadcast swing of his arm he squanders over his Indian field a sum far greater than the clear produce of the whole hereditary revenue of the kingdom of Ireland.
%[4]
\footnote{ The whole of the net Irish hereditary revenue is, on a medium of the last seven years, about 330,000l. yearly. The revenues of all denominations fall short more than 150,000l. yearly of the charges. On the present produce, if Mr. Pitt's scheme was to take place, he might gain from seven to ten thousand pounds a year.}


Strange as this scheme of conduct in ministry is, and inconsistent with all just policy, it is still true to itself, and faithful to its own perverted order. Those who are bountiful to crimes will be rigid to merit and penurious to service. Their penury is even held out as a blind and cover to their prodigality. The economy of injustice is to furnish resources for the fund of corruption. Then they pay off their protection to great crimes and great criminals by being inexorable to the paltry frailties of little men; and these modern flagellants are sure, with a rigid fidelity, to whip their own enormities on the vicarious back of every small offender.

It is to draw your attention to economy of quite another order, it is to animadvert on offences of a far different description, that my honorable friend has brought before you the motion of this day. It is to perpetuate the abuses which are subverting the fabric of your empire, that the motion is opposed. It is, therefore, with reason (and if he has power to carry himself through, I commend his prudence) that the right honorable gentleman makes his stand at the very outset, and boldly refuses all Parliamentary information. Let him admit but one step towards inquiry, and he is undone. You must be ignorant, or he cannot be safe. But before his curtain is let down, and the shades of eternal night shall veil our Eastern dominions from our view, permit me, Sir, to avail myself of the means which were furnished in anxious and inquisitive times to demonstrate out of this single act of the present minister what advantages you are to derive from permitting the greatest concern of this nation to be separated from the cognizance, and exempted even out of the competence, of Parliament. The greatest body of your revenue, your most numerous armies, your most important commerce, the richest sources of your public credit, (contrary to every idea of the known, settled policy of England,) are on the point of being converted into a mystery of state. You are going to have one half of the globe hid even from the common liberal curiosity of an English gentleman. Here a grand revolution commences. Mark the period, and mark the circumstances. In most of the capital changes that are recorded in the principles and system of any government, a public benefit of some kind or other has been pretended. The revolution commenced in something plausible, in something which carried the appearance at least of punishment of delinquency or correction of abuse. But here, in the very moment of the conversion of a department of British government into an Indian mystery, and in the very act in which the change commences, a corrupt private interest is set up in direct opposition to the necessities of the nation. A diversion is made of millions of the public money from the public treasury to a private purse. It is not into secret negotiations for war, peace, or alliance that the House of Commons is forbidden to inquire. It is a matter of account; it is a pecuniary transaction; it is the demand of a suspected steward upon ruined tenants and an embarrassed master that the Commons of Great Britain are commanded not to inspect. The whole tenor of the right honorable gentleman's argument is consonant to the nature of his policy. The system of concealment is fostered by a system of falsehood. False facts, false colors, false names of persons and things, are its whole support.

Sir, I mean to follow the right honorable gentleman over that field of deception, clearing what he has purposely obscured, and fairly stating what it was necessary for him to misrepresent. For this purpose, it is necessary you should know, with some degree of distinctness, a little of the locality, the nature, the circumstances, the magnitude of the pretended debts on which this marvellous donation is founded, as well as of the persons from whom and by whom it is claimed.

Madras, with its dependencies, is the second (but with a long interval, the second) member of the British empire in the East. The trade of that city, and of the adjacent territory, was not very long ago among the most flourishing in Asia. But since the establishment of the British power it has wasted away under an uniform gradual decline, insomuch that in the year 1779 not one merchant of eminence was to be found in the whole country.
%[5]
\footnote{ Mr. Smith's Examination before the Select Committee. Appendix, No. 2.}
 During this period of decay, about six hundred thousand sterling pounds a year have been drawn off by English gentlemen on their private account, by the way of China alone.
%[6]
\footnote{ Appendix, No. 2.}
 If we add four hundred thousand, as probably remitted through other channels, and in other mediums, that is, in jewels, gold, and silver, directly brought to Europe, and in bills upon the British and foreign companies, you will scarcely think the matter overrated. If we fix the commencement of this extraction of money from the Carnatic at a period no earlier than the year 1760, and close it in the year 1780, it probably will not amount to a great deal less than twenty millions of money.

During the deep, silent flow of this steady stream of wealth which set from India into Europe, it generally passed on with no adequate observation; but happening at some periods to meet rifts of rocks that checked its course, it grew more noisy and attracted more notice. The pecuniary discussions caused by an accumulation of part of the fortunes of their servants in a debt from the Nabob of Arcot was the first thing which very particularly called for, and long engaged, the attention of the Court of Directors. This debt amounted to eight hundred and eighty thousand pounds sterling, and was claimed, for the greater part, by English gentlemen residing at Madras. This grand capital, settled at length by order at ten per cent, afforded an annuity of eighty-eight thousand pounds.
%[7]
\footnote{ Fourth Report, Mr. Dundas's Committee, p. 4.}


Whilst the Directors were digesting their astonishment at this information, a memorial was presented to them from three gentlemen, informing them that their friends had lent, likewise, to merchants of Canton in China, a sum of not more than one million sterling. In this memorial they called upon the Company for their assistance and interposition with the Chinese government for the recovery of the debt. This sum lent to Chinese merchants was at twenty-four per cent, which would yield, if paid, an annuity of two hundred and forty thousand pounds.
%[8]
\footnote{ A witness examined before the Committee of Secrecy says that eighteen per cent was the usual interest, but he had heard that more had been given. The above is the account which Mr. B. received.}


Perplexed as the Directors were with these demands, you may conceive, Sir, that they did not find themselves very much disembarrassed by being made acquainted that they must again exert their influence for a new reserve of the happy parsimony of their servants, collected into a second debt from the Nabob of Arcot, amounting to two millions four hundred thousand pounds, settled at an interest of twelve per cent. This is known by the name of the Consolidation of 1777, as the former of the Nabob's debts was by the title of the Consolidation of 1767. To this was added, in a separate parcel, a little reserve, called the Cavalry Debt, of one hundred and sixty thousand pounds, at the same interest. The whole of these four capitals, amounting to four millions four hundred and forty thousand pounds, produced at their several rates, annuities amounting to six hundred and twenty-three thousand pounds a year: a good deal more than one third of the clear land-tax of England, at four shillings in the pound; a good deal more than double the whole annual dividend of the East India Company, the nominal masters to the proprietors in these funds. Of this interest, three hundred and eighty-three thousand two hundred pounds a year stood chargeable on the public revenues of the Carnatic.

Sir, at this moment, it will not be necessary to consider the various operations which the capital and interest of this debt have successively undergone. I shall speak to these operations when I come particularly to answer the right honorable gentleman on each of the heads, as he has thought proper to divide them. But this was the exact view in which these debts first appeared to the Court of Directors, and to the world. It varied afterwards. But it never appeared in any other than a most questionable shape. When this gigantic phantom of debt first appeared before a young minister, it naturally would have justified some degree of doubt and apprehension. Such a prodigy would have filled any common man with superstitious fears. He would exorcise that shapeless, nameless form, and by everything sacred would have adjured it to tell by what means a small number of slight individuals, of no consequence or situation, possessed of no lucrative offices, without the command of armies or the known administration of revenues, without profession of any kind, without any sort of trade sufficient to employ a peddler, could have, in a few years, (as to some, even in a few months,) amassed treasures equal to the revenues of a respectable kingdom? Was it not enough to put these gentlemen, in the novitiate of their administration, on their guard, and to call upon them for a strict inquiry, (if not to justify them in a reprobation of those demands without any inquiry at all,) that, when all England, Scotland, and Ireland had for years been witness to the immense sums laid out by the servants of the Company in stocks of all denominations, in the purchase of lands, in the buying and building of houses, in the securing quiet seats in Parliament or in the tumultuous riot of contested elections, in wandering throughout the whole range of those variegated modes of inventive prodigality which sometimes have excited our wonder, sometimes roused our indignation, that, after all, India was four millions still in debt to them? India in debt to them! For what? Every debt, for which an equivalent of some kind or other is not given, is, on the face of it, a fraud. What is the equivalent they have given? What equivalent had they to give? What are the articles of commerce, or the branches of manufacture, which those gentlemen have carried hence to enrich India? What are the sciences they beamed out to enlighten it? What are the arts they introduced to cheer and to adorn it? What are the religious, what the moral institutions they have taught among that people, as a guide to life, or as a consolation when life is to be no more, that there is an eternal debt, a debt "still paying, still to owe," which must be bound on the present generation in India, and entailed on their mortgaged posterity forever? A debt of millions, in favor of a set of men whose names, with few exceptions, are either buried in the obscurity of their origin and talents or dragged into light by the enormity of their crimes!

In my opinion the courage of the minister was the most wonderful part of the transaction, especially as he must have read, or rather the right honorable gentleman says he has read for him, whole volumes upon the subject. The volumes, by the way, are not by one tenth part so numerous as the right honorable gentleman has thought proper to pretend, in order to frighten you from inquiry; but in these volumes, such as they are, the minister must have found a full authority for a suspicion (at the very least) of everything relative to the great fortunes made at Madras. What is that authority? Why, no other than the standing authority for all the claims which the ministry has thought fit to provide for,—the grand debtor,—the Nabob of Arcot himself. Hear that prince, in the letter written to the Court of Directors, at the precise period whilst the main body of these debts were contracting. In his letter he states himself to be, what undoubtedly he is, a most competent witness to this point. After speaking of the war with Hyder Ali in 1768 and 1769, and of other measures which he censures, (whether right or wrong it signifies nothing,) and into which he says he had been led by the Company's servants, he proceeds in this manner:—"If all these things were against the real interests of the Company, they are ten thousand times more against mine, and against the prosperity of my country and the happiness of my people; for your interests and mine are the same. What were they owing to, then? To the private views of a few individuals, who have enriched themselves at the expense of your influence and of my country: for your servants HAVE NO TRADE IN THIS COUNTRY, neither do you pay them high wages; yet in a few years they return to England with many lacs of pagodas. How can you or I account for such immense fortunes acquired in so short a time, without any visible means of getting them?"

When he asked this question, which involves its answer, it is extraordinary that curiosity did not prompt the Chancellor of the Exchequer to that inquiry which might come in vain recommended to him by his own act of Parliament. Does not the Nabob of Arcot tell us, in so many words, that there was no fair way of making the enormous sums sent by the Company's servants to England? And do you imagine that there was or could be more honesty and good faith in the demands for what remained behind in India? Of what nature were the transactions with himself? If you follow the train of his information, you must see, that, if these great sums were at all lent, it was not property, but spoil, that was lent; if not lent, the transaction was not a contract, but a fraud. Either way, if light enough could not be furnished to authorize a full condemnation of these demands, they ought to have been left to the parties, who best knew and understood each other's proceedings. It was not necessary that the authority of government should interpose in favor of claims whose very foundation was a defiance of that authority, and whose object and end was its entire subversion.

It may be said that this letter was written by the Nabob of Arcot in a moody humor, under the influence of some chagrin. Certainly it was; but it is in such humors that truth comes out. And when he tells you, from his own knowledge, what every one must presume, from the extreme probability of the thing, whether he told it or not, one such testimony is worth a thousand that contradict that probability, when the parties have a better understanding with each other, and when they have a point to carry that may unite them in a common deceit.

If this body of private claims of debt, real or devised, were a question, as it is falsely pretended, between the Nabob of Arcot, as debtor, and Paul Benfield and his associates, as creditors, I am sure I should give myself but little trouble about it. If the hoards of oppression were the fund for satisfying the claims of bribery and peculation, who would wish to interfere between such litigants? If the demands were confined to what might be drawn from the treasures which the Company's records uniformly assert that the Nabob is in possession of, or if he had mines of gold or silver or diamonds, (as we know that he has none,) these gentlemen might break open his hoards or dig in his mines without any disturbance from me. But the gentlemen on the other side of the House know as well as I do, and they dare not contradict me, that the Nabob of Arcot and his creditors are not adversaries, but collusive parties, and that the whole transaction is under a false color and false names. The litigation is not, nor ever has been, between their rapacity and his hoarded riches. No: it is between him and them combining and confederating, on one side, and the public revenues, and the miserable inhabitants of a ruined country, on the other. These are the real plaintiffs and the real defendants in the suit. Refusing a shilling from his hoards for the satisfaction of any demand, the Nabob of Arcot is always ready, nay, he earnestly, and with eagerness and passion, contends for delivering up to these pretended creditors his territory and his subjects. It is, therefore, not from treasuries and mines, but from the food of your unpaid armies, from the blood withheld from the veins and whipped out of the backs of the most miserable of men, that we are to pamper extortion, usury, and peculation, under the false names of debtors and creditors of state.

The great patron of these creditors, (to whose honor they ought to erect statues,) the right honorable gentleman,
%[9]
\footnote{ Mr. Dundas.}
 in stating the merits which recommended them to his favor, has ranked them under three grand divisions. The first, the creditors of 1767; then the creditors of the cavalry loan; and lastly, the creditors of the loan in 1777. Let us examine them, one by one, as they pass in review before us.

The first of these loans, that of 1767, he insists, has an indisputable claim upon the public justice. The creditors, he affirms, lent their money publicly; they advanced it with the express knowledge and approbation of the Company; and it was contracted at the moderate interest of ten per cent. In this loan, the demand is, according to him, not only just, but meritorious in a very high degree: and one would be inclined to believe he thought so, because he has put it last in the provision he has made for these claims.

I readily admit this debt to stand the fairest of the whole; for, whatever may be my suspicions concerning a part of it, I can convict it of nothing worse than the most enormous usury. But I can convict, upon the spot, the right honorable gentleman of the most daring misrepresentation in every one fact, without any exception, that he has alleged in defence of this loan, and of his own conduct with regard to it. I will show you that this debt was never contracted with the knowledge of the Company; that it had not their approbation; that they received the first intelligence of it with the utmost possible surprise, indignation, and alarm.

So for from being previously apprised of the transaction from its origin, it was two years before the Court of Directors obtained any official intelligence of it. "The dealings of the servants with the Nabob were concealed from the first, until they were found out" (says Mr. Sayer, the Company's counsel) "by the report of the country." The Presidency, however, at last thought proper to send an official account. On this the Directors tell them, "To your great reproach, it has been concealed from us. We cannot but suspect this debt to have had its weight in your proposed aggrandizement of Mahomed Ali [the Nabob of Arcot]; but whether it has or has not, certain it is you are guilty of an high breach of duty in concealing it from us."

These expressions, concerning the ground of the transaction, its effect, and its clandestine nature, are in the letters bearing date March 17, 1769. After receiving a more full account, on the 23d March, 1770, they state, that "Messrs. John Pybus, John Call, and James Bourchier, as trustees for themselves and others of the Nabob's private creditors, had proved a deed of assignment upon the Nabob and his son of FIFTEEN districts of the Nabob's country, the revenues of which yielded, in time of peace, eight lacs of pagodas [320,000l. sterling] annually; and likewise an assignment of the yearly tribute paid the Nabob from the Rajah of Tanjore, amounting to four lacs of rupees [40,000l.]." The territorial revenue at that time possessed by these gentlemen, without the knowledge or consent of their masters, amounted to three hundred and sixty thousand pounds sterling annually. They were making rapid strides to the entire possession of the country, when the Directors, whom the right honorable gentleman states as having authorized these proceedings, were kept in such profound ignorance of this royal acquisition of territorial revenue by their servants, that in the same letter they say, "This assignment was obtained by three of the members of your board in January, 1767; yet we do not find the least trace of it upon your Consultations until August, 1768, nor do any of your letters to us afford any information relative to such transactions till the 1st of November, 1768. By your last letters of the 8th of May, 1769, you bring the whole proceedings to light in one view."

As to the previous knowledge of the Company, and its sanction to the debts, you see that this assertion of that knowledge is utterly unfounded. But did the Directors approve of it, and ratify the transaction, when it was known? The very reverse. On the same 3d of March, the Directors declare, "upon an impartial examination of the whole conduct of our late Governor and Council of Fort George [Madras], and on the fullest consideration, that the said Governor and Council have, in notorious violation of the trust reposed in them, manifestly preferred the interest of private individuals to that of the Company, in permitting the assignment of the revenues of certain valuable districts, to a very large amount, from the Nabob to individuals"; and then, highly aggravating their crimes, they add,—"We order and direct that you do examine, in the most impartial manner, all the above-mentioned transactions, and that you punish, by suspension, degradation, dismission, or otherwise, as to you shall seem meet, all and every such servant or servants of the Company who may by you be found guilty of any of the above offences." "We had" (say the Directors) "the mortification to find that the servants of the Company, who had been raised, supported, and owed their present opulence to the advantages gained in such service, have in this instance most unfaithfully betrayed their trust, abandoned the Company's interest, and prostituted its influence to accomplish the purposes of individuals, whilst the interest of the Company is almost wholly neglected, and payment to us rendered extremely precarious." Here, then, is the rock of approbation of the Court of Directors, on which the right honorable gentleman says this debt was founded. Any member, Mr. Speaker, who should come into the House, on my reading this sentence of condemnation of the Court of Directors against their unfaithful servants, might well imagine that he had heard an harsh, severe, unqualified invective against the present ministerial Board of Control. So exactly do the proceedings of the patrons of this abuse tally with those of the actors in it, that the expressions used in the condemnation of the one may serve for the reprobation of the other, without the change of a word.

To read you all the expressions of wrath and indignation fulminated in this dispatch against the meritorious creditors of the right honorable gentleman, who according to him have been so fully approved by the Company, would be to read the whole.

The right honorable gentleman, with an address peculiar to himself, every now and then slides in the Presidency of Madras, as synonymous to the Company. That the Presidency did approve the debt is certain. But the right honorable gentleman, as prudent in suppressing as skilful in bringing forward his matter, has not chosen to tell you that the Presidency were the very persons guilty of contracting this loan,—creditors themselves, and agents and trustees for all the other creditors. For this the Court of Directors accuse them of breach of trust; and for this the right honorable gentleman considers them as perfectly good authority for those claims. It is pleasant to hear a gentleman of the law quote the approbation of creditors as an authority for their own debt.

How they came to contract the debt to themselves, how they came to act as agents for those whom they ought to have controlled, is for your inquiry. The policy of this debt was announced to the Court of Directors by the very persons concerned in creating it. "Till very lately," say the Presidency, "the Nabob placed his dependence on the Company. Now he has been taught by ill advisers that an interest out of doors may stand him in good stead. He has been made to believe that his private creditors have power and interest to overrule the Court of Directors."
%[10]
\footnote{ For the threats of the creditors, and total subversion of the authority of the Company in favor of the Nabob's power and the increase thereby of his evil dispositions, and the great derangement of all public concerns, see Select Committee Fort St. George's letters, 21st November, 1769, and January 31st, 1770; September 11, 1772; and Governor Bourchier's letters to the Nabob of Arcot, 21st November, 1769, and December 9th, 1769.}
 The Nabob was not misinformed. The private creditors instantly qualified a vast number of votes; and having made themselves masters of the Court of Proprietors, as well as extending a powerful cabal in other places as important, they so completely overturned the authority of the Court of Directors at home and abroad, that this poor, baffled government was soon obliged to lower its tone. It was glad to be admitted into partnership with its own servants. The Court of Directors, establishing the debt which they had reprobated as a breach of trust, and which was planned for the subversion of their authority, settled its payments on a par with those of the public; and even so were not able to obtain peace, or even equality in their demands. All the consequences lay in a regular and irresistible train. By employing their influence for the recovery of this debt, their orders, issued in the same breath, against creating new debts, only animated the strong desires of their servants to this prohibited prolific sport, and it soon produced a swarm of sons and daughters, not in the least degenerated from the virtue of their parents.

From that moment the authority of the Court of Directors expired in the Carnatic, and everywhere else. "Every man," says the Presidency, "who opposes the government and its measures, finds an immediate countenance from the Nabob; even our discarded officers, however unworthy, are received into the Nabob's service."
%[11]
\footnote{ "He [the Nabob] is in a great degree the cause of our present inability, by diverting the revenues of the Carnatic through private channels." "Even this peshcush [the Tanjore tribute], circumstanced as he and we are, he has assigned over to others, who now set themselves in opposition to the Company."—Consultations, October 11, 1769, on the 12th communicated to the Nabob.}
 It was, indeed, a matter of no wonderful sagacity to determine whether the Court of Directors, with their miserable salaries to their servants, of four or five hundred pounds a year, or the distributor of millions, was most likely to be obeyed. It was an invention beyond the imagination of all the speculatists of our speculating age, to see a government quietly settled in one and the same town, composed of two distinct members: one to pay scantily for obedience, and the other to bribe high for rebellion and revolt.

The next thing which recommends this particular debt to the right honorable gentleman is, it seems, the moderate interest of ten per cent. It would be lost labor to observe on this assertion. The Nabob, in a long apologetic letter
%[12]
\footnote{ Nabob's letter to Governor Palk. Papers published by the Directors in 1775; and papers printed by the same authority, 1781.}
 for the transaction between him and the body of the creditors, states the fact as I shall state it to you. In the accumulation of this debt, the first interest paid was from thirty to thirty-six per cent; it was then brought down to twenty-five per cent; at length it was reduced to twenty; and there it found its rest. During the whole process, as often as any of these monstrous interests fell into an arrear, (into which they were continually falling,) the arrear, formed into a new capital,
%[13]
\footnote{ See papers printed by order of a General Court in 1780, pp. 222 and 224; as also Nabob's letter to Governor Dupré, 19th July, 1771: "I have taken up loans by which I have suffered a loss of upwards of a crore of pagodas [four millions sterling] by interest on an heavy interest." Letter 15th January, 1772: "Notwithstanding I have taken much trouble, and have made many payments to my creditors, yet the load of my debt, which became so great by interest and compound interest, is not cleared."}
 was added to the old, and the same interest of twenty per cent accrued upon both. The Company, having got some scent of the enormous usury which prevailed at Madras, thought it necessary to interfere, and to order all interests to be lowered to ten per cent. This order, which contained no exception, though it by no means pointed particularly to this class of debts, came like a thunderclap on the Nabob. He considered his political credit as ruined; but to find a remedy to this unexpected evil, he again added to the old principal twenty per cent interest accruing for the last year. Thus a new fund was formed; and it was on that accumulation of various principals, and interests heaped upon interests, not on the sum originally lent, as the right honorable gentleman would make you believe, that ten per cent was settled on the whole.

When you consider the enormity of the interest at which these debts were contracted, and the several interests added to the principal, I believe you will not think me so skeptical, if I should doubt whether for this debt of 880,000l. the Nabob ever saw 100,000l. in real money. The right honorable gentleman suspecting, with all his absolute dominion over fact, that he never will be able to defend even this venerable patriarchal job, though sanctified by its numerous issue, and hoary with prescriptive years, has recourse to recrimination, the last resource of guilt. He says that this loan of 1767 was provided for in Mr. Fox's India bill; and judging of others by his own nature and principles, he more than insinuates that this provision was made, not from any sense of merit in the claim, but from partiality to General Smith, a proprietor, and an agent for that debt. If partiality could have had any weight against justice and policy with the then ministers and their friends, General Smith had titles to it. But the right honorable gentleman knows as well as I do, that General Smith was very far from looking on himself as partially treated in the arrangements of that time; indeed, what man dared to hope for private partiality in that sacred plan for relief to nations?

It is not necessary that the right honorable gentleman should sarcastically call that time to our recollection. Well do I remember every circumstance of that memorable period. God forbid I should forget it! O illustrious disgrace! O victorious defeat! May your memorial be fresh and new to the latest generations! May the day of that generous conflict be stamped in characters never to be cancelled or worn out from the records of time! Let no man hear of us, who shall not hear, that, in a struggle against the intrigues of courts and the perfidious levity of the multitude, we fell in the cause of honor, in the cause of our country, in the cause of human nature itself! But if fortune should be as powerful over fame as she has been prevalent over virtue, at least our conscience is beyond her jurisdiction. My poor share in the support of that great measure no man shall ravish from me. It shall be safely lodged in the sanctuary of my heart,—never, never to be torn from thence, but with those holds that grapple it to life.

I say, I well remember that bill, and every one of its honest and its wise provisions. It is not true that this debt was ever protected or enforced, or any revenue whatsoever set apart for it. It was left in that bill just where it stood: to be paid or not to be paid out of the Nabob's private treasures, according to his own discretion. The Company had actually given it their sanction, though always relying for its validity on the sole security of the faith of him
%[14]
\footnote{ The Nabob of Arcot.}
 who without their knowledge or consent entered into the original obligation. It had no other sanction; it ought to have had no other. So far was Mr. Fox's bill from providing funds for it, as this ministry have wickedly done for this, and for ten times worse transactions, out of the public estate, that an express clause immediately preceded, positively forbidding any British subject from receiving assignments upon any part of the territorial revenue, on any pretence whatsoever.
%[15]
\footnote{ Appendix, No. 3.}


You recollect, Mr. Speaker, that the Chancellor of the Exchequer strongly professed to retain every part of Mr. Fox's bill which was intended to prevent abuse; but in his India bill, which (let me do justice) is as able and skilful a performance, for its own purposes, as ever issued from the wit of man, premeditating this iniquity,—

\begin{verse}
Hoc ipsum ut strueret, Trojamque aperiret Achivis,—
\end{verse}

expunged this essential clause, broke down the fence which was raised to cover the public property against the rapacity of his partisans, and thus levelling every obstruction, he made a firm, broad highway for sin and death, for usury and oppression, to renew their ravages throughout the devoted revenues of the Carnatic.

The tenor, the policy, and the consequences of this debt of 1767 are in the eyes of ministry so excellent, that its merits are irresistible; and it takes the lead to give credit and countenance to all the rest. Along with this chosen body of heavy-armed infantry, and to support it in the line, the right honorable gentleman has stationed his corps of black cavalry. If there be any advantage between this debt and that of 1769, according to him the cavalry debt has it. It is not a subject of defence: it is a theme of panegyric. Listen to the right honorable gentleman, and you will find it was contracted to save the country,—to prevent mutiny in armies,—to introduce economy in revenues; and for all these honorable purposes, it originated at the express desire and by the representative authority of the Company itself.

First let me say a word to the authority. This debt was contracted, not by the authority of the Company, not by its representatives, (as the right honorable gentleman has the unparalleled confidence to assert,) but in the ever-memorable period of 1777, by the usurped power of those who rebelliously, in conjunction with the Nabob of Arcot, had overturned the lawful government of Madras. For that rebellion this House unanimously directed a public prosecution. The delinquents, after they had subverted government, in order to make to themselves a party to support them in their power, are universally known to have dealt jobs about to the right and to the left, and to any who were willing to receive them. This usurpation, which the right honorable gentleman well knows was brought about by and for the great mass of these pretended debts, is the authority which is set up by him to represent the Company,—to represent that Company which, from the first moment of their hearing of this corrupt and fraudulent transaction to this hour, have uniformly disowned and disavowed it.

So much for the authority. As to the facts, partly true, and partly colorable, as they stand recorded, they are in substance these. The Nabob of Arcot, as soon as he had thrown off the superiority of this country by means of these creditors, kept up a great army which he never paid. Of course his soldiers were generally in a state of mutiny.
%[16]
\footnote{ See Mr. Dundas's 1st, 2d, and 3d Reports.}
 The usurping Council say that they labored hard with their master, the Nabob, to persuade him to reduce these mutinous and useless troops. He consented; but, as usual, pleaded inability to pay them their arrears. Here was a difficulty. The Nabob had no money; the Company had no money; every public supply was empty. But there was one resource which no season has ever yet dried up in that climate. The soucars were at hand: that is, private English money-jobbers offered their assistance. Messieurs Taylor, Majendie, and Call proposed to advance the small sum of 160,000l. to pay off the Nabob's black cavalry, provided the Company's authority was given for their loan. This was the great point of policy always aimed at, and pursued through a hundred devices by the servants at Madras. The Presidency, who themselves had no authority for the functions they presumed to exercise, very readily gave the sanction of the Company to those servants who knew that the Company, whose sanction was demanded, had positively prohibited all such transactions.

However, so far as the reality of the dealing goes, all is hitherto fair and plausible; and here the right honorable gentleman concludes, with commendable prudence, his account of the business. But here it is I shall beg leave to commence my supplement: for the gentleman's discreet modesty has led him to cut the thread of the story somewhat abruptly. One of the most essential parties is quite forgotten. Why should the episode of the poor Nabob be omitted? When that prince chooses it, nobody can tell his story better. Excuse me, if I apply again to my book, and give it you from the first hand: from the Nabob himself.

"Mr. Stratton became acquainted with this, and got Mr. Taylor and others to lend me four lacs of pagodas towards discharging the arrears of pay of my troops. Upon this, I wrote a letter of thanks to Mr. Stratton; and upon the faith of this money being paid immediately, I ordered many of my troops to be discharged by a certain day, and lessened the number of my servants. Mr. Taylor, \&c., some time after acquainted me, that they had no ready money, but they would grant teeps payable in four months. This astonished me; for I did not know what might happen, when the sepoys were dismissed from my service. I begged of Mr. Taylor and the others to pay this sum to the officers of my regiments at the time they mentioned; and desired the officers, at the same time, to pacify and persuade the men belonging to them that their pay would be given to them at the end of four months, and that, till those arrears were discharged, their pay should be continued to them. Two years are nearly expired since that time, but Mr. Taylor has not yet entirely discharged the arrears of those troops, and I am obliged to continue their pay from that time till this. I hoped to have been able, by this expedient, to have lessened the number of my troops, and discharged the arrears due to them, considering the trifle of interest to Mr. Taylor and the others as no great matter; but instead of this, I am oppressed with the burden of pay due to those troops, and the interest, which is going on to Mr. Taylor from the day the teeps were granted to him." What I have read to you is an extract of a letter from the Nabob of the Carnatic to Governor Rumbold, dated the 22d, and received the 24th of March, 1779.
%[17]
\footnote{ See further Consultations, 3d February, 1778.}


Suppose his Highness not to be well broken in to things of this kind, it must, indeed, surprise so known and established a bond-vender as the Nabob of Arcot, one who keeps himself the largest bond-warehouse in the world, to find that he was now to receive in kind: not to take money for his obligations, but to give his bond in exchange for the bond of Messieurs Taylor, Majendie, and Call, and to pay, besides, a good, smart interest, legally twelve per cent, (in reality, perhaps, twenty or twenty-four per cent,) for this exchange of paper. But his troops were not to be so paid, or so disbanded. They wanted bread, and could not live by cutting and shuffling of bonds. The Nabob still kept the troops in service, and was obliged to continue, as you have seen, the whole expense to exonerate himself from which he became indebted to the soucars.

Had it stood here, the transaction would have been of the most audacious strain of fraud and usury perhaps ever before discovered, whatever might have been practised and concealed. But the same authority (I mean the Nabob's) brings before you something, if possible, more striking. He states, that, for this their paper, he immediately handed over to these gentlemen something very different from paper,—that is, the receipt of a territorial revenue, of which, it seems, they continued as long in possession as the Nabob himself continued in possession of anything. Their payments, therefore, not being to commence before the end of four months, and not being completed in two years, it must be presumed (unless they prove the contrary) that their payments to the Nabob were made out of the revenues they had received from his assignment. Thus they condescended to accumulate a debt of 160,000l. with an interest of twelve per cent, in compensation for a lingering payment to the Nabob of 160,000l. of his own money.

Still we have not the whole. About two years after the assignment of those territorial revenues to these gentlemen, the Nabob receives a remonstrance from his chief manager in a principal province, of which this is the tenor. "The entire revenue of those districts is by your Highness's order set apart to discharge the tunkaws [assignments] granted to the Europeans. The gomastahs [agents] of Mr. Taylor to Mr. De Fries are there in order to collect those tunkaws; and as they receive all the revenue that is collected, your Highness's troops have seven or eight months' pay due, which they cannot receive, and are thereby reduced to the greatest distress. In such times it is highly necessary to provide for the sustenance of the troops, that they may be ready to exert themselves in the service of your Highness."

Here, Sir, you see how these causes and effects act upon one another. One body of troops mutinies for want of pay; a debt is contracted to pay them; and they still remain unpaid. A territory destined to pay other troops is assigned for this debt; and these other troops fall into the same state of indigence and mutiny with the first. Bond is paid by bond; arrear is turned into new arrear; usury engenders new usury; mutiny, suspended in one quarter, starts up in another; until all the revenues and all the establishments are entangled into one inextricable knot of confusion, from which they are only disengaged by being entirely destroyed. In that state of confusion, in a very few months after the date of the memorial I have just read to you, things were found, when the Nabob's troops, famished to feed English soucars, instead of defending the country, joined the invaders, and deserted in entire bodies to Hyder Ali.
%[18]
\footnote{ Mr. Dundas's 1st Report, pp. 26, 29, and Appendix, No. 2, 10, 18, for the mutinous state and desertion of the Nabob's troops for want of pay. See also Report IV. of the same committee.}


The manner in which this transaction was carried on shows that good examples are not easily forgot, especially by those who are bred in a great school. One of those splendid examples give me leave to mention, at a somewhat more early period; because one fraud furnishes light to the discovery of another, and so on, until the whole secret of mysterious iniquity bursts upon you in a blaze of detection. The paper I shall read you is not on record. If you please, you may take it on my word. It is a letter written from one of undoubted information in Madras to Sir John Clavering, describing the practice that prevailed there, whilst the Company's allies were under sale, during the time of Governor Winch's administration.

"One mode," says Clavering's correspondent, "of amassing money at the Nabob's cost is curious. He is generally in arrears to the Company. Here the Governor, being cash-keeper, is generally on good terms with the banker, who manages matters thus. The Governor presses the Nabob for the balance due from him; the Nabob flies to his banker for relief; the banker engages to pay the money, and grants his notes accordingly, which he puts in the cash-book as ready money; the Nabob pays him an interest for it at two and three per cent per mensem, till the tunkaws he grants on the particular districts for it are paid. Matters in the mean time are so managed that there is no call for this money for the Company's service till the tunkaws become due. By this means not a cash is advanced by the banker, though he receives a heavy interest from the Nabob, which is divided as lawful spoil."

Here, Mr. Speaker, you have the whole art and mystery, the true free-mason secret, of the profession of soucaring; by which a few innocent, inexperienced young Englishmen, such as Mr. Paul Benfield, for instance, without property upon which any one would lend to themselves a single shilling, are enabled at once to take provinces in mortgage, to make princes their debtors, and to become creditors for millions.

But it seems the right honorable gentleman's favorite soucar cavalry have proved the payment before the Mayor's Court at Madras! Have they so? Why, then, defraud our anxiety and their characters of that proof? Is it not enough that the charges which I have laid before you have stood on record against these poor injured gentlemen for eight years? Is it not enough that they are in print by the orders of the East India Company for five years? After these gentlemen have borne all the odium of this publication and all the indignation of the Directors with such unexampled equanimity, now that they are at length stimulated into feeling are you to deny them their just relief? But will the right honorable gentleman be pleased to tell us how they came not to give this satisfaction to the Court of Directors, their lawful masters, during all the eight years of this litigated claim? Were they not bound, by every tie that can bind man, to give them this satisfaction? This day, for the first time, we hear of the proofs. But when were these proofs offered? In what cause? Who were the parties? Who inspected, who contested this belated account? Let us see something to oppose to the body of record which appears against them. The Mayor's Court! the Mayor's Court! Pleasant! Does not the honorable gentleman know that the first corps of creditors (the creditors of 1767) stated it as a sort of hardship to them, that they could not have justice at Madras, from the impossibility of their supporting their claims in the Mayor's Court? Why? Because, say they, the members of that court were themselves creditors, and therefore could not sit as judges.
%[19]
\footnote{ Memorial from the creditors to the Governor and Council, 22d January, 1770.}
 Are we ripe to say that no creditor under similar circumstances was member of the court, when the payment which is the ground of this cavalry debt was put in proof?
%[20]
\footnote{ In the year 1778, Mr. James Call, one of the proprietors of this specific debt, was actually mayor. (Appendix to 2d Report of Mr. Dundas's committee, No. 65.) The only proof which appeared on the inquiry instituted in the General Court of 1781 was an affidavit of the lenders themselves, deposing (what nobody ever denied) that they had engaged and agreed to pay—not that they had paid—the sum of 160,000l. This was two years after the transaction; and the affidavit is made before George Proctor, mayor, an attorney for certain of the old creditors.—Proceedings of the President and Council of Fort St. George, 22d February, 1779.}
 Nay, are we not in a manner compelled to conclude that the court was so constituted, when we know there is scarcely a man in Madras who has not some participation in these transactions? It is a shame to hear such proofs mentioned, instead of the honest, vigorous scrutiny which the circumstances of such an affair so indispensably call for.

But his Majesty's ministers, indulgent enough to other scrutinies, have not been satisfied with authorizing the payment of this demand without such inquiry as the act has prescribed; but they have added the arrear of twelve per cent interest, from the year 1777 to the year 1784, to make a new capital, raising thereby 160 to 294,000l. Then they charge a new twelve per cent on the whole from that period, for a transaction in which it will be a miracle if a single penny will be ever found really advanced from the private stock of the pretended creditors.

In this manner, and at such an interest, the ministers have thought proper to dispose of 294,000l. of the public revenues, for what is called the Cavalry Loan. After dispatching this, the right honorable gentleman leads to battle his last grand division, the consolidated debt of 1777. But having exhausted all his panegyric on the two first, he has nothing at all to say in favor of the last. On the contrary, he admits that it was contracted in defiance of the Company's orders, without even the pretended sanction of any pretended representatives. Nobody, indeed, has yet been found hardy enough to stand forth avowedly in its defence. But it is little to the credit of the age, that what has not plausibility enough to find an advocate has influence enough to obtain a protector. Could any man expect to find that protector anywhere? But what must every man think, when he finds that protector in the chairman of the Committee of Secrecy
%[21]
\footnote{ Right Honorable Henry Dundas.}
, who had published to the House, and to the world, the facts that condemn these debts, the orders that forbid the incurring of them, the dreadful consequences which attended them? Even in his official letter, when he tramples on his Parliamentary report, yet his general language is the same. Read the preface to this part of the ministerial arrangement, and you would imagine that this debt was to be crushed, with all the weight of indignation which could fall from a vigilant guardian of the public treasury upon those who attempted to rob it. What must be felt by every man who has feeling, when, after such a thundering preamble of condemnation, this debt is ordered to be paid without any sort of inquiry into its authenticity,—without a single step taken to settle even the amount of the demand,—without an attempt so much as to ascertain the real persons claiming a sum which rises in the accounts from one million three hundred thousand pound sterling to two million four hundred thousand pound, principal money,
%[22]
\footnote{ Appendix to the 4th Report of Mr. Dundas's committee, No 15.}
—without an attempt made to ascertain the proprietors, of whom no list has ever yet been laid before the Court of Directors,—of proprietors who are known to be in a collusive shuffle, by which they never appear to be the same in any two lists handed about for their own particular purposes?

My honorable friend who made you the motion has sufficiently exposed the nature of this debt. He has stated to you, that its own agents, in the year 1781, in the arrangement they proposed to make at Calcutta, were satisfied to have twenty-five per cent at once struck off from the capital of a great part of this debt, and prayed to have a provision made for this reduced principal, without any interest at all. This was an arrangement of their own, an arrangement made by those who best knew the true constitution of their own debt, who knew how little favor it merited,
%[23]
\footnote{ "No sense of the common danger, in case of a war, can prevail on him [the Nabob of Arcot] to furnish the Company with what is absolutely necessary to assemble an army, though it is beyond a doubt that money to a large amount is now hoarded up in his coffers at Chepauk; and tunkaws are granted to individuals, upon some of his most valuable countries, for payment of part of those debts which he has contracted, and which certainly will not bear inspection, as neither debtor nor creditors have ever had the confidence to submit the accounts to our examination, though they expressed a wish to consolidate the debts under the auspices of this government, agreeably to a plan they had formed."—Madras Consultations, 20th July, 1778. Mr. Dundas's Appendix to 2nd Report, 143. See also last Appendix to ditto Report, No. 376, B.}
 and how little hopes they had to find any persons in authority abandoned enough to support it as it stood.

But what corrupt men, in the fond imaginations of a sanguine avarice, had not the confidence to propose, they have found a Chancellor of the Exchequer in England hardy enough to undertake for them. He has cheered their drooping spirits. He has thanked the peculators for not despairing of their commonwealth. He has told them they were too modest. He has replaced the twenty-five per cent which, in order to lighten themselves, they had abandoned in their conscious terror. Instead of cutting off the interest, as they had themselves consented to do, with the fourth of the capital, he has added the whole growth of four years' usury of twelve per cent to the first overgrown principal; and has again grafted on this meliorated stock a perpetual annuity of six per cent, to take place from the year 1781. Let no man hereafter talk of the decaying energies of Nature. All the acts and monuments in the records of peculation, the consolidated corruption of ages, the patterns of exemplary plunder in the heroic times of Roman iniquity, never equalled the gigantic corruption of this single act. Never did Nero, in all the insolent prodigality of despotism, deal out to his prætorian guards a donation fit to be named with the largess showered down by the bounty of our Chancellor of the Exchequer on the faithful band of his Indian sepoys.

The right honorable gentleman
%[24]
\footnote{ Transcriber's note: Footnote missing in original text.}
 lets you freely and voluntarily into the whole transaction. So perfectly has his conduct confounded his understanding, that he fairly tells you that through the course of the whole business he has never conferred with any but the agents of the pretended creditors. After this, do you want more to establish a secret understanding with the parties,—to fix, beyond a doubt, their collusion and participation in a common fraud?

If this were not enough, he has furnished you with other presumptions that are not to be shaken. It is one of the known indications of guilt to stagger and prevaricate in a story, and to vary in the motives that are assigned to conduct. Try these ministers by this rule. In their official dispatch, they tell the Presidency of Madras that they have established the debt for two reasons: first, because the Nabob (the party indebted) does not dispute it; secondly, because it is mischievous to keep it longer afloat, and that the payment of the European creditors will promote circulation in the country. These two motives (for the plainest reasons in the world) the right honorable gentleman has this day thought fit totally to abandon. In the first place, he rejects the authority of the Nabob of Arcot. It would, indeed, be pleasant to see him adhere to this exploded testimony. He next, upon grounds equally solid, abandons the benefits of that circulation which was to be produced by drawing out all the juices of the body. Laying aside, or forgetting, these pretences of his dispatch, he has just now assumed a principle totally different, but to the full as extraordinary. He proceeds upon a supposition that many of the claims may be fictitious. He then finds, that, in a case where many valid and many fraudulent claims are blended together, the best course for their discrimination is indiscriminately to establish them all. He trusts, (I suppose,) as there may not be a fund sufficient for every description of creditors, that the best warranted claimants will exert themselves in bringing to light those debts which will not bear an inquiry. What he will not do himself he is persuaded will be done by others; and for this purpose he leaves to any person a general power of excepting to the debt. This total change of language and prevarication in principle is enough, if it stood alone, to fix the presumption of unfair dealing. His dispatch assigns motives of policy, concord, trade, and circulation: his speech proclaims discord and litigations, and proposes, as the ultimate end, detection.

But he may shift his reasons, and wind and turn as he will, confusion waits him at all his doubles. Who will undertake this detection? Will the Nabob? But the right honorable gentleman has himself this moment told us that no prince of the country can by any motive be prevailed upon to discover any fraud that is practised upon him by the Company's servants. He says what (with the exception of the complaint against the Cavalry Loan) all the world knows to be true: and without that prince's concurrence, what evidence can be had of the fraud of any the smallest of these demands? The ministers never authorized any person to enter into his exchequer and to search his records. Why, then, this shameful and insulting mockery of a pretended contest? Already contests for a preference have arisen among these rival bond-creditors. Has not the Company itself struggled for a preference for years, without any attempt at detection of the nature of those debts with which they contended? Well is the Nabob of Arcot attended to in the only specific complaint he has ever made. He complained of unfair dealing in the Cavalry Loan. It is fixed upon him with interest on interest; and this loan is excepted from all power of litigation.

This day, and not before, the right honorable gentleman thinks that the general establishment of all claims is the surest way of laying open the fraud of some of them. In India this is a reach of deep policy. But what would be thought of this mode of acting on a demand upon the Treasury in England? Instead of all this cunning, is there not one plain way open,—that is, to put the burden of the proof on those who make the demand? Ought not ministry to have said to the creditors, "The person who admits your debt stands excepted to as evidence; he stands charged as a collusive party, to hand over the public revenues to you for sinister purposes. You say, you have a demand of some millions on the Indian Treasury; prove that you have acted by lawful authority; prove, at least, that your money has been bonâ fide advanced; entitle yourself to my protection by the fairness and fulness of the communications you make"? Did an honest creditor ever refuse that reasonable and honest test?

There is little doubt that several individuals have been seduced by the purveyors to the Nabob of Arcot to put their money (perhaps the whole of honest and laborious earnings) into their hands, and that at such high interest as, being condemned at law, leaves them at the mercy of the great managers whom they trusted. These seduced creditors are probably persons of no power or interest either in England or India, and may be just objects of compassion. By taking, in this arrangement, no measures for discrimination and discovery, the fraudulent and the fair are in the first instance confounded in one mass. The subsequent selection and distribution is left to the Nabob. With him the agents and instruments of his corruption, whom he sees to be omnipotent in England, and who may serve him in future, as they have done in times past, will have precedence, if not an exclusive preference. These leading interests domineer, and have always domineered, over the whole. By this arrangement, the persons seduced are made dependent on their seducers; honesty (comparative honesty at least) must become of the party of fraud, and must quit its proper character and its just claims, to entitle itself to the alms of bribery and peculation.

But be these English creditors what they may, the creditors most certainly not fraudulent are the natives, who are numerous and wretched indeed: by exhausting the whole revenues of the Carnatic, nothing is left for them. They lent bonâ fide; in all probability they were even forced to lend, or to give goods and service for the Nabob's obligations. They had no trusts to carry to his market. They had no faith of alliances to sell. They had no nations to betray to robbery and ruin. They had no lawful government seditiously to overturn; nor had they a governor, to whom it is owing that you exist in India, to deliver over to captivity, and to death in a shameful prison.
%[25]
\footnote{ Lord Pigot}


These were the merits of the principal part of the debt of 1777, and the universally conceived causes of its growth; and thus the unhappy natives are deprived of every hope of payment for their real debts, to make provision for the arrears of unsatisfied bribery and treason. You see in this instance that the presumption of guilt is not only no exception to the demands on the public treasury, but with these ministers it is a necessary condition to their support. But that you may not think this preference solely owing to their known contempt of the natives, who ought with every generous mind to claim their first charities, you will find the same rule religiously observed with Europeans too. Attend, Sir, to this decisive case. Since the beginning of the war, besides arrears of every kind, a bond-debt has been contracted at Madras, uncertain in its amount, but represented from four hundred thousand pound to a million sterling. It stands only at the low interest of eight per cent. Of the legal authority on which this debt was contracted, of its purposes for the very being of the state, of its publicity and fairness, no doubt has been entertained for a moment. For this debt no sort of provision whatever has been made. It is rejected as an outcast, whilst the whole undissipated attention of the minister has been employed for the discharge of claims entitled to his favor by the merits we have seen.

I have endeavored to find out, if possible, the amount of the whole of those demands, in order to see how much, supposing the country in a condition to furnish the fund, may remain to satisfy the public debt and the necessary establishments. But I have been foiled in my attempt.

About one fourth, that is, about 220,000l., of the loan of 1767 remains unpaid. How much interest is in arrear I could never discover: seven or eight years' at least, which would make the whole of that debt about 396,000l. This stock, which the ministers in their instructions to the Governor of Madras state as the least exceptionable, they have thought proper to distinguish by a marked severity, leaving it the only one on which the interest is not added to the principal to beget a new interest.

The Cavalry Loan, by the operation of the same authority, is made up to 294,000l.; and this 294,000l., made up of principal and interest, is crowned with a new interest of twelve per cent.

What the grand loan, the bribery loan of 1777, may be is amongst the deepest mysteries of state. It is probably the first debt ever assuming the title of Consolidation that did not express what the amount of the sum consolidated was. It is little less than a contradiction in terms. In the debt of the year 1767 the sum was stated in the act of consolidation, and made to amount to 880,000l. capital. When this consolidation of 1777 was first announced at the Durbar, it was represented authentically at 2,400,000l. In that, or rather in a higher state, Sir Thomas Rumbold found and condemned it.
%[26]
\footnote{ In Sir Thomas Rumbold's letter to the Court of Directors, March 15th, 1778, he represents it as higher, in the following manner:—"How shall I paint to you my astonishment, on my arrival here, when I was informed, that, independent of this four lacs of pagodas [the Cavalry Loan], independent of the Nabob's debt to his old creditors, and the money due to the Company, he had contracted a debt to the enormous amount of sixty-three lacs of pagodas [2,520,000l.]. I mention this circumstance to you with horror; for the creditors being in general servants of the Company renders my task, on the part of the Company, difficult and invidious." "I have freed the sanction of this government from so corrupt a transaction. It is in my mind the most venal of all proceedings to give the Company's protection to debts that cannot bear the light; and though it appears exceedingly alarming, that a country on which you are to depend for resources should be so involved as to be nearly three years' revenue in debt,—in a country, too, where one year's revenue can never be called secure, by men who know anything of the politics of this part of India." "I think it proper to mention to you, that, although the Nabob reports his private debt to amount to upwards of sixty lacs, yet I understand that it is not quite so much." Afterwards Sir Thomas Rumbold recommended this debt to the favorable attention of the Company, but without any sufficient reason for his change of disposition. However, he went no further.}
 It afterwards fell into such a terror as to sweat away a million of its weight at once; and it sunk to 1,400,000l.
%[27]
\footnote{ Nabob's proposals, November 25th, 1778; and memorial of the creditors, March 1st, 1779.}
 However, it never was without a resource for recruiting it to its old plumpness. There was a sort of floating debt of about four or five hundred thousand pounds more ready to be added, as occasion should require.

In short, when you pressed this sensitive-plant, it always contracted its dimensions. When the rude hand of inquiry was withdrawn, it expanded in all the luxuriant vigor of its original vegetation. In the treaty of 1781, the whole of the Nabob's debt to private Europeans is by Mr. Sulivan, agent to the Nabob and his creditors, stated at 2,800,000l., which, if the Cavalry Loan and the remains of the debt of 1767 be subtracted, leaves it nearly at the amount originally declared at the Durbar in 1777: but then there is a private instruction to Mr. Sulivan, which, it seems, will reduce it again to the lower standard of 1,400,000l.

Failing in all my attempts, by a direct account, to ascertain the extent of the capital claimed, (where in all probability no capital was ever advanced,) I endeavored, if possible, to discover it by the interest which was to be paid. For that purpose, I looked to the several agreements for assigning the territories of the Carnatic to secure the principal and interest of this debt. In one of them,
%[28]
\footnote{ Nabob's proposals to his new consolidated creditors, November 25th, 1778.}
 I found, in a sort of postscript, by way of an additional remark, (not in the body of the obligation,) the debt represented at 1,400,000l.: but when I computed the sums to be paid for interest by instalments in another paper, I found they produced an interest of two millions, at twelve per cent; and the assignment supposed, that, if these instalments might exceed, they might also fall short of, the real provision for that interest.
%[29]
\footnote{ Paper signed by the Nabob, 6th January, 1780.}
 Another instalment-bond was afterwards granted: in that bond the interest exactly tallies with a capital of 1,400,000l.:
%[30]
\footnote{ Kistbundi to July 31, 1780.}
 but pursuing this capital through the correspondence, I lost sight of it again, and it was asserted that this instalment-bond was considerably short of the interest that ought to be computed to the time mentioned.
%[31]
\footnote{ Governor's letter to the Nabob, 25th July, 1779.}


Here are, therefore, two statements of equal authority, differing at least a million from each other; and as neither persons claiming, nor any special sum as belonging to each particular claimant, is ascertained in the instruments of consolidation, or in the installment-bonds, a large scope was left to throw in any sums for any persons, as their merits in advancing the interest of that loan might require; a power was also left for reduction, in case a harder hand, or more scanty funds, might be found to require it. Stronger grounds for a presumption of fraud never appeared in any transaction. But the ministers, faithful to the plan of the interested persons, whom alone they thought fit to confer with on this occasion, have ordered the payment of the whole mass of these unknown, unliquidated sums, without an attempt to ascertain them. On this conduct, Sir, I leave you to make your own reflections.

It is impossible (at least I have found it impossible) to fix on the real amount of the pretended debts with which your ministers have thought proper to load the Carnatic. They are obscure; they shun inquiry; they are enormous. That is all you know of them.

That you may judge what chance any honorable and useful end of government has for a provision that comes in for the leavings of these gluttonous demands, I must take it on myself to bring before you the real condition of that abused, insulted, racked, and ruined country; though in truth my mind revolts from it, though you will hear it with horror, and I confess I tremble when I think on these awful and confounding dispensations of Providence. I shall first trouble you with a few words as to the cause.

The great fortunes made in India, in the beginnings of conquest, naturally excited an emulation in all the parts and through the whole succession of the Company's service. But in the Company it gave rise to other sentiments. They did not find the new channels of acquisition flow with equal riches to them. On the contrary, the high flood-tide of private emolument was generally in the lowest ebb of their affairs. They began also to fear that the fortune of war might take away what the fortune of war had given. Wars were accordingly discouraged by repeated injunctions and menaces: and that the servants might not be bribed into them by the native princes, they were strictly forbidden to take any money whatsoever from their hands. But vehement passion is ingenious in resources. The Company's servants were not only stimulated, but better instructed by the prohibition. They soon fell upon a contrivance which answered their purposes far better than the methods which were forbidden: though in this also they violated an ancient, but they thought, an abrogated order. They reversed their proceedings. Instead of receiving presents, they made loans. Instead of carrying on wars in their own name, they contrived an authority, at once irresistible and irresponsible, in whose name they might ravage at pleasure; and being thus freed from all restraint, they indulged themselves in the most extravagant speculations of plunder. The cabal of creditors who have been the object of the late bountiful grant from his Majesty's ministers, in order to possess themselves, under the name of creditors and assignees, of every country in India, as fast as it should be conquered, inspired into the mind of the Nabob of Arcot (then a dependant on the Company of the humblest order) a scheme of the most wild and desperate ambition that I believe ever was admitted into the thoughts of a man so situated.
%[32]
\footnote{ Report of the Select Committee, Madras Consultations, January 7, 1771. See also papers published by the order of the Court of Directors in 1776; and Lord Macartney's correspondence with Mr. Hastings and the Nabob of Arcot. See also Mr. Dundas's Appendix, No 376, B. Nabob's propositions through Mr. Sulivan and Assam Khân, Art. 6, and indeed the whole.}
 First, they persuaded him to consider himself as a principal member in the political system of Europe. In the next place, they held out to him, and he readily imbibed, the idea of the general empire of Hindostan. As a preliminary to this undertaking, they prevailed on him to propose a tripartite division of that vast country: one part to the Company; another to the Mahrattas; and the third to himself. To himself he reserved all the southern part of the great peninsula, comprehended under the general name of the Deccan.

On this scheme of their servants, the Company was to appear in the Carnatic in no other light than as a contractor for the provision of armies, and the hire of mercenaries for his use and under his direction. This disposition was to be secured by the Nabob's putting himself under the guaranty of France, and, by the means of that rival nation, preventing the English forever from assuming an equality, much less a superiority, in the Carnatic. In pursuance of this treasonable project, (treasonable on the part of the English,) they extinguished the Company as a sovereign power in that part of India; they withdrew the Company's garrisons out of all the forts and strongholds of the Carnatic; they declined to receive the ambassadors from foreign courts, and remitted them to the Nabob of Arcot; they fell upon, and totally destroyed, the oldest ally of the Company, the king of Tanjore, and plundered the country to the amount of near five millions sterling; one after another, in the Nabob's name, but with English force, they brought into a miserable servitude all the princes and great independent nobility of a vast country.
%[33]
\footnote{ "The principal object of the expedition is, to get money from Tanjore to pay the Nabob's debt: if a surplus, to be applied in discharge of the Nabob's debts to his private creditors." (Consultations, March 20, 1771; and for further lights, Consultations, 12th June, 1771.) "We are alarmed lest this debt to individuals should have been the real motive for the aggrandizement of Mahomed Ali [the Nabob of Arcot], and that we are plunged into a war to put him in possession of the Mysore revenues for the discharge of the debt."—Letter from the Directors, March 17, 1769.}
 In proportion to these treasons and violences, which ruined the people, the fund of the Nabob's debt grew and flourished.

Among the victims to this magnificent plan of universal plunder, worthy of the heroic avarice of the projectors, you have all heard (and he has made himself to be well remembered) of an Indian chief called Hyder Ali Khan. This man possessed the western, as the Company, under the name of the Nabob of Arcot, does the eastern division of the Carnatic. It was among the leading measures in the design of this cabal (according to their own emphatic language) to extirpate this Hyder Ali.
%[34]
\footnote{ Letter from the Nabob, May 1st, 1768; and ditto, 24th April, 1770, 1st October; ditto, 16th September, 1772, 16th March, 1773.}
 They declared the Nabob of Arcot to be his sovereign, and himself to be a rebel, and publicly invested their instrument with the sovereignty of the kingdom of Mysore. But their victim was not of the passive kind. They were soon obliged to conclude a treaty of peace and close alliance with this rebel, at the gates of Madras. Both before and since that treaty, every principle of policy pointed out this power as a natural alliance; and on his part it was courted by every sort of amicable office. But the cabinet council of English creditors would not suffer their Nabob of Arcot to sign the treaty, nor even to give to a prince at least his equal the ordinary titles of respect and courtesy.
%[35]
\footnote{ Letter from the Presidency at Madras to the Court of Directors, 27th June, 1769.}
 From that time forward, a continued plot was carried on within the divan, black and white, of the Nabob of Arcot, for the destruction of Hyder Ali. As to the outward members of the double, or rather treble government of Madras, which had signed the treaty, they were always prevented by some overruling influence (which they do not describe, but which cannot be misunderstood) from performing what justice and interest combined so evidently to enforce.
%[36]
\footnote{ Mr. Dundas's committee. Report L, Appendix, No. 29.}


When at length Hyder Ali found that he had to do with men who either would sign no convention, or whom no treaty and no signature could bind, and who were the determined enemies of human intercourse itself, he decreed to make the country possessed by these incorrigible and predestinated criminals a memorable example to mankind. He resolved, in the gloomy recesses of a mind capacious of such things, to leave the whole Carnatic an everlasting monument of vengeance, and to put perpetual desolation as a barrier between him and those against whom the faith which holds the moral elements of the world together was no protection. He became at length so confident of his force, so collected in his might, that he made no secret whatsoever of his dreadful resolution. Having terminated his disputes with every enemy and every rival, who buried their mutual animosities in their common detestation against the creditors of the Nabob of Arcot, he drew from every quarter whatever a savage ferocity could add to his new rudiments in the arts of destruction; and compounding all the materials of fury, havoc, and desolation into one black cloud, he hung for a while on the declivities of the mountains. Whilst the authors of all these evils were idly and stupidly gazing on this menacing meteor, which blackened all their horizon, it suddenly burst, and poured down the whole of its contents upon the plains of the Carnatic. Then ensued a scene of woe, the like of which no eye had seen, no heart conceived, and which no tongue can adequately tell. All the horrors of war before known or heard of were mercy to that new havoc. A storm of universal fire blasted every field, consumed every house, destroyed every temple. The miserable inhabitants, flying from their flaming villages, in part were slaughtered; others, without regard to sex, to age, to the respect of rank or sacredness of function, fathers torn from children, husbands from wives, enveloped in a whirlwind of cavalry, and amidst the goading spears of drivers, and the trampling of pursuing horses, were swept into captivity in an unknown and hostile land. Those who were able to evade this tempest fled to the walled cities; but escaping from fire, sword, and exile, they fell into the jaws of famine.

The alms of the settlement, in this dreadful exigency, were certainly liberal; and all was done by charity that private charity could do: but it was a people in beggary; it was a nation which stretched out its hands for food. For months together, these creatures of sufferance, whose very excess and luxury in their most plenteous days had fallen short of the allowance of our austerest fasts, silent, patient, resigned, without sedition or disturbance, almost without complaint, perished by an hundred a day in the streets of Madras; every day seventy at least laid their bodies in the streets or on the glacis of Tanjore, and expired of famine in the granary of India. I was going to awake your justice towards this unhappy part of our fellow-citizens, by bringing before you some of the circumstances of this plague of hunger: of all the calamities which beset and waylay the life of man, this comes the nearest to our heart, and is that wherein the proudest of us all feels himself to be nothing more than he is: but I find myself unable to manage it with decorum; these details are of a species of horror so nauseous and disgusting, they are so degrading to the sufferers and to the hearers, they are so humiliating to human nature itself, that, on better thoughts, I find it more advisable to throw a pall over this hideous object, and to leave it to your general conceptions.

For eighteen months,
%[37]
\footnote{ Appendix, No. 4. Report of the Committee of Assigned Revenue.}
 without intermission, this destruction raged from the gates of Madras to the gates of Tanjore; and so completely did these masters in their art, Hyder Ali and his more ferocious son, absolve themselves of their impious vow, that, when the British armies traversed, as they did, the Carnatic for hundreds of miles in all directions, through the whole line of their march they did not see one man, not one woman, not one child, not one four-footed beast of any description whatever. One dead, uniform silence reigned over the whole region. With the inconsiderable exceptions of the narrow vicinage of some few forts, I wish to be understood as speaking literally. I mean to produce to you more than three witnesses, above all exception, who will support this assertion in its full extent. That hurricane of war passed through every part of the central provinces of the Carnatic. Six or seven districts to the north and to the south (and these not wholly untouched) escaped the general ravage.

The Carnatic is a country not much inferior in extent to England. Figure to yourself, Mr. Speaker, the land in whose representative chair you sit; figure to yourself the form and fashion of your sweet and cheerful country from Thames to Trent, north and south, and from the Irish to the German Sea, east and west, emptied and embowelled (may God avert the omen of our crimes!) by so accomplished a desolation. Extend your imagination a little further, and then suppose your ministers taking a survey of this scene of waste and desolation. What would be your thoughts, if you should be informed that they were computing how much had been the amount of the excises, how much the customs, how much the land and malt tax, in order that they should charge (take it in the most favorable light) for public service, upon the relics of the satiated vengeance of relentless enemies, the whole of what England had yielded in the most exuberant seasons of peace and abundance? What would you call it? To call it tyranny sublimed into madness would be too faint an image; yet this very madness is the principle upon which the ministers at your right hand have proceeded in their estimate of the revenues of the Carnatic, when they were providing, not supply for the establishments of its protection, but rewards for the authors of its ruin.

Every day you are fatigued and disgusted with this cant, "The Carnatic is a country that will soon recover, and become instantly as prosperous as ever." They think they are talking to innocents, who will believe, that, by sowing of dragons' teeth, men may come up ready grown and ready armed. They who will give themselves the trouble of considering (for it requires no great reach of thought, no very profound knowledge) the manner in which mankind are increased, and countries cultivated, will regard all this raving as it ought to be regarded. In order that the people, after a long period of vexation and plunder, may be in a condition to maintain government, government must begin by maintaining them. Here the road to economy lies not through receipt, but through expense; and in that country Nature has given no short cut to your object. Men must propagate, like other animals, by the mouth. Never did oppression light the nuptial torch; never did extortion and usury spread out the genial bed. Does any of you think that England, so wasted, would, under such a nursing attendance, so rapidly and cheaply recover? But he is meanly acquainted with either England or India who does not know that England would a thousand times sooner resume population, fertility, and what ought to be the ultimate secretion from both, revenue, than such a country as the Carnatic.

The Carnatic is not by the bounty of Nature a fertile soil. The general size of its cattle is proof enough that it is much otherwise. It is some days since I moved that a curious and interesting map, kept in the India House, should be laid before you.
%[38]
\footnote{ Mr. Barnard's map of the Jaghire}
 The India House is not yet in readiness to send it; I have therefore brought down my own copy, and there it lies for the use of any gentleman who may think such a matter worthy of his attention. It is, indeed, a noble map, and of noble things; but it is decisive against the golden dreams and sanguine speculations of avarice run mad. In addition to what you know must be the case in every part of the world, (the necessity of a previous provision of habitation, seed, stock, capital,) that map will show you that the uses of the influences of Heaven itself are in that country a work of art. The Carnatic is refreshed by few or no living brooks or running streams, and it has rain only at a season; but its product of rice exacts the use of water subject to perpetual command. This is the national bank of the Carnatic, on which it must have a perpetual credit, or it perishes irretrievably. For that reason, in the happier times of India, a number, almost incredible, of reservoirs have been made in chosen places throughout the whole country: they are formed, for the greater part, of mounds of earth and stones, with sluices of solid masonry; the whole constructed with admirable skill and labor, and maintained at a mighty charge. In the territory contained in that map alone, I have been at the trouble of reckoning the reservoirs, and they amount to upwards of eleven hundred, from the extent of two or three acres to five miles in circuit. From these reservoirs currents are occasionally drawn over the fields, and these watercourses again call for a considerable expense to keep them properly scoured and duly levelled. Taking the district in that map as a measure, there cannot be in the Carnatic and Tanjore fewer than ten thousand of these reservoirs of the larger and middling dimensions, to say nothing of those for domestic services, and the use of religious purification. These are not the enterprises of your power, nor in a style of magnificence suited to the taste of your minister. These are the monuments of real kings, who were the fathers of their people,—testators to a posterity which they embraced as their own. These are the grand sepulchres built by ambition,—but by the ambition of an insatiable benevolence, which, not contented with reigning in the dispensation of happiness during the contracted term of human life, had strained, with all the reachings and graspings of a vivacious mind, to extend the dominion of their bounty beyond the limits of Nature, and to perpetuate themselves through generations of generations, the guardians, the protectors, the nourishers of mankind.

Long before the late invasion, the persons who are objects of the grant of public money now before you had so diverted the supply of the pious funds of culture and population, that everywhere the reservoirs were fallen into a miserable decay.
%[39]
\footnote{ See Report IV., Mr. Dundas's committee, p. 46.}
 But after those domestic enemies had provoked the entry of a cruel foreign foe into the country, he did not leave it, until his revenge had completed the destruction begun by their avarice. Few, very few indeed, of these magazines of water that are not either totally destroyed, or cut through with such gaps as to require a serious attention and much cost to reëstablish them, as the means of present subsistence to the people and of future revenue to the state.

What, Sir, would a virtuous and enlightened ministry do, on the view of the ruins of such works before them?—on the view of such a chasm of desolation as that which yawned in the midst of those countries, to the north and south, which still bore some vestiges of cultivation? They would have reduced all their most necessary establishments; they would have suspended the justest payments; they would have employed every shilling derived from the producing to reanimate the powers of the unproductive parts. While they were performing this fundamental duty, whilst they were celebrating these mysteries of justice and humanity, they would have told the corps of fictitious creditors, whose crimes were their claims, that they must keep an awful distance,—that they must silence their inauspicious tongues,—that they must hold off their profane, unhallowed paws from this holy work; they would have proclaimed, with a voice that should make itself heard, that on every country the first creditor is the plough,—that this original, indefeasible claim supersedes every other demand.

This is what a wise and virtuous ministry would have done and said. This, therefore, is what our minister could never think of saying or doing. A ministry of another kind would have first improved the country, and have thus laid a solid foundation for future opulence and future force. But on this grand point of the restoration of the country there is not one syllable to be found in the correspondence of our ministers, from the first to the last; they felt nothing for a land desolated by fire, sword, and famine: their sympathies took another direction; they were touched with pity for bribery, so long tormented with a fruitless itching of its palms; their bowels yearned for usury, that had long missed the harvest of its returning months;
%[40]
\footnote{ Interest is rated in India by the month.}
 they felt for peculation, which had been for so many years raking in the dust of an empty treasury; they were melted into compassion for rapine and oppression, licking their dry, parched, unbloody jaws. These were the objects of their solicitude. These were the necessities for which they were studious to provide.

To state the country and its revenues in their real condition, and to provide for those fictitious claims, consistently with the support of an army and a civil establishment, would have been impossible; therefore the ministers are silent on that head, and rest themselves on the authority of Lord Macartney, who, in a letter to the Court of Directors, written in the year 1781, speculating on what might be the result of a wise management of the countries assigned by the Nabob of Arcot, rates the revenue, as in time of peace, at twelve hundred thousand pounds a year, as he does those of the king of Tanjore (which had not been assigned) at four hundred and fifty. On this Lord Macartney grounds his calculations, and on this they choose to ground theirs. It was on this calculation that the ministry, in direct opposition to the remonstrances of the Court of Directors, have compelled that miserable enslaved body to put their hands to an order for appropriating the enormous sum of 480,000l. annually, as a fund for paying to their rebellious servants a debt contracted in defiance of their clearest and most positive injunctions.

The authority and information of Lord Macartney is held high on this occasion, though it is totally rejected in every other particular of this business. I believe I have the honor of being almost as old an acquaintance as any Lord Macartney has. A constant and unbroken friendship has subsisted between us from a very early period; and I trust he thinks, that, as I respect his character, and in general admire his conduct, I am one of those who feel no common interest in his reputation. Yet I do not hesitate wholly to disallow the calculation of 1781, without any apprehension that I shall appear to distrust his veracity or his judgment. This peace estimate of revenue was not grounded on the state of the Carnatic, as it then, or as it had recently, stood. It was a statement of former and better times. There is no doubt that a period did exist, when the large portion of the Carnatic held by the Nabob of Arcot might be fairly reputed to produce a revenue to that, or to a greater amount. But the whole had so melted away by the slow and silent hostility of oppression and mismanagement, that the revenues, sinking with the prosperity of the country, had fallen to about 800,000l. a year, even before an enemy's horse had imprinted his hoof on the soil of the Carnatic. From that view, and independently of the decisive effects of the war which ensued, Sir Eyre Coote conceived that years must pass before the country could be restored to its former prosperity, and production. It was that state of revenue (namely, the actual state before the war) which the Directors have opposed to Lord Macartney's speculation. They refused to take the revenues for more than 800,000l. In this they are justified by Lord Macartney himself, who, in a subsequent letter, informs the court that his sketch is a matter of speculation; it supposes the country restored to its ancient prosperity, and the revenue to be in a course of effective and honest collection. If, therefore, the ministers have gone wrong, they were not deceived by Lord Macartney: they were deceived by no man. The estimate of the Directors is nearly the very estimate furnished by the right honorable gentleman himself, and published to the world in one of the printed reports of his own committee;
%[41]
\footnote{ Mr. Dundas's committee. Rep. I. p. 9, and ditto, Rep. IV. 69, where the revenue of 1777 stated only at 22 lacs,—30 lacs stated as the revenue, "supposing the Carnatic to be properly managed."}
 but as soon as he obtained his power, he chose to abandon his account. No part of his official conduct can be defended on the ground of his Parliamentary information.

In this clashing of accounts and estimates, ought not the ministry, if they wished to preserve even appearances, to have waited for information of the actual result of these speculations, before they laid a charge, and such a charge, not conditionally and eventually, but positively and authoritatively, upon a country which they all knew, and which one of them had registered on the records of this House, to be wasted, beyond all example, by every oppression of an abusive government, and every ravage of a desolating war? But that you may discern in what manner they use the correspondence of office, and that thereby you may enter into the true spirit of the ministerial Board of Control, I desire you, Mr. Speaker, to remark, that, through their whole controversy with the Court of Directors, they do not so much as hint at their ever having seen any other paper from Lord Macartney, or any other estimate of revenue than this of 1781. To this they hold. Here they take post; here they intrench themselves.

When I first read this curious controversy between the ministerial board and the Court of Directors, common candor obliged me to attribute their tenacious adherence to the estimate of 1781 to a total ignorance of what had appeared upon the records. But the right honorable gentleman has chosen to come forward with an uncalled-for declaration; he boastingly tells you, that he has seen, read, digested, compared everything,—and that, if he has sinned, he has sinned with his eyes broad open. Since, then, the ministers will obstinately shut the gates of mercy on themselves, let them add to their crimes what aggravations they please. They have, then, (since it must be so,) wilfully and corruptly suppressed the information which they ought to have produced, and, for the support of peculation, have made themselves guilty of spoliation and suppression of evidence.
%[42]
\footnote{ See Appendix, No. 4. statement in the Report of the Committee of Assigned Revenue.}
 The paper I hold in my hand, which totally overturns (for the present, at least) the estimate of 1781, they have no more taken notice of, in their controversy with the Court of Directors, than if it had no existence. It is the report made by a committee appointed at Madras to manage the whole of the six countries assigned to the Company by the Nabob of Arcot. This committee was wisely instituted by Lord Macartney, to remove from himself the suspicion of all improper management in so invidious a trust; and it seems to have been well chosen. This committee has made a comparative estimate of the only six districts which were in a condition to be let to farm. In one set of columns they state the gross and net produce of the districts as let by the Nabob. To that statement they oppose the terms on which the same districts were rented for five years under their authority. Under the Nabob, the gross farm was so high as 570,000l. sterling. What was the clear produce? Why, no more than about 250,000l.; and this was the whole profit of the Nabob's treasury, under his own management of all the districts which were in a condition to be let to farm on the 27th of May, 1782. Lord Macartney's leases stipulated a gross produce of no more than about 530,000l.; but then the estimated net amount was nearly double the Nabob's. It, however, did not then exceed 480,000l.; and Lord Macartney's commissioners take credit for an annual revenue amounting to this clear sum. Here is no speculation; here is no inaccurate account clandestinely obtained from those who might wish, and were enabled, to deceive. It is the authorized, recorded state of a real, recent transaction. Here is not twelve hundred thousand pound,—not eight hundred. The whole revenue of the Carnatic yielded no more, in May, 1782, than four hundred and eighty thousand pounds: nearly the very precise sum which your minister, who is so careful of the public security, has carried from all descriptions of establishment to form a fund for the private emolument of his creatures.

In this estimate, we see, as I have just observed, the Nabob's farms rated so high as 570,000l. Hitherto all is well: but follow on to the effective net revenue; there the illusion vanishes; and you will not find nearly so much as half the produce. It is with reason, therefore, Lord Macartney invariably, throughout the whole correspondence, qualifies all his views and expectations of revenue, and all his plans for its application, with this indispensable condition, that the management is not in the hands of the Nabob of Arcot. Should that fatal measure take place, he has over and over again told you that he has no prospect of realizing anything whatsoever for any public purpose. With these weighty declarations, confirmed by such a state of indisputable fact before them, what has been done by the Chancellor of the Exchequer and his accomplices? Shall I be believed? They have delivered over those very territories, on the keeping of which in the hands of the committee the defence of our dominions, and, what was more dear to them, possibly, their own job, depended,—they have delivered back again, without condition, without arrangement, without stipulation of any sort for the natives of any rank, the whole of those vast countries, to many of which he had no just claim, into the ruinous mismanagement of the Nabob of Arcot. To crown all, according to their miserable practice, whenever they do anything transcendently absurd, they preface this their abdication of their trust by a solemn declaration that they were not obliged to it by any principle of policy or any demand of justice whatsoever.

I have stated to you the estimated produce of the territories of the Carnatic in a condition to be farmed in 1782, according to the different managements into which they might fall; and this estimate the ministers have thought proper to suppress. Since that, two other accounts have been received. The first informs us, that there has been a recovery of what is called arrear, as well as of an improvement of the revenue of one of the six provinces which were let in 1782.
%[43]
\footnote{ The province of Tinnevelly.}
 It was brought about by making a new war. After some sharp actions, by the resolution and skill of Colonel Fullarton several of the petty princes of the most southerly of the unwasted provinces were compelled to pay very heavy rents and tributes, who for a long time before had not paid any acknowledgment. After this reduction, by the care of Mr. Irwin, one of the committee, that province was divided into twelve farms. This operation raised the income of that particular province; the others remain as they were first farmed. So that, instead of producing only their original rent of 480,000l., they netted, in about two years and a quarter, 1,320,000l. sterling, which would be about 660,000l. a year, if the recovered arrear was not included. What deduction is to be made on account of that arrear I cannot determine, but certainly what would reduce the annual income considerably below the rate I have allowed.

The second account received is the letting of the wasted provinces of the Carnatic. This I understand is at a growing rent, which may or may not realize what it promises; but if it should answer, it will raise the whole, at some future time, to 1,200,000l.

You must here remark, Mr. Speaker, that this revenue is the produce of all the Nabob's dominions. During the assignment, the Nabob paid nothing, because the Company had all. Supposing the whole of the lately assigned territory to yield up to the most sanguine expectations of the right honorable gentleman, and suppose 1,200,000l. to be annually realized, (of which we actually know of no more than the realizing of six hundred thousand,) out of this you must deduct the subsidy and rent which the Nabob paid before the assignment,—namely, 340,000l. a year. This reduces back the revenue applicable to the new distribution made by his Majesty's ministers to about 800,000l. Of that sum five eighths are by them surrendered to the debts. The remaining three are the only fund left for all the purposes so magnificently displayed in the letter of the Board of Control: that is, for a new-cast peace establishment, a now fund for ordnance and fortifications, and a large allowance for what they call "the splendor of the Durbar."

You have heard the account of these territories as they stood in 1782. You have seen the actual receipt since the assignment in 1781, of which I reckon about two years and a quarter productive. I have stated to you the expectation from the wasted part. For realizing all this you may value yourselves on the vigor and diligence of a governor and committee that have done so much. If these hopes from the committee are rational, remember that the committee is no more. Your ministers, who have formed their fund for these debts on the presumed effect of the committee's management, have put a complete end to that committee. Their acts are rescinded; their leases are broken; their renters are dispersed. Your ministers knew, when they signed the death-warrant of the Carnatic, that the Nabob would not only turn all these unfortunate farmers of revenue out of employment, but that he has denounced his severest vengeance against them, for acting under British authority. With a knowledge of this disposition, a British Chancellor of the Exchequer and Treasurer of the Navy, incited by no public advantage, impelled by no public necessity, in a strain of the most wanton perfidy which has ever stained the annals of mankind, have delivered over to plunder, imprisonment, exile, and death itself, according to the mercy of such execrable tyrants as Amir-ul-Omrah and Paul Benfield, the unhappy and deluded souls who, untaught by uniform example, were still weak enough to put their trust in English faith.
%[44]
\footnote{ Appendix, No. 5.}
 They have gone farther: they have thought proper to mock and outrage their misery by ordering them protection and compensation. From what power is this protection to be derived, and from what fund is this compensation to arise? The revenues are delivered over to their oppressor; the territorial jurisdiction, from whence that revenue is to arise, and under which they live, is surrendered to the same iron hands: and that they shall be deprived of all refuge and all hope, the minister has made a solemn, voluntary declaration that he never will interfere with the Nabob's internal government.
%[45]
\footnote{ See extract of their letter in the Appendix, No. 9.}


The last thing considered by the Board of Control among the debts of the Carnatic was that arising to the East India Company, which, after the provision for the cavalry, and the consolidation of 1777, was to divide the residue of the fund of 480,000l. a year with the lenders of 1767. This debt the worthy chairman, who sits opposite to me, contends to be three millions sterling. Lord Macartney's account of 1781 states it to be at that period 1,200,000l. The first account of the Court of Directors makes it 900,000l. This, like the private debt, being without any solid existence, is incapable of any distinct limits. Whatever its amount or its validity may be, one thing is clear: it is of the nature and quality of a public debt. In that light nothing is provided for it, but an eventual surplus to be divided with one class of the private demands, after satisfying the two first classes. Never was a more shameful postponing a public demand, which, by the reason of the thing, and the uniform practice of all nations, supersedes every private claim.

Those who gave this preference to private claims consider the Company's as a lawful demand; else why did they pretend to provide for it? On their own principles they are condemned.

But I, Sir, who profess to speak to your understanding and to your conscience, and to brush away from this business all false colors, all false appellations, as well as false facts, do positively deny that the Carnatic owes a shilling to the Company,—whatever the Company may be indebted to that undone country. It owes nothing to the Company, for this plain and simple reason: the territory charged with the debt is their own. To say that their revenues fall short, and owe them money, is to say they are in debt to themselves, which is only talking nonsense. The fact is, that, by the invasion of an enemy, and the ruin of the country, the Company, either in its own name, or in the names of the Nabob of Arcot and Rajah of Tanjore, has lost for several years what it might have looked to receive from its own estate. If men were allowed to credit themselves upon such principles, any one might soon grow rich by this mode of accounting. A flood comes down upon a man's estate in the Bedford Level of a thousand pounds a year, and drowns his rents for ten years. The Chancellor would put that man into the hands of a trustee, who would gravely make up his books, and for this loss credit himself in his account for a debt due to him of 10,000l. It is, however, on this principle the Company makes up its demands on the Carnatic. In peace they go the full length, and indeed more than the full length, of what the people can bear for current establishments; then they are absurd enough to consolidate all the calamities of war into debts,—to metamorphose the devastations of the country into demands upon its future production. What is this but to avow a resolution utterly to destroy their own country, and to force the people to pay for their sufferings to a government which has proved unable to protect either the share of the husbandman or their own? In every lease of a farm, the invasion of an enemy, instead of forming a demand for arrear, is a release of rent: nor for that release is it at all necessary to show that the invasion has left nothing to the occupier of the soil; though in the present case it would be too easy to prove that melancholy fact.
%[46]
\footnote{ "It is certain that the incursion of a few of Hyder's horse into the Jaghire, in 1767, cost the Company upwards of pagodas 27,000, in allowances for damages."—Consultations, February 11th, 1771.}
 I therefore applauded my right honorable friend, who, when he canvassed the Company's accounts, as a preliminary to a bill that ought not to stand on falsehood of any kind, fixed his discerning eye and his deciding hand on these debts of the Company from the Nabob of Arcot and Rajah of Tanjore, and at one stroke expunged them all, as utterly irrecoverable: he might have added, as utterly unfounded.

On these grounds I do not blame the arrangement this day in question, as a preference given to the debt of individuals over the Company's debt. In my eye it is no more than the preference of a fiction over a chimera; but I blame the preference given to those fictitious private debts over the standing defence and the standing government. It is there the public is robbed. It is robbed in its army; it is robbed in its civil administration; it is robbed in its credit; it is robbed in its investment, which forms the commercial connection between that country and Europe. There is the robbery.

But my principal objection lies a good deal deeper. That debt to the Company is the pretext under which all the other debts lurk and cover themselves. That debt forms the foul, putrid mucus in which are engendered the whole brood of creeping ascarides, all the endless involutions, the eternal knot, added to a knot of those inexpugnable tape-worms which devour the nutriment and eat up the bowels of India.
%[47]
\footnote{ Proceeding at Madras, 11th February, 1769, and throughout the correspondence on this subject; particularly Consultations, October 4th, 1769, and the creditors' memorial, 20th January, 1770.}
 It is necessary, Sir, you should recollect two things. First, that the Nabob's debt to the Company carries no interest. In the next place, you will observe, that, whenever the Company has occasion to borrow, she has always commanded whatever she thought fit at eight per cent. Carrying in your mind these two facts, attend to the process with regard to the public and private debt, and with what little appearance of decency they play into each other's hands a game of utter perdition to the unhappy natives of India. The Nabob falls into an arrear to the Company. The Presidency presses for payment. The Nabob's answer is, "I have no money." Good! But there are soucars who will supply you on the mortgage of your territories. Then steps forward some Paul Benfield, and, from his grateful compassion to the Nabob, and his filial regard to the Company, he unlocks the treasures of his virtuous industry, and, for a consideration of twenty-four or thirty-six per cent on a mortgage of the territorial revenue, becomes security to the Company for the Nabob's arrear.

All this intermediate usury thus becomes sanctified by the ultimate view to the Company's payment. In this case, would not a plain man ask this plain question of the Company: If you know that the Nabob must annually mortgage his territories to your servants to pay his annual arrear to you, why is not the assignment or mortgage made directly to the Company itself? By this simple, obvious operation, the Company would be relieved and the debt paid, without the charge of a shilling interest to that prince. But if that course should be thought too indulgent, why do they not take that assignment with such interest to themselves as they pay to others, that is, eight per cent? Or if it were thought more advisable (why it should I know not) that he must borrow, why do not the Company lend their own credit to the Nabob for their own payment? That credit would not be weakened by the collateral security of his territorial mortgage. The money might still be had at eight per cent. Instead of any of these honest and obvious methods, the Company has for years kept up a show of disinterestedness and moderation, by suffering a debt to accumulate to them from the country powers without any interest at all; and at the same time have seen before their eyes, on a pretext of borrowing to pay that debt, the revenues of the country charged with an usury of twenty, twenty-four, thirty-six, and even eight-and-forty per cent, with compound interest,
%[48]
\footnote{ Appendix, No. 7.}
 for the benefit of their servants. All this time they know that by having a debt subsisting without any interest, which is to be paid by contracting a debt on the highest interest, they manifestly render it necessary to the Nabob of Arcot to give the private demand a preference to the public; and, by binding him and their servants together in a common cause, they enable him to form a party to the utter ruin of their own authority and their own affairs. Thus their false moderation, and their affected purity, by the natural operation of everything false and everything affected, becomes pander and bawd to the unbridled debauchery and licentious lewdness of usury and extortion.

In consequence of this double game, all the territorial revenues have at one time or other been covered by those locusts, the English soucars. Not one single foot of the Carnatic has escaped them: a territory as large as England. During these operations what a scene has that country presented!
%[49]
\footnote{ For some part of these usurious transactions, see Consultation, 28th January, 1781; and for the Nabob's excusing his oppressions on account of these debts, Consultation, 26th November, 1770. "Still I undertook, first, the payment of the money belonging to the Company, who are my kind friends, and by borrowing, and mortgaging my jewels, \&c., by taking from every one of my servants, in proportion to their circumstances, by fresh severities also on my country, notwithstanding its distressed state, as you know."—The Board's remark is as follows: after controverting some of the facts, they say, "That his countries are oppressed is most certain, but not from real necessity; his debts, indeed, have afforded him a constant pretence for using severities and cruel oppressions."}
 The usurious European assignee supersedes the Nabob's native farmer of the revenue; the farmer flies to the Nabob's presence to claim his bargain; whilst his servants murmur for wages, and his soldiers mutiny for pay. The mortgage to the European assignee is then resumed, and the native farmer replaced,—replaced, again to be removed on the new clamor of the European assignee.
%[50]
\footnote{ See Consultation, 28th January, 1781, where it is asserted, and not denied, that the Nabob's farmers of revenue seldom continue for three months together. From this the state of the country may be easily judged of.}
 Every man of rank and landed fortune being long since extinguished, the remaining miserable last cultivator, who grows to the soil, after having his back scored by the farmer, has it again flayed by the whip of the assignee, and is thus, by a ravenous, because a short-lived succession of claimants, lashed from oppressor to oppressor, whilst a single drop of blood is left as the means of extorting a single grain of corn. Do not think I paint. Far, very far, from it: I do not reach the fact, nor approach to it. Men of respectable condition, men equal to your substantial English yeomen, are daily tied up and scourged to answer the multiplied demands of various contending and contradictory titles, all issuing from one and the same source. Tyrannous exaction brings on servile concealment; and that again calls forth tyrannous coercion. They move in a circle, mutually producing and produced; till at length nothing of humanity is left in the government, no trace of integrity, spirit, or manliness in the people, who drag out a precarious and degraded existence under this system of outrage upon human nature. Such is the effect of the establishment of a debt to the Company, as it has hitherto been managed, and as it ever will remain, until ideas are adopted totally different from those which prevail at this time.

Your worthy ministers, supporting what they are obliged to condemn, have thought fit to renew the Company's old order against contracting private debts in future. They begin by rewarding the violation of the ancient law; and then they gravely reenact provisions, of which they have given bounties for the breach. This inconsistency has been well exposed.
%[51]
\footnote{ In Mr. Fox's speech.}
 But what will you say to their having gone the length of giving positive directions for contracting the debt which they positively forbid?

I will explain myself. They order the Nabob, out of the revenues of the Carnatic, to allot four hundred and eighty thousand pounds a year, as a fund for the debts before us. For the punctual payment of this annuity, they order him to give soucar security.
%[52]
\footnote{ The amended letter, Appendix, No. 9.}
 When a soucar, that is, a money-dealer, becomes security for any native prince, the course is for the native prince to counter-secure the money-dealer, by making over to him in mortgage a portion of his territory equal to the sum annually to be paid, with an interest of at least twenty-four per cent. The point fit for the House to know is, who are these soucars to whom this security on the revenues in favor of the Nabob's creditors is to be given? The majority of the House, unaccustomed to these transactions, will hear with astonishment that these soucars are no other than the creditors themselves. The minister, not content with authorizing these transactions in a manner and to an extent unhoped for by the rapacious expectations of usury itself, loads the broken back of the Indian revenues, in favor of his worthy friends, the soucars, with an additional twenty-four per cent for being security to themselves for their own claims, for condescending to take the country in mortgage to pay to themselves the fruits of their own extortions.

The interest to be paid for this security, according to the most moderate strain of soucar demand, comes to 118,000l. a year, which, added to the 480,000l. on which it is to accrue, will make the whole charge amount to 598,000l. a year,—as much as even a long peace will enable those revenues to produce. Can any one reflect for a moment on all those claims of debt, which the minister exhausts himself in contrivances to augment with new usuries, without lifting up his hands and eyes in astonishment at the impudence both of the claim and of the adjudication? Services of some kind or other these servants of the Company must have done, so great and eminent that the Chancellor of the Exchequer cannot think that all they have brought home is half enough. He hallooes after them, "Gentlemen, you have forgot a large packet behind you, in your hurry; you have not sufficiently recovered yourselves; you ought to have, and you shall have, interest upon interest upon a prohibited debt that is made up of interest upon interest. Even this is too little. I have thought of another character for you, by which you may add something to your gains: you shall be security to yourselves; and hence will arise a new usury, which shall efface the memory of all the usuries suggested to you by your own dull inventions."

I have done with the arrangement relative to the Carnatic. After this it is to little purpose to observe on what the ministers have done to Tanjore. Your ministers have not observed even form and ceremony in their outrageous and insulting robbery of that country, whose only crime has been its early and constant adherence to the power of this, and the suffering of an uniform pillage in consequence of it. The debt of the Company from the Rajah of Tanjore is just of the same stuff with that of the Nabob of Arcot.

The subsidy from Tanjore, on the arrear of which this pretended debt (if any there be) has accrued to the Company, is not, like that paid by the Nabob of Arcot, a compensation for vast countries obtained, augmented, and preserved for him; not the price of pillaged treasuries, ransacked houses, and plundered territories: it is a large grant, from a small kingdom not obtained by our arms; robbed, not protected, by our power; a grant for which no equivalent was ever given, or pretended to be given. The right honorable gentleman, however, bears witness in his reports to the punctuality of the payments of this grant of bounty, or, if you please, of fear. It amounts to one hundred and sixty thousand pounds sterling net annual subsidy. He bears witness to a further grant of a town and port, with an annexed district of thirty thousand pound a year, surrendered to the Company since the first donation. He has not borne witness, but the fact is, (he will not deny it,) that in the midst of war, and during the ruin and desolation of a considerable part of his territories, this prince made many very large payments. Notwithstanding these merits and services, the first regulation of ministry is to force from him a territory of an extent which they have not yet thought proper to ascertain,
%[53]
\footnote{ Appendix, No. 8.}
 for a military peace establishment the particulars of which they have not yet been pleased to settle.

The next part of their arrangement is with regard to war. As confessedly this prince had no share in stirring up any of the former wars, so all future wars are completely out of his power; for he has no troops whatever, and is under a stipulation not so much as to correspond with any foreign state, except through the Company. Yet, in case the Company's servants should be again involved in war, or should think proper again to provoke any enemy, as in times past they have wantonly provoked all India, he is to be subjected to a new penalty. To what penalty? Why, to no less than the confiscation of all his revenues. But this is to end with the war, and they are to be faithfully returned? Oh, no! nothing like it. The country is to remain under confiscation until all the debt which the Company shall think fit to incur in such war shall be discharged: that is to say, forever. His sole comfort is, to find his old enemy, the Nabob of Arcot, placed in the very same condition.

The revenues of that miserable country were, before the invasion of Hyder, reduced to a gross annual receipt of three hundred and sixty thousand pound.
%[54]
\footnote{ Mr. Petrie's evidence before the Select Committee, Appendix, No. 7.}
 From this receipt the subsidy I have just stated is taken. This again, by payments in advance, by extorting deposits of additional sums to a vast amount for the benefit of their soucars, and by an endless variety of other extortions, public and private, is loaded with a debt, the amount of which I never could ascertain, but which is large undoubtedly, generating an usury the most completely ruinous that probably was ever heard of: that is, forty-eight per cent, payable monthly, with compound interest.
%[55]
\footnote{ Appendix, No. 7.}


Such is the state to which the Company's servants have reduced that country. Now come the reformers, restorers, and comforters of India. What have they done? In addition to all these tyrannous exactions, with all these ruinous debts in their train, looking to one side of an agreement whilst they wilfully shut their eyes to the other, they withdraw from Tanjore all the benefits of the treaty of 1762, and they subject that nation to a perpetual tribute of forty thousand a year to the Nabob of Arcot: a tribute never due, or pretended to be due, to him, even when he appeared to be something; a tribute, as things now stand, not to a real potentate, but to a shadow, a dream, an incubus of oppression. After the Company has accepted in subsidy, in grant of territory, in remission of rent, as a compensation for their own protection, at least two hundred thousand pound a year, without discounting a shilling for that receipt, the ministers condemn this harassed nation to be tributary to a person who is himself, by their own arrangement, deprived of the right of war or peace, deprived of the power of the sword, forbid to keep up a single regiment of soldiers, and is therefore wholly disabled from all protection of the country which is the object of the pretended tribute. Tribute hangs on the sword. It is an incident inseparable from real, sovereign power. In the present case, to suppose its existence is as absurd as it is cruel and oppressive. And here, Mr. Speaker, you have a clear exemplification of the use of those false names and false colors which the gentlemen who have lately taken possession of India choose to lay on for the purpose of disguising their plan of oppression. The Nabob of Arcot and Rajah of Tanjore have, in truth and substance, no more than a merely civil authority, held in the most entire dependence on the Company. The Nabob, without military, without federal capacity, is extinguished as a potentate; but then he is carefully kept alive as an independent and sovereign power, for the purpose of rapine and extortion,—for the purpose of perpetuating the old intrigues, animosities, usuries, and corruptions.

It was not enough that this mockery of tribute was to be continued without the correspondent protection, or any of the stipulated equivalents, but ten years of arrear, to the amount of 400,000l. sterling, is added to all the debts to the Company and to individuals, in order to create a new debt, to be paid (if at all possible to be paid in whole or in part) only by new usuries,—and all this for the Nabob of Arcot, or rather for Mr. Benfield and the corps of the Nabob's creditors and their soucars. Thus these miserable Indian princes are continued in their seats for no other purpose than to render them, in the first instance, objects of every species of extortion, and, in the second, to force them to become, for the sake of a momentary shadow of reduced authority, a sort of subordinate tyrants, the ruin and calamity, not the fathers and cherishers, of their people.

But take this tribute only as a mere charge (without title, cause, or equivalent) on this people; what one step has been taken to furnish grounds for a just calculation and estimate of the proportion of the burden and the ability? None,—not an attempt at it. They do not adapt the burden to the strength, but they estimate the strength of the bearers by the burden they impose. Then what care is taken to leave a fund sufficient to the future reproduction of the revenues that are to bear all these loads? Every one, but tolerably conversant in Indian affairs, must know that the existence of this little kingdom depends on its control over the river Cavery. The benefits of Heaven to any community ought never to be connected with political arrangements, or made to depend on the personal conduct of princes, in which the mistake, or error, or neglect, or distress, or passion of a moment, on either side, may bring famine on millions, and ruin an innocent nation perhaps for ages. The means of the subsistence of mankind should be as immutable as the laws of Nature, let power and dominion take what course they may.—Observe what has been done with regard to this important concern. The use of this river is, indeed, at length given to the Rajah, and a power provided for its enjoyment at his own charge; but the means of furnishing that charge (and a mighty one it is) are wholly out off. This use of the water, which ought to have no more connection than clouds and rains and sunshine with the politics of the Rajah, the Nabob, or the Company, is expressly contrived as a means of enforcing demands and arrears of tribute. This horrid and unnatural instrument of extortion had been a distinguishing feature in the enormities of the Carnatic politics, that loudly called for reformation. But the food of a whole people is by the reformers of India conditioned on payments from its prince, at a moment that he is overpowered with a swarm of their demands, without regard to the ability of either prince or people. In fine, by opening an avenue to the irruption of the Nabob of Arcot's creditors and soucars, whom every man, who did not fall in love with oppression and corruption on an experience of the calamities they produced, would have raised wall before wall and mound before mound to keep from a possibility of entrance, a more destructive enemy than Hyder Ali is introduced into that kingdom. By this part of their arrangement, in which they establish a debt to the Nabob of Arcot, in effect and substance, they deliver over Tanjore, bound hand and foot, to Paul Benfield, the old betrayer, insulter, oppressor, and scourge of a country which has for years been an object of an unremitted, but, unhappily, an unequal struggle, between the bounties of Providence to renovate and the wickedness of mankind to destroy.

The right honorable gentleman
%[56]
\footnote{ Mr. Dundas.}
 talks of his fairness in determining the territorial dispute between the Nabob of Arcot and the prince of that country, when he superseded the determination of the Directors, in whom the law had vested the decision of that controversy. He is in this just as feeble as he is in every other part. But it is not necessary to say a word in refutation of any part of his argument. The mode of the proceeding sufficiently speaks the spirit of it. It is enough to fix his character as a judge, that he never heard the Directors in defence of their adjudication, nor either of the parties in support of their respective claims. It is sufficient for me that he takes from the Rajah of Tanjore by this pretended adjudication, or rather from his unhappy subjects, 40,000l. a year of his and their revenue, and leaves upon his and their shoulders all the charges that can be made on the part of the Nabob, on the part of his creditors, and on the part of the Company, without so much as hearing him as to right or to ability. But what principally induces me to leave the affair of the territorial dispute between the Nabob and the Rajah to another day is this,—that, both the parties being stripped of their all, it little signifies under which of their names the unhappy, undone people are delivered over to the merciless soucars, the allies of that right honorable gentleman and the Chancellor of the Exchequer. In them ends the account of this long dispute of the Nabob of Arcot and the Rajah of Tanjore.

The right honorable gentleman is of opinion that his judgment in this case can be censured by none but those who seem to act as if they were paid agents to one of the parties. What does he think of his Court of Directors? If they are paid by either of the parties, by which of them does he think they are paid? He knows that their decision has been directly contrary to his. Shall I believe that it does not enter into his heart to conceive that any person can steadily and actively interest himself in the protection of the injured and oppressed without being well paid for his service? I have taken notice of this sort of discourse some days ago, so far as it may be supposed to relate to me. I then contented myself, as I shall now do, with giving it a cold, though a very direct contradiction. Thus much I do from respect to truth. If I did more, it might be supposed, by my anxiety to clear myself, that I had imbibed the ideas which, for obvious reasons, the right honorable gentleman wishes to have received concerning all attempts to plead the cause of the natives of India, as if it were a disreputable employment. If he had not forgot, in his present occupation, every principle which ought to have guided him, and I hope did guide him, in his late profession, he would have known that he who takes a fee for pleading the cause of distress against power, and manfully performs the duty he has assumed, receives an honorable recompense for a virtuous service. But if the right honorable gentleman will have no regard to fact in his insinuations or to reason in his opinions, I wish him at least to consider, that, if taking an earnest part with regard to the oppressions exercised in India, and with regard to this most oppressive case of Tanjore in particular, can ground a presumption of interested motives, he is himself the most mercenary man I know. His conduct, indeed, is such that he is on all occasions the standing testimony against himself. He it was that first called to that case the attention of the House; the reports of his own committee are ample and affecting upon that subject;
%[57]
\footnote{ See Report IV., Committee of Secrecy, pp. 73 and 74; and Appendix, in sundry places.}
 and as many of us as have escaped his massacre must remember the very pathetic picture he made of the sufferings of the Tanjore country, on the day when he moved the unwieldy code of his Indian resolutions. Has he not stated over and over again, in his reports, the ill treatment of the Rajah of Tanjore (a branch of the royal house of the Mahrattas, every injury to whom the Mahrattas felt as offered to themselves) as a main cause of the alienation of that people from the British power? And does he now think that to betray his principles, to contradict his declarations, and to become himself an active instrument in those oppressions which he had so tragically lamented, is the way to clear himself of having been actuated by a pecuniary interest at the time when he chose to appear full of tenderness to that ruined nation?

The right honorable gentleman is fond of parading on the motives of others, and on his own. As to himself, he despises the imputations of those who suppose that anything corrupt could influence him in this his unexampled liberality of the public treasure. I do not know that I am obliged to speak to the motives of ministry, in the arrangements they have made of the pretended debts of Arcot and Tanjore. If I prove fraud and collusion with regard to public money on those right honorable gentlemen, I am not obliged to assign their motives; because no good motives can be pleaded in favor of their conduct. Upon that case I stand; we are at issue; and I desire to go to trial. This, I am sure, is not loose railing, or mean insinuation, according to their low and degenerate fashion, when they make attacks on the measures of their adversaries. It is a regular and juridical course; and unless I choose it, nothing can compel me to go further.

But since these unhappy gentlemen have dared to hold a lofty tone about their motives, and affect to despise suspicion, instead of being careful not to give cause for it, I shall beg leave to lay before you some general observations on what I conceive was their duty in so delicate a business.

If I were worthy to suggest any line of prudence to that right honorable gentleman, I would tell him that the way to avoid suspicion in the settlement of pecuniary transactions, in which great frauds have been very strongly presumed, is, to attend to these few plain principles:—First, to hear all parties equally, and not the managers for the suspected claimants only; not to proceed in the dark, but to act with as much publicity as possible; not to precipitate decision; to be religious in following the rules prescribed in the commission under which we act; and, lastly, and above all, not to be fond of straining constructions, to force a jurisdiction, and to draw to ourselves the management of a trust in its nature invidious and obnoxious to suspicion, where the plainest letter of the law does not compel it. If these few plain rules are observed, no corruption ought to be suspected; if any of them are violated, suspicion will attach in proportion; if all of them are violated, a corrupt motive of some kind or other will not only be suspected, but must be violently presumed.

The persons in whose favor all these rules have been violated, and the conduct of ministers towards them, will naturally call for your consideration, and will serve to lead you through a series and combination of facts and characters, if I do not mistake, into the very inmost recesses of this mysterious business. You will then be in possession of all the materials on which the principles of sound jurisprudence will found, or will reject, the presumption of corrupt motives, or, if such motives are indicated, will point out to you of what particular nature the corruption is.

Our wonderful minister, as you all know, formed a new plan, a plan insigne, recens, indictum ore alio, a plan for supporting the freedom of our Constitution by court intrigues, and for removing its corruptions by Indian delinquency. To carry that bold, paradoxical design into execution, sufficient funds and apt instruments became necessary. You are perfectly sensible that a Parliamentary reform occupies his thoughts day and night, as an essential member in this extraordinary project. In his anxious researches upon this subject, natural instinct, as well as sound policy, would direct his eyes and settle his choice on Paul Benfield. Paul Benfield is the grand Parliamentary reformer, the reformer to whom the whole choir of reformers bow, and to whom even the right honorable gentleman himself must yield the palm: for what region in the empire, what city, what borough, what county, what tribunal in this kingdom is not full of his labors? Others have been only speculators; he is the grand practical reformer; and whilst the Chancellor of the Exchequer pledges in vain the man and the minister, to increase the provincial members, Mr. Benfield has auspiciously and practically begun it. Leaving far behind him even Lord Camelford's generous design of bestowing Old Sarum on the Bank of England, Mr. Benfield has thrown in the borough of Cricklade to reinforce the county representation. Not content with this, in order to station a steady phalanx for all future reforms, this public-spirited usurer, amidst his charitable toils for the relief of India, did not forget the poor, rotten Constitution of his native country. For her, he did not disdain to stoop to the trade of a wholesale upholsterer for this House,—to furnish it, not with the faded tapestry figures of antiquated merit, such as decorate, and may reproach, some other houses, but with real, solid, living patterns of true modern virtue. Paul Benfield made (reckoning himself) no fewer than eight members in the last Parliament. What copious streams of pure blood must he not have transfused into the veins of the present!

But what is even more striking than the real services of this new-imported patriot is his modesty. As soon as he had conferred this benefit on the Constitution, he withdrew himself from our applause. He conceived that the duties of a member of Parliament (which with the elect faithful, the true believers, the Islam of Parliamentary reform, are of little or no merit, perhaps not much better than specious sins) might he as well attended to in India as in England, and the means of reformation to Parliament itself be far better provided. Mr. Benfield was therefore no sooner elected than he set off for Madras, and defrauded the longing eyes of Parliament. We have never enjoyed in this House the luxury of beholding that minion of the human race, and contemplating that visage which has so long reflected the happiness of nations.

It was therefore not possible for the minister to consult personally with this great man. What, then, was he to do? Through a sagacity that never failed him in these pursuits, he found out, in Mr. Benfield's representative, his exact resemblance. A specific attraction, by which he gravitates towards all such characters, soon brought our minister into a close connection with Mr. Benfield's agent and attorney, that is, with the grand contractor, (whom I name to honor,) Mr. Richard Atkinson,—a name that will be well remembered as long as the records of this House, as long as the records of the British Treasury, as long as the monumental debt of England, shall endure.

This gentleman, Sir, acts as attorney for Mr. Paul Benfield. Every one who hears me is well acquainted with the sacred friendship and the steady mutual attachment that subsists between him and the present minister. As many members as chose to attend in the first session of this Parliament can best tell their own feelings at the scenes which were then acted. How much that honorable gentleman was consulted in the original frame and fabric of the bill, commonly called Mr. Pitt's India Bill, is matter only of conjecture, though by no means difficult to divine. But the public was an indignant witness of the ostentation with which the measure was made his own, and the authority with which he brought up clause after clause, to stuff and fatten the rankness of that corrupt act. As fast as the clauses were brought up to the table, they were accepted. No hesitation, no discussion. They were received by the new minister, not with approbation, but with implicit submission. The reformation may be estimated by seeing who was the reformer. Paul Benfield's associate and agent was held up to the world as legislator of Hindostan. But it was necessary to authenticate the coalition between the men of intrigue in India and the minister of intrigue in England by a studied display of the power of this their connecting link. Every trust, every honor, every distinction, was to be heaped upon him. He was at once made a Director of the India Company, made an alderman of London, and to be made, if ministry could prevail, (and I am sorry to say how near, how very near, they were prevailing,) representative of the capital of this kingdom. But to secure his services against all risk, he was brought in for a ministerial borough. On his part, he was not wanting in zeal for the common cause. His advertisements show his motives, and the merits upon which he stood. For your minister, this worn-out veteran submitted to enter into the dusty field of the London contest; and you all remember that in the same virtuous cause he submitted to keep a sort of public office or counting-house, where the whole business of the last general election was managed. It was openly managed by the direct agent and attorney of Benfield. It was managed upon Indian principles and for an Indian interest. This was the golden cup of abominations,—this the chalice of the fornications of rapine, usury, and oppression, which was held out by the gorgeous Eastern harlot,—which so many of the people, so many of the nobles of this land had drained to the very dregs. Do you think that no reckoning was to follow this lewd debauch? that no payment was to be demanded for this riot of public drunkenness and national prostitution? Here, you have it here before you! The principal of the grand election-manager must be indemnified; accordingly, the claims of Benfield and his crew must be put above all inquiry.

For several years Benfield appeared as the chief proprietor, as well as the chief agent, director, and controller of this system of debt. The worthy chairman of the Company has stated the claims of this single gentleman on the Nabob of Arcot as amounting to five hundred thousand pound.
%[58]
\footnote{ Mr. Smith's protest.}
 Possibly at the time of the chairman's state they might have been as high. Eight hundred thousand pound had been mentioned some time before;
%[59]
\footnote{ Madras correspondence on this subject.}
 and, according to the practice of shifting the names of creditors in these transactions, and reducing or raising the debt itself at pleasure, I think it not impossible that at one period the name of Benfield might have stood before those frightful figures. But my best information goes to fix his share no higher than four hundred thousand pounds. By the scheme of the present ministry for adding to the principal twelve per cent from the year 1777 to the year 1781, four hundred thousand pounds, that smallest of the sums ever mentioned for Mr. Benfield, will form a capital of 592,000l. at six per cent. Thus, besides the arrears of three years, amounting to 106,500l., (which, as fast as received, may be legally lent out at twelve per cent,) Benfield has received, by the ministerial grant before you, an annuity of 35,520l. a year, charged on the public revenues.

Our mirror of ministers of finance did not think this enough for the services of such a friend as Benfield. He found that Lord Macartney, in order to frighten the Court of Directors from the project of obliging the Nabob to give soucar security for his debt, assured them, that, if they should take that step, Benfield
%[60]
\footnote{ Appendix, No 6.}
 would infallibly be the soucar, and would thereby become the entire master of the Carnatic. What Lord Macartney thought sufficient to deter the very agents and partakers with Benfield in his iniquities was the inducement to the two right honorable gentlemen to order this very soucar security to be given, and to recall Benfield to the city of Madras from the sort of decent exile into which he had been relegated by Lord Macartney. You must therefore consider Benfield as soucar security for 480,000l. a year, which, at twenty-four per cent, (supposing him contented with that profit,) will, with the interest of his old debt, produce an annual income of 149,520l. a year.

Here is a specimen of the new and pure aristocracy created by the right honorable gentleman,
%[61]
\footnote{ Right Honorable William Pitt.}
 as the support of the crown and Constitution against the old, corrupt, refractory, natural interests of this kingdom; and this is the grand counterpoise against all odious coalitions of these interests. A single Benfield outweighs them all: a criminal, who long since ought to have fattened the region kites with his offal, is by his Majesty's ministers enthroned in the government of a great kingdom, and enfeoffed with an estate which in the comparison effaces the splendor of all the nobility of Europe. To bring a little more distinctly into view the true secret of this dark transaction, I beg you particularly to advert to the circumstances which I am going to place before you.

The general corps of creditors, as well as Mr. Benfield himself, not looking well into futurity, nor presaging the minister of this day, thought it not expedient for their common interest that such a name as his should stand at the head of their list. It was therefore agreed amongst them that Mr. Benfield should disappear, by making over his debt to Messrs. Taylor, Majendie, and Call, and should in return be secured by their bond.

The debt thus exonerated of so great a weight of its odium, and otherwise reduced from its alarming bulk, the agents thought they might venture to print a list of the creditors. This was done for the first time in the year 1783, during the Duke of Portland's administration. In this list the name of Benfield was not to be seen. To this strong negative testimony was added the further testimony of the Nabob of Arcot. That prince
%[62]
\footnote{ Appendix, No. 10.}
 (or rather Mr. Benfield for him) writes to the Court of Directors a letter
%[63]
\footnote{ Dated 13th October. For further illustration of the style in which these letters were written, and the principles on which they proceed, see letters from the Nabob to the Court of Directors, dated August 16th and September 7th, 1783, delivered by Mr. James Macpherson, minister to the Nabob, January 14, 1784. Appendix, No. 10.}
 full of complaints and accusations against Lord Macartney, conveyed in such terms as were natural for one of Mr. Benfield's habits and education to employ. Amongst the rest he is made to complain of his Lordship's endeavoring to prevent an intercourse of politeness and sentiment between him and Mr. Benfield; and to aggravate the affront, he expressly declares Mr. Benfield's visits to be only on account of respect and of gratitude, as no pecuniary transaction subsisted between them.

Such, for a considerable space of time, was the outward form of the loan of 1777, in which Mr. Benfield had no sort of concern. At length intelligence arrived at Madras, that this debt, which had always been renounced by the Court of Directors, was rather like to become the subject of something more like a criminal inquiry than of any patronage or sanction from Parliament. Every ship brought accounts, one stronger than the other, of the prevalence of the determined enemies of the Indian system. The public revenues became an object desperate to the hopes of Mr. Benfield; he therefore resolved to fall upon his associates, and, in violation of that faith which subsists among those who have abandoned all other, commences a suit in the Mayor's Court against Taylor, Majendie, and Call, for the bond given to him, when he agreed to disappear for his own benefit as well as that of the common concern. The assignees of his debt, who little expected the springing of this mine, even from such an engineer as Mr. Benfield, after recovering their first alarm, thought it best to take ground on the real state of the transaction. They divulged the whole mystery, and were prepared to plead that they had never received from Mr. Benfield any other consideration for the bond than a transfer, in trust for himself, of his demand on the Nabob of Arcot. An universal indignation arose against the perfidy of Mr. Benfield's proceeding; the event of the suit was looked upon as so certain, that Benfield was compelled to retreat as precipitately as he had advanced boldly; he gave up his bond, and was reinstated in his original demand, to wait the fortune of other claimants. At that time, and at Madras, this hope was dull indeed; but at home another scene was preparing.

It was long before any public account of this discovery at Madras had arrived in England, that the present minister and his Board of Control thought fit to determine on the debt of 1777. The recorded proceedings at this time knew nothing of any debt to Benfield. There was his own testimony, there was the testimony of the list, there was the testimony of the Nabob of Arcot, against it. Yet such was the ministers' feeling of the true secret of this transaction, that they thought proper, in the teeth of all these testimonies, to give him license to return to Madras. Here the ministers were under some embarrassment. Confounded between their resolution of rewarding the good services of Benfield's friends and associates in England, and the shame of sending that notorious incendiary to the court of the Nabob of Arcot, to renew his intrigues against the British government, at the time they authorize his return, they forbid him, under the severest penalties, from any conversation with the Nabob or his ministers: that is, they forbid his communication with the very person on account of his dealings with whom they permit his return to that city. To overtop this contradiction, there is not a word restraining him from the freest intercourse with the Nabob's second son, the real author of all that is done in the Nabob's name; who, in conjunction with this very Benfield, has acquired an absolute dominion over that unhappy man, is able to persuade him to put his signature to whatever paper they please, and often without any communication of the contents. This management was detailed to them at full length by Lord Macartney, and they cannot pretend ignorance of it.
%[64]
\footnote{ Appendix, No. 6.}


I believe, after this exposure of facts, no man can entertain a doubt of the collusion of ministers with the corrupt interest of the delinquents in India. Whenever those in authority provide for the interest of any person, on the real, but concealed state of his affairs, without regard to his avowed, public, and ostensible pretences, it must be presumed that they are in confederacy with him, because they act for him on the same fraudulent principles on which he acts for himself. It is plain that the ministers were fully apprised of Benfield's real situation, which he had used means to conceal, whilst concealment answered his purposes. They were, or the person on whom they relied was, of the cabinet council of Benfield, in the very depth of all his mysteries. An honest magistrate compels men to abide by one story. An equitable judge would not hear of the claim of a man who had himself thought proper to renounce it. With such a judge his shuffling and prevarication would have damned his claims; such a judge never would have known, but in order to animadvert upon, proceedings of that character.

I have thus laid before you, Mr. Speaker, I think with sufficient clearness, the connection of the ministers with Mr. Atkinson at the general election; I have laid open to you the connection of Atkinson with Benfield; I have shown Benfield's employment of his wealth in creating a Parliamentary interest to procure a ministerial protection; I have set before your eyes his large concern in the debt, his practices to hide that concern from the public eye, and the liberal protection which he has received from the minister. If this chain of circumstances does not lead you necessarily to conclude that the minister has paid to the avarice of Benfield the services done by Benfield's connections to his ambition, I do not know anything short of the confession of the party that can persuade you of his guilt. Clandestine and collusive practice can only be traced by combination and comparison of circumstances. To reject such combination and comparison is to reject the only means of detecting fraud; it is, indeed, to give it a patent and free license to cheat with impunity.

I confine myself to the connection of ministers, mediately or immediately, with only two persons concerned in this debt. How many others, who support their power and greatness within and without doors, are concerned originally, or by transfers of these debts, must be left to general opinion. I refer to the reports of the Select Committee for the proceedings of some of the agents in these affairs, and their attempts, at least, to furnish ministers with the means of buying General Courts, and even whole Parliaments, in the gross.
%[65]
\footnote{ Second Report of Select (General Smith's) Committee.}


I know that the ministers will think it little less than acquittal, that they are not charged with having taken to themselves some part of the money of which they have made so liberal a donation to their partisans, though the charge may be indisputably fixed upon the corruption of their politics. For my part, I follow their crimes to that point to which legal presumptions and natural indications lead me, without considering what species of evil motive tends most to aggravate or to extenuate the guilt of their conduct. But if I am to speak my private sentiments, I think that in a thousand cases for one it would be far less mischievous to the public, and full as little dishonorable to themselves, to be polluted with direct bribery, than thus to become a standing auxiliary to the oppression, usury, and peculation of multitudes, in order to obtain a corrupt support to their power. It is by bribing, not so often by being bribed, that wicked politicians bring rum on mankind. Avarice is a rival to the pursuits of many. It finds a multitude of checks, and many opposers, in every walk of life. But the objects of ambition are for the few; and every person who aims at indirect profit, and therefore wants other protection than innocence and law, instead of its rival, becomes its instrument. There is a natural allegiance and fealty due to this domineering, paramount evil, from all the vassal vices, which acknowledge its superiority, and readily militate under its banners; and it is under that discipline alone that avarice is able to spread to any considerable extent, or to render itself a general, public mischief. It is therefore no apology for ministers, that they have not been bought by the East India delinquents, but that they have only formed an alliance with them for screening each other from justice, according to the exigence of their several necessities. That they have done so is evident; and the junction of the power of office in England with the abuse of authority in the East has not only prevented even the appearance of redress to the grievances of India, but I wish it may not be found to have dulled, if not extinguished, the honor, the candor, the generosity, the good-nature, which used formerly to characterize the people of England. I confess, I wish that some more feeling than I have yet observed for the sufferings of our fellow-creatures and fellow-subjects in that oppressed part of the world had manifested itself in any one quarter of the kingdom, or in any one large description of men.

That these oppressions exist is a fact no more denied than it is resented as it ought to be. Much evil has been done in India under the British authority. What has been done to redress it? We are no longer surprised at anything. We are above the unlearned and vulgar passion of admiration. But it will astonish posterity, when they read our opinions in our actions, that, after years of inquiry, we have found out that the sole grievance of India consisted in this, that the servants of the Company there had not profited enough of their opportunities, nor drained it sufficiently of its treasures,—when they shall hear that the very first and only important act of a commission specially named by act of Parliament is, to charge upon an undone country, in favor of a handful of men in the humblest ranks of the public service, the enormous sum of perhaps four millions of sterling money.

It is difficult for the most wise and upright government to correct the abuses of remote, delegated power, productive of unmeasured wealth, and protected by the boldness and strength of the same ill-got riches. These abuses, full of their own wild native vigor, will grow and flourish under mere neglect. But where the supreme authority, not content with winking at the rapacity of its inferior instruments, is so shameless and corrupt as openly to give bounties and premiums for disobedience to its laws,—when it will not trust to the activity of avarice in the pursuit of its own gains,—when it secures public robbery by all the careful jealousy and attention with which it ought to protect property from such violence,—the commonwealth then is become totally perverted from its purposes; neither God nor man will long endure it; nor will it long endure itself. In that case, there is an unnatural infection, a pestilential taint, fermenting in the constitution of society, which fever and convulsions of some kind or other must throw off, or in which the vital powers, worsted in an unequal struggle, are pushed back upon themselves, and, by a reversal of their whole functions, fester to gangrene, to death,—and instead of what was but just now the delight and boast of the creation, there will be cast out in the face of the sun a bloated, putrid, noisome carcass, full of stench and poison, an offence, a horror, a lesson to the world.

In my opinion, we ought not to wait for the fruitless instruction of calamity to inquire into the abuses which bring upon us ruin in the worst of its forms, in the loss of our fame and virtue. But the right honorable gentleman
%[66]
\footnote{ Mr. Dundas.}
 says, in answer to all the powerful arguments of my honorable friend, "that this inquiry is of a delicate nature, and that the state will suffer detriment by the exposure of this transaction." But it is exposed; it is perfectly known in every member, in every particle, and in every way, except that which may lead to a remedy. He knows that the papers of correspondence are printed, and that they are in every hand.

He and delicacy are a rare and a singular coalition. He thinks that to divulge our Indian politics may be highly dangerous. He! the mover, the chairman, the reporter of the Committee of Secrecy! he, that brought forth in the utmost detail, in several vast, printed folios, the most recondite parts of the politics, the military, the revenues of the British empire in India! With six great chopping bastards,
%[67]
\footnote{ Six Reports of the Committee of Secrecy.}
 each as lusty as an infant Hercules, this delicate creature blushes at the sight of his new bridegroom, assumes a virgin delicacy; or, to use a more fit, as well as a more poetic comparison, the person so squeamish, so timid, so trembling lest the winds of heaven should visit too roughly, is expanded to broad sunshine, exposed like the sow of imperial augury, lying in the mud with all the prodigies of her fertility about her, as evidence of her delicate amours,—

\begin{verse}
Triginta capitum fœtus enixa jacebat,\\
Alba, solo recubans, albi circum ubera nati.
\end{verse}

Whilst discovery of the misgovernment of others led to his own power, it was wise to inquire, it was safe to publish: there was then no delicacy; there was then no danger. But when his object is obtained, and in his imitation he has outdone the crimes that he had reprobated in volumes of reports and in sheets of bills of pains and penalties, then concealment becomes prudence, and it concerns the safety of the state that we should not know, in a mode of Parliamentary cognizance, what all the world knows but too well, that is, in what manner he chooses to dispose of the public revenues to the creatures of his politics.

The debate has been long, and as much so on my part, at least, as on the part of those who have spoken before me. But long as it is, the more material half of the subject has hardly been touched on: that is, the corrupt and destructive system to which this debt has been rendered subservient, and which seems to be pursued with at least as much vigor and regularity as ever. If I considered your ease or my own, rather than the weight and importance of this question, I ought to make some apology to you, perhaps some apology to myself, for having detained your attention so long. I know on what ground I tread. This subject, at one time taken up with so much fervor and zeal, is no longer a favorite in this House. The House itself has undergone a great and signal revolution. To some the subject is strange and uncouth; to several, harsh and distasteful; to the relics of the last Parliament it is a matter of fear and apprehension. It is natural for those who have seen their friends sink in the tornado which raged during the late shift of the monsoon, and have hardly escaped on the planks of the general wreck, it is but too natural for them, as soon as they make the rocks and quicksands of their former disasters, to put about their new-built barks, and, as much as possible, to keep aloof from this perilous lee shore.

But let us do what we please to put India from our thoughts, we can do nothing to separate it from our public interest and our national reputation. Our attempts to banish this importunate duty will only make it return upon us again and again, and every time in a shape more unpleasant than the former. A government has been fabricated for that great province; the right honorable gentleman says that therefore you ought not to examine into its conduct. Heavens! what an argument is this! We are not to examine into the conduct of the Direction, because it is an old government; we are not to examine into this Board of Control, because it is a new one. Then we are only to examine into the conduct of those who have no conduct to account for. Unfortunately, the basis of this new government has been laid on old, condemned delinquents, and its superstructure is raised out of prosecutors turned into protectors. The event has been such as might be expected. But if it had been otherwise constituted, had it been constituted even as I wished, and as the mover of this question had planned, the better part of the proposed establishment was in the publicity of its proceedings, in its perpetual responsibility to Parliament. Without this check, what is our government at home, even awed, as every European government is, by an audience formed of the other states of Europe, by the applause or condemnation of the discerning and critical company before which it acts? But if the scene on the other side of the globe, which tempts, invites, almost compels, to tyranny and rapine, be not inspected with the eye of a severe and unremitting vigilance, shame and destruction must ensue. For one, the worst event of this day, though it may deject, shall not break or subdue me. The call upon us is authoritative. Let who will shrink back, I shall be found at my post. Baffled, discountenanced, subdued, discredited, as the cause of justice and humanity is, it will be only the dearer to me. Whoever, therefore, shall at any time bring before you anything towards the relief of our distressed fellow-citizens in India, and towards a subversion of the present most corrupt and oppressive system for its government, in me shall find a weak, I am afraid, but a steady, earnest, and faithful assistant.

%FOOTNOTES:
% [1] Right Honorable Henry Dundas.

% [2] Sir Thomas Rumbold, late Governor of Madras.

% [3] Appendix, No. 1.

% [4] The whole of the net Irish hereditary revenue is, on a medium of the last seven years, about 330,000l. yearly. The revenues of all denominations fall short more than 150,000l. yearly of the charges. On the present produce, if Mr. Pitt's scheme was to take place, he might gain from seven to ten thousand pounds a year.

% [5] Mr. Smith's Examination before the Select Committee. Appendix, No. 2.

% [6] Appendix, No. 2.

% [7] Fourth Report, Mr. Dundas's Committee, p. 4.

% [8] A witness examined before the Committee of Secrecy says that eighteen per cent was the usual interest, but he had heard that more had been given. The above is the account which Mr. B. received.

% [9] Mr. Dundas.

% [10] For the threats of the creditors, and total subversion of the authority of the Company in favor of the Nabob's power and the increase thereby of his evil dispositions, and the great derangement of all public concerns, see Select Committee Fort St. George's letters, 21st November, 1769, and January 31st, 1770; September 11, 1772; and Governor Bourchier's letters to the Nabob of Arcot, 21st November, 1769, and December 9th, 1769.

% [11] "He [the Nabob] is in a great degree the cause of our present inability, by diverting the revenues of the Carnatic through private channels." "Even this peshcush [the Tanjore tribute], circumstanced as he and we are, he has assigned over to others, who now set themselves in opposition to the Company."—Consultations, October 11, 1769, on the 12th communicated to the Nabob.

% [12] Nabob's letter to Governor Palk. Papers published by the Directors in 1775; and papers printed by the same authority, 1781.

% [13] See papers printed by order of a General Court in 1780, pp. 222 and 224; as also Nabob's letter to Governor Dupré, 19th July, 1771: "I have taken up loans by which I have suffered a loss of upwards of a crore of pagodas [four millions sterling] by interest on an heavy interest." Letter 15th January, 1772: "Notwithstanding I have taken much trouble, and have made many payments to my creditors, yet the load of my debt, which became so great by interest and compound interest, is not cleared."

% [14] The Nabob of Arcot.

% [15] Appendix, No. 3.

% [16] See Mr. Dundas's 1st, 2d, and 3d Reports.

% [17] See further Consultations, 3d February, 1778.

% [18] Mr. Dundas's 1st Report, pp. 26, 29, and Appendix, No. 2, 10, 18, for the mutinous state and desertion of the Nabob's troops for want of pay. See also Report IV. of the same committee.

% [19] Memorial from the creditors to the Governor and Council, 22d January, 1770.

% [20] In the year 1778, Mr. James Call, one of the proprietors of this specific debt, was actually mayor. (Appendix to 2d Report of Mr. Dundas's committee, No. 65.) The only proof which appeared on the inquiry instituted in the General Court of 1781 was an affidavit of the lenders themselves, deposing (what nobody ever denied) that they had engaged and agreed to pay—not that they had paid—the sum of 160,000l. This was two years after the transaction; and the affidavit is made before George Proctor, mayor, an attorney for certain of the old creditors.—Proceedings of the President and Council of Fort St. George, 22d February, 1779.

% [21] Right Honorable Henry Dundas.

% [22] Appendix to the 4th Report of Mr. Dundas's committee, No 15.

% [23] "No sense of the common danger, in case of a war, can prevail on him [the Nabob of Arcot] to furnish the Company with what is absolutely necessary to assemble an army, though it is beyond a doubt that money to a large amount is now hoarded up in his coffers at Chepauk; and tunkaws are granted to individuals, upon some of his most valuable countries, for payment of part of those debts which he has contracted, and which certainly will not bear inspection, as neither debtor nor creditors have ever had the confidence to submit the accounts to our examination, though they expressed a wish to consolidate the debts under the auspices of this government, agreeably to a plan they had formed."—Madras Consultations, 20th July, 1778. Mr. Dundas's Appendix to 2nd Report, 143. See also last Appendix to ditto Report, No. 376, B.

% [24] Transcriber's note: Footnote missing in original text.

% [25] Lord Pigot

% [26] In Sir Thomas Rumbold's letter to the Court of Directors, March 15th, 1778, he represents it as higher, in the following manner:—"How shall I paint to you my astonishment, on my arrival here, when I was informed, that, independent of this four lacs of pagodas [the Cavalry Loan], independent of the Nabob's debt to his old creditors, and the money due to the Company, he had contracted a debt to the enormous amount of sixty-three lacs of pagodas [2,520,000l.]. I mention this circumstance to you with horror; for the creditors being in general servants of the Company renders my task, on the part of the Company, difficult and invidious." "I have freed the sanction of this government from so corrupt a transaction. It is in my mind the most venal of all proceedings to give the Company's protection to debts that cannot bear the light; and though it appears exceedingly alarming, that a country on which you are to depend for resources should be so involved as to be nearly three years' revenue in debt,—in a country, too, where one year's revenue can never be called secure, by men who know anything of the politics of this part of India." "I think it proper to mention to you, that, although the Nabob reports his private debt to amount to upwards of sixty lacs, yet I understand that it is not quite so much." Afterwards Sir Thomas Rumbold recommended this debt to the favorable attention of the Company, but without any sufficient reason for his change of disposition. However, he went no further.

% [27] Nabob's proposals, November 25th, 1778; and memorial of the creditors, March 1st, 1779.

% [28] Nabob's proposals to his new consolidated creditors, November 25th, 1778.

% [29] Paper signed by the Nabob, 6th January, 1780.

% [30] Kistbundi to July 31, 1780.

% [31] Governor's letter to the Nabob, 25th July, 1779.

% [32] Report of the Select Committee, Madras Consultations, January 7, 1771. See also papers published by the order of the Court of Directors in 1776; and Lord Macartney's correspondence with Mr. Hastings and the Nabob of Arcot. See also Mr. Dundas's Appendix, No 376, B. Nabob's propositions through Mr. Sulivan and Assam Khân, Art. 6, and indeed the whole.

% [33] "The principal object of the expedition is, to get money from Tanjore to pay the Nabob's debt: if a surplus, to be applied in discharge of the Nabob's debts to his private creditors." (Consultations, March 20, 1771; and for further lights, Consultations, 12th June, 1771.) "We are alarmed lest this debt to individuals should have been the real motive for the aggrandizement of Mahomed Ali [the Nabob of Arcot], and that we are plunged into a war to put him in possession of the Mysore revenues for the discharge of the debt."—Letter from the Directors, March 17, 1769.

% [34] Letter from the Nabob, May 1st, 1768; and ditto, 24th April, 1770, 1st October; ditto, 16th September, 1772, 16th March, 1773.

% [35] Letter from the Presidency at Madras to the Court of Directors, 27th June, 1769.

% [36] Mr. Dundas's committee. Report L, Appendix, No. 29.

% [37] Appendix, No. 4. Report of the Committee of Assigned Revenue.

% [38] Mr. Barnard's map of the Jaghire

% [39] See Report IV., Mr. Dundas's committee, p. 46.

% [40] Interest is rated in India by the month.

% [41] Mr. Dundas's committee. Rep. I. p. 9, and ditto, Rep. IV. 69, where the revenue of 1777 stated only at 22 lacs,—30 lacs stated as the revenue, "supposing the Carnatic to be properly managed."

% [42] See Appendix, No. 4. statement in the Report of the Committee of Assigned Revenue.

% [43] The province of Tinnevelly.

% [44] Appendix, No. 5.

% [45] See extract of their letter in the Appendix, No. 9.

% [46] "It is certain that the incursion of a few of Hyder's horse into the Jaghire, in 1767, cost the Company upwards of pagodas 27,000, in allowances for damages."—Consultations, February 11th, 1771.

% [47] Proceeding at Madras, 11th February, 1769, and throughout the correspondence on this subject; particularly Consultations, October 4th, 1769, and the creditors' memorial, 20th January, 1770.

% [48] Appendix, No. 7.

% [49] For some part of these usurious transactions, see Consultation, 28th January, 1781; and for the Nabob's excusing his oppressions on account of these debts, Consultation, 26th November, 1770. "Still I undertook, first, the payment of the money belonging to the Company, who are my kind friends, and by borrowing, and mortgaging my jewels, \&c., by taking from every one of my servants, in proportion to their circumstances, by fresh severities also on my country, notwithstanding its distressed state, as you know."—The Board's remark is as follows: after controverting some of the facts, they say, "That his countries are oppressed is most certain, but not from real necessity; his debts, indeed, have afforded him a constant pretence for using severities and cruel oppressions."

% [50] See Consultation, 28th January, 1781, where it is asserted, and not denied, that the Nabob's farmers of revenue seldom continue for three months together. From this the state of the country may be easily judged of.

% [51] In Mr. Fox's speech.

% [52] The amended letter, Appendix, No. 9.

% [53] Appendix, No. 8.

% [54] Mr. Petrie's evidence before the Select Committee, Appendix, No. 7.

% [55] Appendix, No. 7.

% [56] Mr. Dundas.

% [57] See Report IV., Committee of Secrecy, pp. 73 and 74; and Appendix, in sundry places.

% [58] Mr. Smith's protest.

% [59] Madras correspondence on this subject.

% [60] Appendix, No 6.

% [61] Right Honorable William Pitt.

% [62] Appendix, No. 10.

% [63] Dated 13th October. For further illustration of the style in which these letters were written, and the principles on which they proceed, see letters from the Nabob to the Court of Directors, dated August 16th and September 7th, 1783, delivered by Mr. James Macpherson, minister to the Nabob, January 14, 1784. Appendix, No. 10.

% [64] Appendix, No. 6.

% [65] Second Report of Select (General Smith's) Committee.

% [66] Mr. Dundas.

% [67] Six Reports of the Committee of Secrecy.

\clearpage
%%%%%%%%%%%%%%%%%%%%%%%%%%%%%%%%%%%%%%%%%
\begin{center}
  \textbf{\large APPENDIX} \par 
\end{center}
\phantomsection\addcontentsline{toc}{section}{APPENDIX}
%APPENDIX.

\PRLsep
%%%%%%%%%%%%%%%%%%%%%%%%%%%%%%%%%%%%%%%%%
\begin{center}
  \textbf{\large No. 1 \\Clause of Mr Pitt's Bill} \par 
\end{center}
%\phantomsection\addcontentsline{toc}{subsection}{APPENDIX}

%No. 1.

%CLAUSES OF MR PITT'S BILL.

%Referred to from p. 17.

\begin{itquote}
Appointing Commissioners to inquire into the Fees, Gratuities, Perquisites, Emoluments, which are, or have been lately, received in the several Public Offices therein mentioned; to examine into any Abuses which may exist in the same, \&c.
\end{itquote}

And be it further enacted, that it shall and may be lawful to and for the said commissioners, or any two of them, and they are hereby empowered, authorized, and required, to examine upon oath (which oath they, or any two of them, are hereby authorized to administer) the several persons, of all descriptions, belonging to any of the offices or departments before mentioned, and all other persons whom the said commissioners, or any two of them, shall think fit to examine, touching the business of each office or department, and the fees, gratuities, perquisites, and emoluments taken therein, and touching all other matters and things necessary for the execution of the powers vested in the said commissioners by this act; all which persons are hereby required and directed punctually to attend the said commissioners, at such time and place as they, or any two of them, shall appoint, and also to observe and execute such orders and directions as the said commissioners, or any two of them, shall make or give for the purposes before mentioned.

And be it enacted by the authority aforesaid, that the said commissioners, or any two of them, shall be and are hereby empowered to examine into any corrupt and fraudulent practices, or other misconduct, committed by any person or persons concerned in the management of any of the offices or departments hereinbefore mentioned; and for the better execution of this present act, the said commissioners, or any two of them, are hereby authorized to meet and sit, from time to time, in such place or places as they shall find most convenient, with, or without adjournment, and to send their precept or precepts, under their hands and seals, for any person or persons whatsoever, and for such books, papers, writings, or records, as they shall judge necessary for their information, relating to any of the offices or departments hereinbefore mentioned; and all bailiffs, constables, sheriffs, and other his Majesty's officers, are hereby required to obey and execute such orders and precepts aforesaid as shall be sent to them, or any of them, by the said commissioners, or any two of them, touching the premises.

\PRLsep
%%%%%%%%%%%%%%%%%%%%%%%%%%%%%%%%%%%%%%%%%
\begin{center}
  \textbf{\large No. 2 \\Nabob of Arcot's Debts} \par 
\end{center}

%No. 2.

%Referred to from p. 22.

%NABOB OF ARCOT'S DEBTS.

Mr. George Smith being asked, Whether the debts of the Nabob of Arcot have increased since he knew Madras? he said, Yes, they have. He distinguishes his debts into two sorts: those contracted before the year 1766, and those contracted from that year to the year in which he left Madras.—Being asked, What he thinks is the original amount of the old debts? he said, Between twenty-three and twenty-four lacs of pagodas, as well as he can recollect.—Being asked, What was the amount of that debt when he left Madras? he said, Between four and five lacs of pagodas, as he understood.—Being asked, What was the amount of the new debt when he left Madras? he said, In November, 1777, that debt amounted, according to the Nabob's own account, and published at Chepauk, his place of residence, to sixty lacs of pagodas, independent of the old debt, on which debt of sixty lacs of pagodas the Nabob did agree to pay an interest of twelve per cent per annum.—Being asked, Whether this debt was approved of by the Court of Directors? he said, He does not know it was.—Being asked, Whether the old debt was recognized by the Court of Directors? he said, Yes, it has been; and the Court of Directors have sent out repeated orders to the President and Council of Madras to enforce its recovery and payment.—Being asked, If the interest upon the new debt is punctually paid? he said, It was not during his residence at Madras, from 1777 to 1779, in which period he thinks no more than five per cent interest was paid, in different dividends of two and one per cent.—Being asked, What is the usual course taken by the Nabob concerning the arrears of interest? he said, Not having ever lent him moneys himself, he cannot fully answer as to the mode of settling the interest with him.

Being asked, Whether he has reason to believe the sixty lacs of pagodas was all principal money really and truly advanced to the Nabob of Arcot, or a fictitious capital, made up of obligations given by him, where no money or goods were received, or which was increased by the uniting into it a greater interest than the twelve per cent expressed to be due on the capital? he said, He has no reason to believe that the sum of sixty lacs of pagodas was lent in money or goods to the Nabob, because that sum he thinks is of more value than all the money, goods, and chattels in the settlement; but he does not know in what mode or manner this debt of the Nabob's was incurred or accumulated.—Being asked, Whether it was not a general and well-grounded opinion at Madras, that a great part of this sum was accumulated by obligations, and was for services performed or to be performed for the Nabob? he said, He has heard that a part of this debt was given for the purposes mentioned in the above question, but he does not know that it was so.—Being asked, Whether it was the general opinion of the settlement? he said, He cannot say that it was the general opinion, but it was the opinion of a considerable part of the settlement.—Being asked, Whether it was the declared opinion of those that were concerned in the debt, or those that were not? he said, It was the opinion of both parties, at least such of them as he conversed with.—Being asked, Whether he has reason to believe that the interest really paid by the Nabob, upon obligations given, or money lent, did not frequently exceed twelve per cent? he said, Prior to the 1st of August, 1774, he had had reason to believe that a higher interest than twelve per cent was paid by the Nabob on moneys lent to him; but from and after that period, when the last act of Parliament took place in India, he does not know that more than twelve per cent had been paid by the Nabob, or received from him.—Being asked, Whether it is not his opinion that the Nabob has paid more than twelve per cent for money due since the 1st of August, 1774? he said, He has heard that he has, but he does not know it.—Being asked, Whether he has been told so by any considerable and weighty authority, that was like to know? he said, He has been so informed by persons who he believes had a very good opportunity of knowing it.—Being asked, Whether he was ever told so by the Nabob of Arcot himself? he said, He does not recollect that the Nabob of Arcot directly told him so, but from what he said he did infer that he paid a higher interest than twelve per cent.

Mr. Smith being asked, Whether, in the course of trade, he ever sold anything to the Nabob of Arcot? he said, In the year 1775 he did sell to the Nabob of Arcot pearls to the amount of 32,500 pagodas, for which the Nabob gave him an order or tankah on the country of Tanjore, payable in six months, without interest.—Being asked, Whether, at the time he asked the Nabob his price for the pearls, the Nabob beat down that price, as dealers commonly do? he said, No; so far from it, he offered him more than he asked by 1000 pagodas, and which he rejected.—Being asked, Whether, in settling a transaction of discount with the Nabob's agent, he was not offered a greater discount than 12l. per cent? he said, In discounting a soucar's bill for 180,000 pagodas, the Nabob's agent did offer him a discount of twenty-four per cent per annum, saving that it was the usual rate of discount paid by the Nabob; but which he would not accept of, thinking himself confined by the act of Parliament limiting the interest of moneys to twelve per cent, and accordingly he discounted the bill at twelve per cent per annum only.—Being asked, Whether he does not think those offers were made him because the Nabob thought he was a person of some consequence in the settlement? he said, Being only a private merchant, he apprehends that the offer was made to him more from its being a general practice than from any opinion of his importance.

\PRLsep
%%%%%%%%%%%%%%%%%%%%%%%%%%%%%%%%%%%%%%%%%
\begin{center}
  \textbf{\large No. 3}
  \\ \textit{A Bill for the Better Government of the Territorial Possessions and Dependencies in India.}
  \par 
\end{center}
\centerline{[ONE OF MR FOX'S INDIA BILLS]}
\vspace{0.3cm}

%No. 3.

%Referred to from p. 38.

%A Bill for the Better Government of the Territorial Possessions and Dependencies in India.

%[ONE OF MR FOX'S INDIA BILLS.]

And be it further enacted by the authority aforesaid, that the Nabob of Arcot, the Rajah of Tanjore, or any other native protected prince in India, shall not assign, mortgage, or pledge any territory or land whatsoever, or the produce or revenue thereof, to any British subject whatsoever; neither shall it be lawful to and for any British subject whatsoever to take or receive any such assignment, mortgage, or pledge; and the same are hereby declared to be null and void; and all payments or deliveries of produce or revenue, under any such assignment, shall and may be recovered back, by such native prince paying or delivering the same, from the person or persons receiving the same, or his or their representatives.

\PRLsep
%%%%%%%%%%%%%%%%%%%%%%%%%%%%%%%%%%%%%%%%%
\begin{center}
  \textbf{\large No. 4} \par 
\end{center}
%No. 4.

%Referred to from pp. 64 and 73.

\centerline{(COPY.)}

\hfill 27th May, 1782.

\noindent
\textit{Letter from the Committee of Assigned Revenue, to the President and Select Committee, dated 27th May, 1782; with Comparative Statement, and Minute thereon.}
\vspace{0.3cm}

\noindent
To the Right Honorable LORD MACARTNEY, K.B., President, and Governor, \&c., Select Committee of Fort St. George.
\vspace{0.3cm}

\noindent
MY LORD, AND GENTLEMEN,—

Although we have, in obedience to your commands of the 5th January, regularly laid before you our proceedings at large, and have occasionally addressed you upon such points as required your resolutions or orders for our guidance, we still think it necessary to collect and digest in a summary report those transactions in the management of the assigned revenue which have principally engaged our attention, and which, upon the proceeding, are too much intermixed with ordinary occurrences to be readily traced and understood.

Such a report may be formed with the greater propriety at this time, when your Lordship, \&c., have been pleased to conclude your arrangements for the rent of several of the Nabob's districts. Our aim in it is briefly to explain the state of the Carnatic at the period of the Nabob's assignment,—the particular causes which existed to the prejudice of that assignment, after it was made,—and the measures which your Lordship, \&c., have, upon our recommendation, adopted for removing those causes, and introducing a more regular and beneficial system of management in the country.

Hyder Ali having entered the Carnatic with his whole force, about the middle of July, 1780, and employed fire and sword in its destruction for near eighteen months before the Nabob's assignment took place, it will not be difficult to conceive the state of the country at that period. In those provinces which were fully exposed to the ravages of horse, scarce a vestige remained either of population or agriculture: such of the miserable inhabitants as escaped the fury of the sword were either carried into the Mysore country or left to struggle under the horrors of famine. The Arcot and Trichinopoly districts began early to feel the effects of this desolating war. Tinnevelly, Madura, and Ramnadaporum, though little infested with Hyder's troops, became a prey to the incursions of the Polygars, who stripped them of the greatest part of the revenues. Ongole, Nellore, and Palnaud, the only remaining districts, had suffered, but in a small degree.

The misfortunes of war, however, were not the only evils which the Carnatic experienced. The Nabob's aumildars, and other servants, appear to have taken advantage of the general confusion to enrich themselves. A very small part of the revenue was accounted for; and so high were the ordinary expenses of every district, that double the apparent produce of the whole country would not have satisfied them.

In this state, which we believe is no way exaggerated, the Company took charge of the assigned countries. Their prospect of relief from the heavy burdens of the war was, indeed, but little advanced by the Nabob's concession; and the revenues of the Carnatic seemed in danger of being irrecoverably lost, unless a speedy and entire change of system could be adopted.

On our minutes of the 21st January we treated the subject of the assignment at some length, and pointed out the mischiefs which, in addition to the effects of the war, had arisen from what we conceived to be wrong and oppressive management. We used the freedom to suggest an entire alteration in the mode of realizing the revenues. We proposed a considerable and immediate reduction of expenses, and a total change of the principal aumildars who had been employed under the Nabob.

Our ideas had the good fortune to receive your approbation; but the removal of the Nabob's servants being thought improper at that particular period of the collections, we employed our attention chiefly in preserving what revenue was left the country, and acquiring such materials as might lead to a more perfect knowledge of its former and present state.

These pursuits, as we apprehended, met with great obstructions from the conduct of the Nabob's servants. The orders they received were evaded under various pretexts; no attention was paid to the strong and repeated applications made to them for the accounts of their management; and their attachment to the Company's interest appeared, in every instance, so feeble, that we saw no prospect whatever of success, but in the appointment of renters under the Company's sole authority.

Upon this principle, we judged it expedient to recommend that such of the Nabob's districts as were in a state to be farmed out might be immediately let by a public advertisement, issued in the Company's name, and circulated through every province of the Carnatic; and, with the view of encouraging bidders, we proposed that the countries might be advertised for the whole period of the Nabob's assignment, and the security of the Company's protection promised in the fullest manner to such persons as might become renters.

This plan had the desired effect; and the attempts which were secretly made to counteract it afforded an unequivocal proof of its necessity: but the advantages resulting from it were more pleasingly evinced by the number of proposals that were delivered, and by the terms which were in general offered for the districts intended to be farmed out.

Having so far attained the purposes of the assignment, our attention was next turned to the heavy expenses entailed upon the different provinces; and here, we confess, our astonishment was raised to the highest pitch. In the Trichinopoly country the standing disbursements appeared, by the Nabob's own accounts, to be one lac of rupees more than the receipts. In other districts the charges were not in so high a proportion, but still rated on a most extravagant scale; and we saw, by every account that was brought before us, the absolute necessity of retrenching considerably in all the articles of expense.

Our own reason, aided by such inquiries as we were able to make, suggested the alterations we have recommended to your Lordship, \&c., under this head. You will observe that we have not acted sparingly, but we chose rather, in cases of doubt, to incur the hazard of retrenching too much than too little; because it would be easier, after any stated allowance for expenses, to add what might be necessary than to diminish. We hope, however, there will be no material increase in the articles, as they now stand.

One considerable charge upon the Nabob's country was for extraordinary sibbendies, sepoys, and horsemen, who appeared to us to be a very unnecessary incumbrance on the revenue. Your Lordship, \&c., have determined to receive such of these people as will enlist into the Company's service, and discharge the rest. This measure will not only relieve the country of a heavy burden, but tend greatly to fix in the Company that kind of authority which is requisite for the due collection of the revenues.

In consequence of your determination respecting the Nabob's sepoys, \&c., every charge under that head has been struck out of our account of expenses. If the whole number of these people be enlisted by the Company, there will probably be no more than sufficient to complete their ordinary military establishment. But should the present reduction of the Nabob's artillery render it expedient, after the war, to make any addition to the Company's establishment for the purposes of the assigned countries, the expense of such addition, whatever it be, must be deducted from the present account of savings.

In considering the charges of the several districts, in order to establish better regulations, we were careful to discriminate those incurred for troops, kept or supposed to be kept up for the defence of the country, from those of the sibbendy, servants, \&c., for the cultivation of the lands and the collection of the revenues, as well as to pay attention, to such of the established customs of the country, ancient privileges of the inhabitants, and public charities, as were necessarily allowed, and appeared proper to be continued, but which, under the Nabob's government, were not only rated much higher, but had been blended under one confused and almost unintelligible title of expenses of the districts: so joined, perhaps, to afford pleas and means of secreting and appropriating great part of the revenues to other purposes than fairly appeared; and certainly betraying the utmost neglect and mismanagement, as giving latitude for every species of fraud and oppression. Such a system has, in the few latter years of the Nabob's necessities, brought all his countries into that situation from which nothing but the most rigid economy, strict observance of the conduct of managers, and the most conciliating attention to the rights of the inhabitants can possibly recover them.

It now only remains for us to lay before your Lordship, \&c., the inclosed statement of the sums at which the districts lately advertised have been let, compared with the accounts of their produce delivered by the Nabob, and entered on our proceedings of the 21st January,—likewise a comparative view of the former and present expenses.

The Nabob's accounts of the produce of these districts state, as we have some reason to think, the sums which former renters engaged to pay to him, (and which were seldom, if ever, made good,) and not the sums actually produced by the districts; yet we have the satisfaction to observe that the present aggregate rents, upon an average, are equal to those accounts. Your Lordship, \&c., cannot, indeed, expect, that, in the midst of the danger, invasion, and distress which assail the Carnatic on every side, the renters now appointed will be able at present to fulfil the terms of their leases; but we trust, from the measures we have taken, that very little, if any, of the actual collections will be lost, even during the war,—and that, on the return of peace and tranquillity, the renters will have it in their power fully to perform their respective agreements.

We much regret that the situation of the Arcot province will not admit of the same settlement which has been made for the other districts; but the enemy being in possession of the capital, together with several other strongholds, and having entirely desolated the country, there is little room to hope for more from it than a bare subsistence to the few garrisons we have left there.

We shall not fail to give our attention towards obtaining every information respecting this province that the present times will permit, and to take the first opportunity to propose such arrangements for the management as we may think eligible.

We have the honor to be

\hspace{1in} Your most obedient humble servants,

\hspace{3in} CHARLES OAKLEY,

\hspace{3in} EYLES IRWIN,

\hspace{3in} HALL PLUMER,

\hspace{3in} DAVID HALIBURTON,

\hspace{3in} GEORGE MOUBRAY.

FORT ST. GEORGE, 27th May, 1782.

\hspace{2.5in} A true copy.

\hspace{3in} J. HUDLESTON, Sec.

% COMPARATIVE STATEMENT of the Revenues and Expenses of the Nellore, Ongole, Palnaud, Trichinopoly, Madura, and Tinnevelly Countries, while in the Hands of the Nabob, with those of the same Countries on the Terms of the Leases lately granted for Four Years, to commence with the Beginning of the Phazeley, 1192, or the 12th July, 1782. Abstracted from the Accounts received from the Nabob, and from the Rents stipulated for and Expenses allowed by the present Leases.

% GROSS REVENUE.	EXPENSES.	NET REVENUE.
% Annual Gross Rent by the Nabob's Account.
% Average of the Four Years immediately preceding the present War.	Annual Rent by the present Leases, at an Average of Four Years.	Annual Expenses by the Nabob's Accounts.	Annual Expenses allowed by the present Leases at an Estimate.	Reduction in the Annual Expenses.	Net Revenue by the Nabob's Accounts.	Net Revenue by the present Leases.	Increase of Net Revenue.
%  	Star Pagodas.	Star Pagodas.	Star Pagodas.	Star Pagodas.	Star Pagodas.	Star Pagodas.	Star Pagodas.	Star Pagodas.
% Nellore and Sarapilly	3,22,830	3,61,900	1,98,794	33,000	1,65,794	1,24,036	3,28,900	2,04,864
% Ongole	1,10,967
%[68]
%\footnote{ In this statement, the Ongole country, though it is included under the head of gross revenue, has been let for a certain sum, exclusive of charges. If the expenses specified in the Nabob's vassool accounts for this district are added, the present gross revenue even would appear to exceed the Nabob's; and as the country is only let for one year, there may hereafter be an increase of its revenue.}
%	55,000	88,254	...	88,254	22,713	55,000	32,287
% Palnaud	51,355	53,500	25,721	5,698	20,023	25,634	47,802	22,168
% Trichinopoly	2,89,993
%[69]
%\footnote{ The Trichinopoly countries let for the above sum, exclusive of the expenses of sibbendy and saderwared, amounting, by the Nabob's accounts, to rupees 1,30,00 per annum, which are to be defrayed by the renter. And the jaghires of Amir-ul-Omrah and the Begum are not included in the present lease.}
%	2,73,214	2,82,148	13,143	2,63,005	7,845	2,54,071	2,46,226
% Madura	1,02,756	60,290	63,710	12,037	51,673	39,046	48,253	9,207
% Tinnevelly	5,65,537	5,79,713	1,64,098	70,368	93,730	4,01,439	5,09,345	1,07,906	
% Total	14,43,438	13,83,617	8,22,725	1,40,246	6,82,479	6,20,713	12,43,371	6,22,658	
% N.B. In this statement, Madras Pagodas are calculated at 10 per cent Batta; Chuckrums at two thirds of a Porto Novo Pagoda, which are reckoned at 115 per 100 Star Pagodas; and Rupees at 350 per 100 Star Pagodas. To avoid fractions, the nearest integral numbers have been taken.

% Signed,

% CHARLES OAKLEY,
% EYLES IRWIN,
% HALL PLUMER,
% DAVID HALIBURTON,
% GEORGE MOUBRAY.

% FORT ST. GEORGE, 27th May, 1782.

\PRLsep
%%%%%%%%%%%%%%%%%%%%%%%%%%%%%%%%%%%%%%%%%
\begin{center}
  \textbf{\large No. 5}
  \\\textit{Case of certain Persons renting the Assigned Lands wider the Authority of the East India Company.}
  \par 
\end{center}
%No. 5.

%Referred to from p. 73.

%Case of certain Persons renting the Assigned Lands wider the Authority of the East India Company.

Extract of a Letter from the President and Council of Fort 

\centerline{St. George, 25th May, 1783.}

\vspace{0.2cm}
One of them [the renters], Ram Chunder Raus, was, indeed, one of those unfortunate rajahs whose country, by being near to the territories of the Nabob, forfeited its title to independence, and became the prey of ambition and cupidity. This man, though not able to resist the Company's arms, employed in such a deed at the Nabob's instigation, had industry and ability. He acquired, by a series of services, even the confidence of the Nabob, who suffered him to rent apart of the country of which he had deprived him of the property. This man had afforded no motive for his rejection by the Nabob, but that of being ready to engage with the Company: a motive most powerful, indeed, but not to be avowed.

[This is the person whom the English instruments of the Nabob of Arcot have had the audacity to charge with a corrupt transaction with Lord Macartney, and, in support of that charge, to produce a forged letter from his Lordship's steward. The charge and letter the reader may see in this Appendix, under the proper head. It is asserted by the unfortunate prince above mentioned, that the Company first settled on the coast of Coromandel under the protection of one of his ancestors. If this be true, (and it is far from unlikely,) the world must judge of the return the descendant has met with. The case of another of the victims given up by the ministry, though not altogether so striking as the former, is worthy of attention. It is that of the renter of the Province of Nellore.]

It is, with a wantonness of falsehood, and indifference to detection, asserted to you, in proof of the validity of the Nabob's objections, that this man's failures had already forced us to remove him: though in fact he has continued invariably in office; though our greatest supplies have been received from him; and that, in the disappointment of your remittances [the remittances from Bengal] and of other resources, the specie sent us from Nellore alone has sometimes enabled us to carry on the public business; and that the present expedition against the French must, without this assistance from the assignment, have been laid aside, or delayed until it might have become too late.

[This man is by the ministry given over to the mercy of persons capable of making charges on him "with a wantonness of falsehood, and indifference to detection." What is likely to happen to him and the rest of the victims may appear by the following.]

\vspace{0.75cm}
\centerline{\textit{Letter to the Governor-General and Council, March 13th, 1782.}}

\vspace{0.2cm}
The speedy termination, to which the people were taught to look, of the Company's interference in the revenues, and the vengeance denounced against those who, contrary to the mandate of the Durbar, should be connected with them, as reported by Mr. Sullivan, may, as much as the former exactions and oppressions of the Nabob in the revenue, as reported by the commander-in-chief, have deterred some of the fittest men from offering to be concerned in it.

The timid disposition of the Hindoo natives of this country was not likely to be insensible to the specimen of that vengeance given by his Excellency the Amir, who, upon the mere rumor, that a Bramin, of the name of Appagee Row, had given proposals to the Company for the rentership of Vellore, had the temerity to send for him, and to put him in confinement.

A man thus seized by the Nabob's sepoys within the walls of Madras gave a general alarm, and government found it necessary to promise the protection of the Company, in order to calm the apprehensions of the people.

\PRLsep
%%%%%%%%%%%%%%%%%%%%%%%%%%%%%%%%%%%%%%%%%
\begin{center}
  \textbf{\large No. 6
  } \par 
\end{center}
\textit{Extract of a Letter from the Council and Select Committee at Fort St. George, to the Governor-General and Council, dated 25th May, 1783.}
\vspace{0.3cm}

%No. 6.

%Referred to from pp. 101 and 105.

%Extract of a Letter from the Council and Select Committee at Fort St. George, to the Governor-General and Council, dated 25th May, 1783.

In the prosecution of our duty, we beseech you to consider, as an act of strict and necessary justice, previous to reiteration of your orders for the surrender of the assignment, how far it would be likely to affect third persons who do not appear to have committed any breach of their engagements. You command us to compel our aumils to deliver over their respective charges as shall be appointed by the Nabob, or to retain their trust under his sole authority, if he shall choose to confirm them. These aumils are really renters; they were appointed in the room of the Nabob's aumils, and contrary to his wishes; they have already been rejected by him, and are therefore not likely to be confirmed by him. They applied to this government, in consequence of public advertisements in our name, as possessing in this instance the joint authority of the Nabob and the Company, and have entered into mutual and strict covenants with us, and we with them, relative to the certain districts not actually in the possession of the enemy; by which covenants, as they are bound to the punctual payment of their rents and due management of the country, so we, and our constituents, and the public faith, are in like manner bound to maintain them in the enjoyment of their leases, during the continuance of the term. That term was for five years, agreeably to the words of the assignment, which declare that the time of renting shall be for three or five years, as the Governor shall settle with the renters.—Their leases cannot be legally torn from them. Nothing but their previous breach of a part could justify our breach of the whole. Such a stretch and abuse of power would, indeed, not only savor of the assumption of sovereignty, but of arbitrary and oppressive despotism. In the present contest, whether the Nabob be guilty, or we be guilty, the renters are not guilty. Whichever of the contending parties has broken the condition of the assignment, the renters have not broken the condition of their leases. These men, in conducting the business of the assignment, have acted in opposition to the designs of the Nabob, in despite of the menaces denounced against all who should dare to oppose the mandates of the Durbar justice. Gratitude and humanity require that provision should be made by you, before you set the Nabob's ministers loose on the country, for the protection of the victims devoted to their vengeance.

Mr. Benfield, to secure the permanency of his power, and the perfection of his schemes, thought it necessary to render the Nabob an absolute stranger to the state of his affairs. He assured his Highness that full justice was not done to the strength of his sentiments and the keenness of his attacks, in the translations that were made by the Company's servants from the original Persian of his letters. He therefore proposed to him that they should for the future be transmitted in English.—Of the English language or writing his Highness or the Amir cannot read one word, though the latter can converse in it with sufficient fluency. The Persian language, as the language of the Mahomedan conquerors, and of the court of Delhi, as an appendage or signal of authority, was at all times particularly affected by the Nabob. It is the language of all acts of state, and all public transactions, among the Mussulman chiefs of Hindostan. The Nabob thought to have gained no inconsiderable point, in procuring the correspondence from our predecessors to the Rajah of Tanjore to be changed from the Mahratta language, which that Hindoo prince understands, to the Persian, which he disclaims understanding. To force the Rajah to the Nabob's language was gratifying the latter with a new species of subserviency. He had formerly contended with considerable anxiety, and, it was thought, no inconsiderable cost, for particular forms of address to be used towards him in that language. But all of a sudden, in favor of Mr. Benfield, he quits his former affections, his habits, his knowledge, his curiosity, the increasing mistrust of age, to throw himself upon the generous candor, the faithful interpretation, the grateful return, and eloquent organ of Mr. Benfield!—Mr. Benfield relates and reads what he pleases to his Excellency the Amir-ul-Omrah; his Excellency communicates with the Nabob, his father, in the language the latter understands. Through two channels so pure, the truth must arrive at the Nabob in perfect refinement; through this double trust, his Highness receives whatever impression it may be convenient to make on him: he abandons his signature to whatever paper they tell him contains, in the English language, the sentiments with which they had inspired him. He thus is surrounded on every side. He is totally at their mercy, to believe what is not true, and to subscribe to what he does not mean. There is no system so new, so foreign to his intentions, that they may not pursue in his name, without possibility of detection: for they are cautious of who approach him, and have thought prudent to decline, for him, the visits of the Governor, even upon the usual solemn and acceptable occasion of delivering to his Highness the Company's letters. Such is the complete ascendency gained by Mr. Benfield. It may be partly explained by the facts observed already, some years ago, by Mr. Benfield himself, in regard to the Nabob, of the infirmities natural to his advanced age, joined to the decays of his constitution. To this ascendency, in proportion as it grew, must chiefly be ascribed, if not the origin, at least the continuance and increase, of the Nabob's disunion with this Presidency: a disunion which creates the importance and subserves the resentments of Mr. Benfield; and an ascendency which, if you effect the surrender of the assignment, will entirely leave the exercise of power and accumulation of fortune at his boundless discretion: to him, and to the Amir-ul-Omrah, and to Seyd Assam Cawn, the assignment would in fact be surrendered. HE WILL (IF ANY) BE THE SOUCAR SECURITY; and security in this country is counter-secured by possession. You would not choose to take the assignment from the Company, to give it to individuals. Of the impropriety of its returning to the Nabob, Mr. Benfield would now again argue from his former observations, that, under his Highness's management, his country declined, his people emigrated, his revenues decreased, and his country was rapidly approaching to a state of political insolvency. Of Seyd Assam Cawn we judge only from the observations this letter already contains. But of the other two persons [Amir-ul-Omrah and Mr. Benfield] we undertake to declare, not as parties in a cause, or even as voluntary witnesses, but as executive officers, reporting to you, in the discharge of our duty, and under the impression of the sacred obligation which binds us to truth, as well as to justice, that, from every observation of their principles and dispositions, and every information of their character and conduct, they have prosecuted projects to the injury and danger of the Company and individuals; that it would be improper to trust, and dangerous to employ them, in any public or important situation; that the tranquillity of the Carnatic requires a restraint to the power of the Amir; and that the Company, whose service and protection Mr. Benfield has repeatedly and recently forfeited, would be more secure against danger and confusion, if he were removed from their several Presidencies.

[After the above solemn declaration from so weighty an authority, the principal object of that awful and deliberate warning, instead of being "removed from the several Presidencies," is licensed to return to one of the principal of those Presidencies, and the grand theatre of the operations on account of which the Presidency recommends his total removal. The reason given is, for the accommodation of that very debt which has been the chief instrument of his dangerous practices, and the main cause of all the confusions in the Company's government.]

\PRLsep
%%%%%%%%%%%%%%%%%%%%%%%%%%%%%%%%%%%%%%%%%
\begin{center}
  \textbf{\large No. 7} \par 
\end{center}
\textit{Extracts from the Evidence of Mr. Petrie, late Resident for the Company at Tanjore, given to the Select Committee, relative to the Revenues and State of the Country, \&c., \&c.}
\vspace{0.3cm}
%No. 7.

%Referred to from pp. 82, 88, and 89.

%Extracts from the Evidence of Mr. Petrie, late Resident for the Company at Tanjore, given to the Select Committee, relative to the Revenues and State of the Country, \&c., \&c.

\hfill 9th May, 1782.

William Petrie, Esq., attending according to order, was asked, In what station he was in the Company's service? he said, He went to India in the year 1765, a writer upon the Madras establishment: he was employed, during the former war with Hyder Ali, in the capacity of paymaster and commissary to part of the army, and was afterwards paymaster and commissary to the army in the first siege of Tanjore, and the subsequent campaigns; then secretary to the Secret Department from 1772 to 1775; he came to England in 1775, and returned again to Madras the beginning of 1778; he was resident at the durbar of the Rajah of Tanjore from that time to the month of May; and from that time to January, 1780, was chief of Nagore and Carrical, the first of which was received from the Rajah of Tanjore, and the second was taken from the French.—Being asked, Who sent him to Tanjore? he said, Sir Thomas Rumbold, and the Secret Committee.—Being then asked, Upon what errand? he said, He went first up with a letter from the Company to the Rajah of Tanjore: he was directed to give the Rajah the strongest assurances that he should be kept in possession of his country, and every privilege to which he had been restored; he was likewise directed to negotiate with the Rajah of Tanjore for the cession of the seaport and district of Nagore in lieu of the town and district of Devicotta, which he had promised to Lord Pigot: these were the principal, and, to the best of his recollection at present, the only objects in view, when he was first sent up to Tanjore. In the course of his stay at Tanjore, other matters of business occurred between the Company and the Rajah, which came under his management as resident at that durbar.—Being asked, Whether the Rajah did deliver up to him the town and the annexed districts of Nagore voluntarily, or whether he was forced to it? he said, When he made the first proposition to the Rajah, agreeable to the directions he had received from the Secret Committee at Madras, in the most free, open, and liberal manner, the Rajah told him the seaport of Nagore was entirely at the service of his benefactors, the Company, and that he was happy in having that opportunity of testifying his gratitude to them. These may be supposed to be words of course; but, from every experience which he had of the Rajah's mind and conduct, whilst he was at Tanjore, he has reason to believe that his declarations of gratitude to the Company were perfectly sincere. He speaks to the town of Nagore at present, and a certain district,—not of the districts to the amount of which they afterwards received. The Rajah asked him, To what amount he expected a jaghire to the Company? And the witness further said, That he acknowledged to the committee that he was not instructed upon that head; that he wrote for orders to Madras, and was directed to ask the Rajah for a jaghire to a certain amount; that this gave rise to a long negotiation, the Rajah representing to him his inability to make such a gift to the Company as the Secret Committee at Madras seemed to expect; while he (the witness) on the other hand, was directed to make as good a bargain as he could for the Company. From the view that he then took of the Rajah's finances, from the situation of his country, and from the load of debt which pressed hard upon him, he believes he at different times, in his correspondence with the government, represented the necessity of their being moderate in their demands, and it was at last agreed to accept of the town of Nagore, valued at a certain annual revenue, and a jaghire annexed to the town, the whole amounting to 250,000 rupees.—Being asked, Whether it did turn out so valuable? he said, He had not a doubt but it would turn out more, as it was let for more than that to farmers at Madras, if they had managed the districts properly; but they were strangers to the manners and customs of the people; when they came down, they oppressed the inhabitants, and threw the whole district into confusion; the inhabitants, many of them, left the country, and deserted the cultivation of their lands; of course the farmers were disappointed of their collections, and they have since failed, and the Company have lost a considerable part of what the farmers were to pay for the jaghire.—Being asked, Who these farmers were? he said, One of them was the renter of the St. Thomé district, near Madras, and the other, and the most responsible, was a Madras dubash.—Being asked, Whom he was dubash to? he said, To Mr. Cass-major.

Being asked, Whether the lease was made upon higher terms than the district was rated to him by the Rajah? he said, It was.—Being then asked, What reason was assigned why the district was not kept under the former management by aumildars, or let to persons in the Tanjore country acquainted with the district? he said, No reasons were assigned: he was directed from Madras to advertise them to be let to persons of the country; but before he received any proposal, he received accounts that they were let at Madras, in consequence of public advertisements which had been made there: he believes, indeed, there were very few men in those districts responsible enough to have been intrusted with the management of those lands.—Being asked, Whether, at the time he was authorized to negotiate for Nagore in the place of Devicotta, Devicotta was given up to the Rajah? he said, No.—Being asked, Whether the Rajah of Tanjore did not frequently desire that the districts of Arnee and Hanamantagoody should be restored to him, agreeable to treaty, and the Company's orders to Lord Pigot? he said, Many a time; and he transmitted his representations regularly to Madras.—Being then asked, Whether those places were restored to him? he said, Not while he was in India.

Being asked, Whether he was not authorized and required by the Presidency at Madras to demand a large sum of money over and above the four lacs of pagodas that were to be annually paid by a grant of the Rajah, made in the time of Lord Pigot? he said, He was: to the amount, he believes, of four lacs of pagodas, commonly known by the name of deposit-money.—Being asked, Whether the Rajah did not frequently plead his inability to pay that money? he said, He did every time he mentioned it, and complained loudly of the demand.—Being asked, Whether he thinks those complaints were well founded? he says, He thinks the Rajah of Tanjore was not only not in a state of ability to pay the deposit-money, but that the annual payment of four lacs of pagodas was more than his revenues could afford.—Being asked, Whether he was not frequently obliged to borrow money, in order to pay the instalments of the annual payments, and such parts as he paid of the deposit? he said, Yes, he was.—Being asked, Where he borrowed the money? he said, He believes principally from soucars or native bankers, and some at Madras, as he told him.—Being asked, Whether he told him that his credit was very good, and that he borrowed upon moderate interest? he said, That he told him he found great difficulties in raising money, and was obliged to borrow at a most exorbitant interest, even some of it at forty-eight per cent, and he believes not a great deal under it. He desired him (the witness) to speak to one of the soucars or bankers at Tanjore to accommodate him with a loan of money: that man showed him an account between him and the Rajah, from which it appeared that he charged forty-eight per cent, besides compound interest.—Being asked, Whether the sums duo were large? he said, Yes, they were considerable; though he does not recollect the amount.—Being asked, Whether the banker lent the money? he said, He would not, unless the witness could procure him payment of his old arrears.

Being asked, What notice did the government of Madras take of the king of Tanjore's representations of the state of his affairs, and his inability to pay? he said, He does not recollect, that, in their correspondence with him, there was any reasoning upon the subject; and in his correspondence with Sir Thomas Rumbold, upon the amount of the jaghire, he seemed very desirous of adapting the demand of government to the Rajah's circumstances; but, whilst he stayed at Tanjore, the Rajah was not exonerated from any part of his burdens.—Being asked, Whether they ever desired the Rajah to make up a statement of his accounts, disbursements, debts, and payments to the Company, in order to ascertain whether the country was able to pay the increasing demands upon it? he said, Through him he is certain they never did.—Being then asked, If he ever heard whether they did through any one else? he said, He never did.

Being asked, Whether the Rajah is not bound to furnish the cultivators of land with seed for their crops, according to the custom of the country? he said, The king of Tanjore, as proprietor of the land, always makes advances of money for seed for the cultivation of the land.—Being then asked, If money beyond his power of furnishing should be extorted from him, might it not prevent, in the first instance, the means of cultivating the country? he said, It certainly does; he knows it for a fact; and he knows, that, when he left the country, there were several districts which were uncultivated from that cause.—Being asked, Whether it is not necessary to be at a considerable expense in order to keep up the mounds and watercourses? he said, A very considerable one annually.—Being asked, What would be the consequence, if money should fail for that? he said, In the first instance, the country would be partially supplied with water, some districts would be overflowed, and others would be parched.—Being asked, Whether there is not a considerable dam called the Anicut, on the keeping up of which the prosperity of the country greatly depends, and which requires a great expense? he said, Yes, there is: the whole of the Tanjore country is admirably well supplied with water, nor can he conceive any method could be fallen upon more happily adapted to the cultivation and prosperity of the country; but, as the Anicut is the source of that prosperity, any injury done to that must essentially affect all the other works in the country: it is a most stupendous piece of masonry, but, from the very great floods, frequently requiring repairs, which if neglected, not only the expense of repairing must be greatly increased, but a general injury done to the whole country.—Being asked, Whether that dam has been kept in as good preservation since the prevalence of the English government as before? he said, From his own knowledge he cannot tell, but from everything he has read or heard of the former prosperity and opulence of the kings of Tanjore, he should suppose not.—Being asked, Whether he does not know of several attempts that have been made to prevent the repair, and even to damage the work? he said, The Rajah himself frequently complained of that to him, and he has likewise heard it from others at Tanjore.—Being asked, Who it was that attempted those acts of violence? he said, He was told it was the inhabitants of the Nabob's country adjoining to the Anicut.—Being asked, Whether they were not set on or instigated by the Nabob? he answered, The Rajah said so.—And being asked, What steps the President and Council took to punish the authors and prevent those violences? he said, To the best of his recollection, the Governor told him he would make inquiries into it, but he does not know that any inquiries were made; that Sir Thomas Rumbold, the Governor, informed him that he had laid his representations with respect to the Anicut before the Nabob, who denied that his people had given any interruption to the repairs of that work.

\vspace{0.3cm}
\hfill 10th May.

Being asked, What he thinks the real clear receipt of the revenues of Tanjore were worth when he left it? he said, He cannot say what was the net amount, as he does not know the expense of the Rajah's collection; but while he was at Tanjore, he understood from the Rajah himself, and from his ministers, that the gross collection did not exceed nine lacs of pagodas (360,000l.).—Being asked, Whether he thinks the country could pay the eight lacs of pagodas which had been demanded to be paid in the course of one year? he said, Clearly not.—Being asked, Whether there was not an attempt made to remove the Rajah's minister, upon some delay in payment of the deposit? he said, The Governor of Madras wrote to that effect, which he represented to the Rajah.—Being asked, Who was mentioned to succeed to the minister that then was, in case he should be removed? he said, When Sir Hector Munro came afterwards to Tanjore, the old daubiere was mentioned, and recommended to the Rajah as successor to his then dewan.—Being asked, Of what age was the daubiere at that time? he said, Of a very great age: upwards of fourscore.—Being asked, Whether a person called Kanonga Saba Pilla was not likewise named? he said, Yes, he was: he was recommended by Sir Thomas Rumbold; and one recommendation, as well as I can recollect, went through me.—Being asked, What was the reason of his being recommended? he said, He undertook to pay off the Rajah's debts, and to give security for the regular payment of the Rajah's instalments to the Company.—Being asked, Whether he offered to give any security for preserving the country from oppression, and for supporting the dignity of the Rajah and his people? he said, He does not know that he did, or that it was asked of him.—Being asked, Whether he was a person agreeable to the Rajah? he said, He was not.—Being asked, Whether he was not a person who had fled out of the country to avoid the resentment of the Rajah? he said, He was.—Being asked, Whether he was not charged by the Rajah with malpractices, and breach of trust relative to his effects? he said, He was; but he told the Governor that he would account for his conduct, and explain everything to the satisfaction of the Rajah.—Being asked, Whether the Rajah did not consider this man as in the interest of his enemies, and particularly of the Nabob of Arcot and Mr. Benfield? he said, He does not recollect that he did mention that to him: he remembers to have heard him complain of a transaction between Kanonga Saba Pilla and Mr. Benfield; but he told him he had been guilty of a variety of malpractices in his administration, that he had oppressed the people, and defrauded him.—Being asked, In what branch of business the Rajah had formerly employed him? he said, He was at one time, he believes, renter of the whole country, was supposed to have great influence with the Rajah, and was in fact dewan some time.—Being asked, Whether the nomination of that man was not particularly odious to the Rajah? he said, He found the Rajah's mind so exceedingly averse to that man, that he believes he would almost as soon have submitted to his being deposed as to submit to the nomination of that man to be his prime-minister.

\vspace{0.3cm}
\hfill 13th May.

Mr. Petrie being asked, Whether he was informed by the Rajah, or by others, at Tanjore or Madras, that Mr. Benfield, whilst he managed the revenues at Tanjore, during the usurpation of the Nabob, did not treat the inhabitants with great rigor? he said, He did hear from the Rajah that Mr. Benfield did treat the inhabitants with rigor during the time he had anything to do with the administration of the revenues of Tanjore.—Being asked, If he recollects in what particulars? he said, The Rajah particularly complained that grain had been delivered out to the inhabitants, for the purposes of cultivation, at a higher price than the market price of grain in the country; he cannot say the actual difference of price, but it struck him at the time as something very considerable.—Being asked, Whether that money was all recovered from the inhabitants? he said, The Rajah of Tanjore told him that the money was all recovered from the inhabitants.—Being asked, Whether he did not hear that the Nabob exacted from the country of Tanjore, whilst he was in possession of it? he said, From the accounts which he received at Tanjore of the revenues for a number of years past, it appeared that the Nabob collected from the country, while he was in possession, rather more than sixteen lacs of pagodas annually; whereas, when he was at Tanjore, it did not yield more than nine lacs.—Being asked, From whence that difference arose? he said, When Tanjore was conquered for the Nabob, he has been told that many thousand of the native inhabitants fled from the country, some into the country of Mysore, and others into the dominions of the Mahrattas; he understood from the same authority, that, while the Nabob was in possession of the country, many inhabitants from the Carnatic, allured by the superior fertility and opulence of Tanjore, and encouraged by the Nabob, took up their residence there, which enabled the Nabob to cultivate the whole country; and upon the restoration of the Rajah, he has heard that the Carnatic inhabitants were carried back to their own country, which left a considerable blank in the population, which was not replaced while he was there, principally owing to an opinion which prevailed through the country that the Rajah's government was not to be permanent, but that another revolution was fast approaching. During the Nabob's government, the price of grain was considerably higher (owing to a very unusual scarcity in the Carnatic) than when he was in Tanjore.—Being asked, Whether he was ever in the Marawar country? he said, Yes; he was commissary to the army in that expedition.—Being asked, Whether that country was much wasted by the war? he said, Plunder was not permitted to the army, nor did the country suffer from its operations, except in causing many thousands of the inhabitants, who had been employed in the cultivation of the country, to leave it.—Being asked, Whether he knows what is done with the palace and inhabitants of Ramnaut? he said, The town was taken by storm, but not plundered by the troops; it was immediately delivered up to the Nabob's eldest son.—Being asked, Whether great riches were not supposed to be in that palace and temple? he said, It was universally believed so.—Being asked, What account was given of them? he said, He cannot tell; everything remained in the possession of the Nabob.—Being asked, What became of the children and women of the family of the prince of that country? he said, The Rajah was a minor; the government was in the hands of the Ranny, his mother: from general report he has heard they were carried to Trichinopoly, and placed in confinement there.—Being asked, Whether he perceived any difference in the face of the Carnatic when he first knew it and when he last knew it? he said, He thinks he did, particularly in its population.—Being asked, Whether it was better or worse? he said, It was not so populous.—Being asked, What is the condition of the Nabob's eldest son? he said, He was in the Black Town of Madras, when he left the country.—Being asked, Whether he was entertained there in a manner suitable to his birth and expectations? he said, No: he lived there without any of those exterior marks of splendor which princes of his rank in India are particularly fond of.—Being asked, Whether he has not heard that his appointments were poor and mean? he said, He has heard that they were not equal to his rank and expectations.—Being asked, Whether he had any share in the government? he said, He believes none: for some years past the Nabob has delegated most of the powers of government to his second son.—Being asked, Whether the Rajah did not complain to him of the behavior of Mr. Benfield to himself personally; and what were the particulars? he said, He did so, and related to him the following particulars. About fifteen days after Lord Pigot's confinement, Mr. Benfield came to Tanjore, and delivered the Rajah two letters from the then Governor, Mr. Stratton,—one public, and the other private. He demanded an immediate account of the presents which had been made to Lord Pigot, payment of the tunkahs which he (Mr. Benfield) had received from the Nabob upon the country, and that the Rajah should only write such letters to the Madras government as Mr. Benfield should approve and give to him. The Rajah answered, that he did not acknowledge the validity of any demands made by the Nabob upon the country; that those tunkahs related to accounts which he (the Rajah) had no concern with; that he never had given Lord Pigot any presents, but Lord Pigot had given him many; and that as to his correspondence with the Madras government, he would not trouble Mr. Benfield, because he would write his letters himself. That the Rajah told the witness, that by reason of this answer he was much threatened, in consequence of which he desired Colonel Harper, who then commanded at Tanjore, to be present at his next interview with Mr. Benfield; when Mr. Benfield denied many parts of the preceding conversation, and threw the blame upon his interpreter, Comroo. When Mr. Benfield found (as the Rajah informed him) that he could not carry these points which had brought him to Tanjore, he prepared to set off for Madras; that the Rajah sent him a letter which he had drawn out in answer to one which Mr. Benfield had brought him; that Mr. Benfield disapproved of the answer, and returned it by Comroo to the durbar, who did not deliver it into the Rajah's hands, but threw it upon the ground, and expressed himself improperly to him.

Being asked, Whether it was at the king of Tanjore's desire, that such persons as Mr. Benfield and Comroo had been brought into his presence? he said, The Rajah told him, that, when Lord Pigot came to Tanjore, to restore him to his dominions, Comroo, without being sent for, or desired to come to the palace, had found means to get access to his person: he made an offer of introducing Mr. Benfield to the Rajah, which he declined.—Being asked, Whether the military officer commanding there protected the Rajah from the intrusion of such people? he said, The Rajah did not tell him that he called upon the military officer to prevent these intrusions, but that he desired Colonel Harper to be present as a witness to what might pass between him and Mr. Benfield.—Being asked, If it is usual for persons of the conditions and occupations of Mr. Benfield and Comroo to intrude themselves into the presence of the princes of the country, and to treat them with such freedom? he said, Certainly it is not: less there than in any other country.—Being asked, Whether the king of Tanjore has no ministers to whom application might be made to transact such business as Mr. Benfield and Comroo had to do in the country? he said, Undoubtedly: his minister is the person whose province it is to transact that business.—Being asked, Before the invasion of the British troops into Tanjore, what would have been the consequence, if Mr. Benfield had intruded himself into the Rajah's presence, and behaved in that manner? he said, He could not say what would have been the consequence; but the attempt would have been madness, and could not have happened.—Being asked, Whether the Rajah had not particular exceptions to Comroo, and thought he had betrayed him in very essential points? he said, Yes, he had.—Being asked, Whether the Rajah has not been apprised that the Company have made stipulations that their servants should not interfere in the concerns of his government? he said, He signified it to the Rajah, that it was the Company's positive orders, and that any of their servants so interfering would incur their highest displeasure.

\PRLsep
%%%%%%%%%%%%%%%%%%%%%%%%%%%%%%%%%%%%%%%%%
\begin{center}
  \textbf{\large No. 8} \par 
\end{center}
\textit{Commissioners' Amended Clauses for the Fort St. George Dispatch, relative to the Indeterminate Mights and Pretensions of the Nabob of Arcot and Rajah of Tanjore.}
\vspace{0.3cm}

%No. 8.

%Referred to from p. 87, \&c.

%Commissioners' Amended Clauses for the Fort St. George Dispatch, relative to the Indeterminate Mights and Pretensions of the Nabob of Arcot and Rajah of Tanjore.

In our letter of the 28th January last we stated the reasonableness of our expectation that certain contributions towards the expenses of the war should be made by the Rajah of Tanjore. Since writing that letter, we have received one from the Rajah, of the 15th of October last, which contains at length his representations of his inability to make such further payment. We think it unnecessary here to discuss whether these representations are or are not exaggerated, because, from the explanations we have given of our wishes for a new arrangement in future, both with the Nabob of Arcot and the Rajah of Tanjore, and the directions we have given you to carry that arrangement into execution, we think it impolitic to insist upon any demands upon the Rajah for the expenses of the late war, beyond the sum of four lacs of pagodas annually: such a demand might tend to interrupt the harmony which should prevail between the Company and the Rajah, and impede the great objects of the general system we have already so fully explained to you.

But although it is not our opinion that any further claim should be made on the Rajah for his share of the extraordinary expenses of the late war, it is by no means our intention in any manner to affect the just claim which the Nabob has on the Rajah for the arrears due to him on account of peshcush, for the regular payment of which we became guaranty by the treaty of 1762; but we have already expressed to you our hopes that the Nabob may be induced to allow these arrears and the growing payments, when due, to be received by the Company, and carried in discharge of his debt to us. You are at the same time to use every means to convince him, that, when this debt shall be discharged, it is our intention, as we are bound by the above treaty, to exert ourselves to the utmost of our power to insure the constant and regular payment of it into his own hands.

We observe, by the plan sent to us by our Governor of Fort St. George, on the 30th October, 1781, that an arrangement is there proposed for the receipt of those arrears from the Rajah in three years.

We are unable to decide how far this proposal may be consistent with the present state of the Rajah's resources; but we direct you to use all proper means to bring these arrears to account as soon as possible, consistently with a due attention to this consideration.

\vspace{0.3cm}
\centerline{CLAUSES H.}

You will observe, that, by the 38th section of the late act of Parliament, it is enacted, that, for settling upon a permanent foundation the present indeterminate rights of the Nabob of Arcot and the Rajah of Tanjore with respect to each other, we should take into our immediate consideration the said indeterminate rights and pretensions, and take and pursue such measures as in our judgment and discretion shall be best calculated to ascertain and settle the same, according to the principles and the terms and stipulations contained in the treaty of 1762 between the said Nabob and the said Rajah.

On a retrospect of the proceedings transmitted to us from your Presidency, on the subject of the disputes which have heretofore arisen between the Nabob and the Rajah, we find the following points remain unadjusted, viz.

1st, Whether the jaghire of Arnee shall be enjoyed by the Nabob, or delivered up, either to the Rajah, or the descendants of Tremaul Row, the late jaghiredar.

2d, Whether the fort and district of Hanamantagoody, which is admitted by both parties to be within the Marawar, ought to be possessed by the Nabob, or to be delivered up by him to the Rajah.

3rd, To whom the government share of the crop of the Tanjore country, of the year 1775-6, properly belongs.

Lastly, Whether the Rajah has a right, by usage and custom, or ought, from the necessity of the case, to be permitted to repair such part of the Anicut, or dam and banks of the Cavery, as lie within the district of Trichinopoly, and to take earth and sand in the Trichinopoly territory for the repairs of the dam and banks within either or both of those districts.

In order to obtain a complete knowledge of the facts and circumstances relative to the several points in dispute, and how far they are connected with the treaty of 1762, we have with great circumspection examined into all the materials before us on these subjects, and will proceed to state to you the result of our inquiries and deliberations.

The objects of the treaty of 1762 appear to be restricted to the arrears of tribute to be paid to the Nabob for his past claims, and to the quantum of the Rajah's future tribute or peshcush; the cancelling of a certain bond given by the Rajah's father to the father of the Nabob; the confirmation to the Rajah of the districts of Coveladdy and Elangaud, and the restoration of Tremaul Row to his jaghire of Arnee, in condescension to the Rajah's request, upon certain stipulations, viz., that the fort of Arnee and Doby Gudy should be retained by the Nabob; that Tremaul Row should not erect any fortress, walled pagoda, or other stronghold, nor any wall round his dwelling-house exceeding eight feet high or two feet thick, and should in all things behave himself with due obedience to the government; and that he should pay yearly, in the month of July, unto the Nabob or his successors, the sum of ten thousand rupees: the Rajah thereby becoming the security for Tremaul Row, that he should in all things demean and behave himself accordingly, and pay yearly the stipulated sum.

Upon a review of this treaty, the only point now in dispute, which appears to us to be so immediately connected with it as to bring it within the strict line of our duty to ascertain and settle according to the terms and stipulations of the treaty, is that respecting Arnee. For, although the other points enumerated may in some respects have a relation to that treaty, yet, as they are foreign to the purposes expressed in it, and could not be in the contemplation of the contracting parties at the time of making it, those disputes cannot in our comprehension fall within the line of description of rights and pretensions to be now ascertained and settled by us, according to any of the terms and stipulations of it.

In respect to the jaghire of Arnee, we do not find that our records afford us any satisfactory information by what title the Rajah claims it, or what degree of relationship or connection has subsisted between the Rajah and the Killadar of Arnee, save only that by the treaty of 1762 the former became the surety for Tremaul Row's performance of his engagements specified therein, as the conditions for his restoration to that jaghire; on the death of Tremaul Row, we perceive that he was succeeded by his widow, and after her death, by his grandson Seneewasarow, both of whom were admitted to the jaghire by the Nabob.

From your Minutes of Consultation of the 31st October, 1770, and the Nabob's letter to the President of the 21st March, 1771, and the two letters from Rajah Beerbur Atchenur Punt (who we presume was then the Nabob's manager at Arcot) of the 16th and 18th March, referred to in the Nabob's letter, and transmitted therewith to the President, we observe, that, previous to the treaty of 1762, Mr. Pigot concurred in the expediency of the Nabob's taking possession of this jaghire, on account of the troublesome and refractory behavior of the Arnee braminees, by their affording protection to all disturbers, who, by reason of the little distance between Arnee and Arcot, fled to the former, and were there protected, and not given up, though demanded;—that, though the jaghire was restored in 1762, it was done under such conditions and restrictions as were thought best calculated to preserve the peace and good order of the place and due obedience to government;—that, nevertheless, the braminees (quarrelling among themselves) did afterwards, in express violation of the treaty, enlist and assemble many thousand sepoys, and other troops; that they erected gaddies and other small forts, provided themselves with wall-pieces, small guns, and other warlike stores, and raised troubles and disturbances in the neighborhood of the city of Arcot and the forts of Arnee and Shaw Gaddy; and that, finally, they imprisoned the hircarrahs of the Nabob, sent with his letters and instructions, in pursuance of the advice of your board, to require certain of the braminees to repair to the Nabob at Chepauk, and, though peremptorily required to repair thither, paid no regard to those, or to any other orders from the circar.

By the 13th article contained in the instructions given by the Nabob to Mr. Dupré, as the basis for negotiating the treaty made with the Rajah in 1771, the Nabob required that the Arnee district should be delivered up to the circar, because the braminees had broken the conditions which they were to have observed. In the answers given by the Rajah to these propositions, he says, "I am to give up to the circar the jaghire district of Arnee"; and on the 7th of November, 1771, the Rajah, by letter to Seneewasarow, who appears by your Consultations and country correspondence to have been the grandson of Tremaul Row, and to have been put in possession of the jaghire at your recommendation, (on the death of his grandmother,) writes, acquainting him that he had given the Arnee country, then in his (Seneewasarow's) possession, to the Nabob, to whose aumildars Seneewasarow was to deliver up the possession of the country. And in your letter to us of the 28th February, 1772, you certified the district of Arnee to be one of the countries acquired by this treaty, and to be of the estimated value of two lacs of rupees per annum.

In our orders dated the 12th of April, 1775, we declared our determination to replace the Rajah upon the throne of his ancestors, upon certain terms and conditions, to be agreed upon for the mutual benefit of himself and the Company, without infringing the rights of the Nabob. We declared that our faith stood pledged by the treaty of 1762 to obtain payment of the Rajah's tribute to the Nabob, and that for the insuring such payment the fort of Tanjore should be garrisoned by our troops. We directed that you should pay no regard to the article of the treaty of 1771 which respected the alienation of part of the Rajah's dominions; and we declared, that, if the Nabob had not a just title to those territories before the conclusion of the treaty, we denied that he obtained any right thereby, except such temporary sovereignty, for securing the payment of his expenses, as is therein mentioned.

These instructions appear to have been executed in the month of April, 1776; and by your letter of the 14th May following you certified to us that the Rajah had been put into the possession of the whole country his father held in 1762, when the treaty was concluded with the Nabob; but we do not find that you came to any resolution, either antecedent or subsequent to this advice, either for questioning or impeaching the right of the Nabob to the sovereignty of Arnee, or expressive of any doubt of his title to it. Nevertheless, we find, that, although the Board passed no such resolution, yet your President, in his letter to the Nabob of the 30th July and 24th August, called upon his Highness to give up the possession of Arnee to the Rajah; and the Rajah himself, in several letters to us, particularly in those of 21st October, 1776, and the 7th of June, 1777, expressed his expectation of our orders for delivering up that fort and district to him; and so recently as the 15th of October, 1783, he reminds us of his former application, and states, that the country of Arnee being guarantied to him by the Company, it of course is his right, but that it has not been given up to him, and he therefore earnestly entreats our orders for putting him into the possession of it. We also observe by your letter of the 14th of October, 1779, that the Rajah had not then accounted for the Nabob's peshcush since his restoration, but had assigned as a reason for his withdrawing it, that the Nabob had retained from him the district of Arnee, with a certain other district, (Hanamantagoody,) which is made the subject of another part of our present dispatches.

We have thus stated to you the result of our inquiry into the grounds of the dispute relative to Arnee; and as the research has offered no evidence in support of the Rajah's claim, nor even any lights whereby we can discover in what degree of relationship, by consanguinity, caste, or other circumstances, the Rajah now stands, or formerly stood, with the Killadar of Arnee, or the nature of his connection with or command over that district, or the authority he exercised or assumed previous to the treaty of 1771, we should think ourselves highly reprehensible in complying with the Rajah's request,—and the more so, as it is expressly stated, in the treaty of 1762, that this fort and district were then in the possession of the Nabob, as well as the person of the jaghiredar, on account of his disobedience, and were restored him by the Nabob, in condescension to the Rajah's request, upon such terms and stipulations as could not, in our judgment, have been imposed by the one or submitted to by the other, if the sovereignty of the one or the dependency of the other had been at that time a matter of doubt.

Although these materials have not furnished us with evidence in support of the Rajah's claim, they are far from satisfactory to evince the justice of or the political necessity for the Nabob's continuing to withhold the jaghire from the descendants of Tremaul Row; his hereditary right to that jaghire seems to us to have been fully recognized by the stipulations of the treaty of 1762, and so little doubted, that, on his death, his widow was admitted by the Nabob to hold it, on account, as may be presumed, of the nonage of his grandson and heir, Seneewasarow, who appears to have been confirmed in the jaghire, on her death, by the Nabob, as the lineal heir and successor to his grandfather.

With respect to Seneewasarow, it does not appear, by any of the Proceedings in our possession, that he was concerned in the misconduct of the braminees, complained of by the Nabob in the year 1770, which rendered it necessary for his Highness to take the jaghire into his own hands, or that he was privy to or could have prevented those disturbances.

We therefore direct, that, if the heir of Tremaul Row is not at present in possession of the jaghire, and has not, by any violation of the treaty, or act of disobedience, incurred a forfeiture thereof, he be forthwith restored to the possession of it, according to the terms and stipulations of the treaty of 1762. But if any powerful motive of regard to the peace and tranquillity of the Carnatic shall in your judgment render it expedient to suspend the execution of these orders, in that case you are with all convenient speed to transmit to us your proceedings thereupon, with the full state of the facts, and of the reasons which have actuated your conduct.

We have before given it as our opinion that the stipulations of the treaty of 1762 do not apply to the points remaining to be decided. But the late act of Parliament having, from the nature of our connection with the two powers in the Carnatic, pointed out the expediency, and even necessity, of settling the several matters in dispute between them by a speedy and permanent arrangement, we now proceed to give you our instructions upon, the several other heads of disputes before enumerated.

With respect to the fort and district of Hanamantagoody, we observe, that, on the restoration of the Rajah in 1776, you informed us in your letter of the 14th of May, That the Rajah had been put into possession of the whole of the country his father held in 1762, when the treaty was concluded with the Nabob; and on the 25th of June you came to the resolution of putting the Rajah into possession of Hanamantagoody, on the ground of its appearing, on reference to the Nabob's instructions to Mr. Dupré in June, 1762, to his reply, and to the Rajah's representations of 25th March, 1771, that Hanamantagoody was actually in the hands of the late Rajah at the time of making the treaty of 1762. We have referred as well to those papers as to all the other proceedings on this subject, and must confess they fall very short of demonstrating to us the truth of that fact. And we find, by the Secret Consultations of Fort William of the 7th of August, 1776, that the same doubt was entertained by our Governor-General and Council.

But whether, in point of fact, the late Rajah was or was not in possession of Hanamantagoody in 1762, it is notorious that the Nabob had always claimed the dominion of the countries of which this fort and district are a part.

We observe that the Nabob is now in the actual possession of this fort and district; and we are not warranted, by any document we have seen, to concur with the wishes of the Rajah to dispossess him.

With regard to the government share of the crop of 1775-6, we observe by the dobeer's memorandum, recited in your Consultations of the 13th of May, 1776, that it was the established custom of the Tanjore country to gather in the harvest and complete the collections within the month of March, but that, for the causes therein particularly stated, the harvest (and of course the collection of the government share of the crop) was delayed till the month of March was over. We also observe that the Rajah was not restored to his kingdom until the 11th of April, 1776; and from hence we infer, that, if the harvest and collection had been finished at the usual time, the Nabob (being then sovereign of the country) would have received the full benefit of that year's crop.

Although the harvest and collection were delayed beyond the usual time, yet we find by the Proceedings of your government, and particularly by Mr. Mackay's Minute of the 29th of May, 1776, and also by the dobeer's account, that the greatest part of the grain was cut down whilst the Nabob remained in the government of the country.

It is difficult, from the contradictory allegations on the subject, to ascertain what was the precise amount of the collections made after the Nabob ceased to have the possession of the country. But whatever it was, it appears from General Stuart's letter of the 2d of April, 1777, that it had been asserted with good authority that the far greater part of the government share of the crop was plundered by individuals, and never came to account in the Rajah's treasury.

Under all the circumstances of this case, we must be of opinion that the government share of the crop of 1776 belonged to the Nabob, as the then reigning sovereign of the kingdom of Tanjore, he being, de facto, in the full and absolute possession of the government thereof; and consequently that the assignments made by him of the government share of the crop were valid.

Nevertheless, we would by no means be understood by this opinion to suggest that any further demands ought to be made upon the Rajah, in respect of such parts of the government share of the crop as were collected by his people.

For, on the contrary, after so great a length of time as hath elapsed, we should think it highly unjust that the Rajah should be now compelled either to pay the supposed balances, whatever they may be, or be called upon to render a specific account of the collection made by his people.

The Rajah has already, in his letter to Governor Stratton of the 21st of April, 1777, given his assurance, that the produce of the preceding year, accounted for to him, was little more than one lac of pagodas; and as you have acquainted us, by your letter of the 14th of October, 1779, that the Rajah has actually paid into our treasury one lac of pagodas, by way of deposit, on account of the Nabob's claims to the crop, till our sentiments should be known, we direct you to surcease any further demands from the Rajah on that account.

We learn by the Proceedings, and particularly by the Nabob's letter to Lord Pigot of the 6th of July, 1776, that the Nabob, previous to the restoration of the Rajah, actually made assignments or granted tunkaws of the whole of his share of the crop to his creditors and troops; and that your government, (entertaining the same opinion as we do upon the question of right to that share,) by letter to the Rajah of the 20th of August, 1776, recommended to him "to restore to Mr. Benfield (one of the principal assignees or tunkaw-holders of the Nabob) the grain of the last year, which was in possession of his people, and said to be forcibly taken from them,—and farther, to give Mr. Benfield all reasonable assistance in recovering such debts as should appear to have been justly due to him from the inhabitants; and acquainted the Rajah that it had been judged by a majority of the Council that it was the Company's intention to let the Nabob have the produce of the crop of 1776, but that you had no intention that the Rajah should be accountable for more than the government share, whatever that might be; and that you did not mean to do more than recommend to him to see justice done, leaving the manner and time to himself." Subsequent representations appear to have been made to the Rajah by your government on the same subject, in favor of the Nabob's mortgages.

In answer to these applications, the Rajah, in his letter to Mr. Stratton of the 12th January, 1777, acquainted you "that he had given orders respecting the grain which Mr. Benfield had heaped up in his country; and with regard to the money due to him by the farmers, that he had desired Mr. Benfield to bring accounts of it, that he might limit a time for the payment of it proportionably to their ability, and that the necessary orders for stopping this money out of the inhabitants' share of the crop had been sent to the ryots and aumildars; that Mr. Benfield's gomastah was then present there, and oversaw his affairs; and that in everything that was just he (the Rajah) willingly obeyed our Governor and Council."

Our opinion being that the Rajah ought to be answerable for no more than the amount of what he admits was collected by his people for the government share of the crop; and the Proceedings before us not sufficiently explaining whether, in the sum which the Rajah, by his before-mentioned letter of the 21st April, 1777, admits to have collected, are included those parts of the government share of the crop which were taken by his people from Mr. Benfield, or from any other of the assignees or tunkaw-holders; and uninformed, as we also are, what compensation the Rajah has or has not made to Mr. Benfield, or any other of the parties from whom the grain was taken by the Rajah's people; or whether, by means of the Rajah's refusal so to do, or from any other circumstance, any of the persons dispossessed of their grain may have had recourse to the Nabob for satisfaction: we are, for these reasons, incompetent to form a proper judgment what disposition ought in justice to be made of the one lac of pagodas deposited by the Rajah. But as our sentiments and intentions are so fully expressed upon the whole subject, we presume you, who are upon the spot, can have no doubt or difficulty in making such an application of the deposit as will be consistent with those principles of justice whereon our sentiments are founded. But should any such difficulty suggest itself, you will suspend any application of the deposit, until you have fully explained the same to us, and have received our further orders.

With respect to the repairs of the Anicut and banks of the Cavery we have upon various occasions fully expressed to you our sentiments, and in particular in our general letter of the 4th July, 1777, we referred you to the investigation and correspondence on that subject of the year 1764, and to the report made by Mr. James Bourchier, on his personal survey of the waters, and to several letters of the year 1765 and 1767; we also, by our said general letter, acquainted you that it appeared to us perfectly reasonable that the Rajah should be permitted to repair those banks, and the Anicut, in the same manner as had been practised in times past; and we directed you to establish such regulations, by reference to former usage, for keeping the said banks in repair, as would be effectual, and remove all cause of complaint in future.

Notwithstanding such our instructions, the Rajah, in his letter to us of the 15th October, 1783, complains of the destruction of the Anicut; and as the cultivation of the Tanjore country appears, by all the surveys and reports of our engineers employed on that service, to depend altogether on a supply of water by the Cavery, which can only be secured by keeping the Anicut and banks in repair, we think it necessary to repeat to you our orders of the 4th July, 1777, on the subject of those repairs.

And further, as it appears by the survey and report of Mr. Pringle, that those repairs are attended with a much heavier expense, when done with materials taken from the Tanjore district, than with those of Trichinopoly, and that the last-mentioned materials are far preferable to the other, it is our order, that, if any occurrences should make it necessary or expedient, you apply to the Nabob, in our name, to desire that his Highness will permit proper spots of ground to be set out, and bounded by proper marks on the Trichinopoly side, where the Rajah and his people may at all times take sand and earth sufficient for these repairs; and that his Highness will grant his lease of such spots of land for a certain term of years to the Company, at a reasonable annual rent, to the intent that through you the cultivation of the Tanjore country may be secured, without infringing or impairing the rights of the Nabob.

If any attempts have been or shall be hereafter made to divert the water from the Cavery into the Coleroon, by contracting the current of the Upper or Lower Cavery, by planting long grass, as mentioned in Mr. Pringle's report, or by any other means, we have no doubt his Highness, on a proper representation to him in our name, will prevent his people from taking any measures detrimental to the Tanjore country, in the prosperity of which his Highness, as well as the Company, is materially interested.

Should you succeed in reconciling the Nabob to this measure, we think it but just that the proposed lease shall remain no longer in force than whilst the Rajah shall be punctual in the payment of the annual peshcush to the Nabob, as well as the rent to be reserved for the spots of ground. And in order effectually to remove all future occasions of jealousy and complaint between the parties,—that the Rajah, on the one hand, may be satisfied that all necessary works for the cultivation of his country will be made and kept in repair, and that the Nabob, on the other hand, may be satisfied that no encroachment on his rights can be made, nor any works detrimental to the fertility of his country erected,—we think it proper that it should be recommended to the parties, as a part of the adjustment of this very important point, that skilful engineers, appointed by the Company, be employed at the Rajah's expense to conduct all the necessary works, with the strictest attention to the respective rights and interests of both parties. This will remove every probability of injury or dispute. But should either party unexpectedly conceive themselves to be injured, immediate redress might be obtained by application to the government of Madras, under whose appointment the engineer will act, without any discussion between the parties, which might disturb that harmony which it is so much the wish of the Company to establish and preserve, as essential to the prosperity and peace of the Carnatic.

Having now, in obedience to the directions of the act of Parliament, upon the fullest consideration of the indeterminate rights and pretensions of the Nabob and Rajah, pointed out such measures and arrangements as in our judgment and discretion will be best calculated to ascertain and settle the same, we hope, that, upon a candid consideration of the whole system, although each of the parties may feel disappointed in our decision on particular points, they will be convinced that we have been guided in our investigation by principles of strict justice and impartiality, and that the most anxious attention has been paid to the substantial interests of both parties, and such a general and comprehensive plan of arrangement proposed as will most effectually prevent all future dissatisfaction.

Approved by the Board.

\hspace{3in} HENRY DUNDAS,

\hspace{3in} WALSINGHAM,

\hspace{3in} W.W. GRENVILLE,

\hspace{3in} MULGRAVE.

WHITEHALL, October 27, 1784.

\PRLsep
%%%%%%%%%%%%%%%%%%%%%%%%%%%%%%%%%%%%%%%%%
\begin{center}
  \textbf{\large No. 9} \par 
\end{center}
\textit{Extract of a Letter from the Court of Directors to the President and Council of Fort St. George, as amended and approved by the Board of Control.}
\vspace{0.3cm}

%No. 9.

%Referred to from pp. 78 and 85.

%Extract of a Letter from the Court of Directors to the President and Council of Fort St. George, as amended and approved by the Board of Control.

We have taken into our consideration the several advices and papers received from India, relative to the assignment of the revenues of the Carnatic, from the conclusion of the Bengal treaty to the date of your letter in October, 1783, together with the representations of the Nabob of the Carnatic upon that subject; and although we might contend that the agreement should subsist till we are fully reimbursed his Highness's proportion of the expenses of the war, yet, from a principle of moderation, and personal attachment to our old ally, his Highness the Nabob of the Carnatic, for whose dignity and happiness we are ever solicitous, and to cement more strongly, if possible, that mutual harmony and confidence which our connection makes so essentially necessary for our reciprocal safety and welfare, and for removing from his mind every idea of secret design on our part to lessen his authority over the internal government of the Carnatic, and the collection and administration of its revenues, we have resolved that the assignment shall be surrendered; and we do accordingly direct our President, in whose name the assignment was taken, without delay, to surrender the same to his Highness. But while we have adopted this resolution, we repose entire confidence in his Highness, that, actuated by the same motives of liberality, and feelings of old friendship and alliance, he will cheerfully and instantly accede to such arrangements as are necessary to be adopted for our common safety, and for preserving the respect, rights, and interests we enjoy in the Carnatic. The following are the heads and principles of such an arrangement as we are decisively of opinion must be adopted for these purposes, viz.

That, for making a provision for discharging the Nabob's just debts to the Company and individuals, (for the payment of which his Highness has so frequently expressed the greatest solicitude,) the Nabob shall give soucar security for the punctual payment, by instalments, into the Company's treasury, of twelve lacs of pagodas per annum, (as voluntarily proposed by his Highness,) until those debts, with interest, shall be discharged; and shall also consent that the equitable provision lately made by the British legislature for the liquidation of those debts, and such resolutions and determinations as we shall hereafter make, under the authority of that provision for the liquidation and adjustment of the said debts, bonâ fide incurred, shall be carried into full force and effect.

Should any difficulty arise between his Highness and our government of Fort St. George, in respect to the responsibility of the soucar security, or the times and terms of the instalments, it is our pleasure that you pay obedience to the orders and resolutions of our Governor-General and Council of Bengal in respect thereto, not doubting but the Nabob will in such case consent to abide by the determination of our said supreme government.

Although, from the great confidence we repose in the honor and integrity of the Nabob, and from an earnest desire not to subject him to any embarrassment on this occasion, we have not proposed any specific assignment of territory or revenue for securing the payments aforesaid, we nevertheless think it our duty, as well to the private creditors, whose interests in this respect have been so solemnly intrusted to us by the late act of Parliament, as from regard to the debt due to the Company, to insist on a declaration, that, in the event of the failure of the security proposed, or in default of payment at the stipulated periods, we reserve to ourselves full right to demand of the Nabob such additional security, by assignment on his country, as shall be effectual for answering the purposes of the agreement.

After having conciliated the mind of the Nabob to this measure, and adjusted the particulars, you are to carry the same into execution by a formal deed between his Highness and the Company, according to the tenor of these instructions.

As the administration of the British interests and connections in India has in some respects assumed a new shape by the late act of Parliament, and a general peace in India has been happily accomplished, the present appears to us to be the proper period, and which cannot without great imprudence be omitted, to settle and arrange, by a just and equitable treaty, a plan for the future defence and protection of the Carnatic, both in time of peace and war, on a solid and lasting foundation.

For the accomplishment of this great and necessary object, we direct you, in the name of the Company, to use your utmost endeavors to impress the expediency of, and the good effects to be derived from this measure, so strongly upon the minds of the Nabob and the Rajah of Tanjore, as to prevail upon them, jointly or separately, to enter into one or more treaty or treaties with the Company, grounded on this principle of equity: That all the contracting parties shall be bound to contribute jointly to the support of the military force and garrisons, as well in peace as in war.

That the military peace establishment shall be forthwith settled and adjusted by the Company, in pursuance of the authority and directions given to them by the late act of Parliament.

As the payment of the troops and garrisons, occasional expenses in the repairs and improvements of fortifications, and other services incidental to a military establishment, must of necessity be punctual and accurate, no latitude of personal assurance or reciprocal confidence of either of the parties on the other must be accepted or required; but the Nabob and Rajah must of necessity specify particular districts and revenues for securing the due and regular payment of their contributions into the treasury of the Company, with whom the charge of the defence of the coast, and of course the power of the sword, must be exclusively intrusted, with power for the Company, in case of failure or default of such payments at the stipulated times and seasons, to enter upon and possess such districts, and to let the same to renters, to be confirmed by the Nabob and the Rajah respectively; but, trusting that in the execution of this part of the arrangement no undue obstruction will be given by either of those powers, we direct that this part of the treaty be coupled with a most positive assurance, on our part, of our determination to support the dignity and authority of the Nabob and Rajah in the exclusive administration of the civil government and revenues of their respective countries;—and further, that, in case of any hostility committed against the territories of either of the contracting parties on the coast of Coromandel, the whole revenues of their respective territories shall be considered as one common stock, to be appropriated in the common cause of their defence; that the Company, on their part, shall engage to refrain, during the war, from the application of any part of their revenues to any commercial purposes whatsoever, but apply the whole, save only the ordinary charges of their civil government, to the purposes of the war; that the Nabob and the Rajah shall in like manner engage, on their parts, to refrain, during the war, from the application of any part of their revenues, save only what shall be actually necessary for the support of themselves and the civil government of their respective countries, to any other purposes than that of defraying the expenses of such military operations as the Company may find it necessary to carry on for the common safety of their interests on the coast of Coromandel.

And to obviate any difficulties or misunderstanding which might arise from leaving indeterminate the sum necessary to be appropriated for the civil establishment of each of the respective powers, that the sum be now ascertained which is indispensably necessary to be applied to those purposes, and which is to be held sacred under every emergency, and set apart previous to the application of the rest of the revenues, as hereby stipulated, for the purposes of mutual or common defence against any enemy, for clearing the incumbrance which may have become necessarily incurred in addition to the expenditure of those revenues which must be always deemed part of the war establishment. This we think absolutely necessary; as nothing can tend so much to the preservation of peace, and to prevent the renewal of hostilities, as the early putting the finances of the several powers upon a clear footing, and the showing to all other powers that the Company, the Nabob, and the Rajah are firmly united in one common cause, and combined in one system of permanent and vigorous defence, for the preservation of their respective territories and the general tranquillity.

That the whole aggregate revenue of the contracting parties shall, during the war, be under the application of the Company, and shall continue as long after the war as shall be necessary, to discharge the burdens contracted by it; but it must be declared that this provision shall in no respect extend to deprive either the Nabob or the Rajah of the substantial authority necessary to the collection of the revenues of their respective countries. But it is meant that they shall faithfully perform the conditions of this arrangement; and if a division of any part of the revenues to any other than the stipulated purposes shall take place, the Company shall be entitled to take upon themselves the collection of the revenue.

The Company are to engage, during the time they shall administer the revenues, to produce to the other contracting parties regular accounts of the application thereof to the purposes stipulated by the treaty, and faithfully apply them in support of the war.

And, lastly, as the defence of the Carnatic is thus to rest with the Company, the Nabob shall be satisfied of the propriety of avoiding all unnecessary expense, and will therefore agree not to maintain a greater number of troops than shall be necessary for the support of his dignity and the splendor of the durbar, which number shall be specified in the treaty; and if any military aid is requisite for the security and collection of his revenues, other than the fixed establishment employed to enforce the ordinary collections and preserve the police of the country, the Company must be bound to furnish him with such aid: the Rajah of Tanjore must likewise become bound by similar engagements, and be entitled to similar aid.

As, in virtue of the powers vested in Lord Macartney by the agreement of December, 1781, sundry leases, of various periods, have been granted to renters, we direct that you apply to the Nabob, in our name, for his consent that they may be permitted to hold their leases to the end of the stipulated term; and we have great reliance
%[70]
\footnote{ For the ground of this "great reliance," see the papers in this Appendix, No. 5. as also the Nabob's letters to the Court of Directors in this Appendix, No. 10.}
 on the liberality and spirit of accommodation manifested by the Nabob on so many occasions, that he will be disposed to acquiesce in a proposition so just and reasonable. But if, contrary to our expectations, his Highness should be impressed with any particular aversion to comply with this proposition, we do not desire you to insist upon it as an essential part of the arrangement to take place between us; but, in that event, you must take especial care to give such indemnification to the renters for any loss they may sustain as you judge to be reasonable.

It equally concerns the honor of our government, that such natives as may have been put in any degree of authority over the collections, in consequence of the deed of assignment, and who have proved faithful to their trust, shall not suffer inconvenience on account of their fidelity.

Having thus given our sentiments at large, as well for the surrender of the assignment as with regard to those arrangements which we think necessary to adopt in consequence thereof, we cannot dismiss this subject without expressing our highest approbation of the ability, moderation, and command of temper with which our President at Madras has conducted himself in the management of a very delicate and embarrassing situation. His conduct, and that of the Select Committee of Fort St. George, in the execution of the trust delegated to Lord Macartney by the Nabob Mahomed Ali, has been vigorous and effectual, for the purpose of realizing as great a revenue, at a crisis of necessity, as the nature of the case admitted; and the imputation of corruption, suggested in some of the Proceedings, appears to be totally groundless and unwarranted.

While we find so much to applaud, it is with regret we are induced to advert to anything which may appear worthy of blame: as the step of issuing the Torana Chits in Lord Macartney's own name can only be justified upon the ground of absolute necessity;
%[71]
\footnote{ For the full proof of this necessity, Lord Macartney's whole correspondence on the subject may be referred to. Without the act here condemned, not one of the acts commended in the preceding paragraph could be performed. By referring to the Nabob's letters in this Appendix it will be seen what sort of task a governor has on his hands, who is to use, according to the direction of this letter, "acts of address, civility, and conciliation," and to pay, upon all occasions, the highest attention, to persons who at the very time are falsely, and in the grossest terms, accusing him of peculation, corruption, treason, and every species of malversation in office. The recommendation, under menaces of such behavior, and under such circumstances, conveys a lesson the tendency of which cannot be misunderstood.}
 and as his Lordship had every reason to believe that the demand, when made, would be irksome and disagreeable to the feelings of Mahomed Ali, every precaution ought to have been used and more time allowed for proving that necessity, by previous acts of address, civility, and conciliation, applied for the purposes of obtaining his authority to such a measure. It appears to us that more of this might have been used; and therefore we cannot consider the omission of it as blameless, consistent with our wishes of sanctifying no act contrary to the spirit of the agreement, or derogatory to the authority of the Nabob of the Carnatic, in the exercise of any of his just rights in the government of the people under his authority.

We likewise observe, the Nabob has complained that no official communication was made to him of the peace, for near a month after the cessation of arms took place. This, and every other mark of disrespect to the Nabob, will ever appear highly reprehensible in our eyes; and we direct that you do, upon all occasions, pay the highest attention to him and his family.

Lord Macartney, in his Minute of the 9th of September last, has been fully under our consideration. We shall ever applaud the prudence and foresight of our servants which induces them to collect and communicate to us every opinion, or even ground of suspicion they may entertain, relative to any of the powers in India with whose conduct our interest and the safety of our settlements is essentially connected. At the same time we earnestly recommend that those opinions and speculations be communicated to us with prudence, discretion, and all possible secrecy, and the terms in which they are conveyed be expressed in a manner as little offensive as possible to the powers whom they may concern and into whose hands they may fall.
%[72]
\footnote{ The delicacy here recommended, in the expressions concerning conduct "with which the safety of our settlements is essentially connected," is a lesson of the same nature with the former. Dangerous designs, if truly such, ought to be expressed according to their nature and qualities. And as for the secrecy recommended concerning the designs here alluded to, nothing can be more absurd; as they appear very fully and directly in the papers published by the authority of the Court of Directors in 1775, and may be easily discerned from the propositions for the Bengal treaty, published in the Reports of the Committee of Secrecy, and in the Reports of the Select Committee. The keeping of such secrets too long has been one cause of the Carnatic war, and of the ruin of our affairs in India.}


We next proceed to give you our sentiments respecting the private debts of the Nabob; and we cannot but acknowledge that the origin and justice, both of the loan of 1767, and the loan of 1777, commonly called the Cavalry Loan, appear to us clear and indisputable, agreeable to the true sense and spirit of the late act of Parliament.

In speaking of the loan of 1767, we are to be understood as speaking of the debt as constituted by the original bonds of that year, bearing interest at 10l. per cent; and therefore, if any of the Nabob's creditors, under a pretence that their debts made part of the consolidated debt of 1767, although secured by bonds of a subsequent date, carrying an interest exceeding 10l. per cent, shall claim the benefit of the following orders, we direct that you pay no regard to such claims, without further especial instructions for that purpose.

With respect to the consolidated debt of 1777, it certainly stands upon a less favorable footing. So early as the 27th March, 1769, it was ordered by our then President and Council of Fort St. George, that, for the preventing all persons living under the Company's protection from having any dealings with any of the country powers or their ministers without the knowledge or consent of the Board, an advertisement should be published, by fixing it up at the sea-gate, and sending round a copy to the Company's servants and inhabitants, and to the different subordinates, and our garrisons, and giving it out in general orders, stating therein that the President and Council did consider the irreversible order of the Court of Directors of the year 1714 (whereby their people were prohibited from having any dealings with the country governments in money matters) to be in full force and vigor, and thereby expressly forbidding all servants of the Company, and other Europeans under their jurisdiction, to make loans or have any money transactions with any of the princes or states in India, without special license and permission of the President and Council for the time being, except only in the particular cases there mentioned, and declaring that any wilful deviation therefrom should be deemed a breach of orders, and treated as such. And on the 4th of March, 1778, it was resolved by our President and Council of Fort St George, that the consolidated debt of 1777 was not, on any respect whatever, conducted under the auspices or protection of that government; and on the circumstance of the consolidation of the said debt being made known to us, we did, on the 23rd of December, 1778, write to you in the following terms: "Your account of the Nabob's private debts is very alarming; but from whatever cause or causes those debts have been contracted or increased, we hereby repeat our orders, that the sanction of the Company be on no account given to any kind of security for the payment or liquidation of any part thereof, (except by the express authority of the Court of Directors,) on any account or pretence whatever."

The loan of 1777, therefore, has no sanction or authority from us; and in considering the situation and circumstances of this loan, we cannot omit to observe, that the creditors could not be ignorant how greatly the affairs of the Nabob were at that time deranged, and that his debt to the Company was then very considerable,—the payment of which the parties took the most effectual means to postpone, by procuring an assignment of such specific revenues for the discharge of their own debts as alone could have enabled the Nabob to have discharged that of the Company.

Under all these circumstances, we should be warranted to refuse our aid or protection in the recovery of this loan. But when we consider the inexpediency of keeping the subject of the Nabob's debts longer afloat than is absolutely necessary,—when we consider how much the final conclusion of this business will tend to promote tranquillity, credit, and circulation of property in the Carnatic,—and when we consider that the debtor concurs with the creditor in establishing the justice of those debts consolidated in 1777 into gross sums, for which bonds were given, liable to be transferred to persons different from the original creditors, and having no share or knowledge of the transactions in which the debts originated, and of course how little ground there is to expect any substantial good to result from an unlimited investigation into them, we have resolved so far to recognize the justice of those debts as to extend to them that protection which, upon more forcible grounds, we have seen cause to allow to the other two classes of debts. But although we so far adopt the general presumption in their favor as to admit them to a participation in the manner hereafter directed, we do not mean to debar you from receiving any complaints against those debts of 1777, at the instance either of the Nabob himself, or of other creditors injured by their being so admitted, or by any other persons having a proper interest, or stating reasonable grounds of objection; and if any complaints are offered, we order that the grounds of all such be attentively examined by you, and be transmitted to us, together with the evidence adduced in support of them, for our final decision; and as we have before directed that the sum of twelve lacs of pagodas, to be received annually from the Nabob, should be paid into our treasury, it is our order that the same be distributed according to the following arrangement.

That the debt be made up in the following manner, viz.

The debt consolidated in 1767 to be made up to the end of the year 1784, with the current interest at ten per cent.

The Cavalry Loan to be made up to the same period, with the current interest at twelve per cent.

The debt consolidated in 1777 to be made up to the same period, with the current interest at twelve per cent, to November, 1781, and from thence with the current interest at six per cent.

The twelve lacs annually to be received are then to be applied,—

1. To the growing interest on the Cavalry Loan, at twelve per cent.

2. To the growing interest on the debt of 1777, at six per cent.

The remainder to be equally divided: one half to be applied to the extinction of the Company's debt; the other half to be applied to the payment of growing interest at 10l. per cent, and towards the discharge of the principal of the debt of 1767.

This arrangement to continue till the principal of the debt 1767 is discharged.

The application of the twelve lacs is, then, to be,—

1. To the interest of the debt of 1777, as above. The remainder to be then equally divided,—one half towards the discharge of the current interest and principal of the Cavalry Loan, and the other half towards the discharge of the Company's debt.

When the Cavalry Loan shall be thus discharged, there shall then be paid towards the discharge of the Company's debt seven lacs.

To the growing interest and capital of the 1777 loan, five lacs.

When the Company's debt shall be discharged, the whole is then to be applied in discharge of the debt 1777.

If the Nabob shall be prevailed upon to apply the arrears and growing payments of the Tanjore peshcush in further discharge of his debts, over and above the twelve lacs of pagodas, we direct that the whole of that payment, when made, shall be applied towards the reduction of the Company's debt.

We have laid down these general rules of distribution, as appearing to us founded on justice, and the relative circumstances of the different debts; and therefore we give our authority and protection to them only on the supposition that they who ask our protection acquiesce in the condition upon which it is given; and therefore we expressly order, that, if any creditor of the Nabob, a servant of the Company, or being under our protection, shall refuse to express his acquiescence in these arrangements, he shall not only be excluded from receiving any share of the fund under your distribution, but shall be prohibited from taking any separate measures to recover his debt from the Nabob: it being one great inducement to our adopting this arrangement, that the Nabob shall be relieved from all further disquietude by the importunities of his individual creditors, and be left at liberty to pursue those measures for the prosperity of his country which the embarrassments of his situation have hitherto deprived him of the means of exerting. And we further direct, that, if any creditor shall be found refractory, or disposed to disturb the arrangement we have suggested, he shall be dismissed the service, and sent home to England.

The directions we have given only apply to the three classes of debts which have come under our observation. It has been surmised that the Nabob has of late contracted further debts: if any of these are due to British subjects, we forbid any countenance or protection whatever to be given to them, until the debt is fully investigated, the nature of it reported home, and our special instructions upon it received.

We cannot conclude this subject without adverting in the strongest terms to the prohibitions which have from time to time issued under the authority of different Courts of Directors against any of our servants, or of those under our protection, having any money transactions with any of the country powers, without the knowledge and previous consent of our respective governments abroad. We are happy to find that the Nabob, sensible of the great embarrassments, both to his own and the Company's affairs, which the enormous amount of their private claims have occasioned, is willing to engage not to incur any new debts with individuals, and we think little difficulty will be found in persuading his Highness into a positive stipulation for that purpose. And though the legislature has thus humanely interfered in behalf of such individuals as might otherwise have been reduced to great distress by the past transactions, we hereby, in the most pointed and positive terms, repeat our prohibition upon this subject, and direct that no person, being a servant of the Company, or being under our protection, shall, on any pretence whatever, be concerned in any loan or other money transaction with any of the country powers, unless with the knowledge and express permission of our respective governments. And if any of our servants, or others, being under our protection, shall be discovered in any respect counteracting these orders, we strictly enjoin you to take the first opportunity of sending them home to England, to be punished as guilty of disobedience of orders, and no protection or assistance of the Company shall be given for the recovery of any loans connected with such transactions. Your particular attention to this subject is strictly enjoined; and any connivance on your parts to a breach of our orders upon it will incur our highest displeasure. In order to put an end to those intrigues which have been so successfully carried on at the Nabob's durbar, we repeat our prohibition in the strongest terms respecting any intercourse between British subjects and the Nabob and his family; as we are convinced that such an intercourse has been carried on greatly to the detriment and expense of the Nabob, and merely to the advantage of individuals. We therefore direct that all persons who shall offend against the letter and spirit of this necessary order, whether in the Company's service or under their protection, be forthwith sent to England.

\hspace{1in} Approved by the Board.

\hspace{3in} HENRY DUNDAS,

\hspace{3in} WALSINGHAM,

\hspace{3in} W.W. GRENVILLE,

\hspace{3in} MULGRAVE.

WHITEHALL, 15th Oct. 1784.

\PRLsep

\textit{Extract from the Representation of the Court of Directors of the East India Company.}
\vspace{0.3cm}

\noindent
MY LORDS AND GENTLEMEN,—

It is with extreme concern that we express a difference of opinion with your right honorable board, in this early exercise of your controlling power; but in so novel an institution, it can scarce be thought extraordinary, if the exact boundaries of our respective functions and duties should not at once, on either side, be precisely and familiarly understood, and therefore confide in your justice and candor for believing that we have no wish to invade or frustrate the salutary purposes of your institution, as we on our part are thoroughly satisfied that you have no wish to encroach on the legal powers of the East India Company. We shall proceed to state our objections to such of the amendments as appear to us to be either insufficient, inexpedient, or unwarranted.

6th. Concerning the private debts of the Nabob of Arcot, and the application of the fund of twelve lacs of pagodas per annum.

Under this head you are pleased, in lieu of our paragraphs, to substantiate at once the justice of all those demands which the act requires us to investigate, subject only to a right reserved to the Nabob, or any other party concerned, to question the justice of any debt falling within the last of the three classes. We submit, that at least the opportunity of questioning, within the limited time, the justice of any of the debts, ought to have been fully preserved; and supposing the first and second classes to stand free from imputation, (as we incline to believe they do,) no injury can result to individuals from such discussion: and we further submit to your consideration, how far the express direction of the act to examine the nature and origin of the debts has been by the amended paragraphs complied with; and whether at least the rate of interest, according to which the debts arising from soucar assignment of the land-revenues to the servants of the Company, acting in the capacity of native bankers, have been accumulated, ought not to be inquired into, as well as the reasonableness of the deduction of twenty-five per cent which the Bengal government directed to be made from a great part of the debts on certain conditions. But to your appropriation of the fund our duty requires that we should state our strongest dissent. Our right to be paid the arrears of those expenses by which, almost to our own ruin, we have preserved the country and all the property connected with it from falling a prey to a foreign conqueror, surely stands paramount to all claims for former debts upon the revenues of a country so preserved, even if the legislature had not expressly limited the assistance to be given the private creditors to be such as should be consistent with our own rights. The Nabob had, long before passing the act, by treaty with our Bengal government, agreed to pay us seven lacs of pagodas, as part of the twelve lacs, in liquidation of those arrears; of which seven lacs the arrangement you have been pleased to lay down would take away from us more than the half, and give it to private creditors, of whose demands there are only about a sixth part which do not stand in a predicament that you declare would not entitle them to any aid or protection from us in the recovery thereof, were it not upon grounds of expediency, as will more particularly appear by the annexed estimate. Until our debt shall be discharged, we can by no means consent to give up any part of the seven lacs to the private creditors; and we humbly apprehend that in this declaration we do not exceed the limits of the authority and rights vested in us.

\PRLsep
%%%%%%%%%%%%%%%%%%%%%%%%%%%%%%%%%%%%%%%%%
\begin{center}
  THE RIGHT HONORABLE THE COMMISSIONERS FOR THE AFFAIRS OF INDIA.
  \\ \textit{The Representation of the Court of Directors of the East India Company.}
\par
\end{center}

\noindent
My Lords and Gentlemen,—

The Court, having duly attended to your reasonings and decisions on the subjects of Arnee and Hanamantagoody, beg leave to observe, with due deference to your judgment, that the directions we had given in these paragraphs which did not obtain your approbation still appear to us to have been consistent with justice, and agreeable to the late act of Parliament, which pointed out to us, as we apprehended, the treaty of 1762 as our guide.

\hspace{1in}Signed by order of the said Court,

\hspace{3in}THO. MORTON, Sec.

EAST INDIA HOUSE, the 3rd November, 1784.

\PRLsep

\textit{Extract of a Letter from the Commissioners for the Affairs of India, to the Court of Directors, dated 3rd November, 1784, in Answer to their Remonstrance.}

\centerline{SIXTH ARTICLE.}

We think it proper, considering the particular nature of the subject, to state to you the following remarks on that part of your representation which relates to the plan for the discharging of the Nabob's debts.

1st. You compute the revenue which the Carnatic may be expected to produce only at twenty lacs of pagodas. If we concurred with you in this opinion, we should certainly feel our hopes of advantage to all the parties from this arrangement considerably diminished. But we trust that we are not too sanguine on this head, when we place the greatest reliance on the estimate transmitted to you by your President of Fort St. George, having there the best means of information upon the fact, and stating it with a particular view to the subject matter of these paragraphs. Some allowance, we are sensible, must be made for the difference of collection in the Nabob's hands, but, we trust, not such as to reduce the receipt nearly to what you suppose.

2ndly. In making up the amount of the private debts, you take in compound interest at the different rates specified in our paragraph. This it was not our intention to allow; and lest any misconception should arise on the spot, we have added an express direction that the debts be made up with simple interest only, from the time of their respective consolidation. Clause F f.

3rdly. We have also the strongest grounds to believe that the debts will be in other respects considerably less than they are now computed by you; and consequently, the Company's annual proportion of the twelve lacs will be larger than it appears on your estimate. But even on your own statement of it, if we add to the 150,000l., or 3,75,000 pagodas, (which you take as the annual proportion to be received by the Company for five years to the end of 1789,) the annual amount of the Tanjore peshcush for the same period, and the arrears on the peshcush, (proposed by Lord Macartney to be received in three years,) the whole will make a sum not falling very short of pagodas 35,00,000, the amount of pagodas 7,00,000 per annum for the same period. And if we carry our calculations farther, it will appear, that, both by the plan proposed by the Nabob and adopted in your paragraphs, and by that which we transmitted to you, the debt from the Nabob, if taken at 3,000,000l., will be discharged nearly at the same period, viz., in the course of the eleventh year. We cannot, therefore, be of opinion that there is the smallest ground for objecting to this arrangement, as injurious to the interests of the Company, even if the measure were to be considered on the mere ground of expediency, and with a view only to the wisdom of reëstablishing credit and circulation in a commercial settlement, without any consideration of those motives of attention to the feelings and honor of the Nabob, of humanity to individuals, and of justice to persons in your service and living under your protection, which have actuated the legislature, and which afford not only justifiable, but commendable grounds for your conduct.

Impressed with this conviction, we have not made any alteration in the general outlines of the arrangement which we had before transmitted to you. But, as the amount of the Nabob's revenue is matter of uncertain conjecture, and as it does not appear just to us that any deficiency should fall wholly on any one class of these debts, we have added a direction to your government of Fort St. George, that, if, notwithstanding the provisions contained in our former paragraphs, any deficiency should arise, the payments of what shall be received shall be made in the same proportion which would have obtained in the division of the whole twelve lacs, had they been paid.

\PRLsep
%%%%%%%%%%%%%%%%%%%%%%%%%%%%%%%%%%%%%%%%%
\begin{center}
  \textbf{\large No. 10} \par 
\end{center}
\vspace{0.3cm}

%No. 10.

%Referred to from p. 103.

[The following extracts are subjoined, to show the matter and the style of representation employed by those who have obtained that ascendency over the Nabob of Arcot which is described in the letter marked No. 6 of the present Appendix, and which is so totally destructive of the authority and credit of the lawful British government at Madras. The charges made by these persons have been solemnly denied by Lord Macartney; and to judge from the character of the parties accused and accusing, they are probably void of all foundation. But as the letters are in the name and under the signature of a person of great rank and consequence among the natives,—as they contain matter of the most serious nature,—as they charge the most enormous crimes, and corruptions of the grossest kind, on a British governor,—and as they refer to the Nabob's minister in Great Britain for proof and further elucidation of the matters complained of,—common decency and common policy demanded an inquiry into their truth or falsehood. The writing is obviously the product of some English pen. If, on inquiry, these charges should be made good, (a thing very unlikely,) the party accused would become a just object of animadversion. If they should be found (as in all probability they would be found) false and calumnious, and supported by forgery, then the censure would fall on the accuser; at the same time the necessity would be manifest for proper measures towards the security of government against such infamous accusations. It is as necessary to protect the honest fame of virtuous governors as it is to punish the corrupt and tyrannical. But neither the Court of Directors nor the Board of Control have made any inquiry into the truth or falsehood of these charges. They have covered over the accusers and accused with abundance of compliments; they have insinuated some oblique censures; and they have recommended perfect harmony between the chargers of corruption and peculation and the persons charged with these crimes.]

\vspace{0.3cm}
\textit{13th October, 1782. Extract of a Translation of a Letter from the Nabob of Arcot to the Chairman of the Court of Directors of the East India Company.}
\vspace{0.3cm}

Fatally for me, and for the public interest, the Company's favor and my unbounded confidence have been lavished on a man totally unfit for the exalted station in which he has been placed, and unworthy of the trusts that have been reposed in him. When I speak of one who has so deeply stabbed my honor, my wounds bleed afresh, and I must be allowed that freedom of expression which the galling reflection of my injuries and my misfortunes naturally draws from me. Shall your servants, unchecked, unrestrained, and unpunished, gratify their private views and ambition at the expense of my honor, my peace, and my happiness, and to the ruin of my country, as well as of all your affairs? No sooner had Lord Macartney obtained the favorite object of his ambition than he betrayed the greatest insolence towards me, the most glaring neglect of the common civilities and attentions paid me by all former governors in the worst of times, and even by the most inveterate of my enemies. He insulted my servants, endeavored to defame my character by unjustly censuring my administration, and extended his boundless usurpation to the whole government of my dominions, in all the branches of judicature and police; and, in violation of the express articles of the agreements, proceeded to send renters into the countries, unapproved of by me, men of bad character, and unequal to my management or responsibility. Though he is chargeable with the greatest acts of cruelty, even to the shedding the blood and cutting off the noses and ears of my subjects, by those exercising his authority in the countries, and that even the duties of religion and public worship have been interrupted or prevented, and though he carries on all his business by the arbitrary exertion of military force, yet does he not collect from the countries one fourth of the revenue that should be produced. The statement he pretends to hold forth of expected revenue is totally fallacious, and can never be realized under the management of his Lordship, in the appointment of renters totally disqualified, rapacious, and irresponsible, who are actually embezzling and dissipating the public revenues that should assist in the support of the war. Totally occupied by his private views, and governed by his passions, he has neglected or sacrificed all the essential objects of public good, and by want of coöperation with Sir Eyre Coote, and refusal to furnish the army with the necessary supplies, has rendered the glorious and repeated victories of the gallant general ineffectual to the expulsion of our cruel enemy. To cover his insufficiency, and veil the discredit attendant on his failure in every measure, he throws out the most illiberal expressions, and institutes unjust accusations against me; and in aggravation of all the distresses imposed upon me, he has abetted the meanest calumniators to bring forward false charges against me and my son, Amir-ul-Omrah, in order to create embarrassment, and for the distress of my mind. My papers and writings sent to you must testify to the whole world the malevolence of his designs, and the means that have been used to forward them. He has violently seized and opened all letters addressed to me and my servants, on my public and private affairs. My vackeel, that attended him according to ancient custom, has been ignominiously dismissed from his presence, and not suffered to approach the Government-House. He has in the meanest manner, and as he thought in secret, been tampering and intriguing with my family and relations for the worst of purposes. And if I express the agonies of my mind under these most pointed injuries and oppressions, and complain of the violence and injustice of Lord Macartney, I am insulted by his affected construction that my communications are dictated by the insinuations of others, at the same time that his conscious apprehensions for his misconduct have produced the most abject applications to me to smother my feelings, and entreaties to write in his Lordship's favor to England, and to submit all my affairs to his direction. When his submissions have failed to mould me to his will, he has endeavored to effect his purposes by menaces of his secret influence with those in power in England, which he pretends to assert shall be effectual to confirm his usurpation, and to deprive me, and my family, in succession, of my rights of sovereignty and government forever. To such a length have his passions and violences carried him, that all my family, my dependants, and even my friends and visitors, are persecuted with the strongest marks of his displeasure. Every shadow of authority in my person is taken from me, and respect to my name discouraged throughout the whole country. When an officer of high rank in his Majesty's service was some time since introduced to me by Lord Macartney, his Lordship took occasion to show a personal derision and contempt of me. Mr. Richard Sulivan, who has attended my durbar under the commission of the Governor-General and Council of Bengal, has experienced his resentment; and Mr. Benfield, with whom I have no business, and who, as he has been accustomed to do for many years, has continued to pay me his visits of respect, has felt the weight of his Lordship's displeasure, and has had every unmerited insinuation thrown out against him, to prejudice him, and deter him from paying me his compliments as usual.

Thus, Gentlemen, have you delivered me over to a stranger; to a man unacquainted with government and business, and too opinionated to learn; to a man whose ignorance and prejudices operate to the neglect of every good measure, or the liberal coöperation with any that wish well to the public interests; to a man who, to pursue his own passions, plans, and designs, will certainly ruin all mine, as well as the Company's affairs. His mismanagement and obstinacy have caused the loss of many lacs of my revenues, dissipated and embezzled, and every public consideration sacrificed to his vanity and private views. I beg to offer an instance in proof of my assertions, and to justify the hope I have that you will cause to be made good to me all the losses I have sustained by the maladministration and bad practices of your servants, according to all the account of receipts of former years, and which I made known to Lord Macartney, amongst other papers of information, in the beginning of his management in the collections. The district of Ongole produced annually, upon a medium of many years, 90,000 pagodas; but Lord Macartney, upon receiving a sum of money from Ramchundry
%[73]
\footnote{ See Tellinga letter, at the end of this correspondence.}
 let it out to him, in April last, for the inadequate rent of 50,000 pagodas per annum, diminishing, in this district alone, near half the accustomed revenues. After this manner hath he exercised his powers over the countries, to suit his own purposes and designs; and this secret mode has he taken to reduce the collections.

\PRLsep

\textit{1st November, 1782. Copy of a Letter from the Nabob of Arcot to the Court of Directors, \&c. Received 7th April, 1783.}
\vspace{0.3cm}

The distresses which I have set forth in my former letters are now increased to such an alarming pitch by the imprudent measures of your Governor, and by the arbitrary and impolitic conduct pursued with the merchants and importers of grain, that the very existence of the Fort of Madras seems at stake, and that of the inhabitants of the settlement appears to have been totally overlooked: many thousands have died, and continue hourly to perish of famine, though the capacity of one of your youngest servants, with diligence and attention, by doing justice, and giving reasonable encouragement to the merchants, and by drawing the supplies of grain which the northern countries would have afforded, might have secured us against all those dreadful calamities. I had with much difficulty procured and purchased a small quantity of rice, for the use of myself, my family, and attendants, and with a view of sending off the greatest part of the latter to the northern countries, with a little subsistence in their hands. But what must your surprise be, when you learn that even this rice was seized by Lord Macartney, with a military force! and thus am I unable to provide for the few people I have about me, who are driven to such extremity and misery that it gives me pain to behold them. I have desired permission to get a little rice from the northern countries for the subsistence of my people, without its being liable to seizure by your sepoys: this even has been refused me by Lord Macartney. What must your feelings be, on such wanton cruelty exercised towards me, when you consider, that, of thousands of villages belonging to me, a single one would have sufficed for my subsistence!

\PRLsep

\textit{22d March, 1783. Translation of a Letter from the Nabob of Arcot to the Chairman and Directors of the East India Company. Received from Mr. James Macpherson, 1st January, 1784.}
\vspace{0.3cm}

I am willing to attribute this continued usurpation to the fear of detection in Lord Macartney: he dreads the awful day when the scene of his enormities will be laid open, at my restoration to my country, and when the tongues of my oppressed subjects will be unloosed, and proclaim aloud the cruel tyrannies they have sustained. These sentiments of his Lordship's designs are corroborated by his sending, on the 10th instant, two gentlemen to me and my son, Amir-ul-Omrah; and these gentlemen from Lord Macartney especially set forth to me, and to my son, that all dependence on the power of the superior government of Bengal to enforce the intentions of the Company to restore my country was vain and groundless,—that the Company confided in his Lordship's judgment and discretion, and upon his representations, and that if I, and my son, Amir-ul-Omrah, would enter into friendship with Lord Macartney, and sign a paper declaring all my charges and complaints against him to be false, that his Lordship might be induced to write to England that all his allegations against me and my son were not well founded, and, notwithstanding his declarations to withhold my country, yet, on these considerations, it might be still restored to me.

What must be your feelings for your ancient and faithful friend, on his receiving such insults to his honor and understanding from your principal servant, armed with your authority! From these manoeuvres, amongst thousands I have experienced, the truth must evidently appear to you, that I have not been loaded with those injuries and oppressions from motives of public service, but to answer the private views and interests of his Lordship and his secret agents: some papers to this point are inclosed; others, almost without number, must be submitted to your justice, when time and circumstances shall enable me fully to investigate those transactions. This opportunity will not permit the full representation of my load of injuries and distresses: I beg leave to refer you to my minister, Mr. Macpherson, for the papers, according to the inclosed list, which accompanied my last dispatches by the Rodney, which I fear have failed; and my correspondence with Lord Macartney subsequent to that period, such as I have been able to prepare for this opportunity, are inclosed.

Notwithstanding all the violent acts and declarations of Lord Macartney, yet a consciousness of his own misconduct was the sole incentive to the menaces and overtures he has held out in various shapes. He has been insultingly lavish in his expressions of high respect for my person; has had the insolence to say that all his measures flowed from his affectionate regard alone; has presumed to say that all his enmity and oppression were levelled at my son, Amir-ul-Omrah, to whom he before acknowledged every aid and assistance; and his Lordship being without any just cause or foundation for complaint against us, or a veil to cover his own violences, he has now had recourse to the meanness and has dared to intimate of my son, in order to intimidate me and to strengthen his own wicked purposes, to be in league with our enemies the French. You must doubtless be astonished, no less at the assurance than at the absurdity of such a wicked suggestion.

\vspace{0.5cm}
\centerline{IN THE NABOB'S OWN HAND.}
%\vspace{0.3cm}

P.S. In my own handwriting I acquainted Mr. Hastings, as I now do my ancient friends the Company, with the insult offered to my honor and understanding, in the extraordinary propositions sent to me by Lord Macartney, through two gentlemen, on the 10th instant, so artfully veiled with menaces, hopes, and promises. But how can Lord Macartney add to his enormities, after his wicked and calumniating insinuations, so evidently directed against me and my family, through my faithful, my dutiful, and beloved son, Amir-ul-Omrah, who, you well know, has been ever born and bred amongst the English, whom I have studiously brought up in the warmest sentiments of affection and attachment to them,—sentiments that in his maturity have been his highest ambition to improve, insomuch that he knows no happiness but in the faithful support of our alliance and connection with the English nation?

\PRLsep

\textit{12th August, and Postscript of the 16th August, 1783. Translation of a Letter to the Chairman and Directors of the East India Company. Received from Mr. James Macpherson, 14th January, 1784.}
\vspace{0.3cm}

Your astonishment and indignation will be equally raised with mine, when you hear that your President has dared, contrary to your intention, to continue to usurp the privileges and hereditary powers of the Nabob of the Carnatic, your old and unshaken friend, and the declared ally of the king of Great Britain.

I will not take up your time by enumerating the particular acts of Lord Macartney's violence, cruelty, and injustice: they, indeed, occur too frequently, and fall upon me and my devoted subjects and country too thick, to be regularly related. I refer you to my minister, Mr. James Macpherson, for a more circumstantial account of the oppressions and enormities by which he has brought both mine and the Company's affairs to the brink of destruction. I trust that such flagrant violations of all justice, honor, and the faith of treaties will receive the severest marks of your displeasure, and that Lord Macartney's conduct, in making use of your name and authority as a sanction for the continuance of his usurpation, will be disclaimed with the utmost indignation, and followed with the severest punishment. I conceive that his Lordship's arbitrary retention of my country and government can only originate in his insatiable cravings, in his implacable malevolence against me, and through fear of detection, which must follow the surrender of the Carnatic into my hands, of those nefarious proceedings which are now suppressed by the arm of violence and power.

I did not fail to represent to the supreme government of Bengal the deplorable situation to which I was reduced, and the unmerited persecutions I have unremittingly sustained from Lord Macartney; and I earnestly implored them to stretch forth a saving arm, and interpose that controlling power which was vested in them, to check rapacity and presumption, and preserve the honor and faith of the Company from violation. The Governor-General and Council not only felt the cruelty and injustice I had suffered, but were greatly alarmed for the fatal consequences that might result from the distrust of the country powers in the professions of the English, when they saw the Nabob of the Carnatic, the friend of the Company, and the ally of Great Britain, thus stripped of his rights, his dominions, and his dignity, by the most fraudulent means, and under the mask of friendship. The Bengal government had already heard both the Mahrattas and the Nizam urge, as an objection to an alliance with the English, the faithless behavior of Lord Macartney to a prince whose life had been devoted and whose treasures had been exhausted in their service and support; and they did not hesitate to give positive orders to Lord Macartney for the restitution of my government and authority, on such terms as were not only strictly honorable, but equally advantageous to my friends the Company: for they justly thought that my honor and dignity and sovereign rights were the first objects of my wishes and ambition. But how can I paint my astonishment at Lord Macartney's presumption in continuing his usurpation after their positive and reiterated mandates, and, as if nettled by their interference, which he disdained, in redoubling the fury of his violence, and sacrificing the public and myself to his malice and ungovernable passions?

I am, Gentlemen, at a loss to conceive where his usurpation will stop and have an end. Has he not solemnly declared that the assignment was only made for the support of war? and if neither your instructions nor the orders of his superiors at Bengal were to be considered as effectual, has not the treaty of peace virtually determined the period of his tyrannical administration? But so far from surrendering the Carnatic into my hands, he has, since that event, affixed advertisements to the walls and gates of the Black Town for letting to the best bidder the various districts for the term of three years,—and has continued the Committee of Revenue, which you positively ordered to be abolished, to whom he has allowed enormous salaries, from 6000 to 4000 pagodas per annum, which each member has received from the time of his appointment, though his Lordship well knows that most of them are by your orders disqualified by being my principal creditors.

If those acts of violence and outrage had been productive of public advantage, I conceive his Lordship might have held them forward in extenuation of his conduct; but whilst he cloaks his justification under the veil of your records, it is impossible to refute his assertions or to expose to you their fallacy; and when he is no longer able to support his conduct by argument, he refers to those records, where, I understand, he has exercised all his sophistry and malicious insinuations to render me and my family obnoxious in the eyes of the Company and the British nation. And when the glorious victories of Sir Eyre Coote have been rendered abortive by a constant deficiency of supplies,—and when, since the departure of that excellent general to Bengal, whose loss I must ever regret, a dreadful famine, at the close of last year, occasioned by his Lordship's neglect to lay up a sufficient stock of grain at a proper season, and from his prohibitory orders to private merchants,—and when no exertion has been made, nor advantage gained over the enemy,—when Hyder's death and Tippoo's return to his own dominions operated in no degree for the benefit of our affairs,—in short, when all has been a continued series of disappointment and disgrace under Lord Macartney's management, (and in him alone has the management been vested,)—I want words to convey those ideas of his insufficiency, ignorance, and obstinacy which I am convinced you would entertain, had you been spectators of his ruinous and destructive conduct.

But against me, and my son, Amir-ul-Omrah, has his Lordship's vengeance chiefly been exerted: even the Company's own subordinate zemindars have found better treatment, probably because they were more rich; those of Nizanagoram have been permitted, contrary to your pointed orders, to hold their rich zemindaries at the old disproportionate rate of little more than a sixth part of the real revenue; and my zemindar of Tanjore, though he should have regarded himself equally concerned with us in the event of the war, and from whose fertile country many valuable harvests have been gathered in, which have sold at a vast price, has, I understand, only contributed, last year, towards the public exigencies, the very inconsiderable sum of one lac of pagodas, and a few thousand pagodas' worth of grain.

I am much concerned to acquaint you that ever since the peace a dreadful famine has swept away many thousands of the followers and sepoys' families of the army, from Lord Macartney's neglect to send down grain to the camp, though the roads are crowded with vessels: but his Lordship has been too intent upon his own disgraceful schemes to attend to the wants of the army. The negotiation with Tippoo, which he has set on foot through the mediation of Monsieur Bussy, has employed all his thoughts, and to the attainment of that object he will sacrifice the dearest interests of the Company to gratify his malevolence against me, and for his own private advantages. The endeavor to treat with Tippoo, through the means of the French, must strike you, Gentlemen, as highly improper and impolitic; but it must raise your utmost indignation to hear, that, by intercepted letters from Bussy to Tippoo, as well as from their respective vakeels, and from various accounts from Cuddalore, we have every reason to conclude that his Lordship's secretary, Mr. Staunton, when at Cuddalore, as his agent to settle the cessation of arms with the French, was informed of all their operations and projects, and consequently that Lord Macartney has secretly connived at Monsieur Bussy's recommendation to Tippoo to return into the Carnatic, as the means of procuring the most advantageous terms, and furnishing Lord Macartney with the plea of necessity for concluding a peace after his own manner: and what further confirms the truth of this fact is, that repeated reports, as well as the alarms of the inhabitants to the westward, leave us no reason to doubt that Tippoo is approaching towards us. His Lordship has issued public orders that the garrison store of rice, for which we are indebted to the exertions of the Bengal government, should be immediately disposed of, and has strictly forbid all private grain to be sold; by which act he effectually prohibits all private importation of grain, and may eventually cause as horrid a famine as that which we experienced at the close of last year from the same shortsighted policy and destructive prohibitions of Lord Macartney.

But as he has the fabrication of the records in his own hands, he trusts to those partial representations of his character and conduct, because the signatures of those members of government whom he seldom consults are affixed, as a public sanction; but you may form a just idea of their correctness and propriety, when you are informed that his Lordship, upon my noticing the heavy disbursements made for secret service money, ordered the sums to be struck off, and the accounts to be erased from the cash-book of the Company; and I think I cannot give you a better proof of his management of my country and revenues than by calling your attention to his conduct in the Ongole province, and by referring you to his Lordship's administration of your own jaghire, from whence he has brought to the public account the sum of twelve hundred pagodas for the last year's revenue, yet blazons forth his vast merits and exertions, and expects to receive the thanks of his Committee and Council.

I will beg leave to refer you to my minister, James Macpherson, Esq., for a more particular account of my sufferings and miseries, to whom I have transmitted copies of all papers that passed with his Lordship.

I cannot conclude without calling your attention to the situation of my different creditors, whose claims are the claims of justice, and whose demands I am bound by honor and every moral obligation to discharge; it is not, therefore, without great concern I have heard insinuations tending to question the legality of their right to the payment of those just debts: they proceeded from advances made by them openly and honorably for the support of my own and the public affairs. But I hope the tongue of calumny will never drown the voice of truth and justice; and while that is heard, the wisdom of the English nation cannot fail to accede to an effectual remedy for their distresses, by any arrangement in which their claims may be duly considered and equitably provided for: and for this purpose, my minister, Mr. Macpherson, will readily subscribe, in my name, to any agreement you may think proper to adopt, founded on the same principles with either of the engagements I entered into with the supreme government of Bengal for our mutual interest and advantage.

I always pray for your happiness and prosperity.

\PRLsep

\textit{6th September, and Postscript of 7th September, 1783. Translation of a Letter from the Nabob of Arcot to the Chairman and Directors of the East India Company. Received from Mr. James Macpherson, 14th January, 1784.}
\vspace{0.3cm}

I refer you, Gentlemen, to my inclosed duplicate, as well as to my minister, Mr. Macpherson, for the particulars of my sufferings. There is no word or action of mine that is not perverted; and though it was my intention to have sent my son, Amir-ul-Omrah, who is well versed in my affairs, to Bengal, to impress those gentlemen with a full sense of my situation, yet I find myself obliged to lay it aside, from the insinuations of the calumniating tongue of Lord Macartney, that takes every license to traduce every action of my life and that of my son. I am informed that Lord Macartney, at this late moment, intends to write a letter: I am ignorant of the subject, but fully perceive, that, by delaying to send it till the very eve of the dispatch, he means to deprive me of all possibility of communicating my reply, and forwarding it for the information of my friends in England. Conscious of the weak ground on which he stands, he is obliged to have recourse to these artifices to mislead the judgment, and support for a time his unjustifiable measures by deceit and imposition. I wish only to meet and combat his charges and allegations fairly and openly, and I have repeatedly and urgently demanded to be furnished with copies of those parts of his fabricated records relative to myself; but as he well knows I should refute his sophistry, I cannot be surprised at his refusal, though I lament that it prevents you, Gentlemen, from a clear investigation of his conduct towards me.

Inclosed you have a translation of an arzee from the Killidar of Vellore. I have thousands of the same kind; but this, just now received, will serve to give you some idea of the miseries brought upon this my devoted country, and the wretched inhabitants that remain in it, by the oppressive hand of Lord Macartney's management: nor will the embezzlements of collections thus obtained, when brought before you in proof, appear less extraordinary,—which shall certainly be done in due time.

\PRLsep

\textit{Translation of an Arzee, in the Persian Language, from Uzzim-ul-Doen Cawn, the Killidar of Vellore, to the Nabob, dated 1st September, 1783. Inclosed in the Nabob's Letter to the Court of Directors, September, 1783.}
\vspace{0.3cm}

I have repeatedly represented to your Highness the violences and oppressions exercised by the present aumildar [collector of revenue], of Lord Macartney's appointment, over the few remaining inhabitants of the districts of Vellore, Amboor, Saulguda, \&c.

The outrages and violences now committed are of that astonishing nature as were never known or heard of during the administration of the Circar. Hyder Naik, the cruellest of tyrants, used every kind of oppression in the Circar countries; but even his measures were not like those now pursued. Such of the inhabitants as had escaped the sword and pillage of Hyder Naik, by taking refuge in the woods, and within the walls of Vellore, \&c., on the arrival of Lord Macartney's aumildar to Vellore, and in consequence of his cowle of protection and support, most cheerfully returned to the villages, set about the cultivation of the lands, and with great pains rebuilt their cottages.—But now the aumildar has imprisoned the wives and children of the inhabitants, seized the few jewels that were on the bodies of the women, and then, before the faces of their husbands, flogged them, in order to make them produce other jewels and effects, which he said they had buried somewhere under ground, and to make the inhabitants bring him money, notwithstanding there was yet no cultivation in the country. Terrified with the flagellations, some of them produced their jewels and wearing-apparel of their women, to the amount of ten or fifteen pagodas, which they had hidden; others, who declared they had none, the aumildar flogged their women severely, tied cords around their breasts, and tore the sucking children from their teats, and exposed them to the scorching heat of the sun. Those children died, as did the wife of Ramsoamy, an inhabitant of Bringpoor. Even this could not stir up compassion in the breast of the aumildar. Some of the children that were somewhat large he exposed to sale. In short, the violences of the aumildar are so astonishing, that the people, on seeing the present situation, remember the loss of Hyder with regret. With whomsoever the aumildar finds a single measure of natchinee or rice, he takes it away from him, and appropriates it to the expenses of the sibindy that he keeps up. No revenues are collected from the countries, but from the effects of the poor, wretched inhabitants. Those ryots [yeomen] who intended to return to their habitations, hearing of those violences, have fled for refuge, with their wives and children, into Hyder's country. Every day is ushered in and closed with these violences and disturbances. I have no power to do anything; and who will hear what I have to say? My business is to inform your Highness, who are my master. The people bring their complaints to me, and I tell them I will write to your Highness.
%[74]
\footnote{ The above-recited practices, or practices similar to them, have prevailed in almost every part of the miserable countries on the coast of Coromandel for near twenty years past. That they prevailed as strongly and generally as they could prevail, under the administration of the Nabob, there can be no question, notwithstanding the assertion in the beginning of the above petition; nor will it ever be otherwise, whilst affairs are conducted upon the principles which influence the present system. Whether the particulars here asserted are true or false neither the Court of Directors nor their ministry have thought proper to inquire. If they are true, in order to bring them to affect Lord Macartney, it ought to be proved that the complaint was made to him, and that he had refused redress. Instead of this fair course, the complaint is carried to the Court of Directors.—The above is one of the documents transmitted by the Nabob, in proof of his charge of corruption against Lord Macartney. If genuine, it is conclusive, at least against Lord Macartney's principal agent and manager. If it be a forgery, (as in all likelihood it is,) it is conclusive against the Nabob and his evil counsellors, and folly demonstrates, if anything further were necessary to demonstrate, the necessity of the clause in Mr. Fox's bill prohibiting the residence of the native princes in the Company's principal settlements,—which clause was, for obvious reasons, not admitted into Mr. Pitt's. It shows, too, the absolute necessity of a severe and exemplary punishment on certain of his English evil counsellors and creditors, by whom such practices are carried on.}


\PRLsep

\textit{Translation of a Tellinga Letter from Veira Permaul, Head Dubash to Lord Macartney, in his own Handwriting, to Rajah Ramchunda, the Renter of Ongole. Dated 25th of the Hindoo month Mausay, in the year Plavanamal, corresponding to 5th March, 1782.}
\vspace{0.3cm}

I present my respects to you, and am very well here, wishing to hear frequently of your welfare.

Your peasher Vancatroyloo has brought the Visseel Bakees, and delivered them to me, as also what you sent him for me to deliver to my master, which I have done. My master at first refused to take it, because he is unacquainted with your disposition, or what kind of a person you are. But after I made encomiums on your goodness and greatness of mind, and took my oath to the same, and that it would not become public, but be held as precious as our lives, my master accepted it. You may remain satisfied that I will get the Ongole business settled in your name; I will cause the jamaubundee to be settled agreeable to your desire. It was formerly the Nabob's intention to give this business to you, as the Governor knows full well, but did not at that time agree to it, which you must be well acquainted with.

Your peasher Vancatroyloo is a very careful, good man; he is well experienced in business; he has bound me by an oath to keep all this business secret, and that his own, yours, and my lives are responsible for it. I write this letter to you with the greatest reluctance, and I signified the same to your peasher, and declared that I would not write to you by any means. To this the peasher urged, that, if I did not write to his master, how could he know to whom he (the peasher) delivered the money, and what must his master think of it? Therefore I write you this letter, and send it by my servant Ramanah, accompanied by the peasher's servant, and it will come safe to your hands. After perusal, you will send it back to me immediately: until I receive it, I don't like to eat my victuals or take any sleep. Your peasher took his oath, and urged me to write this for your satisfaction, and has engaged to me that I shall have this letter returned to me in the space of twelve days.

The present Governor is not like the former Governors: he is a very great man in Europe; and all the great men of Europe are much obliged, to him for his condescension in accepting the government of this place. It is his custom, when he makes friendship with any one, to continue it always; and if he is at enmity with any one, he never will desist till he has worked his destruction. He is now exceedingly displeased with the Nabob, and you will understand by-and-by that the Nabob's business cannot be carried on; he (the Nabob) will have no power to do anything in his own affairs: you have, therefore, no room to fear him; you may remain with a contented mind. I desired the Governor to write you a letter for your satisfaction: the Governor said he would do so, when the business was settled. This letter you must peruse as soon as possible, and send it back with all speed by the bearer, Ramadoo, accompanied by three or four of your people, to the end that no accident may happen on the road. These people must be ordered to march in the night only, and to arrive here with the greatest dispatch. You sent ten mangoes for my master and two for me, all of which I have delivered to my master, thinking that ten was not sufficient to present him with. I write this for your information, and salute you with ten thousand respects.

\vspace{0.3cm}
I, Muttu Kistnah, of Madras Patnam, dubash, declare that I perfectly understand the Gentoo language, and do most solemnly affirm that the foregoing is a true translation of the annexed paper writing from the Gentoo language.

(Signed)

Muttu Kistnah.

%FOOTNOTES:
% [68] In this statement, the Ongole country, though it is included under the head of gross revenue, has been let for a certain sum, exclusive of charges. If the expenses specified in the Nabob's vassool accounts for this district are added, the present gross revenue even would appear to exceed the Nabob's; and as the country is only let for one year, there may hereafter be an increase of its revenue.

% [69] The Trichinopoly countries let for the above sum, exclusive of the expenses of sibbendy and saderwared, amounting, by the Nabob's accounts, to rupees 1,30,00 per annum, which are to be defrayed by the renter. And the jaghires of Amir-ul-Omrah and the Begum are not included in the present lease.

% [70] For the ground of this "great reliance," see the papers in this Appendix, No. 5. as also the Nabob's letters to the Court of Directors in this Appendix, No. 10.

% [71] For the full proof of this necessity, Lord Macartney's whole correspondence on the subject may be referred to. Without the act here condemned, not one of the acts commended in the preceding paragraph could be performed. By referring to the Nabob's letters in this Appendix it will be seen what sort of task a governor has on his hands, who is to use, according to the direction of this letter, "acts of address, civility, and conciliation," and to pay, upon all occasions, the highest attention, to persons who at the very time are falsely, and in the grossest terms, accusing him of peculation, corruption, treason, and every species of malversation in office. The recommendation, under menaces of such behavior, and under such circumstances, conveys a lesson the tendency of which cannot be misunderstood.

% [72] The delicacy here recommended, in the expressions concerning conduct "with which the safety of our settlements is essentially connected," is a lesson of the same nature with the former. Dangerous designs, if truly such, ought to be expressed according to their nature and qualities. And as for the secrecy recommended concerning the designs here alluded to, nothing can be more absurd; as they appear very fully and directly in the papers published by the authority of the Court of Directors in 1775, and may be easily discerned from the propositions for the Bengal treaty, published in the Reports of the Committee of Secrecy, and in the Reports of the Select Committee. The keeping of such secrets too long has been one cause of the Carnatic war, and of the ruin of our affairs in India.

% [73] See Tellinga letter, at the end of this correspondence.

% [74] The above-recited practices, or practices similar to them, have prevailed in almost every part of the miserable countries on the coast of Coromandel for near twenty years past. That they prevailed as strongly and generally as they could prevail, under the administration of the Nabob, there can be no question, notwithstanding the assertion in the beginning of the above petition; nor will it ever be otherwise, whilst affairs are conducted upon the principles which influence the present system. Whether the particulars here asserted are true or false neither the Court of Directors nor their ministry have thought proper to inquire. If they are true, in order to bring them to affect Lord Macartney, it ought to be proved that the complaint was made to him, and that he had refused redress. Instead of this fair course, the complaint is carried to the Court of Directors.—The above is one of the documents transmitted by the Nabob, in proof of his charge of corruption against Lord Macartney. If genuine, it is conclusive, at least against Lord Macartney's principal agent and manager. If it be a forgery, (as in all likelihood it is,) it is conclusive against the Nabob and his evil counsellors, and folly demonstrates, if anything further were necessary to demonstrate, the necessity of the clause in Mr. Fox's bill prohibiting the residence of the native princes in the Company's principal settlements,—which clause was, for obvious reasons, not admitted into Mr. Pitt's. It shows, too, the absolute necessity of a severe and exemplary punishment on certain of his English evil counsellors and creditors, by whom such practices are carried on.


%%%%%%%%%%%%%%%%%%%%%%%%%%%%%%%%%%%%%%%%%%%%%%%%%%%%%%%%%%%%%%%%%%%%%%%
\chapter*[Substance of Speech on the Army Estimates]{
Substance of the Speech
\\{\small In the}
\\Debate on the Army Estimates
\\In the House of Commons,
\\{\small On Tuesday, February 9, 1790}
\\{\small Comprehending}
\\A Discussion of the Present Situation of Affairs in France.
}
\addcontentsline{toc}{chapter}{
SUBSTANCE OF SPEECH ON THE ARMY ESTIMATES, February 9, 1790}

Mr. Burke's speech on the report of the army estimates has not been correctly stated in some of the public papers. It is of consequence to him not to be misunderstood. The matter which incidentally came into discussion is of the most serious importance. It is thought that the heads and substance of the speech will answer the purpose sufficiently. If, in making the abstract, through defect of memory in the person who now gives it, any difference at all should be perceived from the speech as it was spoken, it will not, the editor imagines, be found in anything which may amount to a retraction of the opinions he then maintained, or to any softening in the expressions in which they were conveyed.

Mr. Burke spoke a considerable time in answer to various arguments, which had been insisted upon by Mr. Grenville and Mr. Pitt, for keeping an increased peace establishment, and against an improper jealousy of the ministers, in whom a full confidence, subject to responsibility, ought to be placed, on account of their knowledge of the real situation of affairs, the exact state of which it frequently happened that they could not disclose without violating the constitutional and political secrecy necessary to the well-being of their country.

Mr. Burke said in substance, That confidence might become a vice, and jealousy a virtue, according to circumstances. That confidence, of all public virtues, was the most dangerous, and jealousy in an House of Commons, of all public vices, the most tolerable,—- especially where the number and the charge of standing armies in time of peace was the question.

That in the annual Mutiny Bill the annual army was declared to be for the purpose of preserving the balance of power in Europe. The propriety of its being larger or smaller depended, therefore, upon the true state of that balance. If the increase of peace establishments demanded of Parliament agreed with the manifest appearance of the balance, confidence in ministers as to the particulars would be very proper. If the increase was not at all supported by any such appearance, he thought great jealousy might be, and ought to be, entertained on that subject.

That he did not find, on a review of all Europe, that, politically, we stood in the smallest degree of danger from any one state or kingdom it contained, nor that any other foreign powers than our own allies were likely to obtain a considerable preponderance in the scale.

That France had hitherto been our first object in all considerations concerning the balance of power. The presence or absence of France totally varied every sort of speculation relative to that balance.

That France is at this time, in a political light, to be considered as expunged out of the system of Europe. Whether she could ever appear in it again, as a leading power, was not easy to determine; but at present be considered France as not politically existing; and most assuredly it would take up much time to restore her to her former active existence: Gallos quoque in bellis floruisse audivimus might possibly be the language of the rising generation. He did not mean to deny that it was our duty to keep our eye on that nation, and to regulate our preparation by the symptoms of her recovery.

That it was to her strength, not to her form of government, which we were to attend; because republics, as well as monarchies, were susceptible of ambition, jealousy, and anger, the usual causes of war.

But if, while France continued in this swoon, we should go on increasing our expenses, we should certainly make ourselves less a match for her when it became our concern to arm.

It was said, that, as she had speedily fallen, she might speedily rise again. He doubted this. That the fall from an height was with an accelerated velocity; but to lift a weight up to that height again was difficult, and opposed by the laws of physical and political gravitation.

In a political view, France was low indeed. She had lost everything, even to her name.

\begin{verse}
Jacet ingens littore truncus, \\
Avolsumque humeris caput, et sine nomine corpus. 
%[75]
\footnote{ 
Mr. Burke probably had in his mind the remainder of the passage, and was filled with some congenial apprehensions:—

\begin{verse}
Hæc finis Priami fatorum; hic exitus illum\\
Sorte tulit, Trojam incensam et prolapsa videntem\\
Pergama, tot quondam populis terrisque superbum\\
Regnatorem Asiæ. Jacet ingens littore truncus,\\
Avolsumque humeris caput, et sine nomine corpus.\\
At me tum primum sævus circumstetit horror.\\
Obstupui: subiit chari genitoris imago.
\end{verse}
}

\end{verse}

He was astonished at it; he was alarmed at it; he trembled at the uncertainty of all human greatness.

Since the House had been prorogued in the summer much work was done in France. The French had shown themselves the ablest architects of ruin that had hitherto existed in the world. In that very short space of time they had completely pulled down to the ground their monarchy, their church, their nobility, their law, their revenue, their army, their navy, their commerce, their arts, and their manufactures. They had done their business for us as rivals in a way in which twenty Ramillies or Blenheims could never have done it. Were we absolute conquerors, and France to lie prostrate at our feet, we should be ashamed to send a commission to settle their affairs which could impose so hard a law upon the French, and so destructive of all their consequence as a nation, as that they had imposed upon themselves.

France, by the mere circumstance of its vicinity, had been, and in a degree always must be, an object of our vigilance, either with regard to her actual power or to her influence and example. As to the former he had spoken; as to the latter (her example) he should say a few words: for by this example our friendship and our intercourse with that nation had once been, and might again become, more dangerous to us than their worst hostility.

In the last century, Louis the Fourteenth had established a greater and better disciplined military force than ever had been before seen in Europe, and with it a perfect despotism. Though that despotism was proudly arrayed in manners, gallantry, splendor, magnificence, and even covered over with the imposing robes of science, literature, and arts, it was, in government, nothing better than a painted and gilded tyranny,—in religion, a hard, stern intolerance, the fit companion and auxiliary to the despotic tyranny which prevailed in its government. The same character of despotism insinuated itself into every court of Europe,—the same spirit of disproportioned magnificence,—the same love of standing armies, above the ability of the people. In particular, our then sovereigns, King Charles and King James, fell in love with the government of their neighbor, so flattering to the pride of kings. A similarity of sentiments brought on connections equally dangerous to the interests and liberties of their country. It were well that the infection had gone no farther than the throne. The admiration of a government flourishing and successful, unchecked in its operations, and seeming, therefore, to compass its objects more speedily and effectually, gained something upon all ranks of people. The good patriots of that day, however, struggled against it. They sought nothing more anxiously than to break off all communication with France, and to beget a total alienation from its councils and its example,—which, by the animosity prevalent between the abettors of their religious system and the assertors of ours, was in some degree effected.

This day the evil is totally changed in France: but there is an evil there. The disease is altered; but the vicinity of the two countries remains, and must remain; and the natural mental habits of mankind are such, that the present distemper of France is far more likely to be contagious than the old one: for it is not quite easy to spread a passion for servitude among the people; but in all evils of the opposite kind our natural inclinations are flattered. In the case of despotism, there is the fœdum crimen servitutis: in the last, the falsa SPECIES libertatis; and accordingly, as the historian says, pronis auribus accipitur.

In the last age we were in danger of being entangled by the example of France in the net of a relentless despotism. It is not necessary to say anything upon that example. It exists no longer. Our present danger from the example of a people whose character knows no medium is, with regard to government, a danger from anarchy: a danger of being led, through an admiration of successful fraud and violence, to an imitation of the excesses of an irrational, unprincipled, proscribing, confiscating, plundering, ferocious, bloody, and tyrannical democracy. On the side of religion, the danger of their example is no longer from intolerance, but from atheism: a foul, unnatural vice, foe to all the dignity and consolation of mankind; which seems in France, for a long time, to have been embodied into a faction, accredited, and almost avowed.

These are our present dangers from France. But, in his opinion, the very worst part of the example set is in the late assumption of citizenship by the army, and the whole of the arrangement, or rather disarrangement, of their military.

He was sorry that his right honorable friend (Mr. Fox) had dropped even a word expressive of exultation on that circumstance, or that he seemed of opinion that the objection from standing armies was at all lessened by it. He attributed this opinion of Mr. Fox entirely to his known zeal for the best of all causes, liberty. That it was with a pain inexpressible he was obliged to have even the shadow of a difference with his friend, whose authority would always be great with him, and with all thinking people,—Quæ maxima semper censetur nobis, et ERIT quæ maxima semper;—his confidence in Mr. Fox was such, and so ample, as to be almost implicit. That he was not ashamed to avow that degree of docility. That, when the choice is well made, it strengthens, instead of oppressing our intellect. That he who calls in the aid of an equal understanding doubles his own. He who profits of a superior understanding raises his powers to a level with the height of the superior understanding he unites with. He had found the benefit of such a junction, and would not lightly depart from it. He wished almost, on all occasions, that his sentiments were understood to be conveyed in Mr. Fox's words. And that he wished, as amongst the greatest benefits he could wish the country, an eminent share of power to that right honorable gentleman; because he knew that to his great and masterly understanding he had joined the greatest possible degree of that natural moderation which is the best corrective of power: that he was of the most artless, candid, open, and benevolent disposition; disinterested in the extreme; of a temper mild and placable even to a fault; without one drop of gall in his whole constitution.

That the House must perceive, from his coming forward to mark an expression or two of his best friend, how anxious he was to keep the distemper of France from the least countenance in England, where he was sure some wicked persons had shown a strong disposition to recommend an imitation of the French spirit of reform. He was so strongly opposed to any the least tendency towards the means of introducing a democracy like theirs, as well as to the end itself, that, much as it would afflict him, if such a thing could be attempted, and that any friend of his could concur in such measures, (he was far, very far, from believing they could,) he would abandon his best friends, and join with his worst enemies, to oppose either the means or the end,—and to resist all violent exertions of the spirit of innovation, so distant from all principles of true and safe reformation: a spirit well calculated to overturn states, but perfectly unfit to amend them.

That he was no enemy to reformation. Almost every business in which he was much concerned, from the first day he sat in that House to that hour, was a business of reformation; and when he had not been employed in correcting, he had been employed in resisting abuses. Some traces of this spirit in him now stand on their statute-book. In his opinion, anything which unnecessarily tore to pieces the contexture of the state not only prevented all real reformation, but introduced evils which would call, but perhaps call in vain, for new reformation.

That he thought the French nation very unwise. What they valued themselves on was a disgrace to them. They had gloried (and some people in England had thought fit to take share in that glory) in making a Revolution, as if revolutions were good things in themselves. All the horrors and all the crimes of the anarchy which led to their Revolution, which attend its progress, and which may virtually attend it in its establishment, pass for nothing with the lovers of revolutions. The French have made their way, through the destruction of their country, to a bad constitution, when they were absolutely in possession of a good one. They were in possession of it the day the states met in separate orders. Their business, had they been either virtuous or wise, or had been left to their own judgment, was to secure the stability and independence of the states, according to those orders, under the monarch on the throne. It was then their duty to redress grievances.

Instead of redressing grievances, and improving the fabric of their state, to which they were called by their monarch and sent by their country, they were made to take a very different course. They first destroyed all the balances and counterpoises which serve to fix the state and to give it a steady direction, and which furnish sure correctives to any violent spirit which may prevail in any of the orders. These balances existed in their oldest constitution, and in the constitution of this country, and in the constitution of all the countries in Europe. These they rashly destroyed, and then they melted down the whole into one incongruous, ill-connected mass.

When they had done this, they instantly, and with the most atrocious perfidy and breach of all faith among men, laid the axe to the root of all property, and consequently of all national prosperity, by the principles they established and the example they set, in confiscating all the possessions of the Church. They made and recorded a sort of institute and digest of anarchy, called the Rights of Man, in such a pedantic abuse of elementary principles as would have disgraced boys at school: but this declaration of rights was worse than trifling and pedantic in them; as by their name and authority they systematically destroyed every hold of authority by opinion, religious or civil, on the minds of the people. By this mad declaration they subverted the state, and brought on such calamities as no country, without a long war, has ever been known to suffer, and which may in the end produce such a war, and perhaps many such.

With them the question was not between despotism and liberty. The sacrifice they made of the peace and power of their country was not made on the altar of freedom. Freedom, and a better security for freedom than that they have taken, they might have had without any sacrifice at all. They brought themselves into all the calamities they suffer, not that through them they might obtain a British constitution; they plunged themselves headlong into those calamities to prevent themselves from settling into that constitution, or into anything resembling it.

That, if they should perfectly succeed in what they propose, as they are likely enough to do, and establish a democracy, or a mob of democracies, in a country circumstanced like France, they will establish a very bad government,—a very bad species of tyranny.

That the worst effect of all their proceeding was on their military, which was rendered an army for every purpose but that of defence. That, if the question was, whether soldiers were to forget they were citizens, as an abstract proposition, he could have no difference about it; though, as it is usual, when abstract principles are to be applied, much was to be thought on the manner of uniting the character of citizen and soldier. But as applied to the events which had happened in France, where the abstract principle was clothed with its circumstances, he thought that his friend would agree with him, that what was done there furnished no matter of exultation, either in the act or the example. These soldiers were not citizens, but base, hireling mutineers, and mercenary, sordid deserters, wholly destitute of any honorable principle. Their conduct was one of the fruits of that anarchic spirit from the evils of which a democracy itself was to be resorted to, by those who were the least disposed to that form, as a sort of refuge. It was not an army in corps and with discipline, and embodied under the respectable patriot citizens of the state in resisting tyranny. Nothing like it. It was the case of common soldiers deserting from their officers, to join a furious, licentious populace. It was a desertion to a cause the real object of which was to level all those institutions, and to break all those connections, natural and civil, that regulate and hold together the community by a chain of subordination: to raise soldiers against their officers, servants against their masters, tradesmen against their customers, artificers against their employers, tenants against their landlords, curates against their bishops, and children against their parents. That this cause of theirs was not an enemy to servitude, but to society.

He wished the House to consider how the members would like to have their mansions pulled down and pillaged, their persons abused, insulted, and destroyed, their title-deeds brought out and burned before their faces, and themselves and their families driven to seek refuge in every nation throughout Europe, for no other reason than this, that, without any fault of theirs, they were born gentlemen and men of property, and were suspected of a desire to preserve their consideration and their estates. The desertion in France was to aid an abominable sedition, the very professed principle of which was an implacable hostility to nobility and gentry, and whose savage war-whoop was, "A l'Aristocrate!"—by which senseless, bloody cry they animated one another to rapine and murder; whilst abetted by ambitious men of another class, they were crushing everything respectable and virtuous in their nation, and to their power disgracing almost every name by which we formerly knew there was such a country in the world as France.

He knew too well, and he felt as much as any man, how difficult it was to accommodate a standing army to a free constitution, or to any constitution. An armed disciplined body is, in its essence, dangerous to liberty; undisciplined, it is ruinous to society. Its component parts are in the latter case neither good citizens nor good soldiers. What have they thought of in France, under such a difficulty as almost puts the human faculties to a stand? They have put their army under such a variety of principles of duty, that it is more likely to breed litigants, pettifoggers, and mutineers than soldiers.
%[76]
\footnote{ They are Sworn to obey the king, the nation, and the law.}
 They have set up, to balance their crown army, another army, deriving under another authority, called a municipal army,—a balance of armies, not of orders. These latter they have destroyed with every mark of insult and oppression. States may, and they will best, exist with a partition of civil powers. Armies cannot exist under a divided command. This state of things he thought in effect a state of war, or at best but a truce, instead of peace, in the country.

What a dreadful thing is a standing army for the conduct of the whole or any part of which no man is responsible! In the present state of the French crown army, is the crown responsible for the whole of it? Is there any general who can be responsible for the obedience of a brigade, any colonel for that of a regiment, any captain for that of a company? And as to the municipal army, reinforced as it is by the new citizen deserters, under whose command are they? Have we not seen them, not led by, but dragging, their nominal commander, with a rope about his neck, when they, or those whom they accompanied, proceeded to the most atrocious acts of treason and murder? Are any of these armies? Are any of these citizens?

We have in such a difficulty as that of fitting a standing army to the state, he conceived, done much better. We have not distracted our army by divided principles of obedience. We have put them under a single authority, with a simple (our common) oath of fidelity; and we keep the whole under our annual inspection. This was doing all that could be safely done.

He felt some concern that this strange thing called a Revolution in France should be compared with the glorious event commonly called the Revolution in England, and the conduct of the soldiery on that occasion compared with the behavior of some of the troops of France in the present instance. At that period, the Prince of Orange, a prince of the blood-royal in England, was called in by the flower of the English aristocracy to defend its ancient Constitution, and not to level all distinctions. To this prince, so invited, the aristocratic leaders who commanded the troops went over with their several corps, in bodies, to the deliverer of their country. Aristocratic leaders brought up the corps of citizens who newly enlisted in this cause. Military obedience changed its object; but military discipline was not for a moment interrupted in its principle. The troops were ready for war, but indisposed to mutiny.

But as the conduct of the English armies was different, so was that of the whole English nation at that time. In truth, the circumstances of our Revolution (as it is called) and that of France are just the reverse of each other in almost every particular, and in the whole spirit of the transaction. With us it was the case of a legal monarch attempting arbitrary power; in France it is the case of an arbitrary monarch beginning, from whatever cause, to legalize his authority. The one was to be resisted, the other was to be managed and directed; but in neither case was the order of the state to be changed, lest government might be ruined, which ought only to be corrected and legalized. With us we got rid of the man, and preserved the constituent parts of the state. There they get rid of the constituent parts of the state, and keep the man. What we did was in truth and substance, and in a constitutional light, a revolution, not made, but prevented. We took solid securities; we settled doubtful questions; we corrected anomalies in our law. In the stable, fundamental parts of our Constitution we made no revolution,—no, nor any alteration at all. We did not impair the monarchy. Perhaps it might be shown that we strengthened it very considerably. The nation kept the same ranks, the same orders, the same privileges, the same franchises, the same rules for property, the same subordinations, the same order in the law, in the revenue, and in the magistracy,—the same lords, the same commons, the same corporations, the same electors.

The Church was not impaired. Her estates, her majesty, her splendor, her orders and gradations, continued the same. She was preserved in her full efficiency, and cleared only of a certain intolerance, which was her weakness and disgrace. The Church and the State were the same after the Revolution that they were before, but better secured in every part.

Was little done because a revolution was not made in the Constitution? No! Everything was done; because we commenced with reparation, not with ruin. Accordingly, the state flourished. Instead of lying as dead, in a sort of trance, or exposed, as some others, in an epileptic fit, to the pity or derision of the world, for her wild, ridiculous, convulsive movements, impotent to every purpose but that of dashing out her brains against the pavement, Great Britain rose above the standard even of her former self. An era of a more improved domestic prosperity then commenced, and still continues, not only unimpaired, but growing, under the wasting hand of time. All the energies of the country were awakened. England never preserved a firmer countenance or a more vigorous arm to all her enemies and to all her rivals. Europe under her respired and revived. Everywhere she appeared as the protector, assertor, or avenger of liberty. A war was made and supported against fortune itself. The treaty of Ryswick, which first limited the power of France, was soon after made; the grand alliance very shortly followed, which shook to the foundations the dreadful power which menaced the independence of mankind. The states of Europe lay happy under the shade of a great and free monarchy, which knew how to be great without endangering its own peace at home or the internal or external peace of any of its neighbors.

Mr. Burke said he should have felt very unpleasantly, if he had not delivered these sentiments. He was near the end of his natural, probably still nearer the end of his political career. That he was weak and weary, and wished for rest. That he was little disposed to controversies, or what is called a detailed opposition. That at his time of life, if he could not do something by some sort of weight of opinion, natural or acquired, it was useless and indecorous to attempt anything by mere struggle. Turpe senex miles. That he had for that reason little attended the army business, or that of the revenue, or almost any other matter of detail, for some years past. That he had, however, his task. He was far from condemning such opposition; on the contrary, he most highly applauded it, where a just occasion existed for it, and gentlemen had vigor and capacity to pursue it. Where a great occasion occurred, he was, and, while he continued in Parliament, would be, amongst the most active and the most earnest,—as he hoped he had shown on a late event. With respect to the Constitution itself, he wished few alterations in it,—happy if he left it not the worse for any share he had taken in its service.

Mr. Fox then rose, and declared, in substance, that, so far as regarded the French army, he went no farther than the general principle, by which that army showed itself indisposed to be an instrument in the servitude of their fellow-citizens, but did not enter into the particulars of their conduct. He declared that he did not affect a democracy: that he always thought any of the simple, unbalanced governments bad: simple monarchy, simple aristocracy, simple democracy,—he held them all imperfect or vicious; all were bad by themselves; the composition alone was good. That these had been always his principles, in which he had agreed with his friend Mr. Burke,—of whom he had said many kind and flattering things, which Mr. Burke, I take it for granted, will know himself too well to think he merits from anything but Mr. Fox's acknowledged good-nature. Mr. Fox thought, however, that, in many cases, Mr. Burke was rather carried too far by his hatred to innovation.

Mr. Burke said, he well knew that these had been Mr. Fox's invariable opinions; that they were a sure ground for the confidence of his country. But he had been fearful that cabals of very different intentions would be ready to make use of his great name, against his character and sentiments, in order to derive a credit to their destructive machinations.

Mr. Sheridan then rose, and made a lively and eloquent speech against Mr. Burke; in which, among other things, he said that Mr. Burke had libelled the National Assembly of France, and had cast out reflections on such characters as those of the Marquis de La Fayette and Mr. Bailly.

Mr. Burke said, that he did not libel the National Assembly of France, whom he considered very little in the discussion of these matters. That he thought all the substantial power resided in the republic of Paris, whose authority guided, or whose example was followed by, all the republics of France. The republic of Paris had an army under their orders, and not under those of the National Assembly.

N.B. As to the particular gentlemen, I do not remember that Mr. Burke mentioned either of them,—certainly not Mr. Bailly. He alluded, undoubtedly, to the case of the Marquis de La Fayette; but whether what he asserted of him be a libel on him must be left to those who are acquainted with the business.

Mr. Pitt concluded the debate with becoming gravity and dignity, and a reserve on both sides of the question, as related to France, fit for a person in a ministerial situation. He said, that what he had spoken only regarded France when she should unite, which he rather thought she soon might, with the liberty she had acquired, the blessings of law and order. He, too, said several civil things concerning the sentiments of Mr. Burke, as applied to this country.

%FOOTNOTES:
% [75] Mr. Burke probably had in his mind the remainder of the passage, and was filled with some congenial apprehensions:—

%\begin{verse}
%Hæc finis Priami fatorum; hic exitus illum\\
%Sorte tulit, Trojam incensam et prolapsa videntem\\
%Pergama, tot quondam populis terrisque superbum\\
%Regnatorem Asiæ. Jacet ingens littore truncus,\\
%Avolsumque humeris caput, et sine nomine corpus.\\
%At me tum primum sævus circumstetit horror.\\
%Obstupui: subiit chari genitoris imago.
%\end{verse}

% [76] They are Sworn to obey the king, the nation, and the law.

%%%%%%%%%%%%%%%%%%%%%%%%%%%%%%%%%%%%%%%%%%%%%%%%%%%%%%%%%%%%%%%%%%%%%%%
\chapter*[Reflections on the Revolution in France]{
Reflections
\\{\small On the}
\\Revolution in France,
\\{\small And on}
\\The Proceedings in Certain Societies in London Relative to That Event:
\\In a Letter
\\Intended to Have Been Sent to a Gentleman in Paris.
\\{\small 1790}
}
\addcontentsline{toc}{chapter}{REFLECTIONS ON THE REVOLUTION IN FRANCE}

It may not be unnecessary to inform the reader that the following Reflections had their origin in a correspondence between the author and a very young gentleman at Paris, who did him the honor of desiring his opinion upon the important transactions which then, and ever since have, so much occupied the attention of all men. An answer was written some time in the month of October, 1789; but it was kept back upon prudential considerations. That letter is alluded to in the beginning of the following sheets. It has been since forwarded to the person to whom it was addressed. The reasons for the delay in sending it were assigned in a short letter to the same gentleman. This produced on his part a new and pressing application for the author's sentiments.

The author began a second and more full discussion on the subject. This he had some thoughts of publishing early in the last spring; but the matter gaining upon him, he found that what he had undertaken not only far exceeded the measure of a letter, but that its importance required rather a more detailed consideration than at that time he had any leisure to bestow upon it. However, having thrown down his first thoughts in the form of a letter, and, indeed, when he sat down to write, having intended it for a private letter, he found it difficult to change the form of address, when his sentiments had grown into a greater extent and had received another direction. A different plan, he is sensible, might be more favorable to a commodious division and distribution of his matter.

%%%%%%%%%%%%%%%%%%%%%%%%%%%%%%%%%%%%%%%%%
\begin{center}
  \textbf{{\large Reflections} \\On {\large The Revolution in France}} \par 
\end{center}

\PRLsep

Dear Sir,—You are pleased to call again, and with some earnestness, for my thoughts on the late proceedings in France. I will not give you reason to imagine that I think my sentiments of such value as to wish myself to be solicited about them. They are of too little consequence to be very anxiously either communicated or withheld. It was from attention to you, and to you only, that I hesitated at the time when you first desired to receive them. In the first letter I had the honor to write to you, and which at length I send, I wrote neither for nor from any description of men; nor shall I in this. My errors, if any, are my own. My reputation alone is to answer for them.

You see, Sir, by the long letter I have transmitted to you, that, though I do most heartily wish that France may be animated by a spirit of rational liberty, and that I think you bound, in all honest policy, to provide a permanent body in which that spirit may reside, and an effectual organ by which it may act, it is my misfortune to entertain great doubts concerning several material points in your late transactions.

You imagined, when you wrote last, that I might possibly be reckoned among the approvers of certain proceedings in France, from the solemn public seal of sanction they have received from two clubs of gentlemen in London, called the Constitutional Society, and the Revolution Society.

I certainly have the honor to belong to more clubs than one in which the Constitution of this kingdom and the principles of the glorious Revolution are held in high reverence; and I reckon myself among the most forward in my zeal for maintaining that Constitution and those principles in their utmost purity and vigor. It is because I do so that I think it necessary for me that there should be no mistake. Those who cultivate the memory of our Revolution, and those who are attached to the Constitution of this kingdom, will take good care how they are involved with persons who, under the pretext of zeal towards the Revolution and Constitution, too frequently wander from their true principles, and are ready on every occasion to depart from the firm, but cautious and deliberate, spirit which produced the one and which presides in the other. Before I proceed to answer the more material particulars in your letter, I shall beg leave to give you such information as I have been able to obtain of the two clubs which have thought proper, as bodies, to interfere in the concerns of France,—first assuring you that I am not, and that I have never been, a member of either of those societies.

The first, calling itself the Constitutional Society, or Society for Constitutional Information, or by some such title, is, I believe, of seven or eight years' standing. The institution of this society appears to be of a charitable, and so far of a laudable nature: it was intended for the circulation, at the expense of the members, of many books which few others would be at the expense of buying, and which might lie on the hands of the booksellers, to the great loss of an useful body of men. Whether the books so charitably circulated were ever as charitably read is more than I know. Possibly several of them have been exported to France, and, like goods not in request here, may with you have found a market. I have heard much talk of the lights to be drawn from books that are sent from hence. What improvements they have had in their passage (as it is said some liquors are meliorated by crossing the sea) I cannot tell; but I never heard a man of common judgment or the least degree of information speak a word in praise of the greater part of the publications circulated by that society; nor have their proceedings been accounted, except by some of themselves, as of any serious consequence.

Your National Assembly seems to entertain much the same opinion that I do of this poor charitable club. As a nation, you reserved the whole stock of your eloquent acknowledgments for the Revolution Society, when their fellows in the Constitutional were in equity entitled to some share. Since you have selected the Revolution Society as the great object of your national thanks and praises, you will think me excusable in making its late conduct the subject of my observations. The National Assembly of France has given importance to these gentlemen by adopting them; and they return the favor by acting as a committee in England for extending the principles of the National Assembly. Henceforward we must consider them as a kind of privileged persons, as no inconsiderable members in the diplomatic body. This is one among the revolutions which have given splendor to obscurity and distinction to undiscerned merit. Until very lately I do not recollect to have heard of this club. I am quite sure that it never occupied a moment of my thoughts,—nor, I believe, those of any person out of their own set. I find, upon inquiry, that, on the anniversary of the Revolution in 1688, a club of Dissenters, but of what denomination I know not, have long had the custom of hearing a sermon in one of their churches, and that afterwards they spent the day cheerfully, as other clubs do, at the tavern. But I never heard that any public measure or political system, much less that the merits of the constitution of any foreign nation, had been the subject of a formal proceeding at their festivals, until, to my inexpressible surprise, I found them in a sort of public capacity, by a congratulatory address, giving an authoritative sanction to the proceedings of the National Assembly in France.

In the ancient principles and conduct of the club, so far at least as they were declared, I see nothing to which I could take exception. I think it very probable, that, for some purpose, new members may have entered among them,—and that some truly Christian politicians, who love to dispense benefits, but are careful to conceal the hand which distributes the dole, may have made them the instruments of their pious designs. Whatever I may have reason to suspect concerning private management, I shall speak of nothing as of a certainty but what is public.

For one, I should be sorry to be thought directly or indirectly concerned in their proceedings. I certainly take my full share, along with the rest of the world, in my individual and private capacity, in speculating on what has been done, or is doing, on the public stage, in any place, ancient or modern,—in the republic of Rome, or the republic of Paris; but having no general apostolical mission, being a citizen of a particular state, and being bound up, in a considerable degree, by its public will, I should think it at least improper and irregular for me to open a formal public correspondence with the actual government of a foreign nation, without the express authority of the government under which I live.

I should be still more unwilling to enter into that correspondence under anything like an equivocal description, which to many, unacquainted with our usages, might make the address in which I joined appear as the act of persons in some sort of corporate capacity, acknowledged by the laws of this kingdom, and authorized to speak the sense of some part of it. On account of the ambiguity and uncertainty of unauthorized general descriptions, and of the deceit which may be practised under them, and not from mere formality, the House of Commons would reject the most sneaking petition for the most trifling object, under that mode of signature to which you have thrown open the folding-doors of your presence-chamber, and have ushered into your National Assembly with as much ceremony and parade, and with as great a bustle of applause, as if you had been visited by the whole representative majesty of the whole English nation. If what this society has thought proper to send forth had been a piece of argument, it would have signified little whose argument it was. It would be neither the more nor the less convincing on account of the party it came from. But this is only a vote and resolution. It stands solely on authority; and in this case it is the mere authority of individuals, few of whom appear. Their signatures ought, in my opinion, to have been annexed to their instrument. The world would then have the means of knowing how many they are, who they are, and of what value their opinions may be, from their personal abilities, from their knowledge, their experience, or their lead and authority in this state. To me, who am but a plain man, the proceeding looks a little too refined and too ingenious; it has too much the air of a political stratagem, adopted for the sake of giving, under a high-sounding name, an importance to the public declarations of this club, which, when the matter came to be closely inspected, they did not altogether so well deserve. It is a policy that has very much the complexion of a fraud.

I flatter myself that I love a manly, moral, regulated liberty as well as any gentleman of that society, be he who he will; and perhaps I have given as good proofs of my attachment to that cause, in the whole course of my public conduct. I think I envy liberty as little as they do to any other nation. But I cannot stand forward, and give praise or blame to anything which relates to human actions and human concerns on a simple view of the object, as it stands stripped of every relation, in all the nakedness and solitude of metaphysical abstraction. Circumstances (which with some gentlemen pass for nothing) give in reality to every political principle its distinguishing color and discriminating effect. The circumstances are what render every civil and political scheme beneficial or noxious to mankind. Abstractedly speaking, government, as well as liberty, is good; yet could I, in common sense, ten years ago, have felicitated France on her enjoyment of a government, (for she then had a government,) without inquiry what the nature of that government was, or how it was administered? Can I now congratulate the same nation upon its freedom? Is it because liberty in the abstract may be classed amongst the blessings of mankind, that I am seriously to felicitate a madman who has escaped from the protecting restraint and wholesome darkness of his cell on his restoration to the enjoyment of light and liberty? Am I to congratulate a highwayman and murderer who has broke prison upon the recovery of his natural rights? This would be to act over again the scene of the criminals condemned to the galleys, and their heroic deliverer, the metaphysic Knight of the Sorrowful Countenance.

When I see the spirit of liberty in action, I see a strong principle at work; and this, for a while, is all I can possibly know of it. The wild gas, the fixed air, is plainly broke loose: but we ought to suspend our judgment until the first effervescence is a little subsided, till the liquor is cleared, and until we see something deeper than the agitation of a troubled and frothy surface. I must be tolerably sure, before I venture publicly to congratulate men upon a blessing, that they have really received one. Flattery corrupts both the receiver and the giver; and adulation is not of more service to the people than to kings. I should therefore suspend my congratulations on the new liberty of France, until I was informed how it had been combined with government, with public force, with the discipline and obedience of armies, with the collection of an effective and well-distributed revenue, with morality and religion, with solidity and property, with peace and order, with civil and social manners. All these (in their way) are good things, too; and without them, liberty is not a benefit whilst it lasts, and is not likely to continue long. The effect of liberty to individuals is, that they may do what they please: we ought to see what it will please them to do, before we risk congratulations, which may be soon turned into complaints. Prudence would dictate this in the case of separate, insulated, private men. But liberty, when men act in bodies, is power. Considerate people, before they declare themselves, will observe the use which is made of power,—and particularly of so trying a thing as new power in new persons, of whose principles, tempers, and dispositions they have little or no experience, and in situations where those who appear the most stirring in the scene may possibly not be the real movers.

All these considerations, however, were below the transcendental dignity of the Revolution Society. Whilst I continued in the country, from whence I had the honor of writing to you, I had but an imperfect idea of their transactions. On my coming to town, I sent for an account of their proceedings, which had been published by their authority, containing a sermon of Dr. Price, with the Duke de Rochefoucault's and the Archbishop of Aix's letter and several other documents annexed. The whole of that publication, with the manifest design of connecting the affairs of France with those of England, by drawing us into an imitation of the conduct of the National Assembly, gave me a considerable degree of uneasiness. The effect of that conduct upon the power, credit, prosperity, and tranquillity of France became every day more evident. The form of constitution to be settled, for its future polity, became more clear. We are now in a condition to discern with tolerable exactness the true nature of the object held up to our imitation. If the prudence of reserve and decorum dictates silence in some circumstances, in others prudence of a higher order may justify us in speaking our thoughts. The beginnings of confusion with us in England are at present feeble enough; but with you we have seen an infancy still more feeble growing by moments into a strength to heap mountains upon mountains, and to wage war with Heaven itself. Whenever our neighbor's house is on fire, it cannot be amiss for the engines to play a little on our own. Better to be despised for too anxious apprehensions than ruined by too confident a security.

Solicitous chiefly for the peace of my own country, but by no means unconcerned for yours, I wish to communicate more largely what was at first intended only for your private satisfaction. I shall still keep your affairs in my eye, and continue to address myself to you. Indulging myself in the freedom of epistolary intercourse, I beg leave to throw out my thoughts and express my feelings just as they arise in my mind, with very little attention to formal method. I set out with the proceedings of the Revolution Society; but I shall not confine myself to them. Is it possible I should? It looks to me as if I were in a great crisis, not of the affairs of France alone, but of all Europe, perhaps of more than Europe. All circumstances taken together, the French Revolution is the most astonishing that has hitherto happened in the world. The most wonderful things are brought about in many instances by means the most absurd and ridiculous, in the most ridiculous modes, and apparently by the most contemptible instruments. Everything seems out of nature in this strange chaos of levity and ferocity, and of all sorts of crimes jumbled together with all sorts of follies. In viewing this monstrous tragi-comic scene, the most opposite passions necessarily succeed and sometimes mix with each other in the mind: alternate contempt and indignation, alternate laughter and tears, alternate scorn and horror.

It cannot, however, be denied that to some this strange scene appeared in quite another point of view. Into them it inspired no other sentiments than those of exultation and rapture. They saw nothing in what has been done in France but a firm and temperate exertion of freedom,—so consistent, on the whole, with morals and with piety as to make it deserving not only of the secular applause of dashing Machiavelian politicians, but to render it a fit theme for all the devout effusions of sacred eloquence.

On the forenoon of the fourth of November last, Doctor Richard Price, a Non-Conforming minister of eminence, preached at the Dissenting meeting-house of the Old Jewry, to his club or society, a very extraordinary miscellaneous sermon, in which there are some good moral and religious sentiments, and not ill expressed, mixed up with a sort of porridge of various political opinions and reflections: but the Revolution in France is the grand ingredient in the caldron. I consider the address transmitted by the Revolution Society to the National Assembly, through Earl Stanhope, as originating in the principles of the sermon, and as a corollary from them. It was moved by the preacher of that discourse. It was passed by those who came reeking from the effect of the sermon, without any censure or qualification, expressed or implied. If, however, any of the gentlemen concerned shall wish to separate the sermon from the resolution, they know how to acknowledge the one and to disavow the other. They may do it: I cannot.

For my part, I looked on that sermon as the public declaration of a man much connected with literary caballers and intriguing philosophers, with political theologians and theological politicians, both at home and abroad. I know they set him up as a sort of oracle; because, with the best intentions in the world, he naturally philippizes, and chants his prophetic song in exact unison with their designs.

That sermon is in a strain which I believe has not been heard in this kingdom, in any of the pulpits which are tolerated or encouraged in it, since the year 1648,—when a predecessor of Dr. Price, the Reverend Hugh Peters, made the vault of the king's own chapel at St. James's ring with the honor and privilege of the saints, who, with the "high praises of God in their mouths, and a two-edged sword in their hands, were to execute judgment on the heathen, and punishments upon the people; to bind their kings with chains, and their nobles with fetters of iron."
%[77]
\footnote{ Ps. cxlix.}
 Few harangues from the pulpit, except in the days of your League in France, or in the days of our Solemn League and Covenant in England, have ever breathed less of the spirit of moderation than this lecture in the Old Jewry. Supposing, however, that something like moderation were visible in this political sermon, yet politics and the pulpit are terms that have little agreement. No sound ought to be heard in the church but the healing voice of Christian charity. The cause of civil liberty and civil government gains as little as that of religion by this confusion of duties. Those who quit their proper character to assume what does not belong to them are, for the greater part, ignorant both of the character they leave and of the character they assume. Wholly unacquainted with the world, in which they are so fond of meddling, and inexperienced in all its affairs, on which they pronounce with so much confidence, they have nothing of politics but the passions they excite. Surely the church is a place where one day's truce ought to be allowed to the dissensions and animosities of mankind.

This pulpit style, revived after so long a discontinuance, had to me the air of novelty, and of a novelty not wholly without danger. I do not charge this danger equally to every part of the discourse. The hint given to a noble and reverend lay-divine, who is supposed high in office in one of our universities,
%[78]
\footnote{ Discourse on the Love of our Country, Nov. 4, 1789, by Dr. Richard Price, 3d edition, p. 17 and 18.}
 and other lay-divines "of rank and literature," may be proper and seasonable, though somewhat new. If the noble Seekers should find nothing to satisfy their pious fancies in the old staple of the national Church, or in all the rich variety to be found in the well-assorted warehouses of the Dissenting congregations, Dr. Price advises them to improve upon Non-Conformity, and to set up, each of them, a separate meeting-house upon his own particular principles.
%[79]
\footnote{ "Those who dislike that mode of worship which is prescribed by public authority ought, if they can find no worship out of the Church which they approve, to set up a separate worship for themselves; and by doing this, and giving an example of a rational and manly worship, men of weight from their rank and literature may do the greatest service to society and the world."—P. 18, Dr. Price's Sermon.}
 It is somewhat remarkable that this reverend divine should be so earnest for setting up new churches, and so perfectly indifferent concerning the doctrine which may be taught in them. His zeal is of a curious character. It is not for the propagation of his own opinions, but of any opinions. It is not for the diffusion of truth, but for the spreading of contradiction. Let the noble teachers but dissent, it is no matter from whom or from what. This great point once secured, it is taken for granted their religion will be rational and manly. I doubt whether religion would reap all the benefits which the calculating divine computes from this "great company of great preachers." It would certainly be a valuable addition of nondescripts to the ample collection of known classes, genera, and species, which at present beautify the hortus siccus of Dissent. A sermon from a noble duke, or a noble marquis, or a noble earl, or baron bold, would certainly increase and diversify the amusements of this town, which begins to grow satiated with the uniform round of its vapid dissipations. I should only stipulate that these new Mess-Johns in robes and coronets should keep some sort of bounds in the democratic and levelling principles which are expected from their titled pulpits. The new evangelists will, I dare say, disappoint the hopes that are conceived of them. They will not become, literally as well as figuratively, polemic divines,—nor be disposed so to drill their congregations, that they may, as in former blessed times, preach their doctrines to regiments of dragoons and corps of infantry and artillery. Such arrangements, however favorable to the cause of compulsory freedom, civil and religious, may not be equally conducive to the national tranquillity. These few restrictions I hope are no great stretches of intolerance, no very violent exertions of despotism.

But I may say of our preacher, "Utinam nugis tota illa dedisset et tempora sævitiæ." All things in this his fulminating bull are not of so innoxious a tendency. His doctrines affect our Constitution in its vital parts. He tells the Revolution Society, in this political sermon, that his Majesty "is almost the only lawful king in the world, because the only one who owes his crown to the choice of his people." As to the kings of the world, all of whom (except one) this arch-pontiff of the rights of men, with all the plenitude and with more than the boldness of the Papal deposing power in its meridian fervor of the twelfth century, puts into one sweeping clause of ban and anathema, and proclaims usurpers by circles of longitude and latitude over the whole globe, it behooves them to consider how they admit into their territories these apostolic missionaries, who are to tell their subjects they are not lawful kings. That is their concern. It is ours, as a domestic interest of some moment, seriously to consider the solidity of the only principle upon which these gentlemen acknowledge a king of Great Britain to be entitled to their allegiance.

This doctrine, as applied to the prince now on the British throne, either is nonsense, and therefore neither true nor false, or it affirms a most unfounded, dangerous, illegal, and unconstitutional position. According to this spiritual doctor of politics, if his Majesty does not owe his crown to the choice of his people, he is no lawful king. Now nothing can be more untrue than that the crown of this kingdom is so held by his Majesty. Therefore, if you follow their rule, the king of Great Britain, who most certainly does not owe his high office to any form of popular election, is in no respect better than the rest of the gang of usurpers, who reign, or rather rob, all over the face of this our miserable world, without any sort of right or title to the allegiance of their people. The policy of this general doctrine, so qualified, is evident enough. The propagators of this political gospel are in hopes their abstract principle (their principle that a popular choice is necessary to the legal existence of the sovereign magistracy) would be overlooked, whilst the king of Great Britain was not affected by it. In the mean time the ears of their congregations would be gradually habituated to it, as if it were a first principle admitted without dispute. For the present it would only operate as a theory, pickled in the preserving juices of pulpit eloquence, and laid by for future use. Condo et compono quæ mox depromere passim. By this policy, whilst our government is soothed with a reservation in its favor, to which it has no claim, the security which it has in common with all governments, so far as opinion is security, is taken away.

Thus these politicians proceed, whilst little notice is taken of their doctrines; but when they come to be examined upon the plain meaning of their words and the direct tendency of their doctrines, then equivocations and slippery constructions come into play. When they say the king owes his crown to the choice of his people, and is therefore the only lawful sovereign in the world, they will perhaps tell us they mean to say no more than that some of the king's predecessors have been called to the throne by some sort of choice, and therefore he owes his crown to the choice of his people. Thus, by a miserable subterfuge, they hope to render their proposition safe by rendering it nugatory. They are welcome to the asylum they seek for their offence, since they take refuge in their folly. For, if you admit this interpretation, how does their idea of election differ from our idea of inheritance? And how does the settlement of the crown in the Brunswick line, derived from James the First, come to legalize our monarchy rather than that of any of the neighboring countries? At some time or other, to be sure, all the beginners of dynasties were chosen by those who called them to govern. There is ground enough for the opinion that all the kingdoms of Europe were at a remote period elective, with more or fewer limitations in the objects of choice. But whatever kings might have been here or elsewhere a thousand years ago, or in whatever manner the ruling dynasties of England or France may have begun, the king of Great Britain is at this day king by a fixed rule of succession, according to the laws of his country; and whilst the legal conditions of the compact of sovereignty are performed by him, (as they are performed,) he holds his crown in contempt of the choice of the Revolution Society, who have not a single vote for a king amongst them, either individually or collectively: though I make no doubt they would soon erect themselves into an electoral college, if things were ripe to give effect to their claim. His Majesty's heirs and successors, each in his time and order, will come to the crown with the same contempt of their choice with which his Majesty has succeeded to that he wears.

Whatever may be the success of evasion in explaining away the gross error fact, which supposes that his Majesty (though he holds it in concurrence with the wishes) owes his crown to the choice of his people, yet nothing can evade their full, explicit declaration concerning the principle of a right in the people to choose,—which right is directly maintained, and tenaciously adhered to. All the oblique insinuations concerning election bottom in this proposition, and are referable to it. Lest the foundation of the king's exclusive legal title should pass for a mere rant of adulatory freedom, the political divine proceeds dogmatically to assert,
%[80]
\footnote{ P. 34, Discourse on the Love of our Country, by Dr. Price.}
 that, by the principles of the Revolution, the people of England have acquired three fundamental rights, all of which, with him, compose one system, and lie together in one short sentence: namely, that we have acquired a right

1. "To choose our own governors."

2. "To cashier them for misconduct."

3. "To frame a government for ourselves."

This new, and hitherto unheard-of bill of rights, though made in the name of the whole people, belongs to those gentlemen and their faction only. The body of the people of England have no share in it. They utterly disclaim it. They will resist the practical assertion of it with their lives and fortunes. They are bound to do so by the laws of their country, made at the time of that very Revolution which is appealed to in favor of the fictitious rights claimed by the society which abuses its name.

These gentlemen of the Old Jewry, in all their reasonings on the Revolution of 1688, have a revolution which happened in England about forty years before, and the late French Revolution, so much before their eyes and in their hearts, that they are constantly confounding all the three together. It is necessary that we should separate what they confound. We must recall their erring fancies to the acts of the Revolution which we revere, for the discovery of its true principles. If the principles of the Revolution of 1688 are anywhere to be found, it is in the statute called the Declaration of Right. In that most wise, sober, and considerate declaration, drawn up by great lawyers and great statesmen, and not by warm and inexperienced enthusiasts, not one word is said, nor one suggestion made, of a general right "to choose our own governors, to cashier them for misconduct, and to form a government for ourselves."

This Declaration of Right (the act of the 1st of William and Mary, sess. 2, ch. 2) is the corner-stone of our Constitution, as reinforced, explained, improved, and in its fundamental principles forever settled. It is called "An act for declaring the rights and liberties of the subject, and for settling the succession of the crown." You will observe that these rights and this succession are declared in one body, and bound indissolubly together.

A few years after this period, a second opportunity offered for asserting a right of election to the crown. On the prospect of a total failure of issue from King William, and from the princess, afterwards Queen Anne, the consideration of the settlement of the Crown, and of a further security for the liberties of the people, again came before the legislature. Did they this second time make any provision for legalizing the crown on the spurious Revolution principles of the Old Jewry? No. They followed the principles which prevailed in the Declaration of Right; indicating with more precision the persons who were to inherit in the Protestant line. This act also incorporated, by the same policy, our liberties and an hereditary succession in the same act. Instead of a right to choose our own governors, they declared that the succession in that line (the Protestant line drawn from James the First) was absolutely necessary "for the peace, quiet, and security of the realm," and that it was equally urgent on them "to maintain a certainty in the succession thereof, to which the subjects may safely have recourse for their protection." Both these acts, in which are heard the unerring, unambiguous oracles of Revolution policy, instead of countenancing the delusive gypsy predictions of a "right to choose our governors," prove to a demonstration how totally adverse the wisdom of the nation was from turning a case of necessity into a rule of law.

Unquestionably there was at the Revolution, in the person of King William, a small and a temporary deviation from the strict order of a regular hereditary succession; but it is against all genuine principles of jurisprudence to draw a principle from a law made in a special case and regarding an individual person. Privilegium non transit in exemplum. If ever there was a time favorable for establishing the principle that a king of popular choice was the only legal king, without all doubt it was at the Revolution. Its not being done at that time is a proof that the nation was of opinion it ought not to be done at any time. There is no person so completely ignorant of our history as not to know that the majority in Parliament, of both parties, were so little disposed to anything resembling that principle, that at first they were determined to place the vacant crown, not on the head of the Prince of Orange, but on that of his wife, Mary, daughter of King James, the eldest born of the issue of that king, which they acknowledged as undoubtedly his. It would be to repeat a very trite story, to recall to your memory all those circumstances which demonstrated that their accepting King William was not properly a choice; but to all those who did not wish in effect to recall King James, or to deluge their country in blood, and again to bring their religion, laws, and liberties into the peril they had just escaped, it was an act of necessity, in the strictest moral sense in which necessity can be taken.

In the very act in which, for a time, and in a single case, Parliament departed from the strict order of inheritance, in favor of a prince who, though not next, was, however, very near in the line of succession, it is curious to observe how Lord Somers, who drew the bill called the Declaration of Right, has comported himself on that delicate occasion. It is curious to observe with what address this temporary solution of continuity is kept from the eye; whilst all that could be found in this act of necessity to countenance the idea of an hereditary succession is brought forward, and fostered, and made the most of, by this great man, and by the legislature who followed him. Quitting the dry, imperative style of an act of Parliament, he makes the Lords and Commons fall to a pious legislative ejaculation, and declare that they consider it "as a marvellous providence, and merciful goodness of God to this nation, to preserve their said Majesties' royal persons most happily to reign over us on the throne of their ancestors, for which, from the bottom of their hearts, they return their humblest thanks and praises." The legislature plainly had in view the Act of Recognition of the first of Queen Elizabeth, chap. 3rd, and of that of James the First, chap. 1st, both acts strongly declaratory of the inheritable nature of the crown; and in many parts they follow, with a nearly literal precision, the words, and even the form of thanksgiving which is found in these old declaratory statutes.

The two Houses, in the act of King William, did not thank God that they had found a fair opportunity to assert a right to choose their own governors, much less to make an election the only lawful title to the crown. Their having been in a condition to avoid the very appearance of it, as much as possible, was by them considered as a providential escape. They threw a politic, well-wrought veil over every circumstance tending to weaken the rights which in the meliorated order of succession they meant to perpetuate, or which might furnish a precedent for any future departure from what they had then settled forever. Accordingly, that they might not relax the nerves of their monarchy, and that they might preserve a close conformity to the practice of their ancestors, as it appeared in the declaratory statutes of Queen Mary
%[81]
\footnote{ 1st Mary, sess. 3, ch. 1.}
 and Queen Elizabeth, in the next clause they vest, by recognition, in their Majesties all the legal prerogatives of the crown, declaring "that in them they are most fully, rightfully, and entirely invested, incorporated, united, and annexed." In the clause which follows, for preventing questions, by reason of any pretended titles to the crown, they declare (observing also in this the traditionary language, along with the traditionary policy of the nation, and repeating as from a rubric the language of the preceding acts of Elizabeth and James) that on the preserving "a certainty in the SUCCESSION thereof the unity, peace, and tranquillity of this nation doth, under God, wholly depend."

They knew that a doubtful title of succession would but too much resemble an election, and that an election would be utterly destructive of the "unity, peace, and tranquillity of this nation," which they thought to be considerations of some moment. To provide for these objects, and therefore to exclude forever the Old Jewry doctrine of "a right to choose our own governors," they follow with a clause containing a most solemn pledge, taken from the preceding act of Queen Elizabeth,—as solemn a pledge as ever was or can be given in favor of an hereditary succession, and as solemn a renunciation as could be made of the principles by this society imputed to them:—"The Lords Spiritual and Temporal, and Commons, do, in the name of all the people aforesaid, most humbly and faithfully submit themselves, their heirs, and posterities forever; and do faithfully promise that they will stand to, maintain, and defend their said Majesties, and also the limitation of the crown, herein specified and contained, to the utmost of their powers," \&c., \&c.

So far is it from being true that we acquired a right by the Revolution to elect our kings, that, if we had possessed it before, the English nation did at that time most solemnly renounce and abdicate it, for themselves, and for all their posterity forever. These gentlemen may value themselves as much as they please on their Whig principles; but I never desire to be thought a better Whig than Lord Somers, or to understand the principles of the Revolution better than those by whom it was brought about, or to read in the Declaration of Right any mysteries unknown to those whose penetrating style has engraved in our ordinances, and in our hearts, the words and spirit of that immortal law.

It is true, that, aided with the powers derived from force and opportunity, the nation was at that time, in some sense, free to take what course it pleased for filling the throne,—but only free to do so upon the same grounds on which they might have wholly abolished their monarchy, and every other part of their Constitution. However, they did not think such bold changes within their commission. It is, indeed, difficult, perhaps impossible, to give limits to the mere abstract competence of the supreme power, such as was exercised by Parliament at that time; but the limits of a moral competence, subjecting, even in powers more indisputably sovereign, occasional will to permanent reason, and to the steady maxims of faith, justice, and fixed fundamental policy, are perfectly intelligible, and perfectly binding upon those who exercise any authority, under any name, or under any title, in the state. The House of Lords, for instance, is not morally competent to dissolve the House of Commons,—no, nor even to dissolve itself, nor to abdicate, if it would, its portion in the legislature of the kingdom. Though a king may abdicate for his own person, he cannot abdicate for the monarchy. By as strong, or by a stronger reason, the House of Commons cannot renounce its share of authority. The engagement and pact of society, which generally goes by the name of the Constitution, forbids such invasion and such surrender. The constituent parts of a state are obliged to hold their public faith with each other, and with all those who derive any serious interest under their engagements, as much as the whole state is bound to keep its faith with separate communities: otherwise, competence and power would soon be confounded, and no law be left but the will of a prevailing force. On this principle, the succession of the crown has always been what it now is, an hereditary succession by law: in the old line it was a succession by the Common Law; in the new by the statute law, operating on the principles of the Common Law, not changing the substance, but regulating the mode and describing the persons. Both these descriptions of law are of the same force, and are derived from an equal authority, emanating from the common agreement and original compact of the state, communi sponsione reipublicæ, and as such are equally binding on king, and people too, as long as the terms are observed, and they continue the same body politic.

It is far from impossible to reconcile, if we do not suffer ourselves to be entangled in the mazes of metaphysic sophistry, the use both of a fixed rule and an occasional deviation,—the sacredness of an hereditary principle of succession in our government with a power of change in its application in cases of extreme emergency. Even in that extremity, (if we take the measure of our rights by our exercise of them at the Revolution,) the change is to be confined to the peccant part only,—to the part which produced the necessary deviation; and even then it is to be effected without a decomposition of the whole civil and political mass, for the purpose of originating a new civil order out of the first elements of society.

A state without the means of some change is without the means of its conservation. Without such means it might even risk the loss of that part of the Constitution which it wished the most religiously to preserve. The two principles of conservation and correction operated strongly at the two critical periods of the Restoration and Revolution, when England found itself without a king. At both those periods the nation had lost the bond of union in their ancient edifice: they did not, however, dissolve the whole fabric. On the contrary, in both cases they regenerated the deficient part of the old Constitution through the parts which were not impaired. They kept these old parts exactly as they were, that the part recovered might be suited to them. They acted by the ancient organized states in the shape of their old organization, and not by the organic moleculæ of a disbanded people. At no time, perhaps, did the sovereign legislature manifest a more tender regard to that fundamental principle of British constitutional policy than at the time of the Revolution, when it deviated from the direct line of hereditary succession. The crown was carried somewhat out of the line in which it had before moved; but the new line was derived from the same stock. It was still a line of hereditary descent; still an hereditary descent in the same blood, though an hereditary descent qualified with Protestantism. When the legislature altered the direction, but kept the principle, they showed that they held it inviolable.

On this principle, the law of inheritance had admitted some amendment in the old time, and long before the era of the Revolution. Some time after the Conquest great questions arose upon the legal principles of hereditary descent. It became a matter of doubt whether the heir per capita or the heir per stirpes was to succeed; but whether the heir per capita gave way when the heirdom per stirpes took place, or the Catholic heir when the Protestant was preferred, the inheritable principle survived with a sort of immortality through all transmigrations,—

\begin{verse}
\hspace{1.5in}Multosque per annos\\
Stat fortuna domûs, et avi numerantur avorum.
\end{verse}

This is the spirit of our Constitution, not only in its settled course, but in all its revolutions. Whoever came in, or however he came in, whether he obtained the crown by law or by force, the hereditary succession was either continued or adopted.

The gentlemen of the Society for Revolutions see nothing in that of 1688 but the deviation from the Constitution; and they take the deviation from the principle for the principle. They have little regard to the obvious consequences of their doctrine, though they may see that it leaves positive authority in very few of the positive institutions of this country. When such an unwarrantable maxim is once established, that no throne is lawful but the elective, no one act of the princes who preceded this era of fictitious election can be valid. Do these theorists mean to imitate some of their predecessors, who dragged the bodies of our ancient sovereigns out of the quiet of their tombs? Do they mean to attaint and disable backwards all the kings that have reigned before the Revolution, and consequently to stain the throne of England with the blot of a continual usurpation? Do they mean to invalidate, annul, or to call into question, together with the titles of the whole line of our kings, that great body of our statute law which passed under those whom they treat as usurpers? to annul laws of inestimable value to our liberties,—of as great value at least as any which have passed at or since the period of the Revolution? If kings who did not owe their crown to the choice of their people had no title to make laws, what will become of the statute De tallagio non concedendo? of the Petition of Right? of the act of Habeas Corpus? Do these new doctors of the rights of men presume to assert that King James the Second, who came to the crown as next of blood, according to the rules of a then unqualified succession, was not to all intents and purposes a lawful king of England, before he had done any of those acts which were justly construed into an abdication of his crown? If he was not, much trouble in Parliament might have been saved at the period these gentlemen commemorate. But King James was a bad king with a good title, and not an usurper. The princes who succeeded according to the act of Parliament which settled the crown on the Electress Sophia and on her descendants, being Protestants, came in as much by a title of inheritance as King James did. He came in according to the law, as it stood at his accession to the crown; and the princes of the House of Brunswick came to the inheritance of the crown, not by election, but by the law, as it stood at their several accessions, of Protestant descent and inheritance, as I hope I have shown sufficiently.

The law by which this royal family is specifically destined to the succession is the act of the 12th and 13th of King William. The terms of this act bind "us, and our heirs, and our posterity, to them, their heirs, and their posterity," being Protestants, to the end of time, in the same words as the Declaration of Right had bound us to the heirs of King William and Queen Mary. It therefore secures both an hereditary crown and an hereditary allegiance. On what ground, except the constitutional policy of forming an establishment to secure that kind of succession which is to preclude a choice of the people forever, could the legislature have fastidiously rejected the fair and abundant choice which our own country presented to them, and searched in strange lands for a foreign princess, from whose womb the line of our future rulers were to derive their title to govern millions of men through a series of ages?

The Princess Sophia was named in the act of settlement of the 12th and 13th of King William, for a stock and root of inheritance to our kings, and not for her merits as a temporary administratrix of a power which she might not, and in fact did not, herself ever exercise. She was adopted for one reason, and for one only,—because, says the act, "the most excellent Princess Sophia, Electress and Duchess Dowager of Hanover, is daughter of the most excellent Princess Elizabeth, late Queen of Bohemia, daughter of our late sovereign lord King James the First, of happy memory, and is hereby declared to be the next in succession in the Protestant line," \&c., \&c.; "and the crown shall continue to the heirs of her body, being Protestants." This limitation was made by Parliament, that through the Princess Sophia an inheritable line not only was to be continued in future, but (what they thought very material) that through her it was to be connected with the old stock of inheritance in King James the First; in order that the monarchy might preserve an unbroken unity through all ages, and might be preserved (with safety to our religion) in the old approved mode by descent, in which, if our liberties had been once endangered, they had often, through all storms and struggles of prerogative and privilege, been preserved. They did well. No experience has taught us that in any other course or method than that of an hereditary crown our liberties can be regularly perpetuated and preserved sacred as our hereditary right. An irregular, convulsive movement may be necessary to throw off an irregular, convulsive disease. But the course of succession is the healthy habit of the British Constitution. Was it that the legislature wanted, at the act for the limitation of the crown in the Hanoverian line, drawn through the female descendants of James the First, a due sense of the inconveniences of having two or three, or possibly more, foreigners in succession to the British throne? No!—they had a due sense of the evils which might happen from such foreign rule, and more than a due sense of them. But a more decisive proof cannot be given of the full conviction of the British nation that the principles of the Revolution did not authorize them to elect kings at their pleasure, and without any attention to the ancient fundamental principles of our government, than their continuing to adopt a plan of hereditary Protestant succession in the old line, with all the dangers and all the inconveniences of its being a foreign line full before their eyes, and operating with the utmost force upon their minds.

A few years ago I should be ashamed to overload a matter so capable of supporting itself by the then unnecessary support of any argument; but this seditious, unconstitutional doctrine is now publicly taught, avowed, and printed. The dislike I feel to revolutions, the signals for which have so often been given from pulpits,—the spirit of change that is gone abroad,—the total contempt which prevails with you, and may come to prevail with us, of all ancient institutions, when set in opposition to a present sense of convenience, or to the bent of a present inclination,—all these considerations make it not unadvisable, in my opinion, to call back our attention to the true principles of our own domestic laws, that you, my French friend, should begin to know, and that we should continue to cherish them. We ought not, on either side of the water, to suffer ourselves to be imposed upon by the counterfeit wares which some persons, by a double fraud, export to you in illicit bottoms, as raw commodities of British growth, though wholly alien to our soil, in order afterwards to smuggle them back again into this country, manufactured after the newest Paris fashion of an improved liberty.

The people of England will not ape the fashions they have never tried, nor go back to those which they have found mischievous on trial. They look upon the legal hereditary succession of their crown as among their rights, not as among their wrongs,—as a benefit, not as a grievance,—as a security for their liberty, not as a badge of servitude. They look on the frame of their commonwealth, such as it stands, to be of inestimable value; and they conceive the undisturbed succession of the crown to be a pledge of the stability and perpetuity of all the other members of our Constitution.

I shall beg leave, before I go any further, to take notice of some paltry artifices which the abettors of election as the only lawful title to the crown are ready to employ, in order to render the support of the just principles of our Constitution a task somewhat invidious. These sophisters substitute a fictitious cause, and feigned personages, in whose favor they suppose you engaged, whenever you defend the inheritable nature of the crown. It is common with them to dispute as if they were in a conflict with some of those exploded fanatics of slavery who formerly maintained, what I believe no creature now maintains, "that the crown is held by divine, hereditary, and indefeasible right." These old fanatics of single arbitrary power dogmatized as if hereditary royalty was the only lawful government in the world,—just as our new fanatics of popular arbitrary power maintain that a popular election is the sole lawful source of authority. The old prerogative enthusiasts, it is true, did speculate foolishly, and perhaps impiously too, as if monarchy had more of a divine sanction than any other mode of government,—and as if a right to govern by inheritance were in strictness indefeasible in every person who should be found in the succession to a throne, and under every circumstance, which no civil or political right can be. But an absurd opinion concerning the king's hereditary right to the crown does not prejudice one that is rational, and bottomed upon solid principles of law and policy. If all the absurd theories of lawyers and divines were to vitiate the objects in which they are conversant, we should have no law and no religion left in the world. But an absurd theory on one side of a question forms no justification for alleging a false fact or promulgating mischievous maxims on the other.

The second claim of the Revolution Society is "a right of cashiering their governors for misconduct." Perhaps the apprehensions our ancestors entertained of forming such a precedent as that "of cashiering for misconduct" was the cause that the declaration of the act which implied the abdication of King James was, if it had any fault, rather too guarded and too circumstantial.
%[82]
\footnote{ "That King James the Second, having endeavored to subvert the Constitution of the kingdom, by breaking the original contract between king and people, and, by the advice of Jesuits and other wicked persons, having violated the fundamental laws, and having withdrawn himself out of the kingdom, hath abdicated the government, and the throne is thereby vacant."}
 But all this guard, and all this accumulation of circumstances, serves to show the spirit of caution which predominated in the national councils, in a situation in which men irritated by oppression, and elevated by a triumph over it, are apt to abandon themselves to violent and extreme courses; it shows the anxiety of the great men who influenced the conduct of affairs at that great event to make the Revolution a parent of settlement, and not a nursery of future revolutions.

No government could stand a moment, if it could be blown down with anything so loose and indefinite as an opinion of "misconduct." They who led at the Revolution grounded their virtual abdication of King James upon no such light and uncertain principle. They charged him with nothing less than a design, confirmed by a multitude of illegal overt acts, to subvert the Protestant Church and State, and their fundamental, unquestionable laws and liberties: they charged him with having broken the original contrast between king and people. This was more than misconduct. A grave and overruling necessity obliged them to take the step they took, and took with infinite reluctance, as under that most rigorous of all laws. Their trust for the future preservation of the Constitution was not in future revolutions. The grand policy of all their regulations was to render it almost impracticable for any future sovereign to compel the states of the kingdom to have again recourse to those violent remedies. They left the crown, what in the eye and estimation of law it had ever been, perfectly irresponsible. In order to lighten the crown still further, they aggravated responsibility on ministers of state. By the statute of the first of King William, sess. 2d, called "the act for declaring the rights and liberties of the subject, and for settling the succession of the crown," they enacted that the ministers should serve the crown on the terms of that declaration. They secured soon after the frequent meetings of Parliament, by which the whole government would be under the constant inspection and active control of the popular representative and of the magnates of the kingdom. In the next great constitutional act, that of the 12th and 13th of King William, for the further limitation of the crown, and better securing the rights and liberties of the subject, they provided "that no pardon under the great seal of England should be pleadable to an impeachment by the Commons in Parliament." The rule laid down for government in the Declaration of Right, the constant inspection of Parliament, the practical claim of impeachment, they thought infinitely a better security not only for their constitutional liberty, but against the vices of administration, than the reservation of a right so difficult in the practice, so uncertain in the issue, and often so mischievous in the consequences, as that "cashiering their governors."

Dr. Price, in this sermon,
%[83]
\footnote{ P. 23, 23, 24.}
 condemns, very properly, the practice of gross adulatory addresses to kings. Instead of this fulsome style, he proposes that his Majesty should be told, on occasions of congratulation, that "he is to consider himself as more properly the servant than the sovereign of his people." For a compliment, this new form of address does not seem to be very soothing. Those who are servants in name, as well as in effect, do not like to be told of their situation, their duty, and their obligations. The slave in the old play tells his master, "Hæc commemeratio est quasi exprobratio." It is not pleasant as compliment; it is not wholesome as instruction. After all, if the king were to bring himself to echo this new kind of address, to adopt it in terms, and even to take the appellation of Servant of the People as his royal style, how either he or we should be much mended by it I cannot imagine. I have seen very assuming letters signed, "Your most obedient, humble servant." The proudest domination that ever was endured on earth took a title of still greater humility than that which is now proposed for sovereigns by the Apostle of Liberty. Kings and nations were trampled upon by the foot of one calling himself "The Servant of Servants"; and mandates for deposing sovereigns were sealed with the signet of "The Fisherman."

I should have considered all this as no more than a sort of flippant, vain discourse, in which, as in an unsavory fume, several persons suffer the spirit of liberty to evaporate, if it were not plainly in support of the idea, and a part of the scheme, of "cashiering kings for misconduct." In that light it is worth some observation.

Kings, in one sense, are undoubtedly the servants of the people, because their power has no other rational end than that of the general advantage; but it is not true that they are, in the ordinary sense, (by our Constitution, at least,) anything like servants,—the essence of whose situation is to obey the commands of some other, and to be removable at pleasure. But the king of Great Britain obeys no other person; all other persons are individually, and collectively too, under him, and owe to him a legal obedience. The law, which knows neither to flatter nor to insult, calls this high-magistrate, not our servant, as this humble divine calls him, but "our sovereign lord the king"; and we, on our parts, have learned to speak only the primitive language of the law, and not the confused jargon of their Babylonian pulpits.

As he is not to obey us, but we are to obey the law in him, our Constitution has made no sort of provision towards rendering him, as a servant, in any degree responsible. Our Constitution knows nothing of a magistrate like the Justicia of Aragon,—nor of any court legally appointed, nor of any process legally settled, for submitting the king to the responsibility belonging to all servants. In this he is not distinguished from the commons and the lords, who, in their several public capacities, can never be called to an account for their conduct; although the Revolution Society chooses to assert, in direct opposition to one of the wisest and most beautiful parts of our Constitution, that "a king is no more than the first servant of the public, created by it, and responsible to it."

Ill would our ancestors at the Revolution have deserved their fame for wisdom, if they had found no security for their freedom, but in rendering their government feeble in its operations and precarious in its tenure,—if they had been able to contrive no better remedy against arbitrary power than civil confusion. Let these gentlemen state who that representative public is to whom they will affirm the king, as a servant, to be responsible. It will be then time enough for me to produce to them the positive statute law which affirms that he is not.

The ceremony of cashiering kings, of which these gentlemen talk so much at their ease, can rarely, if ever, be performed without force. It then becomes a case of war, and not of constitution. Laws are commanded to hold their tongues amongst arms; and tribunals fall to the ground with the peace they are no longer able to uphold. The Revolution of 1688 was obtained by a just war, in the only case in which any war, and much more a civil war, can be just. "Justa bella quibus NECESSARIA." The question of dethroning, or, if these gentlemen, like the phrase better, "cashiering kings," will always be, as it has always been, an extraordinary question of state, and wholly out of the law: a question (like all other questions of state) of dispositions, and of means, and of probable consequences, rather than of positive rights. As it was not made for common abuses, so it is not to be agitated by common minds. The speculative line of demarcation, where obedience ought to end and resistance must begin, is faint, obscure, and not easily definable. It is not a single act or a single event which determines it. Governments must be abused and deranged indeed, before it can be thought of; and the prospect of the future must be as bad as the experience of the past. When things are in that lamentable condition, the nature of the disease is to indicate the remedy to those whom Nature has qualified to administer in extremities this critical, ambiguous, bitter potion to a distempered state. Times and occasions and provocations will teach their own lessons. The wise will determine from the gravity of the case; the irritable, from sensibility to oppression; the high-minded, from disdain and indignation at abusive power in unworthy hands; the brave and bold, from the love of honorable danger in a generous cause: but, with or without right, a revolution will be the very last resource of the thinking and the good.

The third head of right asserted by the pulpit of the Old Jewry, namely, the "right to form a government for ourselves," has, at least, as little countenance from anything done at the Revolution, either in precedent or principle, as the two first of their claims. The Revolution was made to preserve our ancient indisputable laws and liberties, and that ancient constitution of government which is our only security for law and liberty. If you are desirous of knowing the spirit of our Constitution, and the policy which predominated in that great period which has secured it to this hour, pray look for both in our histories, in our records, in our acts of Parliament and journals of Parliament, and not in the sermons of the Old Jewry, and the after-dinner toasts of the Revolution Society. In the former you will find other ideas and another language. Such a claim is as ill-suited to our temper and wishes as it is unsupported by any appearance of authority. The very idea of the fabrication of a new government is enough to fill us with disgust and horror. We wished at the period of the Revolution, and do now wish, to derive all we possess as an inheritance from our forefathers. Upon that body and stock of inheritance we have taken care not to inoculate any scion alien to the nature of the original plant. All the reformations we have hitherto made have proceeded upon the principle of reference to antiquity; and I hope, nay, I am persuaded, that all those which possibly may be made hereafter will be carefully formed upon analogical precedent, authority, and example.

Our oldest reformation is that of Magna Charta. You will see that Sir Edward Coke, that great oracle of our law, and indeed all the great men who follow him, to Blackstone,
%[84]
\footnote{ See Blackstone's Magna Charta, printed at Oxford, 1759.}
 are industrious to prove the pedigree of our liberties. They endeavor to prove that the ancient charter, the Magna Charta of King John, was connected with another positive charter from Henry the First, and that both the one and the other were nothing more than a reaffirmance of the still more ancient standing law of the kingdom. In the matter of fact, for the greater part, these authors appear to be in the right; perhaps not always: but if the lawyers mistake in some particulars, it proves my position still the more strongly; because it demonstrates the powerful prepossession towards antiquity with which the minds of all our lawyers and legislators, and of all the people whom they wish to influence, have been always filled, and the stationary policy of this kingdom in considering their most sacred rights and franchises as an inheritance.

In the famous law of the 3rd of Charles the First, called the Petition of Right, the Parliament says to the king, "Your subjects have inherited this freedom": claiming their franchises, not on abstract principles, "as the rights of men," but as the rights of Englishmen, and as a patrimony derived from their forefathers. Selden, and the other profoundly learned men who drew this Petition of Right, were as well acquainted, at least, with all the general theories concerning the "rights of men" as any of the discoursers in our pulpits or on your tribune: full as well as Dr. Price, or as the Abbé Sièyes. But, for reasons worthy of that practical wisdom which superseded their theoretic science, they preferred this positive, recorded, hereditary title to all which can be dear to the man and the citizen to that vague, speculative right which exposed their sure inheritance to be scrambled for and torn to pieces by every wild, litigious spirit.

The same policy pervades all the laws which have since been made for the preservation of our liberties. In the 1st of William and Mary, in the famous statute called the Declaration of Right, the two Houses utter not a syllable of "a right to frame a government for themselves." You will see that their whole care was to secure the religion, laws, and liberties that had been long possessed, and had been lately endangered. "Taking
%[85]
\footnote{ 1 W. and M.}
 into their most serious consideration the best means for making such an establishment that their religion, laws, and liberties might not be in danger of being again subverted," they auspicate all their proceedings by stating as some of those best means, "in the first place," to do "as their ancestors in like cases have usually done for vindicating their ancient rights and liberties, to declare";—and then they pray the king and queen "that it may be declared and enacted that all and singular the rights and liberties asserted and declared are the true ancient and indubitable rights and liberties of the people of this kingdom."

You will observe, that, from Magna Charta to the Declaration of Right, it has been the uniform policy of our Constitution to claim and assert our liberties as an entailed inheritance derived to us from our forefathers, and to be transmitted to our posterity,—as an estate specially belonging to the people of this kingdom, without any reference whatever to any other more general or prior right. By this means our Constitution preserves an unity in so great a diversity of its parts. We have an inheritable crown, an inheritable peerage, and a House of Commons and a people inheriting privileges, franchises, and liberties from a long line of ancestors.

This policy appears to me to be the result of profound reflection,—or rather the happy effect of following Nature, which is wisdom without reflection, and above it. A spirit of innovation is generally the result of a selfish temper and confined views. People will not look forward to posterity, who never look backward to their ancestors. Besides, the people of England well know that the idea of inheritance furnishes a sure principle of conservation, and a sure principle of transmission, without at all excluding a principle of improvement. It leaves acquisition free; but it secures what it acquires. Whatever advantages are obtained by a state proceeding on these maxims are locked fast as in a sort of family settlement, grasped as in a kind of mortmain forever. By a constitutional policy working after the pattern of Nature, we receive, we hold, we transmit our government and our privileges, in the same manner in which we enjoy and transmit our property and our lives. The institutions of policy, the goods of fortune, the gifts of Providence, are handed down to us, and from us, in the same course and order. Our political system is placed in a just correspondence and symmetry with the order of the world, and with the mode of existence decreed to a permanent body composed of transitory parts,—wherein, by the disposition of a stupendous wisdom, moulding together the great mysterious incorporation of the human race, the whole, at one time, is never old or middle-aged or young, but, in a condition of unchangeable constancy, moves on through the varied tenor of perpetual decay, fall, renovation, and progression. Thus, by preserving the method of Nature in the conduct of the state, in what we improve we are never wholly new, in what we retain we are never wholly obsolete. By adhering in this manner and on those principles to our forefathers, we are guided, not by the superstition of antiquarians, but by the spirit of philosophic analogy. In this choice of inheritance we have given to our frame of polity the image of a relation in blood: binding up the Constitution of our country with our dearest domestic ties; adopting our fundamental laws into the bosom of our family affections; keeping inseparable, and cherishing with the warmth of all their combined and mutually reflected charities, our state, our hearths, our sepulchres, and our altars.

Through the same plan of a conformity to Nature in our artificial institutions, and by calling in the aid of her unerring and powerful instincts to fortify the fallible and feeble contrivances of our reason, we have derived several other, and those no small benefits, from considering our liberties in the light of an inheritance. Always acting as if in the presence of canonized forefathers, the spirit of freedom, leading in itself to misrule and excess, is tempered with an awful gravity. This idea of a liberal descent inspires us with a sense of habitual native dignity, which prevents that upstart insolence almost inevitably adhering to and disgracing those who are the first acquirers of any distinction. By this means our liberty becomes a noble freedom. It carries an imposing and majestic aspect. It has a pedigree and illustrating ancestors. It has its bearings and its ensigns armorial. It has its gallery of portraits, its monumental inscriptions, its records, evidences, and titles. We procure reverence to our civil institutions on the principle upon which Nature teaches us to revere individual men: on account of their age, and on account of those from whom they are descended. All your sophisters cannot produce anything better adapted to preserve a rational and manly freedom than the course that we have pursued, who have chosen our nature rather than our speculations, our breasts rather than our inventions, for the great conservatories and magazines of our rights and privileges.

You might, if you pleased, have profited of our example, and have given to your recovered freedom a correspondent dignity. Your privileges, though discontinued, were not lost to memory. Your Constitution, it is true, whilst you were out of possession, suffered waste and dilapidation; but you possessed in some parts the walls, and in all the foundations, of a noble and venerable castle. You might have repaired those walls; you might have built on those old foundations. Your Constitution was suspended before it was perfected; but you had the elements of a Constitution very nearly as good as could be wished. In your old states you possessed that variety of parts corresponding with the various descriptions of which your community was happily composed; you had all that combination and all that opposition of interests, you had that action and counteraction, which, in the natural and in the political world, from the reciprocal struggle of discordant powers draws out the harmony of the universe. These opposed and conflicting interests, which you considered as so great a blemish in your old and in our present Constitution, interpose a salutary check to all precipitate resolutions. They render deliberation a matter, not of choice, but of necessity; they make all change a subject of compromise, which naturally begets moderation; they produce temperaments, preventing the sore evil of harsh, crude, unqualified reformations, and rendering all the headlong exertions of arbitrary power, in the few or in the many, forever impracticable. Through that diversity of members and interests, general liberty had as many securities as there were separate views in the several orders; whilst by pressing down the whole by the weight of a real monarchy, the separate parts would have been prevented from warping and starting from their allotted places.

You had all these advantages in your ancient states; but you chose to act as if you had never been moulded into civil society, and had everything to begin anew. You began ill, because you began by despising everything that belonged to you. You set up your trade without a capital. If the last generations of your country appeared without much lustre in your eyes, you might have passed them by, and derived your claims from a more early race of ancestors. Under a pious predilection for those ancestors, your imaginations would have realized in them a standard of virtue and wisdom beyond the vulgar practice of the hour; and you would have risen with the example to whose imitation you aspired. Respecting your forefathers, you would have been taught to respect yourselves. You would not have chosen to consider the French as a people of yesterday, as a nation of low-born, servile wretches until the emancipating year of 1789. In order to furnish, at the expense of your honor, an excuse to your apologists here for several enormities of yours, you would not have been content to be represented as a gang of Maroon slaves, suddenly broke loose from the house of bondage, and therefore to be pardoned for your abuse of the liberty to which you were not accustomed, and were ill fitted. Would it not, my worthy friend, have been wiser to have you thought, what I for one always thought you, a generous and gallant nation, long misled to your disadvantage by your high and romantic sentiments of fidelity, honor, and loyalty; that events had been unfavorable to you, but that you were not enslaved through any illiberal or servile disposition; that, in your most devoted submission, you were actuated by a principle of public spirit; and that it was your country you worshipped, in the person of your king? Had you made it to be understood, that, in the delusion of this amiable error, you had gone further than your wise ancestors,—that you were resolved to resume your ancient privileges, whilst you preserved the spirit of your ancient and your recent loyalty and honor; or if, diffident of yourselves, and not clearly discerning the almost obliterated Constitution of your ancestors, you had looked to your neighbors in this land, who had kept alive the ancient principles and models of the old common law of Europe, meliorated and adapted to its present state,—by following wise examples you would have given new examples of wisdom to the world. You would have rendered the cause of liberty venerable in the eyes of every worthy mind in every nation. You would have shamed despotism from the earth, by showing that freedom was not only reconcilable, but, as, when well disciplined, it is, auxiliary to law. You would have had an unoppressive, but a productive revenue. You would have had a flourishing commerce to feed it. You would have had a free Constitution, a potent monarchy, a disciplined army, a reformed and venerated clergy,—a mitigated, but spirited nobility, to lead your virtue, not to overlay it; you would have had a liberal order of commons, to emulate and to recruit that nobility; you would have had a protected, satisfied, laborious, and obedient people, taught to seek and to recognize the happiness that is to be found by virtue in all conditions,—in which consists the true moral equality of mankind, and not in that monstrous fiction which, by inspiring false ideas and vain expectations into men destined to travel in the obscure walk of laborious life, serves only to aggravate and embitter that real inequality which it never can remove, and which the order of civil life establishes as much for the benefit of those whom it must leave in an humble state as those whom it is able to exalt to a condition more splendid, but not more happy. You had a smooth and easy career of felicity and glory laid open to you, beyond anything recorded in the history of the world; but you have shown that difficulty is good for man.

Compute your gains; see what is got by those extravagant and presumptuous speculations which have taught your leaders to despise all their predecessors, and all their contemporaries, and even to despise themselves, until the moment in which they became truly despicable. By following those false lights, France has bought undisguised calamities at a higher price than any nation has purchased the most unequivocal blessings. France has bought poverty by crime. France has not sacrificed her virtue to her interest; but she has abandoned her interest, that she might prostitute her virtue. All other nations have begun the fabric of a new government, or the reformation of an old, by establishing originally, or by enforcing with greater exactness, some rites or other of religion. All other people have laid the foundations of civil freedom in severer manners, and a system of a more austere and masculine morality. France, when she let loose the reins of regal authority, doubled the license of a ferocious dissoluteness in manners, and of an insolent irreligion in opinions and practices,—and has extended through all ranks of life, as if she were communicating some privilege, or laying open some secluded benefit, all the unhappy corruptions that usually were the disease of wealth and power. This is one of the new principles of equality in France.

France, by the perfidy of her leaders, has utterly disgraced the tone of lenient council in the cabinets of princes, and disarmed it of its most potent topics. She has sanctified the dark, suspicious maxims of tyrannous distrust, and taught kings to tremble at (what will hereafter be called) the delusive plausibilities of moral politicians. Sovereigns will consider those who advise them to place an unlimited confidence in their people as subverters of their thrones,—as traitors who aim at their destruction, by leading their easy good-nature, under specious pretences, to admit combinations of bold and faithless men into a participation of their power. This alone (if there were nothing else) is an irreparable calamity to you and to mankind. Remember that your Parliament of Paris told your king, that, in calling the states together, he had nothing to fear but the prodigal excess of their zeal in providing for the support of the throne. It is right that these men should hide their heads. It is right that they should bear their part in the ruin which their counsel has brought on their sovereign and their country. Such sanguine declarations tend to lull authority asleep,—to encourage it rashly to engage in perilous adventures of untried policy,—to neglect those provisions, preparations, and precautions which distinguish benevolence from imbecility, and without which no man can answer for the salutary effect of any abstract plan of government or of freedom. For want of these, they have seen the medicine of the state corrupted into its poison. They have seen the French rebel against a mild and lawful monarch, with more fury, outrage, and insult than ever any people has been known to rise against the most illegal usurper or the most sanguinary tyrant. Their resistance was made to concession; their revolt was from protection; their blow was aimed at a hand holding out graces, favors, and immunities.

This was unnatural. The rest is in order. They have found their punishment in their success. Laws overturned; tribunals subverted; industry without vigor; commerce expiring; the revenue unpaid, yet the people impoverished; a church pillaged, and a state not relieved; civil and military anarchy made the constitution of the kingdom; everything human and divine sacrificed to the idol of public credit, and national bankruptcy the consequence; and, to crown all, the paper securities of new, precarious, tottering power, the discredited paper securities of impoverished fraud and beggared rapine, held out as a currency for the support of an empire, in lieu of the two great recognized species that represent the lasting, conventional credit of mankind, which disappeared and hid themselves in the earth from whence they came, when the principle of property, whose creatures and representatives they are, was systematically subverted.

Were all these dreadful things necessary? Were they the inevitable results of the desperate struggle of determined patriots, compelled to wade through blood and tumult to the quiet shore of a tranquil and prosperous liberty? No! nothing like it. The fresh ruins of France, which shock our feelings wherever we can turn our eyes, are not the devastation of civil war: they are the sad, but instructive monuments of rash and ignorant counsel in time of profound peace. They are the display of inconsiderate and presumptuous, because unresisted and irresistible authority. The persons who have thus squandered away the precious treasure of their crimes, the persons who have made this prodigal and wild waste of public evils, (the last stake reserved for the ultimate ransom of the state,) have met in their progress with little, or rather with no opposition at all. Their whole march was more like a triumphal procession than the progress of a war. Their pioneers have gone before them, and demolished and laid everything level at their feet. Not one drop of their blood have they shed in the cause of the country they have ruined. They have made no sacrifices to their projects of greater consequence than their shoe-buckles, whilst they were imprisoning their king, murdering their fellow-citizens, and bathing in tears and plunging in poverty and distress thousands of worthy men and worthy families. Their cruelty has not even been the base result of fear. It has been the effect of their sense of perfect safety, in authorizing treasons, robberies, rapes, assassinations, slaughters, and burnings, throughout their harassed land. But the cause of all was plain from the beginning.

This unforced choice, this fond election of evil, would appear perfectly unaccountable, if we did not consider the composition of the National Assembly: I do not mean its formal constitution, which, as it now stands, is exceptionable enough, but the materials of which in a great measure it is composed, which is of ten thousand times greater consequence than all the formalities in the world. If we were to know nothing of this assembly but by its title and function, no colors could paint to the imagination anything more venerable. In that light, the mind of an inquirer, subdued by such an awful image as that of the virtue and wisdom of a whole people collected into one focus, would pause and hesitate in condemning things even of the very worst aspect. Instead of blamable, they would appear only mysterious. But no name, no power, no function, no artificial institution whatsoever, can make the men, of whom any system of authority is composed, any other than God, and Nature, and education, and their habits of life have made them. Capacities beyond these the people have not to give. Virtue and wisdom may be the objects of their choice; but their choice confers neither the one nor the other on those upon whom they lay their ordaining hands. They have not the engagement of Nature, they have not the promise of Revelation for any such powers.

After I had read over the list of the persons and descriptions elected into the Tiers État, nothing which they afterwards did could appear astonishing. Among them, indeed, I saw some of known rank, some of shining talents; but of any practical experience in the state not one man was to be found. The best were only men of theory. But whatever the distinguished few may have been, it is the substance and mass of the body which constitutes its character, and must finally determine its direction. In all bodies, those who will lead must also, in a considerable degree, follow. They must conform their propositions to the taste, talent, and disposition of those whom they wish to conduct: therefore, if an assembly is viciously or feebly composed in a very great part of it, nothing but such a supreme degree of virtue as very rarely appears in the world, and for that reason cannot enter into calculation, will prevent the men of talents disseminated through it from becoming only the expert instruments of absurd projects. If, what is the more likely event, instead of that unusual degree of virtue, they should be actuated by sinister ambition and a lust of meretricious glory, then the feeble part of the assembly, to whom at first they conform, becomes, in its turn, the dupe and instrument of their designs. In this political traffic, the leaders will be obliged to bow to the ignorance of their followers, and the followers to become subservient to the worst designs of their leaders.

To secure any degree of sobriety in the propositions made by the leaders in any public assembly, they ought to respect, in some degree perhaps to fear, those whom they conduct. To be led any otherwise than blindly, the followers must be qualified, if not for actors, at least for judges; they must also be judges of natural weight and authority. Nothing can secure a steady and moderate conduct in such assemblies, but that the body of them should be respectably composed, in point of condition in life, of permanent property, of education, and of such habits as enlarge and liberalize the understanding.

In the calling of the States-General of France, the first thing that struck me was a great departure from the ancient course. I found the representation for the third estate composed of six hundred persons. They were equal in number to the representatives of both the other orders. If the orders were to act separately, the number would not, beyond the consideration of the expense, be of much moment. But when it became apparent that the three orders were to be melted down into one, the policy and necessary effect of this numerous representation became obvious. A very small desertion from either of the other two orders must throw the power of both into the hands of the third. In fact, the whole power of the state was soon resolved into that body. Its due composition became, therefore, of infinitely the greater importance.

Judge, Sir, of my surprise, when I found that a very great proportion of the Assembly (a majority, I believe, of the members who attended) was composed of practitioners in the law. It was composed, not of distinguished magistrates, who had given pledges to their country of their science, prudence, and integrity,—not of leading advocates, the glory of the bar,—not of renowned professors in universities,—but for the far greater part, as it must in such a number, of the inferior, unlearned, mechanical, merely instrumental members of the profession. There were distinguished exceptions; but the general composition was of obscure provincial advocates, of stewards of petty local jurisdictions, country attorneys, notaries, and the whole train of the ministers of municipal litigation, the fomenters and conductors of the petty war of village vexation. From the moment I read the list, I saw distinctly, and very nearly as it has happened, all that was to follow.

The degree of estimation in which any profession is held becomes the standard of the estimation in which the professors hold themselves. Whatever the personal merits of many individual lawyers might have been, (and in many it was undoubtedly very considerable,) in that military kingdom no part of the profession had been much regarded, except the highest of all, who often united to their professional offices great family splendor, and were invested with great power and authority. These certainly were highly respected, and even with no small degree of awe. The next rank was not much esteemed; the mechanical part was in a very low degree of repute.

Whenever the supreme authority is vested in a body so composed, it must evidently produce the consequences of supreme authority placed in the hands of men not taught habitually to respect themselves,—who had no previous fortune in character at stake,—who could not be expected to bear with moderation or to conduct with discretion a power which they themselves, more than any others, must be surprised to find in their hands. Who could flatter himself that these men, suddenly, and as it were by enchantment, snatched from the humblest rank of subordination, would not be intoxicated with their unprepared greatness? Who could conceive that men who are habitually meddling, daring, subtle, active, of litigious dispositions and unquiet minds, would easily fall back into their old condition of obscure contention, and laborious, low, and unprofitable chicane? Who could doubt but that, at any expense to the state, of which they understood nothing, they must pursue their private interests, which they understood but too well? It was not an event depending on chance or contingency. It was inevitable; it was necessary; it was planted in the nature of things. They must join (if their capacity did not permit them to lead) in any project which could procure to them a litigious constitution,—which could lay open to them those innumerable lucrative jobs which follow in the train of all great convulsions and revolutions in the state, and particularly in all great and violent permutations of property. Was it to be expected that they would attend to the stability of property, whose existence had always depended upon whatever rendered property questionable, ambiguous, and insecure? Their objects would be enlarged with their elevation; but their disposition, and habits, and mode of accomplishing their designs must remain the same.

Well! but these men were to be tempered and restrained by other descriptions, of more sober minds and more enlarged understandings. Were they, then, to be awed by the supereminent authority and awful dignity of a handful of country clowns, who have seats in that assembly, some of whom are said not to be able to read and write,—and by not a greater number of traders, who, though somewhat more instructed, and more conspicuous in the order of society, had never known anything beyond their counting-house? No! both these descriptions were more formed to be overborne and swayed by the intrigues and artifices of lawyers than to become their counterpoise. With such a dangerous disproportion, the whole must needs be governed by them.

To the faculty of law was joined a pretty considerable proportion of the faculty of medicine. This faculty had not, any more than that of the law, possessed in France its just estimation. Its professors, therefore, must have the qualities of men not habituated to sentiments of dignity. But supposing they had ranked as they ought to do, and as with us they do actually, the sides of sick-beds are not the academies for forming statesmen and legislators. Then came the dealers in stocks and funds, who must be eager, at any expense, to change their ideal paper wealth for the more solid substance of land. To these were joined men of other descriptions, from whom as little knowledge of or attention to the interests of a great state was to be expected, and as little regard to the stability of any institution,—men formed to be instruments, not controls.—Such, in general, was the composition of the Tiers État in the National Assembly; in which was scarcely to be perceived the slightest traces of what we call the natural landed interest of the country.

We know that the British House of Commons, without shutting its doors to any merit in any class, is, by the sure operation of adequate causes, filled with everything illustrious in rank, in descent, in hereditary and in acquired opulence, in cultivated talents, in military, civil, naval, and politic distinction, that the country can afford. But supposing, what hardly can be supposed as a case, that the House of Commons should be composed in the same manner with the Tiers État in France,—would this dominion of chicane be borne with patience, or even conceived without horror? God forbid I should insinuate anything derogatory to that profession which is another priesthood, administering the rights of sacred justice! But whilst I revere men in the functions which belong to them, and would do as much as one man can do to prevent their exclusion from any, I cannot, to flatter them, give the lie to Nature. They are good and useful in the composition; they must be mischievous, if they preponderate so as virtually to become the whole. Their very excellence in their peculiar functions may be far from a qualification for others. It cannot escape observation, that, when men are too much confined to professional and faculty habits, and, as it were, inveterate in the recurrent employment of that narrow circle, they are rather disabled than qualified for whatever depends on the knowledge of mankind, on experience in mixed affairs, on a comprehensive, connected view of the various, complicated, external, and internal interests which go to the formation of that multifarious thing called a State.

After all, if the House of Commons were to have an wholly professional and faculty composition, what is the power of the House of Commons, circumscribed and shut in by the immovable barriers of laws, usages, positive rules of doctrine and practice, counterpoised by the House of Lords, and every moment of its existence at the discretion of the crown to continue, prorogue, or dissolve us? The power of the House of Commons, direct or indirect, is, indeed, great: and long may it be able to preserve its greatness, and the spirit belonging to true greatness, at the full!—and it will do so, as long as it can keep the breakers of law in India from becoming the makers of law for England. The power, however, of the House of Commons, when least diminished, is as a drop of water in the ocean, compared to that residing in a settled majority of your National Assembly. That assembly, since the destruction of the orders, has no fundamental law, no strict convention, no respected usage to restrain it. Instead of finding themselves obliged to conform to a fixed constitution, they have a power to make a constitution which shall conform to their designs. Nothing in heaven or upon earth can serve as a control on them. What ought to be the heads, the hearts, the dispositions, that are qualified, or that dare, not only to make laws under a fixed constitution, but at one heat to strike out a totally new constitution for a great kingdom, and in every part of it, from the monarch on the throne to the vestry of a parish? But

\centerline{"Fools rush in where angels fear to tread."}

In such a state of unbounded power, for undefined and undefinable purposes, the evil of a moral and almost physical inaptitude of the man to the function must be the greatest we can conceive to happen in the management of human affairs.

Having considered the composition of the third estate, as it stood in its original frame, I took a view of the representatives of the clergy. There, too, it appeared that full as little regard was had to the general security of property, or to the aptitude of the deputies for their public purposes, in the principles of their election. That election was so contrived as to send a very large proportion of mere country curates to the great and arduous work of new-modelling a state: men who never had seen the state so much as in a picture; men who knew nothing of the world beyond the bounds of an obscure village; who, immersed in hopeless poverty, could regard all property, whether secular or ecclesiastical, with no other eye than that of envy; among whom must be many who, for the smallest hope of the meanest dividend in plunder, would readily join in any attempts upon a body of wealth in which they could hardly look to have any share, except in a general scramble. Instead of balancing the power of the active chicaners in the other assembly, these curates must necessarily become the active coadjutors, or at best the passive instruments, of those by whom they had been habitually guided in their petty village concerns. They, too, could hardly be the most conscientious of their kind, who, presuming upon their incompetent understanding, could intrigue for a trust which led them from their natural relation to their flocks, and their natural spheres of action, to undertake the regeneration of kingdoms. This preponderating weight, being added to the force of the body of chicane in the Tiers État, completed that momentum of ignorance, rashness, presumption, and lust of plunder, which nothing has been able to resist.

To observing men it must have appeared from the beginning, that the majority of the third estate, in conjunction with such a deputation from the clergy as I have described, whilst it pursued the destruction of the nobility, would inevitably become subservient to the worst designs of individuals in that class. In the spoil and humiliation of their own order these individuals would possess a sure fund for the pay of their new followers. To squander away the objects which made the happiness of their fellows would be to them no sacrifice at all. Turbulent, discontented men of quality, in proportion as they are puffed up with personal pride and arrogance, generally despise their own order. One of the first symptoms they discover of a selfish and mischievous ambition is a profligate disregard of a dignity which they partake with others. To be attached to the subdivision, to love the little platoon we belong to in society, is the first principle (the germ, as it were) of public affections. It is the first link in the series by which we proceed towards a love to our country and to mankind. The interest of that portion of social arrangement is a trust in the hands of all those who compose it; and as none but bad men would justify it in abuse, none but traitors would barter it away for their own personal advantage.

There were, in the time of our civil troubles in England, (I do not know whether you have any such in your Assembly in France,) several persons, like the then Earl of Holland, who by themselves or their families had brought an odium on the throne by the prodigal dispensation of its bounties towards them, who afterwards joined in the rebellions arising from the discontents of which they were themselves the cause: men who helped to subvert that throne to which they owed, some of them, their existence, others all that power which they employed to ruin their benefactor. If any bounds are set to the rapacious demands of that sort of people, or that others are permitted to partake in the objects they would engross, revenge and envy soon fill up the craving void that is left in their avarice. Confounded by the complication of distempered passions, their reason is disturbed; their views become vast and perplexed,—to others inexplicable, to themselves uncertain. They find, on all sides, bounds to their unprincipled ambition in any fixed order of things; but in the fog and haze of confusion all is enlarged, and appears without any limit.

When men of rank sacrifice all ideas of dignity to an ambition without a distinct object, and work with low instruments and for low ends, the whole composition becomes low and base. Does not something like this now appear in France? Does it not produce something ignoble and inglorious: a kind of meanness in all the prevalent policy; a tendency in all that is done to lower along with individuals all the dignity and importance of the state? Other revolutions have been conducted by persons who, whilst they attempted or affected changes in the commonwealth, sanctified their ambition by advancing the dignity of the people whose peace they troubled. They had long views. They aimed at the rule, not at the destruction of their country. They were men of great civil and great military talents, and if the terror, the ornament of their age. They were not like Jew brokers contending with each other who could best remedy with fraudulent circulation and depreciated paper the wretchedness and ruin brought on their country by their degenerate councils. The compliment made to one of the great bad men of the old stamp (Cromwell) by his kinsman, a favorite poet of that time, shows what it was he proposed, and what indeed to a great degree he accomplished in the success of his ambition:—

\begin{verse}
Still as you rise, the state, exalted too, \\
Finds no distemper whilst 't is changed by you; \\
Changed like the world's great scene, when without noise \\
The rising sun night's vulgar lights destroys.
\end{verse}

These disturbers were not so much like men usurping power as asserting their natural place in society. Their rising was to illuminate and beautify the world. Their conquest over their competitors was by outshining them. The hand, that, like a destroying angel, smote the country, communicated to it the force and energy under which it suffered. I do not say, (God forbid!) I do not say that the virtues of such men were to be taken as a balance to their crimes; but they were some corrective to their effects. Such was, as I said, our Cromwell. Such were your whole race of Guises, Condés, and Colignys. Such the Richelieus, who in more quiet times acted in the spirit of a civil war. Such, as better men, and in a less dubious cause, were your Henry the Fourth, and your Sully, though nursed in civil confusions, and not wholly without some of their taint. It is a thing to be wondered at, to see how very soon France, when she had a moment to respire, recovered and emerged from the longest and most dreadful civil war that ever was known in any nation. Why? Because, among all their massacres, they had not slain the mind in their country. A conscious dignity, a noble pride, a generous sense of glory and emulation, was not extinguished. On the contrary, it was kindled and inflamed. The organs also of the state, however shattered, existed. All the prizes of honor and virtue, all the rewards, all the distinctions, remained. But your present confusion, like a palsy, has attacked the fountain of life itself. Every person in your country, in a situation to be actuated by a principle of honor, is disgraced and degraded, and can entertain no sensation of life, except in a mortified and humiliated indignation. But this generation will quickly pass away. The next generation of the nobility will resemble the artificers and clowns, and money-jobbers, usurers, and Jews, who will be always their fellows, sometimes their masters. Believe me, Sir, those who attempt to level never equalize. In all societies consisting of various descriptions of citizens, some description must be uppermost. The levellers, therefore, only change and pervert the natural order of things: they load the edifice of society by setting up in the air what the solidity of the structure requires to be on the ground. The associations of tailors and carpenters, of which the republic (of Paris, for instance) is composed, cannot be equal to the situation into which, by the worst of usurpations, an usurpation on the prerogatives of Nature, you attempt to force them.

The Chancellor of France, at the opening of the States, said, in a tone of oratorial flourish, that all occupations were honorable. If he meant only that no honest employment was disgraceful, he would not have gone beyond the truth. But in asserting that anything is honorable, we imply some distinction in its favor. The occupation of a hair-dresser, or of a working tallow-chandler, cannot be a matter of honor to any person,—to say nothing of a number of other more servile employments. Such descriptions of men ought not to suffer oppression from the state; but the state suffers oppression, if such as they, either individually or collectively, are permitted to rule. In this you think you are combating prejudice, but you are at war with Nature.
%[86]
\footnote{ Ecclesiasticus, chap, xxxviii. ver. 24, 25. "The wisdom of a learned man cometh by opportunity of leisure: and he that hath little business shall become wise. How can he get wisdom that holdeth the plough, and that glorieth in the goad; that driveth oxen, and is occupied in their labors, and whose talk is of bullocks?"

Ver. 27. "So every carpenter and workmaster, that laboreth night and day," \&c.

Ver. 33. "They shall not be sought for in public counsel, nor sit high in the congregation: they shall not sit on the judge's seat, nor understand the sentence of judgment: they cannot declare justice and judgment, and they shall not be found where parables are spoken."

Ver. 34. "But they will maintain the state of the world."

I do not determine whether this book be canonical, as the Gallican Church (till lately) has considered it, or apocryphal, as here it is taken. I am sure it contains a great deal of sense and truth.
}

I do not, my dear Sir, conceive you to be of that sophistical, captious spirit, or of that uncandid dullness, as to require, for every general observation or sentiment, an explicit detail of the correctives and exceptions which reason will presume to be included in all the general propositions which come from reasonable men. You do not imagine that I wish to confine power, authority, and distinction to blood and names and titles. No, Sir. There is no qualification for government but virtue and wisdom, actual or presumptive. Wherever they are actually found, they have, in whatever state, condition, profession, or trade, the passport of Heaven to human place and honor. Woe to the country which would madly and impiously reject the service of the talents and virtues, civil, military, or religious, that are given to grace and to serve it; and would condemn to obscurity everything formed to diffuse lustre and glory around a state! Woe to that country, too, that, passing into the opposite extreme, considers a low education, a mean, contracted view of things, a sordid, mercenary occupation, as a preferable title to command! Everything ought to be open,—but not indifferently to every man. No rotation, no appointment by lot, no mode of election operating in the spirit of sortition or rotation, can be generally good in a government conversant in extensive objects; because they have no tendency, direct or indirect, to select the man with a view to the duty, or to accommodate the one to the other. I do not hesitate to say that the road to eminence and power, from obscure condition, ought not to be made too easy, nor a thing too much of course. If rare merit be the rarest of all rare things, it ought to pass through some sort of probation. The temple of honor ought to be seated on an eminence. If it be opened through virtue, let it be remembered, too, that virtue is never tried but by some difficulty and some struggle.

Nothing is a due and adequate representation of a state, that does not represent its ability, as well as its property. But as ability is a vigorous and active principle, and as property is sluggish, inert, and timid, it never can be safe from the invasions of ability, unless it be, out of all proportion, predominant in the representation. It must be represented, too, in great masses of accumulation, or it is not rightly protected. The characteristic essence of property, formed out of the combined principles of its acquisition and conservation, is to be unequal. The great masses, therefore, which excite envy, and tempt rapacity, must be put out of the possibility of danger. Then they form a natural rampart about the lesser properties in all their gradations. The same quantity of property which is by the natural course of things divided among many has not the same operation. Its defensive power is weakened as it is diffused. In this diffusion each man's portion is less than what, in the eagerness of his desires, he may flatter himself to obtain by dissipating the accumulations of others. The plunder of the few would, indeed, give but a share inconceivably small in the distribution to the many. But the many are not capable of making this calculation; and those who lead them to rapine never intend this distribution.

The power of perpetuating our property in our families is one of the most valuable and interesting circumstances belonging to it, and that which tends the most to the perpetuation of society itself. It makes our weakness subservient to our virtue; it grafts benevolence even upon avarice. The possessors of family wealth, and of the distinction which attends hereditary possession, (as most concerned in it,) are the natural securities for this transmission. With us the House of Peers is formed upon this principle. It is wholly composed of hereditary property and hereditary distinction, and made, therefore, the third of the legislature, and, in the last event, the sole judge of all property in all its subdivisions. The House of Commons, too, though not necessarily, yet in fact, is always so composed, in the far greater part. Let those large proprietors be what they will, (and they have their chance of being amongst the best,) they are, at the very worst, the ballast in the vessel of the commonwealth. For though hereditary wealth, and the rank which goes with it, are too much idolized by creeping sycophants, and the blind, abject admirers of power, they are too rashly slighted in shallow speculations of the petulant, assuming, short-sighted coxcombs of philosophy. Some decent, regulated preëminence, some preference (not exclusive appropriation) given to birth, is neither unnatural, nor unjust, nor impolitic.

It is said that twenty-four millions ought to prevail over two hundred thousand. True; if the constitution of a kingdom be a problem of arithmetic. This sort of discourse does well enough with the lamp-post for its second: to men who may reason calmly it is ridiculous The will of the many, and their interest, must very often differ; and great will be the difference when they make an evil choice. A government of five hundred country attorneys and obscure curates is not good for twenty-four millions of men, though it were chosen by eight-and-forty millions; nor is it the better for being guided by a dozen of persons of quality who have betrayed their trust in order to obtain that power. At present, you seem in everything to have strayed out of the high road of Nature. The property of France does not govern it. Of course property is destroyed, and rational liberty has no existence. All you have got for the present is a paper circulation, and a stock-jobbing constitution: and as to the future, do you seriously think that the territory of France, upon the republican system of eighty-three independent municipalities, (to say nothing of the parts that compose them,) can ever be governed as one body, or can ever be set in motion by the impulse of one mind? When the National Assembly has completed its work, it will have accomplished its ruin. These commonwealths will not long bear a state of subjection to the republic of Paris. They will not bear that this one body should monopolize the captivity of the king, and the dominion over the assembly calling itself national. Each will keep its own portion of the spoil of the Church to itself; and it will not suffer either that spoil, or the more just fruits of their industry, or the natural produce of their soil, to be sent to swell the insolence or pamper the luxury of the mechanics of Paris. In this they will see none of the equality, under the pretence of which they have been tempted to throw off their allegiance to their sovereign, as well as the ancient constitution of their country. There can be no capital city in such a constitution as they have lately made. They have forgot, that, when they framed democratic governments, they had virtually dismembered their country. The person whom they persevere in calling king has not power left to him by the hundredth part sufficient to hold together this collection of republics. The republic of Paris will endeavor, indeed, to complete the debauchery of the army, and illegally to perpetuate the Assembly, without resort to its constituents, as the means of continuing its despotism. It will make efforts, by becoming the heart of a boundless paper circulation, to draw everything to itself: but in vain. All this policy in the end will appear as feeble as it is now violent.

If this be your actual situation, compared to the situation to which you were called, as it were by the voice of God and man, I cannot find it in my heart to congratulate you on the choice you have made, or the success which has attended your endeavors. I can as little recommend to any other nation a conduct grounded on such principles and productive of such effects. That I must leave to those who can see further into your affairs than I am able to do, and who best know how far your actions are favorable to their designs. The gentlemen of the Revolution Society, who were so early in their congratulations, appear to be strongly of opinion that there is some scheme of politics relative to this country, in which your proceedings may in some way be useful. For your Dr. Price, who seems to have speculated himself into no small degree of fervor upon this subject, addresses his auditors in the following very remarkable words:—"I cannot conclude without recalling particularly to your recollection a consideration which I have more than once alluded to, and which probably your thoughts have been all along anticipating; a consideration with which my mind is impressed more than can express: I mean the consideration of the favorableness of the present times to all exertions in the cause of liberty."

It is plain that the mind of this political preacher was at the time big with some extraordinary design; and it is very probable that the thoughts of his audience, who understood him better than I do, did all along run before him in his reflection, and in the whole train of consequences to which it led.

Before I read that sermon, I really thought I had lived in a free country; and it was an error I cherished, because it gave me a greater liking to the country I lived in. I was, indeed, aware that a jealous, ever-waking vigilance, to guard the treasure of our liberty, not only from invasion, but from decay and corruption, was our best wisdom and our first duty. However, I considered that treasure rather as a possession to be secured than as a prize to be contended for. I did not discern how the present time came to be so very favorable to all exertions in the cause of freedom. The present time differs from any other only by the circumstance of what is doing in France. If the example of that nation is to have an influence on this, I can easily conceive why some of their proceedings which have an unpleasant aspect, and are not quite reconcilable to humanity, generosity, good faith, and justice, are palliated with so much milky good-nature towards the actors, and borne with so much heroic fortitude towards the sufferers. It is certainly not prudent to discredit the authority of an example we mean to follow. But allowing this, we are led to a very natural question:—What is that cause of liberty, and what are those exertions in its favor, to which the example of France is so singularly auspicious? Is our monarchy to be annihilated, with all the laws, all the tribunals, and all the ancient corporations of the kingdom? Is every landmark of the country to be done away in favor of a geometrical and arithmetical constitution? Is the House of Lords to be voted useless? Is Episcopacy to be abolished? Are the Church lands to be sold to Jews and jobbers, or given to bribe new-invented municipal republics into a participation in sacrilege? Are all the taxes to be voted grievances, and the revenue reduced to a patriotic contribution or patriotic presents? Are silver shoe-buckles to be substituted in the place of the land-tax and the malt-tax, for the support of the naval strength of this kingdom? Are all orders, ranks, and distinctions to be confounded, that out of universal anarchy, joined to national bankruptcy, three or four thousand democracies should be formed into eighty-three, and that they may all, by some sort of unknown attractive power, be organized into one? For this great end is the army to be seduced from its discipline and its fidelity, first by every kind of debauchery, and then by the terrible precedent of a donative in the increase of pay? Are the curates to be seduced from their bishops by holding out to them the delusive hope of a dole out of the spoils of their own order? Are the citizens of London to be drawn from their allegiance by feeding them at the expense of their fellow-subjects? Is a compulsory paper currency to be substituted in the place of the legal coin of this kingdom? Is what remains of the plundered stock of public revenue to be employed in the wild project of maintaining two armies to watch over and to fight with each other? If these are the ends and means of the Revolution Society, I admit they are well assorted; and France may furnish them for both with precedents in point.

I see that your example is held out to shame us. I know that we are supposed a dull, sluggish race, rendered passive by finding our situation tolerable, and prevented by a mediocrity of freedom from ever attaining to its full perfection. Your leaders in France began by affecting to admire, almost to adore, the British Constitution; but as they advanced, they came to look upon it with a sovereign contempt. The friends of your National Assembly amongst us have full as mean an opinion of what was formerly thought the glory of their country. The Revolution Society has discovered that the English nation is not free. They are convinced that the inequality in our representation is a "defect in our Constitution so gross and palpable as to make it excellent chiefly in form and theory";
%[87]
\footnote{ Discourse on the Love of our Country, 3rd edit p. 39.}
—that a representation in the legislature of a kingdom is not only the basis of all constitutional liberty in it, but of "all legitimate government; that without it a government is nothing but an usurpation";—that, "when the representation is partial, the kingdom possesses liberty only partially; and if extremely partial, it gives only a semblance; and if not only extremely partial, but corruptly chosen, it becomes a nuisance." Dr. Price considers this inadequacy of representation as our fundamental grievance; and though, as to the corruption of this semblance of representation, he hopes it is not yet arrived to its full perfection of depravity, he fears that "nothing will be done towards gaining for us this essential blessing, until some great abuse of power again provokes our resentment, or some great calamity again alarms our fears, or perhaps till the acquisition of a pure and equal representation by other countries, whilst we are mocked with the shadow, kindles our shame." To this he subjoins a note in these words:—"A representation chosen chiefly by the Treasury, and a few thousands of the dregs of the people, who are generally paid for their votes."

You will smile here at the consistency of those democratists who, when they are not on their guard, treat the humbler part of the community with the greatest contempt, whilst, at the same time, they pretend to make them the depositories of all power. It would require a long discourse to point out to you the many fallacies that lurk in the generality and equivocal nature of the terms "inadequate representation." I shall only say here, in justice to that old-fashioned Constitution under which we have long prospered, that our representation has been found perfectly adequate to all the purposes for which a representation of the people can be desired or devised. I defy the enemies of our Constitution to show the contrary. To detail the particulars in which it is found so well to promote its ends would demand a treatise on our practical Constitution. I state here the doctrine of the revolutionists, only that you and others may see what an opinion these gentlemen entertain of the Constitution of their country, and why they seem to think that some great abuse of power, or some great calamity, as giving a chance for the blessing of a Constitution according to their ideas, would be much palliated to their feelings; you see why they are so much enamored of your fair and equal representation, which being once obtained, the same effects might follow. You see they consider our House of Commons as only "a semblance," "a form," "a theory," "a shadow," "a mockery," perhaps "a nuisance."

These gentlemen value themselves on being systematic, and not without reason. They must therefore look on this gross and palpable defect of representation, this fundamental grievance, (so they call it,) as a thing not only vicious in itself, but as rendering our whole government absolutely illegitimate, and not at all better than a downright usurpation. Another revolution, to get rid of this illegitimate and usurped government, would of course be perfectly justifiable, if not absolutely necessary. Indeed, their principle, if you observe it with any attention, goes much further than to an alteration in the election of the House of Commons; for, if popular representation, or choice, is necessary to the legitimacy of all government, the House of Lords is, at one stroke, bastardized and corrupted in blood. That House is no representative of the people at all, even in "semblance" or "in form." The case of the crown is altogether as bad. In vain the crown may endeavor to screen itself against these gentlemen by the authority of the establishment made on the Revolution. The Revolution, which is resorted to for a title, on their system, wants a title itself. The Revolution is built, according to their theory, upon a basis not more solid than our present formalities, as it was made by a House of Lords not representing any one but themselves, and by a House of Commons exactly such as the present, that is, as they term it, by a mere "shadow and mockery" of representation.

Something they must destroy, or they seem to themselves to exist for no purpose. One set is for destroying the civil power through the ecclesiastical; another for demolishing the ecclesiastic through the civil. They are aware that the worst consequences might happen to the public in accomplishing this double ruin of Church and State; but they are so heated with their theories, that they give more than hints that this ruin, with all the mischiefs that must lead to it and attend it, and which to themselves appear quite certain, would not be unacceptable to them, or very remote from their wishes. A man amongst them of great authority, and certainly of great talents, speaking of a supposed alliance between Church and State, says, "Perhaps we must wait for the fall of the civil powers, before this most unnatural alliance be broken. Calamitous, no doubt, will that time be. But what convulsion in the political world ought to be a subject of lamentation, if it be attended with so desirable an effect?" You see with what a steady eye these gentlemen are prepared to view the greatest calamities which can befall their country!

It is no wonder, therefore, that, with these ideas of everything in their Constitution and government at home, either in Church or State, as illegitimate and usurped, or at best as a vain mockery, they look abroad with an eager and passionate enthusiasm. Whilst they are possessed by these notions, it is vain to talk to them of the practice of their ancestors, the fundamental laws of their country, the fixed form of a Constitution whose merits are confirmed by the solid test of long experience and an increasing public strength and national prosperity. They despise experience as the wisdom of unlettered men; and as for the rest, they have wrought under ground a mine that will blow up, at one grand explosion, all examples of antiquity, all precedents, charters, and acts of Parliament. They have "the rights of men." Against these there can be no prescription; against these no argument is binding: these admit no temperament and no compromise: anything withheld from their full demand is so much of fraud and injustice. Against these their rights of men let no government look for security in the length of its continuance, or in the justice and lenity of its administration. The objections of these speculatists, if its forms do not quadrate with their theories, are as valid against such an old and beneficent government as against the most violent tyranny or the greenest usurpation. They are always at issue with governments, not on a question of abuse, but a question of competency and a question of title. I have nothing to say to the clumsy subtilty of their political metaphysics. Let them be their amusement in the schools.

\begin{verse}
\hspace{1in}Illa se jactet in aula \\
Æolus, et clauso ventorum carcere regnet.
\end{verse}

But let them not break prison to burst like a Levanter, to sweep the earth with their hurricane, and to break up the fountains of the great deep to overwhelm us!

Far am I from denying in theory, full as far is my heart from withholding in practice, (if I were of power to give or to withhold,) the real rights of men. In denying their false claims of right, I do not mean to injure those which are real, and are such as their pretended rights would totally destroy. If civil society be made for the advantage of man, all the advantages for which it is made become his right. It is an institution of beneficence; and law itself is only beneficence acting by a rule. Men have a right to live by that rule; they have a right to justice, as between their fellows, whether their fellows are in politic function or in ordinary occupation. They have a right to the fruits of their industry, and to the means of making their industry fruitful. They have a right to the acquisitions of their parents, to the nourishment and improvement of their offspring, to instruction in life and to consolation in death. Whatever each man can separately do, without trespassing upon others, he has a right to do for himself; and he has a right to a fair portion of all which society, with all its combinations of skill and force, can do in his favor. In this partnership all men have equal rights; but not to equal things. He that has but five shillings in the partnership has as good a right to it as he that has five hundred pounds has to his larger proportion; but he has not a right to an equal dividend in the product of the joint stock. And as to the share of power, authority, and direction which each individual ought to have in the management of the state, that I must deny to be amongst the direct original rights of man in civil society; for I have in my contemplation the civil social man, and no other. It is a thing to be settled by convention.

If civil society be the offspring of convention, that convention must be its law. That convention must limit and modify all the descriptions of constitution which are formed under it. Every sort of legislative, judicial, or executory power are its creatures. They can have no being in any other state of things; and how can any man claim, under the conventions of civil society, rights which do not so much as suppose its existence,—rights which are absolutely repugnant to it? One of the first motives to civil society, and which becomes one of its fundamental rules, is, that no man should be judge in his own cause. By this each person has at once divested himself of the first fundamental right of uncovenanted man, that is, to judge for himself, and to assert his own cause. He abdicates all right to be his own governor. He inclusively, in a great measure, abandons the right of self-defence, the first law of Nature. Men cannot enjoy the rights of an uncivil and of a civil state together. That he may obtain justice, he gives up his right of determining what it is in points the most essential to him. That he may secure some liberty, he makes a surrender in trust of the whole of it.

Government is not made in virtue of natural rights, which may and do exist in total independence of it,—and exist in much greater clearness, and in a much greater degree of abstract perfection: but their abstract perfection is their practical defect. By having a right to everything they want everything. Government is a contrivance of human wisdom to provide for human wants. Men have a right that these wants should be provided for by this wisdom. Among these wants is to be reckoned the want, out of civil society, of a sufficient restraint upon their passions. Society requires not only that the passions of individuals should be subjected, but that even in the mass and body, as well as in the individuals, the inclinations of men should frequently be thwarted, their will controlled, and their passions brought into subjection. This can only be done by a power out of themselves, and not, in the exercise of its function, subject to that will and to those passions which it is its office to bridle and subdue. In this sense the restraints on men, as well as their liberties, are to be reckoned among their rights. But as the liberties and the restrictions vary with times and circumstances, and admit of infinite modifications, they cannot be settled upon any abstract rule; and nothing is so foolish as to discuss them upon that principle.

The moment you abate anything from the full rights of men each to govern himself, and suffer any artificial, positive limitation upon those rights, from that moment the whole organization of government becomes a consideration of convenience. This it is which makes the constitution of a state, and the due distribution of its powers, a matter of the most delicate and complicated skill. It requires a deep knowledge of human nature and human necessities, and of the things which facilitate or obstruct the various ends which are to be pursued by the mechanism of civil institutions. The state is to have recruits to its strength and remedies to its distempers. What is the use of discussing a man's abstract right to food or medicine? The question is upon the method of procuring and administering them. In that deliberation I shall always advise to call in the aid of the farmer and the physician, rather than the professor of metaphysics.

The science of constructing a commonwealth, or renovating it, or reforming it, is, like every other experimental science, not to be taught a priori. Nor is it a short experience that can instruct us in that practical science; because the real effects of moral causes are not always immediate, but that which in the first instance is prejudicial may be excellent in its remoter operation, and its excellence may arise even from the ill effects it produces in the beginning. The reverse also happens; and very plausible schemes, with very pleasing commencements, have often shameful and lamentable conclusions. In states there are often some obscure and almost latent causes, things which appear at first view of little moment, on which a very great part of its prosperity or adversity may most essentially depend. The science of government being, therefore, so practical in itself, and intended for such practical purposes, a matter which requires experience, and even more experience than any person can gain in his whole life, however sagacious and observing he may be, it is with infinite caution that any man ought to venture upon pulling down an edifice which has answered in any tolerable degree for ages the common purposes of society, or on building it up again without having models and patterns of approved utility before his eyes.

These metaphysic rights entering into common life, like rays of light which pierce into a dense medium, are, by the laws of Nature, refracted from their straight line. Indeed, in the gross and complicated mass of human passions and concerns, the primitive rights of men undergo such a variety of refractions and reflections that it becomes absurd to talk of them as if they continued in the simplicity of their original direction. The nature of man is intricate; the objects of society are of the greatest possible complexity: and therefore no simple disposition or direction of power can be suitable either to man's nature or to the quality of his affairs. When I hear the simplicity of contrivance aimed at and boasted of in any new political constitutions, I am at no loss to decide that the artificers are grossly ignorant of their trade or totally negligent of their duty. The simple governments are fundamentally defective, to say no worse of them. If you were to contemplate society in but one point of view, all these simple modes of polity are infinitely captivating. In effect each would answer its single end much more perfectly than the more complex is able to attain all its complex purposes. But it is better that the whole, should be imperfectly and anomalously answered than that while some parts are provided for with great exactness, others might be totally neglected, or perhaps materially injured, by the over-care of a favorite member.

The pretended rights of these theorists are all extremes; and in proportion as they are metaphysically true, they are morally and politically false. The rights of men are in a sort of middle, incapable of definition, but not impossible to be discerned. The rights of men in governments are their advantages; and these are often in balances between differences of good,—in compromises sometimes between good and evil, and sometimes between evil and evil. Political reason is a computing principle: adding, subtracting, multiplying, and dividing, morally, and not metaphysically or mathematically, true moral denominations.

By these theorists the right of the people is almost always sophistically confounded with their power. The body of the community, whenever it can come to act, can meet with no effectual resistance; but till power and right are the same, the whole body of them has no right inconsistent with virtue, and the first of all virtues, prudence. Men have no right to what is not reasonable, and to what is not for their benefit; for though a pleasant writer said, "Liceat perire poetis," when one of them, in cold blood, is said to have leaped into the flames of a volcanic revolution, "ardentem frigidus Ætnam insiluit," I consider such a frolic rather as an unjustifiable poetic license than as one of the franchises of Parnassus; and whether he were poet, or divine, or politician, that chose to exercise this kind of right, I think that more wise, because more charitable, thoughts would urge me rather to save the man than to preserve his brazen slippers as the monuments of his folly.

The kind of anniversary sermons to which a great part of what I write refers, if men are not shamed out of their present course, in commemorating the fact, will cheat many out of the principles and deprive them of the benefits of the Revolution they commemorate. I confess to you, Sir, I never liked this continual talk of resistance and revolution, or the practice of making the extreme medicine of the Constitution its daily bread. It renders the habit of society dangerously valetudinary; it is taking periodical doses of mercury sublimate, and swallowing down repeated provocatives of cantharides to our love of liberty.

This distemper of remedy, grown habitual, relaxes and wears out, by a vulgar and prostituted use, the spring of that spirit which is to be exerted on great occasions. It was in the most patient period of Roman servitude that themes of tyrannicide made the ordinary exercise of boys at school,—cum perimit sævos classis numerosa tyrannos. In the ordinary state of things, it produces in a country like ours the worst effects, even on the cause of that liberty which it abuses with the dissoluteness of an extravagant speculation. Almost all the high-bred republicans of my time have, after a short space, become the most decided, thorough-paced courtiers; they soon left the business of a tedious, moderate, but practical resistance, to those of us whom, in the pride and intoxication of their theories, they have slighted as not much better than Tories. Hypocrisy, of course, delights in the most sublime speculations; for, never intending to go beyond speculation, it costs nothing to have it magnificent. But even in cases where rather levity than fraud was to be suspected in these ranting speculations, the issue has been much the same. These professors, finding their extreme principles not applicable to cases which call only for a qualified, or, as I may say, civil and legal resistance, in such cases employ no resistance at all. It is with them a war or a revolution, or it is nothing. Finding their schemes of politics not adapted to the state of the world in which they live, they often come to think lightly of all public principle, and are ready, on their part, to abandon for a very trivial interest what they find of very trivial value. Some, indeed, are of more steady and persevering natures; but these are eager politicians out of Parliament, who have little to tempt them to abandon their favorite projects. They have some change in the Church or State, or both, constantly in their view. When that is the case, they are always bad citizens, and perfectly unsure connections. For, considering their speculative designs as of infinite value, and the actual arrangement of the state as of no estimation, they are, at best, indifferent about it. They see no merit in the good, and no fault in the vicious management of public affairs; they rather rejoice in the latter, as more propitious to revolution. They see no merit or demerit in any man, or any action, or any political principle, any further than as they may forward or retard their design of change; they therefore take up, one day, the most violent and stretched prerogative, and another time the wildest democratic ideas of freedom, and pass from the one to the other without any sort of regard to cause, to person, or to party.

In France you are now in the crisis of a revolution, and in the transit from one form of government to another: you cannot see that character of men exactly in the same situation in which we see it in this country. With us it is militant, with you it is triumphant; and you know how it can act, when its power is commensurate to its will. I would not be supposed to confine those observations to any description of men, or to comprehend all men of any description within them,—no, far from it! I am as incapable of that injustice as I am of keeping terms with those who profess principles of extremes, and who, under the name of religion, teach little else than wild and dangerous politics. The worst of these politics of revolution is this: they temper and harden the breast, in order to prepare it for the desperate strokes which are sometimes used in extreme occasions. But as these occasions may never arrive, the mind receives a gratuitous taint; and the moral sentiments suffer not a little, when no political purpose is served by the depravation. This sort of people are so taken up with their theories about the rights of man, that they have totally forgot his nature. Without opening one new avenue to the understanding, they have succeeded in stopping up those that lead to the heart. They have perverted in themselves, and in those that attend to them, all the well-placed sympathies of the human breast.

This famous sermon of the Old Jewry breathes nothing but this spirit through all the political part. Plots, massacres, assassinations, seem to some people a trivial price for obtaining a revolution. A cheap, bloodless reformation, a guiltless liberty, appear flat and vapid to their taste. There must be a great change of scene; there must be a magnificent stage effect; there must be a grand spectacle to rouse the imagination, grown torpid with the lazy enjoyment of sixty years' security, and the still unanimating repose of public prosperity. The preacher found them all in the French Revolution. This inspires a juvenile warmth through his whole frame. His enthusiasm kindles as he advances; and when he arrives at his peroration, it is in a full blaze. Then viewing, from the Pisgah of his pulpit, the free, moral, happy, flourishing, and glorious state of France, as in a bird-eye landscape of a promised land, he breaks out into the following rapture:—

"What an eventful period is this! I am thankful that I have lived to it; I could almost say, Lord, now lettest thou thy servant depart in peace, for mine eyes have seen thy salvation.—I have lived to see a diffusion of knowledge which has undermined superstition and error.—I have lived to see the rights of men better understood than ever, and nations panting for liberty which seemed to have lost the idea of it.—I have lived to see thirty millions of people, indignant and resolute, spurning at slavery, and demanding liberty with an irresistible voice; their king led in triumph, and an arbitrary monarch surrendering himself to his subjects."
%[88]
\footnote{ Another of these reverend gentlemen, who was witness to some of the spectacles which Paris has lately exhibited, expresses himself thus:—"A king dragged in submissive triumph by his conquering subjects is one of those appearances of grandeur which seldom rise in the prospect of human affairs, and which, during the remainder of my life, I shall think of with wonder and gratification." These gentlemen agree marvellously in their feelings.}


Before I proceed further, I have to remark that Dr. Price seems rather to overvalue the great acquisitions of light which he has obtained and diffused in this age. The last century appears to me to have been quite as much enlightened. It had, though in a different place, a triumph as memorable as that of Dr. Price; and some of the great preachers of that period partook of it as eagerly as he has done in the triumph of France. On the trial of the Reverend Hugh Peters for high treason, it was deposed, that, when King Charles was brought to London for his trial, the Apostle of Liberty in that day conducted the triumph. "I saw," says the witness, "his Majesty in the coach with six horses, and Peters riding before the king triumphing." Dr. Price, when he talks as if he had made a discovery, only follows a precedent; for, after the commencement of the king's trial, this precursor, the same Dr. Peters, concluding a long prayer at the royal chapel at Whitehall, (he had very triumphantly chosen his place,) said, "I have prayed and preached these twenty years; and now I may say with old Simeon, Lord, now lettest thou thy servant depart in peace, for mine eyes have seen thy salvation."
%[89]
\footnote{ State Trials, Vol. II. p. 360, 363.}
 Peters had not the fruits of his prayer; for he neither departed so soon as he wished, nor in peace. He became (what I heartily hope none of his followers may be in this country) himself a sacrifice to the triumph which he led as pontiff. They dealt at the Restoration, perhaps, too hardly with this poor good man. But we owe it to his memory and his sufferings, that he had as much illumination and as much zeal, and had as effectually undermined all the superstition and error which might impede the great business he was engaged in, as any who follow and repeat after him in this age, which would assume to itself an exclusive title to the knowledge of the rights of men, and all the glorious consequences of that knowledge.

After this sally of the preacher of the Old Jewry, which differs only in place and time, but agrees perfectly with the spirit and letter of the rapture of 1648, the Revolution Society, the fabricators of governments, the heroic band of cashierers of monarchs, electors of sovereigns, and leaders of kings in triumph, strutting with a proud consciousness of the diffusion of knowledge, of which every member had obtained so large a share in the donative, were in haste to make a generous diffusion of the knowledge they had thus gratuitously received. To make this bountiful communication, they adjourned from the church in the Old Jewry to the London Tavern, where the same Dr. Price, in whom the fumes of his oracular tripod were not entirely evaporated, moved and carried the resolution, or address of congratulation, transmitted by Lord Stanhope to the National Assembly of France.

I find a preacher of the Gospel profaning the beautiful and prophetic ejaculation, commonly called "Nunc dimittis," made on the first presentation of our Saviour in the temple, and applying it, with an inhuman and unnatural rapture, to the most horrid, atrocious, and afflicting spectacle that perhaps ever was exhibited to the pity and indignation of mankind. This "leading in triumph," a thing in its best form unmanly and irreligious, which fills our preacher with such unhallowed transports, must shock, I believe, the moral taste of every well-born mind. Several English were the stupefied and indignant spectators of that triumph. It was (unless we have been strangely deceived) a spectacle more resembling a procession of American savages entering into Onondaga after some of their murders called victories, and leading into hovels hung round with scalps their captives overpowered with the scoffs and buffets of women as ferocious as themselves, much more than it resembled the triumphal pomp of a civilized martial nation;—if a civilized nation, or any men who had a sense of generosity, were capable of a personal triumph over the fallen and afflicted.

This, my dear Sir, was not the triumph of France. I must believe, that, as a nation, it overwhelmed you with shame and horror. I must believe that the National Assembly find themselves in a state of the greatest humiliation in not being able to punish the authors of this triumph or the actors in it, and that they are in a situation in which any inquiry they may make upon the subject must be destitute even of the appearance of liberty or impartiality. The apology of that assembly is found in their situation; but when we approve what they must bear, it is in us the degenerate choice of a vitiated mind.

With a compelled appearance of deliberation, they vote under the dominion of a stern necessity. They sit in the heart, as it were, of a foreign republic: they have their residence in a city whose constitution has emanated neither from the charter of their king nor from their legislative power. There they are surrounded by an army not raised either by the authority of their crown or by their command, and which, if they should order to dissolve itself, would instantly dissolve them. There they sit, after a gang of assassins had driven away some hundreds of the members; whilst those who held the same moderate principles, with more patience or better hope, continued every day exposed to outrageous insults and murderous threats. There a majority, sometimes real, sometimes pretended, captive itself, compels a captive king to issue as royal edicts, at third hand, the polluted nonsense of their most licentious and giddy coffee-houses. It is notorious that all their measures are decided before they are debated. It is beyond doubt, that, under the terror of the bayonet, and the lamp-post, and the torch to their houses, they are obliged to adopt all the crude and desperate measures suggested by clubs composed of a monstrous medley of all conditions, tongues, and nations. Among these are found persons in comparison of whom Catiline would be thought scrupulous, and Cethegus a man of sobriety and moderation. Nor is it in these clubs alone that the public measures are deformed into monsters. They undergo a previous distortion in academies, intended as so many seminaries for these clubs, which are set up in all the places of public resort. In these meetings of all sorts, every counsel, in proportion as it is daring and violent and perfidious, is taken for the mark of superior genius. Humanity and compassion are ridiculed as the fruits of superstition and ignorance. Tenderness to individuals is considered as treason to the public. Liberty is always to be estimated perfect as property is rendered insecure. Amidst assassination, massacre, and confiscation, perpetrated or meditated, they are forming plans for the good order of future society. Embracing in their arms the carcasses of base criminals, and promoting their relations on the title of their offences, they drive hundreds of virtuous persons to the same end, by forcing them to subsist by beggary or by crime.

The Assembly, their organ, acts before them the farce of deliberation with as little decency as liberty. They act like the comedians of a fair, before a riotous audience; they act amidst the tumultuous cries of a mixed mob of ferocious men, and of women lost to shame, who, according to their insolent fancies, direct, control, applaud, explode them, and sometimes mix and take their seats amongst them,—domineering over them with a strange mixture of servile petulance and proud, presumptuous authority. As they have inverted order in all things, the gallery is in the place of the house. This assembly, which overthrows kings and kingdoms, has not even the physiognomy and aspect of a grave legislative body,—nec color imperii, nec frons erat ulla senatûs. They have a power given to them, like that of the Evil Principle, to subvert and destroy,—but none to construct, except such machines as may be fitted for further subversion and further destruction.

Who is it that admires, and from the heart is attached to national representative assemblies, but must turn with horror and disgust from such a profane burlesque and abominable perversion of that sacred institute? Lovers of monarchy, lovers of republics, must alike abhor it. The members of your Assembly must themselves groan under the tyranny of which they have all the shame, none of the direction, and little of the profit. I am sure many of the members who compose even the majority of that body must feel as I do, notwithstanding the applauses of the Revolution Society. Miserable king! miserable assembly! How must that assembly be silently scandalized with those of their members who could call a day which seemed to blot the sun out of heaven "un beau jour"!
%[90]
\footnote{ 6th of October, 1789.}
 How must they be inwardly indignant at hearing others who thought fit to declare to them, "that the vessel of the state would fly forward in her course towards regeneration with more speed than ever," from the stiff gale of treason and murder which preceded our preacher's triumph! What must they have felt, whilst, with outward patience and inward indignation, they heard of the slaughter of innocent gentlemen in their houses, that "the blood spilled was not the most pure"! What must they have felt, when they were besieged by complaints of disorders which shook their country to its foundations, at being compelled coolly to tell the complainants that they were under the protection of the law, and that they would address the king (the captive king) to cause the laws to be enforced for their protection, when the enslaved ministers of that captive king had formally notified to them that there were neither law nor authority nor power left to protect! What must they have felt at being obliged, as a felicitation on the present new year, to request their captive king to forget the stormy period of the last, on account of the great good which he was likely to produce to his people,—to the complete attainment of which good they adjourned the practical demonstrations of their loyalty, assuring him of their obedience when he should no longer possess any authority to command!

This address was made with much good-nature and affection, to be sure. But among the revolutions in France must be reckoned a considerable revolution in their ideas of politeness. In England we are said to learn manners at second-hand from your side of the water, and that we dress our behavior in the frippery of France. If so, we are still in the old cut, and have not so far conformed to the new Parisian mode of good breeding as to think it quite in the most refined strain of delicate compliment (whether in condolence or congratulation) to say, to the most humiliated creature that crawls upon the earth, that great public benefits are derived from the murder of his servants, the attempted assassination of himself and of his wife, and the mortification, disgrace, and degradation that he has personally suffered. It is a topic of consolation which our ordinary of Newgate would be too humane to use to a criminal at the foot of the gallows. I should have thought that the hangman of Paris, now that he is liberalized by the vote of the National Assembly, and is allowed his rank and arms in the Herald's College of the rights of men, would be too generous, too gallant a man, too full of the sense of his new dignity, to employ that cutting consolation to any of the persons whom the lèze-nation might bring under the administration of his executive powers.

A man is fallen indeed, when he is thus flattered. The anodyne draught of oblivion, thus drugged, is well calculated to preserve a galling wakefulness, and to feed the living ulcer of a corroding memory. Thus to administer the opiate potion of amnesty, powdered with all the ingredients of scorn and contempt, is to hold to his lips, instead of "the balm of hurt minds," the cup of human misery full to the brim, and to force him to drink it to the dregs.

Yielding to reasons at least as forcible as those which were so delicately urged in the compliment on the new year, the king of France will probably endeavor to forget these events and that compliment. But History, who keeps a durable record of all our acts, and exercises her awful censure over the proceedings of all sorts of sovereigns, will not forget either those events, or the era of this liberal refinement in the intercourse of mankind. History will record, that, on the morning of the sixth of October, 1789, the king and queen of France, after a day of confusion, alarm, dismay, and slaughter, lay down, under the pledged security of public faith, to indulge nature in a few hours of respite, and troubled, melancholy repose. From this sleep the queen was first startled by the voice of the sentinel at her door, who cried out to her to save herself by flight,—that this was the last proof of fidelity he could give,—that they were upon him, and he was dead. Instantly he was cut down. A band of cruel ruffians and assassins, reeking with his blood, rushed into the chamber of the queen, and pierced with a hundred strokes of bayonets and poniards the bed, from whence this persecuted woman had but just time to fly almost naked, and, through ways unknown to the murderers, had escaped to seek refuge at the feet of a king and husband not secure of his own life for a moment.

This king, to say no more of him, and this queen, and their infant children, (who once would have been the pride and hope of a great and generous people,) were then forced to abandon the sanctuary of the most splendid palace in the world, which they left swimming in blood, polluted by massacre, and strewed with scattered limbs and mutilated carcasses. Thence they were conducted into the capital of their kingdom. Two had been selected from the unprovoked, unresisted, promiscuous slaughter which was made of the gentlemen of birth and family who composed the king's body-guard. These two gentlemen, with all the parade of an execution of justice, were cruelly and publicly dragged to the block, and beheaded in the great court of the palace. Their heads were stuck upon spears, and led the procession; whilst the royal captives who followed in the train were slowly moved along, amidst the horrid yells, and shrilling screams, and frantic dances, and infamous contumelies, and all the unutterable abominations of the furies of hell, in the abused shape of the vilest of women. After they had been made to taste, drop by drop, more than the bitterness of death, in the slow torture of a journey of twelve miles, protracted to six hours, they were, under a guard composed of those very soldiers who had thus conducted them through this famous triumph, lodged in one of the old palaces of Paris, now converted into a Bastile for kings.

Is this a triumph to be consecrated at altars, to be commemorated with grateful thanksgiving, to be offered to the Divine Humanity with fervent prayer and enthusiastic ejaculation?—These Theban and Thracian orgies, acted in France, and applauded only in the Old Jewry, I assure you, kindle prophetic enthusiasm in the minds but of very few people in this kingdom: although a saint and apostle, who may have revelations of his own, and who has so completely vanquished all the mean superstitions of the heart, may incline to think it pious and decorous to compare it with the entrance into the world of the Prince of Peace, proclaimed in an holy temple by a venerable sage, and not long before not worse announced by the voice of angels to the quiet innocence of shepherds.

At first I was at a loss to account for this fit of unguarded transport. I knew, indeed, that the sufferings of monarchs make a delicious repast to some sort of palates. There were reflections which might serve to keep this appetite within some bounds of temperance. But when I took one circumstance into my consideration, I was obliged to confess that much allowance ought to be made for the society, and that the temptation was too strong for common discretion: I mean, the circumstance of the Io Pæan of the triumph, the animating cry which called for "all the BISHOPS to be hanged on the lamp-posts,"
%[91]
\footnote{ "Tous les Évêques à la lanterne!"}
 might well have brought forth a burst of enthusiasm on the foreseen consequences of this happy day. I allow to so much enthusiasm some little deviation from prudence. I allow this prophet to break forth into hymns of joy and thanksgiving on an event which appears like the precursor of the Millennium, and the projected Fifth Monarchy, in the destruction of all Church establishments. There was, however, (as in all human affairs there is,) in the midst of this joy, something to exercise the patience of these worthy gentlemen, and to try the long-suffering of their faith. The actual murder of the king and queen, and their child, was wanting to the other auspicious circumstances of this "beautiful day". The actual murder of the bishops, though called for by so many holy ejaculations, was also wanting. A group of regicide and sacrilegious slaughter was, indeed, boldly sketched, but it was only sketched. It unhappily was left unfinished, in this great history-piece of the massacre of innocents. What hardy pencil of a great master, from the school of the rights of men, will finish it, is to be seen hereafter. The age has not yet the complete benefit of that diffusion of knowledge that has undermined superstition and error; and the king of France wants another object or two to consign to oblivion, in consideration of all the good which is to arise from his own sufferings, and the patriotic crimes of an enlightened age.
%[92]
\footnote{ It is proper here to refer to a letter written upon this subject by an eyewitness. That eyewitness was one of the most honest, intelligent, and eloquent members of the National Assembly, one of the most active and zealous reformers of the state. He was obliged to secede from the Assembly; and he afterwards became a voluntary exile, on account of the horrors of this pious triumph, and the dispositions of men, who, profiting of crimes, if not causing them, have taken the lead in public affairs.

Extract of M. de Lally Tollendal's Second Letter to a Friend.

"Parlons du parti que j'ai pris; il est bien justifé dans ma conscience.—Ni cette ville coupable, ni cette assemblée plus coupable encore, ne méritoient que je me justifie; mais j'ai à cœur que vous, et les personnes qui pensent comme vous, ne me condamnent pas.—Ma santé, je vous jure, me rendoit mes fonctions impossibles; mais même en les mettant de côté il a été au-dessus de mes forces de supporter plus longtems l'horreur que me causoit ce sang,—ces têtes,—cette reine presque egorgée,—ce roi, amené esclave, entrant à Paris au milieu de ses assassins, et précédé des têtes de ses malheureux gardes,—ces perfides janissaires, ces assassins, ces femmes cannibales,—ce cri de TOUS LES ÉVÊQUES À LA LANTERNE, dans le moment où le roi entre sa capitale avec deux évêques de son conseil dans sa voiture,—un coup de fusil, que j'ai vu tirer dans un des carrosses de la reine,—M. Bailly appellant cela un beau jour,—l'assemblée ayant déclaré froidement le matin, qu'il n'étoit pas de sa dignité d'aller toute entière environner le roi,—M. Mirabeau disant impunément dans cette assemblée, que le vaisseau de l'état, loin d'être arrêté dans sa course, s'élanceroit avec plus de rapidité que jamais vers sa régénération,—M. Barnave, riant avec lui, quand des flots de sang couloient autour de nous,—le vertueux Mounier[A] échappant par miracle à vingt assassins, qui avoient voulu faire de sa tête un trophée de plus: Voilà ce qui me fit jurer de ne plus mettre le pied dans cette caverne d'Antropophages [The National Assembly], où je n'avois plus de force d'élever la voix, où depuis six semaines je l'avois élevée en vain."

"Moi, Mounier, et tous les honnêtes gens, ont pensé que le dernier effort à faire pour le bien étoit d'en sortir. Aucune idée de crainte ne s'est approchée de moi. Je rougirois de m'en défendre. J'avois encore reçû sur la route de la part de ce peuple, moins coupable que ceux qui l'ont enivré de fureur, des acclamations, et des applaudissements, dont d'autres auroient été flattés, et qui m'ont fait frémir. C'est à l'indignation, c'est à l'horreur, c'est aux convulsions physiques, que le seul aspect du sang me fait éprouver que j'ai cédé. On brave une seule mort; on la brave plusieurs fois, quand elle peut être utile. Mais aucune puissance sous le ciel, mais aucune opinion publique ou privée n'ont le droit de me condamner à souffrir inutilement mille supplices par minute, et à périr de désespoir, de rage, au milieu des triomphes, du crime que je n'ai pu arrêter. Ils me proscriront, ils confisqueront mes biens. Je labourerai la terre, et je ne les verrai plus. Voilà ma justification. Vous pourrez la lire, la montrer, la laisser copier; tant pis pour ceux qui ne la comprendront pas; ce ne sera alors moi qui auroit eu tort de la leur donner."

This military man had not so good nerves as the peaceable gentlemen of the Old Jewry.—See Mons. Mounier's narrative of these transactions: a man also of honor and virtue and talents, and therefore a fugitive.
}

Although this work of our new light and knowledge did not go to the length that in all probability it was intended it should be carried, yet I must think that such treatment of any human creatures must be shocking to any but those who are made for accomplishing revolutions. But I cannot stop here. Influenced by the inborn feelings of my nature, and not being illuminated by a single ray of this new-sprung modern light, I confess to you, Sir, that the exalted rank of the persons suffering, and particularly the sex, the beauty, and the amiable qualities of the descendant of so many kings and emperors, with the tender age of royal infants, insensible only through infancy and innocence of the cruel outrages to which their parents were exposed, instead of being a subject of exultation, adds not a little to my sensibility on that most melancholy occasion.

I hear that the august person who was the principal object of our preacher's triumph, though he supported himself, felt much on that shameful occasion. As a man, it became him to feel for his wife and his children, and the faithful guards of his person that were massacred in cold blood about him; as a prince, it became him to feel for the strange and frightful transformation of his civilized subjects, and to be more grieved for them than solicitous for himself. It derogates little from his fortitude, while it adds infinitely to the honor of his humanity. I am very sorry to say it, very sorry indeed, that such personages are in a situation in which it is not unbecoming in us to praise the virtues of the great.

I hear, and I rejoice to hear, that the great lady, the other object of the triumph, has borne that day, (one is interested that beings made for suffering should suffer well,) and that she bears all the succeeding days, that she bears the imprisonment of her husband, and her own captivity, and the exile of her friends, and the insulting adulation of addresses, and the whole weight of her accumulated wrongs, with a serene patience, in a manner suited to her rank and race, and becoming the offspring of a sovereign distinguished for her piety and her courage; that, like her, she has lofty sentiments; that she feels with the dignity of a Roman matron; that in the last extremity she will save herself from the last disgrace; and that, if she must fall, she will fall by no ignoble hand.

It is now sixteen or seventeen years since I saw the queen of France, then the Dauphiness, at Versailles; and surely never lighted on this orb, which she hardly seemed to touch, a more delightful vision. I saw her just above the horizon, decorating and cheering the elevated sphere she just began to move in,—glittering like the morning-star, full of life and splendor and joy. Oh! what a revolution! and what an heart must I have, to contemplate without emotion that elevation and that fall! Little did I dream, when she added titles of veneration to those of enthusiastic, distant, respectful love, that she should ever be obliged to carry the sharp antidote against disgrace concealed in that bosom! little did I dream that I should have lived to see such disasters fallen upon her in a nation of gallant men, in a nation of men of honor, and of cavaliers! I thought ten thousand swords must have leaped from their scabbards to avenge even a look that threatened her with insult. But the age of chivalry is gone. That of sophisters, economists, and calculators has succeeded; and the glory of Europe is extinguished forever. Never, never more, shall we behold that generous loyalty to rank and sex, that proud submission, that dignified obedience, that subordination of the heart, which kept alive, even in servitude itself, the spirit of an exalted freedom! The unbought grace of life, the cheap defence of nations, the nurse of manly sentiment and heroic enterprise, is gone! It is gone, that sensibility of principle, that chastity of honor, which felt a stain like a wound, which inspired courage whilst it mitigated ferocity, which ennobled whatever it touched, and under which vice itself lost half its evil by losing all its grossness!

This mixed system of opinion and sentiment had its origin in the ancient chivalry; and the principle, though varied in its appearance by the varying state of human affairs, subsisted and influenced through a long succession of generations, even to the time we live in. If it should ever be totally extinguished, the loss, I fear, will be great. It is this which has given its character to modern Europe. It is this which has distinguished it under all its forms of government, and distinguished it to its advantage, from the states of Asia, and possibly from those states which flourished in the most brilliant periods of the antique world. It was this, which, without confounding ranks, had produced a noble equality, and handed it down through all the gradations of social life. It was this opinion which mitigated kings into companions, and raised private men to be fellows with kings. Without force or opposition, it subdued the fierceness of pride and power; it obliged sovereigns to submit to the soft collar of social esteem, compelled stern authority to submit to elegance, and gave a domination, vanquisher of laws, to be subdued by manners.

But now all is to be changed. All the pleasing illusions which made power gentle and obedience liberal, which harmonized the different shades of life, and which by a bland assimilation incorporated into politics the sentiments which beautify and soften private society, are to be dissolved by this new conquering empire of light and reason. All the decent drapery of life is to be rudely torn off. All the superadded ideas, furnished from the wardrobe of a moral imagination, which the heart owns and the understanding ratifies, as necessary to cover the defects of our naked, shivering nature, and to raise it to dignity in our own estimation, are to be exploded, as a ridiculous, absurd, and antiquated fashion.

On this scheme of things, a king is but a man, a queen is but a woman, a woman is but an animal,—and an animal not of the highest order. All homage paid to the sex in general as such, and without distinct views, is to be regarded as romance and folly. Regicide, and parricide, and sacrilege, are but fictions of superstition, corrupting jurisprudence by destroying its simplicity. The murder of a king, or a queen, or a bishop, or a father, are only common homicide,—and if the people are by any chance or in any way gainers by it, a sort of homicide much the most pardonable, and into which we ought not to make too severe a scrutiny.

On the scheme of this barbarous philosophy, which is the offspring of cold hearts and muddy understandings and which is as void of solid wisdom as it is destitute of all taste and elegance, laws are to be supported only by their own terrors, and by the concern which each individual may find in them from his own private speculations, or can spare to them from his own private interests. In the groves of their academy, at the end of every visto, you see nothing but the gallows. Nothing is left which engages the affections on the part of the commonwealth. On the principles of this mechanic philosophy, our institutions can never be embodied, if I may use the expression, in persons,—so as to create in us love, veneration, admiration, or attachment. But that sort of reason which banishes the affections is incapable of filling their place. These public affections, combined with manners, are required sometimes as supplements, sometimes as correctives, always as aids to law. The precept given by a wise man, as well as a great critic, for the construction of poems, is equally true as to states:—"Non satis est pulchra esse poemata, dulcia sunto." There ought to be a system of manners in every nation which a well-formed mind would be disposed to relish. To make us love our country, our country ought to be lovely.

But power, of some kind or other, will survive the shock in which manners and opinions perish; and it will find other and worse means for its support. The usurpation, which, in order to subvert ancient institutions, has destroyed ancient principles, will hold power by arts similar to those by which it has acquired it. When the old feudal and chivalrous spirit of fealty, which, by freeing kings from fear, freed both kings and subjects from the precautions of tyranny, shall be extinct in the minds of men, plots and assassinations will be anticipated by preventive murder and preventive confiscation, and that long roll of grim and bloody maxims which form the political code of all power not standing on its own honor and the honor of those who are to obey it. Kings will be tyrants from policy, when subjects are rebels from principle.

When ancient opinions and rules of life are taken away, the loss cannot possibly be estimated. From that moment we have no compass to govern us, nor can we know distinctly to what port we steer. Europe, undoubtedly, taken in a mass, was in a flourishing condition the day on which your Revolution was completed. How much of that prosperous state was owing to the spirit of our old manners and opinions is not easy to say; but as such causes cannot be indifferent in their operation, we must presume, that, on the whole, their operation was beneficial.

We are but too apt to consider things in the state in which we find them, without sufficiently adverting to the causes by which they have been produced, and possibly may be upheld. Nothing is more certain than that our manners, our civilization, and all the good things which are connected with manners and with, civilization, have, in this European world of ours, depended for ages upon two principles, and were, indeed, the result of both combined: I mean the spirit of a gentleman, and the spirit of religion. The nobility and the clergy, the one by profession, and the other by patronage, kept learning in existence, even in the midst of arms and confusions, and whilst governments were rather in their causes than formed. Learning paid back what it received to nobility and to priesthood, and paid it with usury, by enlarging their ideas, and by furnishing their minds. Happy, if they had all continued to know their indissoluble union, and their proper place! Happy, if learning, not debauched by ambition, had been satisfied to continue the instructor, and not aspired to be the master! Along with its natural protectors and guardians, learning will be cast into the mire and trodden down under the hoofs of a swinish multitude.
%[93]
\footnote{ See the fate of Bailly and Condorcet, supposed to be here particularly alluded to. Compare the circumstances of the trial and execution of the former with this prediction.}


If, as I suspect, modern letters owe more than they are always willing to own to ancient manners, so do other interests which we value full as much as they are worth. Even commerce, and trade, and manufacture, the gods of our economical politicians, are themselves perhaps but creatures, are themselves but effects, which, as first causes, we choose to worship. They certainly grew under the same shade in which learning flourished. They, too, may decay with their natural protecting principles. With you, for the present at least, they all threaten to disappear together. Where trade and manufactures are wanting to a people, and the spirit of nobility and religion remains, sentiment supplies, and not always ill supplies, their place; but if commerce and the arts should be lost in an experiment to try how well a state may stand without these old fundamental principles, what sort of a thing must be a nation of gross, stupid, ferocious, and at the same time poor and sordid barbarians, destitute of religion, honor, or manly pride, possessing nothing at present, and hoping for nothing hereafter?

I wish you may not be going fast, and by the shortest cut, to that horrible and disgustful situation. Already there appears a poverty of conception, a coarseness and vulgarity, in all the proceedings of the Assembly and of all their instructors. Their liberty is not liberal. Their science is presumptuous ignorance. Their humanity is savage and brutal.

It is not clear whether in England we learned those grand and decorous principles and manners, of which considerable traces yet remain, from you, or whether you took them from us. But to you, I think, we trace them best. You seem to me to be gentis incunabula nostræ. France has always more or less influenced manners in England; and when your fountain is choked up and polluted, the stream will not run long or not run clear with us, or perhaps with any nation. This gives all Europe, in my opinion, but too close and connected a concern in what is done in France. Excuse me, therefore, if I have dwelt too long on the atrocious spectacle of the sixth of October, 1789, or have given too much scope to the reflections which have arisen in my mind on occasion of the most important of all revolutions, which may be dated from that day: I mean a revolution in sentiments, manners, and moral opinions. As things now stand, with everything respectable destroyed without us, and an attempt to destroy within us every principle of respect, one is almost forced to apologize for harboring the common feelings of men.

Why do I feel so differently from the Reverend Dr. Price, and those of his lay flock who will choose to adopt the sentiments of his discourse?—For this plain reason: Because it is natural I should; because we are so made as to be affected at such spectacles with melancholy sentiments upon the unstable condition of mortal prosperity, and the tremendous uncertainty of human greatness; because in those natural feelings we learn great lessons; because in events like these our passions instruct our reason; because, when kings are hurled from their thrones by the Supreme Director of this great drama, and become the objects of insult to the base and of pity to the good, we behold such disasters in the moral as we should behold a miracle in the physical order of things. We are alarmed into reflection; our minds (as it has long since been observed) are purified by terror and pity; our weak, unthinking pride is humbled under the dispensations of a mysterious wisdom. Some tears might be drawn from me, if such a spectacle were exhibited on the stage. I should be truly ashamed of finding in myself that superficial, theatric sense of painted distress, whilst I could exult over it in real life. With such a perverted mind, I could never venture to show my face at a tragedy. People would think the tears that Garrick formerly, or that Siddons not long since, have extorted from me, were the tears of hypocrisy; I should know them to be the tears of folly.

Indeed, the theatre is a better school of moral sentiments than churches where the feelings of humanity are thus outraged. Poets who have to deal with an audience not yet graduated in the school of the rights of men, and who must apply themselves to the moral constitution of the heart, would not dare to produce such a triumph as a matter of exultation. There, where men follow their natural impulses, they would not bear the odious maxims of a Machiavelian policy, whether applied to the attainment of monarchical or democratic tyranny. They would reject them on the modern, as they once did on the ancient stage, where they could not bear even the hypothetical proposition of such wickedness in the mouth of a personated tyrant, though suitable to the character he sustained. No theatric audience in Athens would bear what has been borne in the midst of the real tragedy of this triumphal day: a principal actor weighing, as it were in scales hung in a shop of horrors, so much actual crime against so much contingent advantage,—and after putting in and out weights, declaring that the balance was on the side of the advantages. They would not bear to see the crimes of new democracy posted as in a ledger against the crimes of old despotism, and the book-keepers of politics finding democracy still in debt, but by no means unable or unwilling to pay the balance. In the theatre, the first intuitive glance, without any elaborate process of reasoning, would show that this method of political computation would justify every extent of crime. They would see, that, on these principles, even where the very worst acts were not perpetrated, it was owing rather to the fortune of the conspirators than to their parsimony in the expenditure of treachery and blood. They would soon see that criminal means, once tolerated, are soon preferred. They present a shorter cut to the object than through the highway of the moral virtues. Justifying perfidy and murder for public benefit, public benefit would soon become the pretext, and perfidy and murder the end,—until rapacity, malice, revenge, and fear more dreadful than revenge, could satiate their insatiable appetites. Such must be the consequences of losing, in the splendor of these triumphs of the rights of men, all natural sense of wrong and right.

But the reverend pastor exults in this "leading in triumph," because, truly, Louis the Sixteenth was "an arbitrary monarch": that is, in other words, neither more nor less than because he was Louis the Sixteenth, and because he had the misfortune to be born king of France, with the prerogatives of which a long line of ancestors, and a long acquiescence of the people, without any act of his, had put him in possession. A misfortune it has indeed turned out to him, that he was born king of France. But misfortune is not crime, nor is indiscretion always the greatest guilt. I shall never think that a prince, the acts of whose whole reign were a series of concessions to his subjects, who was willing to relax his authority, to remit his prerogatives, to call his people to a share of freedom not known, perhaps not desired, by their ancestors,—such a prince, though he should be subject to the common frailties attached to men and to princes, though he should have once thought it necessary to provide force against the desperate designs manifestly carrying on against his person and the remnants of his authority,—though all this should be taken into consideration, I shall be led with great difficulty to think he deserves the cruel and insulting triumph of Paris, and of Dr. Price. I tremble for the cause of liberty, from such an example to kings. I tremble for the cause of humanity, in the unpunished outrages of the most wicked of mankind. But there are some people of that low and degenerate fashion of mind that they look up with a sort of complacent awe and admiration to kings who know to keep firm in their seat, to hold a strict hand over their subjects, to assert their prerogative, and, by the awakened vigilance of a severe despotism, to guard against the very first approaches of freedom. Against such as these they never elevate their voice. Deserters from principle, listed with fortune, they never see any good in suffering virtue, nor any crime in prosperous usurpation.

If it could have been made clear to me that the king and queen of France (those, I mean, who were such before the triumph) were inexorable and cruel tyrants, that they had formed a deliberate scheme for massacring the National Assembly, (I think I have seen something like the latter insinuated in certain publications,) I should think their captivity just. If this be true, much more ought to have been done, but done, in my opinion, in another manner. The punishment of real tyrants is a noble and awful act of justice; and it has with truth been said to be consolatory to the human mind. But if I were to punish a wicked king, I should regard the dignity in avenging the crime. Justice is grave and decorous, and in its punishments rather seems to submit to a necessity than to make a choice. Had Nero, or Agrippina, or Louis the Eleventh, or Charles the Ninth been the subject,—if Charles the Twelfth of Sweden, after the murder of Patkul, or his predecessor, Christina, after the murder of Monaldeschi, had fallen into your hands, Sir, or into mine, I am sure our conduct would have been different.

If the French king, or king of the French, (or by whatever name he is known in the new vocabulary of your Constitution,) has in his own person and that of his queen really deserved these unavowed, but unavenged, murderous attempts, and those frequent indignities more cruel than murder, such a person would ill deserve even that subordinate executory trust which I understand is to be placed in him; nor is he fit to be called chief in a nation which he has outraged and oppressed. A worse choice for such an office in a new commonwealth than that of a deposed tyrant could not possibly be made. But to degrade and insult a man as the worst of criminals, and afterwards to trust him in your highest concerns, as a faithful, honest, and zealous servant, is not consistent in reasoning, nor prudent in policy, nor safe in practice. Those who could make such an appointment must be guilty of a more flagrant breach of trust than any they have yet committed against the people. As this is the only crime in which your leading politicians could have acted inconsistently, I conclude that there is no sort of ground for these horrid insinuations. I think no better of all the other calumnies.

In England, we give no credit to them. We are generous enemies; we are faithful allies. We spurn from us with disgust and indignation the slanders of those who bring us their anecdotes with the attestation of the flower-de-luce on their shoulder. We have Lord George Gordon fast in Newgate; and neither his being a public proselyte to Judaism, nor his having, in his zeal against Catholic priests and all sorts of ecclesiastics, raised a mob (excuse the term, it is still in use here) which pulled down all our prisons, have preserved to him a liberty of which he did not render himself worthy by a virtuous use of it. We have rebuilt Newgate, and tenanted the mansion. We have prisons almost as strong as the Bastile, for those who dare to libel the queens of France. In this spiritual retreat let the noble libeller remain. Let him there meditate on his Talmud, until he learns a conduct more becoming his birth and parts, and not so disgraceful to the ancient religion to which he has become a proselyte,—or until some persons from your side of the water, to please your new Hebrew brethren, shall ransom him. He may then be enabled to purchase, with the old hoards of the synagogue, and a very small poundage on the long compound interest of the thirty pieces of silver, (Dr. Price has shown us what miracles compound interest will perform in 1790 years,) the lands which are lately discovered to have been usurped by the Gallican Church. Send us your Popish Archbishop of Paris, and we will send you our Protestant Rabbin. We shall treat the person you send us in exchange like a gentleman and an honest man, as he is: but pray let him bring with him the fund of his hospitality, bounty, and charity; and, depend upon it, we shall never confiscate a shilling of that honorable and pious fund, nor think of enriching the Treasury with the spoils of the poor-box.

To tell you the truth, my dear Sir, I think the honor of our nation to be somewhat concerned in the disclaimer of the proceedings of this society of the Old Jewry and the London Tavern. I have no man's proxy. I speak only from myself, when I disclaim, as I do with all possible earnestness, all communion with the actors in that triumph, or with the admirers of it. When I assert anything else, as concerning the people of England, I speak from observation, not from authority; but I speak from the experience I have had in a pretty extensive and mixed communication with the inhabitants of this kingdom, of all descriptions and ranks, and after a course of attentive observation, begun in early life, and continued for near forty years. I have often been astonished, considering that we are divided from you but by a slender dike of about twenty-four miles, and that the mutual intercourse between the two countries has lately been very great, to find how little you seem to know of us. I suspect that this is owing to your forming a judgment of this nation from certain publications, which do, very erroneously, if they do at all, represent the opinions and dispositions generally prevalent in England. The vanity, restlessness, petulance, and spirit of intrigue of several petty cabals, who attempt to hide their total want of consequence in bustle and noise, and puffing and mutual quotation of each other, makes you imagine that our contemptuous neglect of their abilities is a general mark of acquiescence in their opinions. No such thing, I assure you. Because half a dozen grasshoppers under a fern make the field ring with their importunate chink, whilst thousands of great cattle reposed beneath the shadow of the British oak chew the cud and are silent, pray do not imagine that those who make the noise are the only inhabitants of the field,—that, of course, they are many in number,—or that, after all, they are other than the little, shrivelled, meagre, hopping, though loud and troublesome insects of the hour.

I almost venture to affirm that not one in a hundred amongst us participates in the "triumph" of the Revolution Society. If the king and queen of France and their children were to fall into our hands by the chance of war, in the most acrimonious of all hostilities, (I deprecate such an event, I deprecate such hostility,) they would be treated with another sort of triumphal entry into London. We formerly have had a king of France in that situation: you have read how he was treated by the victor in the field, and in what manner he was afterwards received in England. Four hundred years have gone over us; but I believe we are not materially changed since that period. Thanks to our sullen resistance to innovation, thanks to the cold sluggishness of our national character, we still bear the stamp of our forefathers. We have not (as I conceive) lost the generosity and dignity of thinking of the fourteenth century; nor as yet have we subtilized ourselves into savages. We are not the converts of Rousseau; we are not the disciples of Voltaire; Helvetius has made no progress amongst us. Atheists are not our preachers; madmen are not our lawgivers. We know that we have made no discoveries, and we think that no discoveries are to be made, in morality,—nor many in the great principles of government, nor in the ideas of liberty, which were understood long before we were born altogether as well as they will be after the grave has heaped its mould upon our presumption, and the silent tomb shall have imposed its law on our pert loquacity. In England we have not yet been completely embowelled of our natural entrails: we still feel within us, and we cherish and cultivate, those inbred sentiments which are the faithful guardians, the active monitors of our duty, the true supporters of all liberal and manly morals. We have not been drawn and trussed, in order that we may be filled, like stuffed birds in a museum, with chaff and rags, and paltry, blurred shreds of paper about the rights of man. We preserve the whole of our feelings still native and entire, unsophisticated by pedantry and infidelity. We have real hearts of flesh and blood beating in our bosoms. We fear God; we look up with awe to kings, with affection to Parliaments, with duty to magistrates, with reverence to priests, and with respect to nobility.
%[94]
\footnote{ The English are, I conceive, misrepresented in a letter published in one of the papers, by a gentleman thought to be a Dissenting minister. When writing to Dr. Price of the spirit which prevails at Paris, he says,—"The spirit of the people in this place has abolished all the proud distinctions which the king and nobles had usurped in their minds: whether they talk of the king, the noble, or the priest, their whole language is that of the most enlightened and liberal amongst the English." If this gentleman means to confine the terms enlightened and liberal to one set of men in England, it may be true. It is not generally so.}
 Why? Because, when such ideas are brought before our minds, it is natural to be so affected; because all other feelings are false and spurious, and tend to corrupt our minds, to vitiate our primary morals, to render us unfit for rational liberty, and, by teaching us a servile, licentious, and abandoned insolence, to be our low sport for a few holidays, to make us perfectly fit for and justly deserving of slavery through the whole course of our lives.

You see, Sir, that in this enlightened age I am bold enough to confess that we are generally men of untaught feelings: that, instead of casting away all our old prejudices, we cherish them to a very considerable degree; and, to take more shame to ourselves, we cherish them because they are prejudices; and the longer they have lasted, and the more generally they have prevailed, the more we cherish them. We are afraid to put men to live and trade each on his own private stock of reason; because we suspect that the stock in each man is small, and that the individuals would do better to avail themselves of the general bank and capital of nations and of ages. Many of our men of speculation, instead of exploding general prejudices, employ their sagacity to discover the latent wisdom which prevails in them. If they find what they seek, (and they seldom fail,) they think it more wise to continue the prejudice, with the reason involved, than to cast away the coat of prejudice, and to leave nothing but the naked reason; because prejudice, with its reason, has a motive to give action to that reason, and an affection which will give it permanence. Prejudice is of ready application in the emergency; it previously engages the mind in a steady course of wisdom and virtue, and does not leave the man hesitating in the moment of decision, skeptical, puzzled, and unresolved. Prejudice renders a man's virtue his habit, and not a series of unconnected acts. Through just prejudice, his duty becomes a part of his nature.

Your literary men, and your politicians, and so do the whole clan of the enlightened among us, essentially differ in these points. They have no respect for the wisdom of others; but they pay it off by a very full measure of confidence in their own. With them it is a sufficient motive to destroy an old scheme of things, because it is an old one. As to the new, they are in no sort of fear with regard to the duration of a building run up in haste; because duration is no object to those who think little or nothing has been done before their time, and who place all their hopes in discovery. They conceive, very systematically, that all things which give perpetuity are mischievous, and therefore they are at inexpiable war with all establishments. They think that government may vary like modes of dress, and with as little ill effect; that there needs no principle of attachment, except a sense of present conveniency, to any constitution of the state. They always speak as if they were of opinion that there is a singular species of compact between them and their magistrates, which binds the magistrate, but which has nothing reciprocal in it, but that the majesty of the people has a right to dissolve it without any reason but its will. Their attachment to their country itself is only so far as it agrees with some of their fleeting projects: it begins and ends with that scheme of polity which falls in with their momentary opinion.

These doctrines, or rather sentiments, seem prevalent with your new statesmen. But they are wholly different from those on which we have always acted in this country.

I hear it is sometimes given out in France, that what is doing among you is after the example of England. I beg leave to affirm that scarcely anything done with you has originated from the practice or the prevalent opinions of this people, either in the act or in the spirit of the proceeding. Let me add, that we are as unwilling to learn these lessons from France as we are sure that we never taught them to that nation. The cabals here who take a sort of share in your transactions as yet consist of but a handful of people. If, unfortunately, by their intrigues, their sermons, their publications, and by a confidence derived from an expected union with the counsels and forces of the French nation, they should draw considerable numbers into their faction, and in consequence should seriously attempt anything here in imitation of what has been done with you, the event, I dare venture to prophesy, will be, that, with some trouble to their country, they will soon accomplish their own destruction. This people refused to change their law in remote ages from respect to the infallibility of Popes, and they will not now alter it from a pious implicit faith in the dogmatism of philosophers,—though the former was armed with the anathema and crusade, and though the latter should act with the libel and the lamp-iron.

Formerly your affairs were your own concern only. We felt for them as men; but we kept aloof from them, because we were not citizens of France. But when we see the model held up to ourselves, we must feel as Englishmen, and, feeling, we must provide as Englishmen. Your affairs, in spite of us, are made a part of our interest,—so far at least as to keep at a distance your panacea or your plague. If it be a panacea, we do not want it: we know the consequences of unnecessary physic. If it be a plague, it is such a plague that the precautions of the most severe quarantine ought to be established against it.

I hear on all hands, that a cabal, calling itself philosophic, receives the glory of many of the late proceedings, and that their opinions and systems are the true actuating spirit of the whole of them. I have heard of no party in England, literary or political, at any time, known by such a description. It is not with you composed of those men, is it? whom the vulgar, in their blunt, homely style, commonly call Atheists and Infidels? If it be, I admit that we, too, have had writers of that description, who made some noise in their day. At present they repose in lasting oblivion. Who, born within the last forty years, has read one word of Collins, and Toland, and Tindal, and Chubb, and Morgan, and that whole race who called themselves Freethinkers? Who now reads Bolingbroke? Who ever read him through? Ask the booksellers of London what is become of all these lights of the world. In as few years their few successors will go to the family vault of "all the Capulets." But whatever they were, or are, with us they were and are wholly unconnected individuals. With us they kept the common nature of their kind, and were not gregarious. They never acted in corps, nor were known as a faction in the state, nor presumed to influence in that name or character, or for the purposes of such a faction, on any of our public concerns. Whether they ought so to exist, and so be permitted to act, is another question. As such cabals have not existed in England, so neither has the spirit of them had any influence in establishing the original frame of our Constitution, or in any one of the several reparations and improvements it has undergone. The whole has been done under the auspices, and is confirmed by the sanctions, of religion and piety. The whole has emanated from the simplicity of our national character, and from a sort of native plainness and directness of understanding, which for a long time characterized those men who have successively obtained authority among us. This disposition still remains,—at least in the great body of the people.

We know, and, what is better, we feel inwardly, that religion is the basis of civil society, and the source of all good, and of all comfort.
%[95]
\footnote{ Sit igitur hoc ab initio persuasum civibus, dominos esse omnium rerum ac moderatores deos; eaque, quæ gerantur, eorum geri vi, ditione, ac numine; eosdemque optime de genere hominum mereri; et qualis quisque sit, quid agat, quid in se admittat, qua mente, qua pietate colat religiones intueri: piorum et impiorum habere rationem. His enim rebus imbutæ mentes haud sane abhorrebunt ab utili et a vera sententia.—Cic. de Legibus, l. 2.}
 In England we are so convinced of this, that there is no rust of superstition, with which the accumulated absurdity of the human mind might have crusted it over in the course of ages, that ninety-nine in a hundred of the people of England would not prefer to impiety. We shall never be such fools as to call in an enemy to the substance of any system to remove its corruptions, to supply its defects, or to perfect its construction. If our religious tenets should ever want a further elucidation, we shall not call on Atheism to explain them. We shall not light up our temple from that unhallowed fire. It will be illuminated with other lights. It will be perfumed with other incense than the infectious stuff which is imported by the smugglers of adulterated metaphysics. If our ecclesiastical establishment should want a revision, it is not avarice or rapacity, public or private, that we shall employ for the audit or receipt or application of its consecrated revenue. Violently condemning neither the Greek nor the Armenian, nor, since heats are subsided, the Roman system of religion, we prefer the Protestant: not because we think it has less of the Christian religion in it, but because, in our judgment, it has more. We are Protestants, not from indifference, but from zeal.

We know, and it is our pride to know, that man is by his constitution a religious animal; that atheism is against, not only our reason, but our instincts; and that it cannot prevail long. But if, in the moment of riot, and in a drunken delirium from the hot spirit drawn out of the alembic of hell, which in France is now so furiously boiling, we should uncover our nakedness, by throwing off that Christian religion which has hitherto been our boast and comfort, and one great source of civilization amongst us, and among many other nations, we are apprehensive (being well aware that the mind will not endure a void) that some uncouth, pernicious, and degrading superstition might take place of it.

For that reason, before we take from our establishment the natural, human means of estimation, and give it up to contempt, as you have done, and in doing it have incurred the penalties you well deserve to suffer, we desire that some other may be presented to us in the place of it. We shall then form our judgment.

On these ideas, instead of quarrelling with establishments, as some do, who have made a philosophy and a religion of their hostility to such institutions, we cleave closely to them. We are resolved to keep an established church, an established monarchy, an established aristocracy, and an established democracy, each in the degree it exists, and in no greater. I shall show you presently how much of each of these we possess.

It has been the misfortune (not, as these gentlemen think it, the glory) of this age, that everything is to be discussed, as if the Constitution of our country were to be always a subject rather of altercation than enjoyment. For this reason, as well as for the satisfaction of those among you (if any such you have among you) who may wish to profit of examples, I venture to trouble you with a few thoughts upon each of these establishments. I do not think they were unwise in ancient Rome, who, when they wished to new-model their laws, sent commissioners to examine the best-constituted republics within their reach.

First I beg leave to speak of our Church Establishment, which is the first of our prejudices,—not a prejudice destitute of reason, but involving in it profound and extensive wisdom. I speak of it first. It is first, and last, and midst in our minds. For, taking ground on that religious system of which we are now in possession, we continue to act on the early received and uniformly continued sense of mankind. That sense not only, like a wise architect, hath built up the august fabric of states, but, like a provident proprietor, to preserve the structure from profanation and ruin, as a sacred temple, purged from all the impurities of fraud and violence and injustice and tyranny, hath solemnly and forever consecrated the commonwealth, and all that officiate in it. This consecration is made, that all who administer in the government of men, in which they stand in the person of God Himself, should have high and worthy notions of their function and destination; that their hope should be full of immortality; that they should not look to the paltry pelf of the moment, nor to the temporary and transient praise of the vulgar, but to a solid, permanent existence, in the permanent part of their nature, and to a permanent fame and glory, in the example they leave as a rich inheritance to the world.

Such sublime principles ought to be infused into persons of exalted situations, and religious establishments provided that may continually revive and enforce them. Every sort of moral, every sort of civil, every sort of politic institution, aiding the rational and natural ties that connect the human understanding and affections to the divine, are not more than necessary, in order to build up that wonderful structure, Man,—whose prerogative it is, to be in a great degree a creature of his own making, and who, when made as he ought to be made, is destined to hold no trivial place in the creation. But whenever man is put over men, as the better nature ought ever to preside, in that case more particularly he should as nearly as possible be approximated to his perfection.

The consecration of the state by a state religious establishment is necessary also to operate with a wholesome awe upon free citizens; because, in order to secure their freedom, they must enjoy some determinate portion of power. To them, therefore, a religion connected with the state, and with their duty towards it, becomes even more necessary than in such societies where the people, by the terms of their subjection, are confined to private sentiments, and the management of their own family concerns. All persons possessing any portion of power ought to be strongly and awfully impressed with an idea that they act in trust, and that they are to account for their conduct in that trust to the one great Master, Author, and Founder of society.

This principle ought even to be more strongly impressed upon the minds of those who compose the collective sovereignty than upon those of single princes. Without instruments, these princes can do nothing. Whoever uses instruments, in finding helps, finds also impediments. Their power is therefore by no means complete; nor are they safe in extreme abuse. Such persons, however elevated by flattery, arrogance, and self-opinion, must be sensible, that, whether covered or not by positive law, in some way or other they are accountable even here for the abuse of their trust. If they are not cut off by a rebellion of their people, they may be strangled by the very janissaries kept for their security against all other rebellion. Thus we have seen the king of France sold by his soldiers for an increase of pay. But where popular authority is absolute and unrestrained, the people have an infinitely greater, because a far better founded, confidence in their own power. They are themselves in a great measure their own instruments. They are nearer to their objects. Besides, they are less under responsibility to one of the greatest controlling powers on earth, the sense of fame and estimation. The share of infamy that is likely to fall to the lot of each individual in public acts is small indeed: the operation of opinion being in the inverse ratio to the number of those who abuse power. Their own approbation of their own acts has to them the appearance of a public judgment in their favor. A perfect democracy is therefore the most shameless thing in the world. As it is the most shameless, it is also the most fearless. No man apprehends in his person that he can be made subject to punishment. Certainly the people at large never ought: for, as all punishments are for example towards the conservation of the people at large, the people at large can never become the subject of punishment by any human hand.
%[96]
\footnote{ Quicquid multis peccatur inultum.}
 It is therefore of infinite importance that they should not be suffered to imagine that their will, any more than that of kings, is the standard of right and wrong. They ought to be persuaded that they are full as little entitled, and far less qualified, with safety to themselves, to use any arbitrary power whatsoever; that therefore they are not, under a false show of liberty, but in truth to exercise an unnatural, inverted domination, tyrannically to exact from those who officiate in the state, not an entire devotion to their interest, which is their right, but an abject submission to their occasional will: extinguishing thereby, in all those who serve them, all moral principle, all sense of dignity, all use of judgment, and all consistency of character; whilst by the very same process they give themselves up a proper, a suitable, but a most contemptible prey to the servile ambition of popular sycophants or courtly flatterers.

When the people have emptied themselves of all the lust of selfish will, which without religion it is utterly impossible they ever should,—when they are conscious that they exercise, and exercise perhaps in a higher link of the order of delegation, the power which to be legitimate must be according to that eternal, immutable law in which will and reason are the same,—they will be more careful how they place power in base and incapable hands. In their nomination to office, they will not appoint to the exercise of authority as to a pitiful job, but as to a holy function; not according to their sordid, selfish interest, nor to their wanton caprice, nor to their arbitrary will; but they will confer that power (which any man may well tremble to give or to receive) on those only in whom they may discern that predominant proportion of active virtue and wisdom, taken together and fitted to the charge, such as in the great and inevitable mixed mass of human imperfections and infirmities is to be found.

When they are habitually convinced that no evil can be acceptable, either in the act or the permission, to Him whose essence is good, they will be better able to extirpate out of the minds of all magistrates, civil, ecclesiastical, or military, anything that bears the least resemblance to a proud and lawless domination.

But one of the first and most leading principles on which the commonwealth and the laws are consecrated is lest the temporary possessors and life-renters in it, unmindful of what they have received from their ancestors, or of what is due to their posterity, should act as if they were the entire masters; that they should not think it amongst their rights to cut off the entail or commit waste on the inheritance, by destroying at their pleasure the whole original fabric of their society: hazarding to leave to those who come after them a ruin instead of an habitation,—and teaching these successors as little to respect their contrivances as they had themselves respected the institutions of their forefathers. By this unprincipled facility of changing the state as often and as much and in as many ways as there are floating fancies or fashions, the whole chain and continuity of the commonwealth would be broken; no one generation could link with the other; men would become little better than the flies of a summer.

And first of all, the science of jurisprudence, the pride of the human intellect, which, with all its defects, redundancies, and errors, is the collected reason of ages, combining the principles of original justice with the infinite variety of human concerns, as a heap of old exploded errors, would be no longer studied. Personal self-sufficiency and arrogance (the certain attendants upon all those who have never experienced a wisdom greater than their own) would usurp the tribunal. Of course no certain laws, establishing invariable grounds of hope and fear, would keep the actions of men in a certain course, or direct them to a certain end. Nothing stable in the modes of holding property or exercising function could form a solid ground on which any parent could speculate in the education of his offspring, or in a choice for their future establishment in the world. No principles would be early worked into the habits. As soon as the most able instructor had completed his laborious course of institution, instead of sending forth his pupil accomplished in a virtuous discipline fitted to procure him attention and respect in his place in society, he would find everything altered, and that he had turned out a poor creature to the contempt and derision of the world, ignorant of the true grounds of estimation. Who would insure a tender and delicate sense of honor to beat almost with the first pulses of the heart, when no man could know what would be the test of honor in a nation continually varying the standard of its coin? No part of life would retain its acquisitions. Barbarism with regard to science and literature, unskilfulness with regard to arts and manufactures, would infallibly succeed to the want of a steady education and settled principle; and thus the commonwealth itself would in a few generations crumble away, be disconnected into the dust and powder of individuality, and at length dispersed to all the winds of heaven.

To avoid, therefore, the evils of inconstancy and versatility, ten thousand times worse than those of obstinacy and the blindest prejudice, we have consecrated the state, that no man should approach to look into its defects or corruptions but with due caution; that he should never dream of beginning its reformation by its subversion; that he should approach to the faults of the state as to the wounds of a father, with pious awe and trembling solicitude. By this wise prejudice we are taught to look with horror on those children of their country who are prompt rashly to hack that aged parent in pieces and put him into the kettle of magicians, in hopes that by their poisonous weeds and wild incantations they may regenerate the paternal constitution and renovate their father's life.

Society is, indeed, a contract. Subordinate contracts for objects of mere occasional interest may be dissolved at pleasure; but the state ought not to be considered as nothing better than a partnership agreement in a trade of pepper and coffee, calico or tobacco, or some other such low concern, to be taken up for a little temporary interest, and to be dissolved by the fancy of the parties. It is to be looked on with other reverence; because it is not a partnership in things subservient only to the gross animal existence of a temporary and perishable nature. It is a partnership in all science, a partnership in all art, a partnership in every virtue and in all perfection. As the ends of such a partnership cannot be obtained in many generations, it becomes a partnership not only between those who are living, but between those who are living, those who are dead, and those who are to be born. Each contract of each particular state is but a clause in the great primeval contract of eternal society, linking the lower with the higher natures, connecting the visible and invisible world, according to a fixed compact sanctioned by the inviolable oath which holds all physical and all moral natures each in their appointed place. This law is not subject to the will of those who, by an obligation above them, and infinitely superior, are bound to submit their will to that law. The municipal corporations of that universal kingdom are not morally at liberty, at their pleasure, and on their speculations of a contingent improvement, wholly to separate and tear asunder the bands of their subordinate community, and to dissolve it into an unsocial, uncivil, unconnected chaos of elementary principles. It is the first and supreme necessity only, a necessity that is not chosen, but chooses, a necessity paramount to deliberation, that admits no discussion and demands no evidence, which alone can justify a resort to anarchy. This necessity is no exception to the rule; because this necessity itself is a part, too, of that moral and physical disposition of things to which man must be obedient by consent or force: but if that which is only submission to necessity should be made the object of choice, the law is broken, Nature is disobeyed, and the rebellious are outlawed, cast forth, and exiled, from this world of reason, and order, and peace, and virtue, and fruitful penitence, into the antagonist world of madness, discord, vice, confusion, and unavailing sorrow.

These, my dear Sir, are, were, and, I think, long will be, the sentiments of not the least learned and reflecting part of this kingdom. They who are included in this description form their opinions on such grounds as such persons ought to form them. The less inquiring receive them from an authority which those whom Providence dooms to live on trust need not be ashamed to rely on. These two sorts of men move in the same direction, though in a different place. They both move with the order of the universe. They all know or feel this great ancient truth:—"Quod illi principi et præpotenti Deo qui omnem hunc mundum regit nihil eorum quæ quidem fiant in terris acceptius quam concilia et coetus hominum jure sociati quæ civitates appellantur." They take this tenet of the head and heart, not from the great name which it immediately bears, nor from the greater from whence it is derived, but from that which alone can give true weight and sanction to any learned opinion, the common nature and common relation of men. Persuaded that all things ought to be done with reference, and referring all to the point of reference to which all should be directed, they think themselves bound, not only as individuals in the sanctuary of the heart, or as congregated in that personal capacity, to renew the memory of their high origin and cast, but also in their corporate character to perform their national homage to the Institutor and Author and Protector of civil society, without which civil society man could not by any possibility arrive at the perfection of which his nature is capable, nor even make a remote and faint approach to it. They conceive that He who gave our nature to be perfected by our virtue willed also the necessary means of its perfection: He willed, therefore, the state: He willed its connection with the source and original archetype of all perfection. They who are convinced of this His will, which is the law of laws and the sovereign of sovereigns, cannot think it reprehensible that this our corporate fealty and homage, that this our recognition of a signiory paramount, I had almost said this oblation of the state itself, as a worthy offering on the high altar of universal praise, should be performed, as all public, solemn acts are performed, in buildings, in music, in decoration, in speech, in the dignity of persons, according to the customs of mankind, taught by their nature,—that is, with modest splendor, with unassuming state, with mild majesty and sober pomp. For those purposes they think some part of the wealth of the country is as usefully employed as it can be in fomenting the luxury of individuals. It is the public ornament. It is the public consolation. It nourishes the public hope. The poorest man finds his own importance and dignity in it, whilst the wealth and pride of individuals at every moment makes the man of humble rank and fortune sensible of his inferiority, and degrades and vilifies his condition. It is for the man in humble life, and to raise his nature, and to put him in mind of a state in which the privileges of opulence will cease, when he will be equal by nature, and may be more than equal by virtue, that this portion of the general wealth of his country is employed and sanctified.

I assure you I do not aim at singularity. I give you opinions which have been accepted amongst us, from very early times to this moment, with a continued and general approbation, and which, indeed, are so worked into my mind that I am unable to distinguish what I have learned from others from the results of my own meditation.

It is on some such principles that the majority of the people of England, far from thinking a religious national establishment unlawful, hardly think it lawful to be without one. In France you are wholly mistaken, if you do not believe us above all other things attached to it, and beyond all other nations; and when this people has acted unwisely and unjustifiably in its favor, (as in some instances they have done, most certainly,) in their very errors you will at least discover their zeal.

This principle runs through the whole system of their polity. They do not consider their Church establishment as convenient, but as essential to their state: not as a thing heterogeneous and separable,—something added for accommodation,—what they may either keep up or lay aside, according to their temporary ideas of convenience. They consider it as the foundation of their whole Constitution, with which, and with every part of which, it holds an indissoluble union. Church and State are ideas inseparable in their minds, and scarcely is the one ever mentioned without mentioning the other.

Our education is so formed as to confirm and fix this impression. Our education is in a manner wholly in the hands of ecclesiastics, and in all stages from infancy to manhood. Even when our youth, leaving schools and universities, enter that most important period of life which begins to link experience and study together, and when with that view they visit other countries, instead of old domestics whom we have seen as governors to principal men from other parts, three fourths of those who go abroad with our young nobility and gentlemen are ecclesiastics: not as austere masters, nor as mere followers; but as friends and companions of a graver character, and not seldom persons as well born as themselves. With them, as relations, they most commonly keep up a close connection through life. By this connection we conceive that we attach our gentlemen to the Church; and we liberalize the Church by an intercourse with the leading characters of the country.

So tenacious are we of the old ecclesiastical modes and fashions of institution, that very little alteration has been made in them since the fourteenth or fifteenth century: adhering in this particular, as in all things else, to our old settled maxim, never entirely nor at once to depart from antiquity. We found these old institutions, on the whole, favorable to morality and discipline; and we thought they were susceptible of amendment, without altering the ground. We thought that they were capable of receiving and meliorating, and above all of preserving, the accessions of science and literature, as the order of Providence should successively produce them. And after all, with this Gothic and monkish education, (for such it is in the groundwork,) we may put in our claim to as ample and as early a share in all the improvements in science, in arts, and in literature, which have illuminated and adorned the modern world, as any other nation in Europe: we think one main cause of this improvement was our not despising the patrimony of knowledge which was left us by our forefathers.

It is from our attachment to a Church establishment, that the English nation did not think it wise to intrust that great fundamental interest of the whole to what they trust no part of their civil or military public service,—that is, to the unsteady and precarious contribution of individuals. They go further. They certainly never have suffered, and never will suffer, the fixed estate of the Church to be converted into a pension, to depend on the Treasury, and to be delayed, withheld, or perhaps to be extinguished by fiscal difficulties: which difficulties may sometimes be pretended for political purposes, and are in fact often brought on by the extravagance, negligence, and rapacity of politicians. The people of England think that they have constitutional motives, as well as religious, against any project of turning their independent clergy into ecclesiastical pensioners of state. They tremble for their liberty, from the influence of a clergy dependent on the crown; they tremble for the public tranquillity, from the disorders of a factious clergy, if it were made to depend upon any other than the crown. They therefore made their Church, like their king and their nobility, independent.

From the united considerations of religion and constitutional policy, from their opinion of a duty to make a sure provision for the consolation of the feeble and the instruction of the ignorant, they have incorporated and identified the estate of the Church with the mass of private property, of which the state is not the proprietor, either for use or dominion, but the guardian only and the regulator. They have ordained that the provision of this establishment might be as stable as the earth on which it stands, and should not fluctuate with the Euripus of funds and actions.

The men of England, the men, I mean, of light and leading in England, whose wisdom (if they have any) is open and direct, would be ashamed, as of a silly, deceitful trick, to profess any religion in name, which by their proceedings they appear to contemn. If by their conduct (the only language that rarely lies) they seemed to regard the great ruling principle of the moral and the natural world as a mere invention to keep the vulgar in obedience, they apprehend that by such a conduct they would defeat the politic purpose they have in view. They would find it difficult to make others believe in a system to which they manifestly gave no credit themselves. The Christian statesmen of this land would, indeed, first provide for the multitude, because it is the multitude, and is therefore, as such, the first object in the ecclesiastical institution, and in all institutions. They have been taught that the circumstance of the Gospel's being preached to the poor was one of the great tests of its true mission. They think, therefore, that those do not believe it who do not take care it should be preached to the poor. But as they know that charity is not confined to any one description, but ought to apply itself to all men who have wants, they are not deprived of a due and anxious sensation of pity to the distresses of the miserable great. They are not repelled, through a fastidious delicacy, at the stench of their arrogance and presumption, from a medicinal attention to their mental blotches and running sores. They are sensible that religious instruction is of more consequence to them than to any others: from the greatness of the temptation to which they are exposed; from the important consequences that attend their faults; from the contagion of their ill example; from the necessity of bowing down the stubborn neck of their pride and ambition to the yoke of moderation and virtue; from a consideration of the fat stupidity and gross ignorance concerning what imports men most to know, which prevails at courts, and at the head of armies, and in senates, as much as at the loom and in the field.

The English people are satisfied, that to the great the consolations of religion are as necessary as its instructions. They, too, are among the unhappy. They feel personal pain and domestic sorrow. In these they have no privilege, but are subject to pay their full contingent to the contributions levied on mortality. They want this sovereign balm under their gnawing cares and anxieties, which, being less conversant about the limited wants of animal life, range without limit, and are diversified by infinite combinations in the wild and unbounded regions of imagination. Some charitable dole is wanting to these, our often very unhappy brethren, to fill the gloomy void that reigns in minds which have nothing on earth to hope or fear; something to relieve in the killing languor and over-labored lassitude of those who have nothing to do; something to excite an appetite to existence in the palled satiety which attends on all pleasures which may be bought, where Nature is not left to her own process, where even desire is anticipated, and therefore fruition defeated by meditated schemes and contrivances of delight, and no interval, no obstacle, is interposed between the wish and the accomplishment.

The people of England know how little influence the teachers of religion are likely to have with the wealthy and powerful of long standing, and how much less with the newly fortunate, if they appear in a manner no way assorted to those with whom they must associate, and over whom they must even exercise, in some cases, something like an authority. What must they think of that body of teachers, if they see it in no part above the establishment of their domestic servants? If the poverty were voluntary, there might be some difference. Strong instances of self-denial operate powerfully on our minds; and a man who has no wants has obtained great freedom and firmness, and even dignity. But as the mass of any description of men are but men, and their poverty cannot be voluntary, that disrespect which attends upon all lay poverty will not depart from the ecclesiastical. Our provident Constitution has therefore taken care that those who are to instruct presumptuous ignorance, those who are to be censors over insolent vice, should neither incur their contempt nor live upon their alms; nor will it tempt the rich to a neglect of the true medicine of their minds. For these reasons, whilst we provide first for the poor, and with a parental solicitude, we have not relegated religion (like something we were ashamed to show) to obscure municipalities or rustic villages. No! we will have her to exalt her mitred front in courts and parliaments. We will have her mixed throughout the whole mass of life, and blended with all the classes of society. The people of England will show to the haughty potentates of the world, and to their talking sophisters, that a free, a generous, an informed nation honors the high magistrates of its Church; that it will not suffer the insolence of wealth and titles, or any other species of proud pretension, to look down with scorn upon what they look up to with reverence, nor presume to trample on that acquired personal nobility which they intend always to be, and which often is, the fruit, not the reward, (for what can be the reward?) of learning, piety, and virtue. They can see, without pain or grudging, an archbishop precede a duke. They can see a bishop of Durham or a bishop of Winchester in possession of ten thousand pounds a year, and cannot conceive why it is in worse hands than estates to the like amount in the hands of this earl or that squire; although it may be true that so many dogs and horses are not kept by the former, and fed with the victuals which ought to nourish the children of the people. It is true, the whole Church revenue is not always employed, and to every shilling, in charity; nor perhaps ought it; but something is generally so employed. It is better to cherish virtue and humanity, by leaving much to free will, even with some loss to the object, than to attempt to make men mere machines and instruments of a political benevolence. The world on the whole will gain by a liberty without which virtue cannot exist.

When once the commonwealth has established the estates of the Church as property, it can consistently hear nothing of the more or the less. Too much and too little are treason against property. What evil can arise from the quantity in any hand, whilst the supreme authority has the full, sovereign superintendence over this, as over any property, to prevent every species of abuse,—and whenever it notably deviates, to give to it a direction agreeable to the purposes of its institution?

In England most of us conceive that it is envy and malignity towards those who are often the beginners of their own fortune, and not a love of the self-denial and mortification of the ancient Church, that makes some look askance at the distinctions and honors and revenues which, taken from no person, are set apart for virtue. The ears of the people of England are distinguishing. They hear these men speak broad. Their tongue betrays them. Their language is in the patois of fraud, in the cant and gibberish of hypocrisy. The people of England must think so, when these praters affect to carry back the clergy to that primitive evangelic poverty which in the spirit ought always to exist in them, (and in us, too, however we may like it,) but in the thing must be varied, when the relation of that body to the state is altered,—when manners, when modes of life, when indeed the whole order of human affairs, has undergone a total revolution. We shall believe those reformers to be then honest enthusiasts, not, as now we think them, cheats and deceivers, when we see them throwing their own goods into common, and submitting their own persons to the austere discipline of the early Church.

With these ideas rooted in their minds, the Commons of Great Britain, in the national emergencies, will never seek their resource from the confiscation of the estates of the Church and poor. Sacrilege and proscription are not among the ways and means of our Committee of Supply. The Jews in Change Alley have not yet dared to hint their hopes of a mortgage on the revenues belonging to the see of Canterbury. I am not afraid that I shall be disavowed, when I assure you that there is not one public man in this kingdom, whom you wish to quote,—no, not one, of any party or description,—who does not reprobate the dishonest, perfidious, and cruel confiscation which the National Assembly has been compelled to make of that property which it was their first duty to protect.

It is with the exultation of a little national pride I tell you that those amongst us who have wished to pledge the societies of Paris in the cup of their abominations have been disappointed. The robbery of your Church has proved a security to the possessions of ours. It has roused the people. They see with horror and alarm that enormous and shameless act of proscription. It has opened, and will more and more open, their eyes upon the selfish enlargement of mind and the narrow liberality of sentiment of insidious men, which, commencing in close hypocrisy and fraud, have ended in open violence and rapine. At home we behold similar beginnings. We are on our guard against similar conclusions.

I hope we shall never be so totally lost to all sense of the duties imposed upon us by the law of social union, as, upon any pretest of public service, to confiscate the goods of a single unoffending citizen. Who but a tyrant (a name expressive of everything which can vitiate and degrade human nature) could think of seizing on the property of men, unaccused, unheard, untried, by whole descriptions, by hundreds and thousands together? Who that had not lost every trace of humanity could think of casting down men of exalted rank and sacred function, some of them of an age to call at once for reverence and compassion,—of casting them down from the highest situation in the commonwealth, wherein they were maintained by their own landed property, to a state of indigence, depression, and contempt?

The confiscators truly have made some allowance to their victims from the scraps and fragments of their own tables, from which they have been so harshly driven, and which have been so bountifully spread for a feast to the harpies of usury. But to drive men from independence to live on alms is itself great cruelty. That which might be a tolerable condition to men in one state of life, and not habituated to other things, may, when all these circumstances are altered, be a dreadful revolution, and one to which a virtuous mind would feel pain in condemning any guilt, except that which would demand the life of the offender. But to many minds this punishment of degradation and infamy is worse than death. Undoubtedly it is an infinite aggravation of this cruel suffering, that the persons who were taught a double prejudice in favor of religion, by education, and by the place they held in the administration of its functions, are to receive the remnants of their property as alms from the profane and impious hands of those who had plundered them of all the rest,—to receive (if they are at all to receive) not from the charitable contributions of the faithful, but from the insolent tenderness of known and avowed atheism, the maintenance of religion, measured out to them on the standard of the contempt in which it is held, and for the purpose of rendering those who receive the allowance vile and of no estimation in the eyes of mankind.

But this act of seizure of property, it seems, is a judgment in law, and not a confiscation. They have, it seems, found out in the academies of the Palais Royal and the Jacobins, that certain men had no right to the possessions which they held under law, usage, the decisions of courts, and the accumulated prescription of a thousand years. They say that ecclesiastics are fictitious persons, creatures of the state, whom at pleasure they may destroy, and of course limit and modify in every particular; that the goods they possess are not properly theirs, but belong to the state which created the fiction; and we are therefore not to trouble ourselves with what they may suffer in their natural feelings and natural persons on account of what is done towards them in this their constructive character. Of what import is it, under what names you injure men, and deprive them of the just emoluments of a profession in which they were not only permitted, but encouraged by the state to engage, and upon the supposed certainty of which emoluments they had formed the plan of their lives, contracted debts, and led multitudes to an entire dependence upon them?

You do not imagine, Sir, that I am going to compliment this miserable distinction of persons with any long discussion. The arguments of tyranny are as contemptible as its force is dreadful. Had not your confiscators by their early crimes obtained a power which secures indemnity to all the crimes of which they have since been guilty, or that they can commit, it is not the syllogism of the logician, but the lash of the executioner, that would have refuted a sophistry which becomes an accomplice of theft and murder. The sophistic tyrants of Paris are loud in their declamations against the departed regal tyrants who in former ages have vexed the world. They are thus bold, because they are safe from the dungeons and iron cages of their old masters. Shall we be more tender of the tyrants of our own time, when we see them acting worse tragedies under our eyes? Shall we not use the same liberty that they do, when we can use it with the same safety, when to speak honest truth only requires a contempt of the opinions of those whose actions we abhor?

This outrage on all the rights of property was at first covered with what, on the system of their conduct, was the most astonishing of all pretexts,—a regard to national faith. The enemies to property at first pretended a most tender, delicate, and scrupulous anxiety for keeping the king's engagements with the public creditor. These professors of the rights of men are so busy in teaching others, that they have not leisure to learn anything themselves; otherwise they would have known that it is to the property of the citizen, and not to the demands of the creditor of the state, that the first and original faith of civil society is pledged. The claim of the citizen is prior in time, paramount in title, superior in equity. The fortunes of individuals, whether possessed by acquisition, or by descent, or in virtue of a participation in the goods of some community, were no part of the creditor's security, expressed or implied. They never so much as entered into his head, when he made his bargain. He well knew that the public, whether represented by a monarch or by a senate, can pledge nothing but the public estate; and it can have no public estate, except in what it derives from a just and proportioned imposition upon the citizens at large. This was engaged, and nothing else could be engaged, to the public creditor. No man can mortgage his injustice as a pawn for his fidelity.

It is impossible to avoid some observation on the contradictions, caused by the extreme rigor and the extreme laxity of this new public faith, which influenced in this transaction, and which influenced not according to the nature of the obligation, but to the description of the persons to whom it was engaged. No acts of the old government of the kings of France are held valid in the National Assembly, except its pecuniary engagements: acts of all others of the most ambiguous legality. The rest of the acts of that royal government are considered in so odious a light that to have a claim under its authority is looked on as a sort of crime. A pension, given as a reward for service to the state, is surely as good a ground of property as any security for money advanced to the state. It is a better; for money is paid, and well paid, to obtain that service. We have, however, seen multitudes of people under this description in France, who never had been deprived of their allowances by the most arbitrary ministers in the most arbitrary times, by this assembly of the rights of men robbed without mercy. They were told, in answer to their claim to the bread earned with their blood, that their services had not been rendered to the country that now exists.

This laxity of public faith is not confined to those unfortunate persons. The Assembly, with perfect consistency, it must be owned, is engaged in a respectable deliberation how far it is bound by the treaties made with other nations under the former government; and their committee is to report which of them they ought to ratify, and which not. By this means they have put the external fidelity of this virgin state on a par with its internal.

It is not easy to conceive upon what rational principle the royal government should not, of the two, rather have possessed the power of rewarding service and making treaties, in virtue of its prerogative, than that of pledging to creditors the revenue of the state, actual and possible. The treasure of the nation, of all things, has been the least allowed to the prerogative of the king of France, or to the prerogative of any king in Europe. To mortgage the public revenue implies the sovereign dominion, in the fullest sense, over the public purse. It goes far beyond the trust even of a temporary and occasional taxation. The acts, however, of that dangerous power (the distinctive mark of a boundless despotism) have been alone held sacred. Whence arose this preference given by a democratic assembly to a body of property deriving its title from the most critical and obnoxious of all the exertions of monarchical authority? Reason can furnish nothing to reconcile inconsistency; nor can partial favor be accounted for upon equitable principles. But the contradiction and partiality which admit no justification are not the less without an adequate cause; and that cause I do not think it difficult to discover.

By the vast debt of France a great moneyed interest has insensibly grown up, and with it a great power. By the ancient usages which prevailed in that kingdom, the general circulation of property, and in particular the mutual convertibility of land into money and of money into land, had always been a matter of difficulty. Family settlements, rather more general and more strict than they are in England, the jus retractûs, the great mass of landed property held by the crown, and, by a maxim of the French law, held unalienably, the vast estates of the ecclesiastic corporations,—all these had kept the landed and moneyed interests more separated in France, less miscible, and the owners of the two distinct species of property not so well disposed to each other as they are in this country.

The moneyed property was long looked on with rather an evil eye by the people. They saw it connected with their distresses, and aggravating them. It was no less envied by the old landed interests,—partly for the same reasons that rendered it obnoxious to the people, but much more so as it eclipsed, by the splendor of an ostentatious luxury, the unendowed pedigrees and naked titles of several among the nobility. Even when the nobility, which represented the more permanent landed interest, united themselves by marriage (which sometimes was the case) with the other description, the wealth which saved the family from ruin was supposed to contaminate and degrade it. Thus the enmities and heart burnings of these parties were increased even by the usual means by which discord is made to cease and quarrels are turned into friendship. In the mean time, the pride of the wealthy men, not noble, or newly noble, increased with its cause. They felt with resentment an inferiority the grounds of which they did not acknowledge. There was no measure to which they were not willing to lend themselves, in order to be revenged of the outrages of this rival pride, and to exalt their wealth to what they considered as its natural rank and estimation. They struck at the nobility through the crown and the Church. They attacked them particularly on the side on which they thought them the most vulnerable,—that is, the possessions of the Church, which, through the patronage of the crown, generally devolved upon the nobility. The bishoprics and the great commendatory abbeys were, with few exceptions, held by that order.

In this state of real, though not always perceived, warfare between the noble ancient landed interest and the new moneyed interest, the greatest, because the most applicable, strength was in the hands of the latter. The moneyed interest is in its nature more ready for any adventure, and its possessors more disposed to new enterprises of any kind. Being of a recent acquisition, it falls in more naturally with any novelties. It is therefore the kind of wealth which will be resorted to by all who wish for change.

Along with the moneyed interest, a new description of men had grown up, with whom that interest soon formed a close and marked union: I mean the political men of letters. Men of letters, fond of distinguishing themselves, are rarely averse to innovation. Since the decline of the life and greatness of Louis the Fourteenth, they were not so much cultivated either by him, or by the Regent, or the successors to the crown; nor were they engaged to the court by favors and emoluments so systematically as during the splendid period of that ostentatious and not impolitic reign. What they lost in the old court protection they endeavored to make up by joining in a sort of incorporation of their own; to which the two academies of France, and afterwards the vast undertaking of the Encyclopædia, carried on by a society of these gentlemen, did not a little contribute.

The literary cabal had some years ago formed something like a regular plan for the destruction of the Christian religion. This object they pursued with a degree of zeal which hitherto had been discovered only in the propagators of some system of piety. They were possessed with a spirit of proselytism in the most fanatical degree,—and from thence, by an easy progress, with the spirit of persecution according to their means.
%[97]
\footnote{ This (down to the end of the first sentence in the next paragraph) and some other parts, here and there, were inserted, on his reading the manuscript, by my lost son.}
 What was not to be done towards their great end by any direct or immediate act might be wrought by a longer process through the medium of opinion. To command that opinion, the first step is to establish a dominion over those who direct it. They contrived to possess themselves, with great method and perseverance, of all the avenues to literary fame. Many of them, indeed, stood high in the ranks of literature and science. The world had done them justice, and in favor of general talents forgave the evil tendency of their peculiar principles. This was true liberality; which they returned by endeavoring to confine the reputation of sense, learning, and taste to themselves or their followers. I will venture to say that this narrow, exclusive spirit has not been less prejudicial to literature and to taste than to morals and true philosophy. These atheistical fathers have a bigotry of their own; and they have learnt to talk against monks with the spirit of a monk. But in some things they are men of the world. The resources of intrigue are called in to supply the defects of argument and wit. To this system of literary monopoly was joined an unremitting industry to blacken and discredit in every way, and by every means, all those who did not hold to their faction. To those who have observed the spirit of their conduct it has long been clear that nothing was wanted but the power of carrying the intolerance of the tongue and of the pen into a persecution which would strike at property, liberty, and life.

The desultory and faint persecution carried on against them, more from compliance with form and decency than with serious resentment, neither weakened their strength nor relaxed their efforts. The issue of the whole was, that, what with opposition, and what with success, a violent and malignant zeal, of a kind hitherto unknown in the world, had taken an entire possession of their minds, and rendered their whole conversation, which otherwise would have been pleasing and instructive, perfectly disgusting. A spirit of cabal, intrigue, and proselytism pervaded all their thoughts, words, and actions. And as controversial zeal soon turns its thoughts on force, they began to insinuate themselves into a correspondence with foreign princes,—in hopes, through their authority, which at first they flattered, they might bring about the changes they had in view. To them it was indifferent whether these changes were to be accomplished by the thunderbolt of despotism or by the earthquake of popular commotion. The correspondence between this cabal and the late king of Prussia will throw no small light upon the spirit of all their proceedings.
%[98]
\footnote{ I do not choose to shock the feeling of the moral reader with any quotation of their vulgar, base, and profane language.}
 For the same purpose for which they intrigued with princes, they cultivated, in a distinguished manner, the moneyed interest of France; and partly through the means furnished by those whose peculiar offices gave them the most extensive and certain means of communication, they carefully occupied all the avenues to opinion.

Writers, especially when they act in a body and with one direction, have great influence on the public mind; the alliance, therefore, of these writers with the moneyed interest
%[99]
\footnote{ Their connection with Turgot and almost all the people of the finance.}
 had no small effect in removing the popular odium and envy which attended that species of wealth. These writers, like the propagators of all novelties, pretended to a great zeal for the poor and the lower orders, whilst in their satires they rendered hateful, by every exaggeration, the faults of courts, of nobility, and of priesthood. They became a sort of demagogues. They served as a link to unite, in favor of one object, obnoxious wealth to restless and desperate poverty.

As these two kinds of men appear principal leaders in all the late transactions, their junction and politics will serve to account, not upon any principles of law or of policy, but as a cause, for the general fury with which all the landed property of ecclesiastical corporations has been attacked, and the great care which, contrary to their pretended principles, has been taken of a moneyed interest originating from the authority of the crown. All the envy against wealth and power was artificially directed against other descriptions of riches. On what other principle than that which I have stated can we account for an appearance so extraordinary and unnatural as that of the ecclesiastical possessions, which had stood so many successions of ages and shocks of civil violences, and were guarded at once by justice and by prejudice, being applied to the payment of debts comparatively recent, invidious, and contracted by a decried and subverted government?

Was the public estate a sufficient stake for the public debts? Assume that it was not, and that a loss must be incurred somewhere. When the only estate lawfully possessed, and which the contracting parties had in contemplation at the time in which their bargain was made, happens to fail, who, according to the principles of natural and legal equity, ought to be the sufferer? Certainly it ought to be either the party who trusted, or the party who persuaded him to trust, or both; and not third parties who had no concern with the transaction. Upon any insolvency, they ought to suffer who were weak enough to lend upon bad security, or they who fraudulently held out a security that was not valid. Laws are acquainted with no other rules of decision. But by the new institute of the rights of men, the only persons who in equity ought to suffer are the only persons who are to be saved harmless: those are to answer the debt who neither were lenders nor borrowers, mortgagers nor mortgagees.

What had the clergy to do with these transactions? What had they to do with any public engagement further than the extent of their own debt? To that, to be sure, their estates were bound to the last acre. Nothing can lead more to the true spirit of the Assembly, which sits for public confiscation with its new equity and its new morality, than an attention to their proceeding with regard to this debt of the clergy. The body of confiscators, true to that moneyed interest for which they were false to every other, have found the clergy competent to incur a legal debt. Of course they declared them legally entitled to the property which their power of incurring the debt and mortgaging the estate implied: recognizing the rights of those persecuted citizens in the very act in which they were thus grossly violated.

If, as I said, any persons are to mate good deficiencies to the public creditor, besides the public at large, they must be those who managed the agreement. Why, therefore, are not the estates of all the comptrollers-general confiscated?
%[100]
\footnote{ All have been confiscated in their turn.}
 Why not those of the long succession of ministers, financiers, and bankers who have been enriched whilst the nation was impoverished by their dealings and their counsels? Why is not the estate of M. Laborde declared forfeited rather than of the Archbishop of Paris, who has had nothing to do in the creation or in the jobbing of the public funds? Or, if you must confiscate old landed estates in favor of the money-jobbers, why is the penalty confined to one description? I do not know whether the expenses of the Duke de Choiseul have left anything of the infinite sums which he had derived from the bounty of his master, during the transactions of a reign which contributed largely, by every species of prodigality in war and peace, to the present debt of France. If any such remains, why is not this confiscated? I remember to have been in Paris during the time of the old government. I was there just after the Duke d'Aiguillon had been snatched (as it was generally thought) from the block by the hand of a protecting despotism. He was a minister, and had some concern in the affairs of that prodigal period. Why do I not see his estate delivered up to the municipalities in which it is situated? The noble family of Noailles have long been servants (meritorious servants I admit) to the crown of France, and have had of course some share in its bounties. Why do I hear nothing of the application of their estates to the public debt? Why is the estate of the Duke de Rochefoucault more sacred than that of the Cardinal de Rochefoucault? The former is, I doubt not, a worthy person; and (if it were not a sort of profaneness to talk of the use, as affecting the title to property) he makes a good use of his revenues; but it is no disrespect to him to say, what authentic information well warrants me in saying, that the use made of a property equally valid, by his brother,
%[101]
\footnote{ Not his brother, nor any near relation; but this mistake does not affect the argument.}
 the Cardinal Archbishop of Rouen, was far more laudable and far more public-spirited. Can one hear of the proscription of such persons, and the confiscation of their effects, without indignation, and horror? He is not a man who does not feel such emotions on such occasions. He does not deserve the name of a free man who will not express them.

Few barbarous conquerors have ever made so terrible a revolution in property. None of the heads of the Roman factions, when they established crudelem illam hastam in all their auctions of rapine, have ever set up to sale the goods of the conquered citizen to such an enormous amount. It must be allowed in favor of those tyrants of antiquity, that what was done by them could hardly be said to be done in cold blood. Their passions were inflamed, their tempers soured, their understandings confused with the spirit of revenge, with the innumerable reciprocated and recent inflictions and retaliations of blood and rapine. They were driven beyond all bounds of moderation by the apprehension of the return of power with the return of property to the families of those they had injured beyond all hope of forgiveness.

These Roman confiscators, who were yet only in the elements of tyranny, and were not instructed in the rights of men to exercise all sorts of cruelties on each other without provocation, thought it necessary to spread a sort of color over their injustice. They considered the vanquished party as composed of traitors, who had borne arms, or otherwise had acted with hostility, against the commonwealth. They regarded them as persons who had forfeited their property by their crimes. With you, in your improved state of the human mind, there was no such formality. You seized upon five millions sterling of annual rent, and turned forty or fifty thousand human creatures out of their houses, because "such was your pleasure." The tyrant Harry the Eighth of England, as he was not better enlightened than the Roman Mariuses and Syllas, and had not studied in your new schools, did not know what an effectual instrument of despotism was to be found in that grand magazine of offensive weapons, the rights of men. When he resolved to rob the abbeys, as the club of the Jacobins have robbed all the ecclesiastics, he began by setting on foot a commission to examine into the crimes and abuses which prevailed in those communities. As it might be expected, his commission reported truths, exaggerations, and falsehoods. But truly or falsely, it reported abuses and offences. However, as abuses might be corrected, as every crime of persons does not infer a forfeiture with regard to communities, and as property, in that dark age, was not discovered to be a creature of prejudice, all those abuses (and there were enough of them) were hardly thought sufficient ground for such a confiscation as it was for his purposes to make. He therefore procured the formal surrender of these estates. All these operose proceedings were adopted by one of the most decided tyrants in the rolls of history, as necessary preliminaries, before he could venture, by bribing the members of his two servile Houses with a share of the spoil, and holding out to them an eternal immunity from taxation, to demand a confirmation of his iniquitous proceedings by an act of Parliament. Had fate reserved him to our times, four technical terms would have done his business, and saved him all this trouble; he needed nothing more than one short form of incantation:—"Philosophy, Light, Liberality, the Rights of Men."

I can say nothing in praise of those acts of tyranny, which no voice has hitherto ever commended under any of their false colors; yet in these false colors an homage was paid by despotism to justice. The power which was above all fear and all remorse was not set above all shame. Whilst shame keeps its watch, virtue is not wholly extinguished in the heart, nor will moderation be utterly exiled from the minds of tyrants.

I believe every honest man sympathizes in his reflections with our political poet on that occasion, and will pray to avert the omen, whenever these acts of rapacious despotism present themselves to his view or his imagination:—

\begin{verse}
"May no such storm \\
Fall on our times, where rain must reform! \\
Tell me, my Muse, what monstrous, dire offence, \\
What crime could any Christian king incense \\
To such a rage? Was't luxury, or lust \\
Was he so temperate, so chaste, so just? \\
Were these their crimes? They were his own much more: \\
But wealth is crime enough to him that's poor."
%[102]
\footnote{ The rest of the passage is this:—

\begin{verse}
Who, having spent the treasures of his crown, \\
Condemns their luxury to feed his own. \\
And yet this act, to varnish o'er the shame \\
Of sacrilege, must bear Devotion's name. \\
No crime so bold, but would be understood \\
A Real, or at least a seeming good. \\
Who fears not to do ill, yet fears the name, \\
And free from conscience, is a slave to fame. \\
Thus he the Church at once protects and spoils: \\
But princes' swords are sharper than their styles. \\
And thus to th' ages past he makes amends, \\
Their charity destroys, their faith defends. \\
Then did Religion in a lazy cell, \\
In empty, airy contemplations, dwell; \\
And like the block, unmovèd lay: but ours, \\
As much too active, like the stork devours. \\
Is there no temperate region can be known \\
Betwixt their frigid and our torrid zone? \\
Could we not wake from that lethargic dream, \\
But to be restless in a worse extreme? \\
And for that lethargy was there no care, \\
But to be cast into a calenture? \\
Can knowledge have no bound, but must advance \\
So far, to make us wish for ignorance, \\
And rather in the dark to grope our way, \\
Than, led by a false guide, to err by day? \\
Who sees these dismal heaps, but would demand \\
What barbarous invader sack'd the land? \\
But when he hears no Goth, no Turk did bring \\
This desolation, but a Christian king, \\
When nothing but the name of zeal appears \\
'Twixt our best actions and the worst of theirs, \\
What does he think our sacrilege would spare, \\
When such th' effects of our devotions are?
\end{verse}

\hfill Cooper's Hill, by Sir JOHN DENHAM.
}

\end{verse}

This same wealth, which is at all times treason and lèze-nation to indigent and rapacious despotism, under all modes of polity, was your temptation to violate property, law, and religion, united in one object. But was the state of France so wretched and undone, that no other resource but rapine remained to preserve its existence? On this point I wish to receive some information. When the States met, was the condition of the finances of France such, that, after economizing, on principles of justice and mercy, through all departments, no fair repartition of burdens upon all the orders could possibly restore them? If such an equal imposition would have been sufficient, you well know it might easily have been made. M. Necker, in the budget which he laid before the orders assembled at Versailles, made a detailed exposition of the state of the French nation.
%[103]
\footnote{ Rapport de Mons. le Directeur-Général des Finances, fait par Ordre du Roi à Versailles. Mai 5, 1789.}


If we give credit to him, it was not necessary to have recourse to any new impositions whatsoever, to put the receipts of France on a balance with its expenses. He stated the permanent charges of all descriptions, including the interest of a new loan of four hundred millions, at 531,444,000 livres; the fixed revenue at 475,294,000: making the deficiency 56,150,000, or short of 2,200,000 l. sterling. But to balance it, he brought forward savings and improvements of revenue (considered as entirely certain) to rather more than the amount of that deficiency; and he concludes with these emphatical words (p. 39):—"Quel pays, Messieurs, que celui, où, sans impôts et avec de simples objets inaperçus, on peut faire disparoître un déficit qui a fait tant de bruit en Europe!" As to the reimbursement, the sinking of debt, and the other great objects of public credit and political arrangement indicated in Monsieur Necker's speech, no doubt could be entertained but that a very moderate and proportioned assessment on the citizens without distinction would have provided for all of them to the fullest extent of their demand.

If this representation of M. Necker was false, then the Assembly are in the highest degree culpable for having forced the king to accept as his minister, and, since the king's deposition, for having employed as their minister, a man who had been capable of abusing so notoriously the confidence of his master and their own: in a matter, too, of the highest moment, and directly appertaining to his particular office. But if the representation was exact, (as, having always, along with you, conceived a high degree of respect for M. Necker, I make no doubt it was,) then what can be said in favor of those who, instead of moderate, reasonable, and general contribution, have in cold blood, and impelled by no necessity, had recourse to a partial and cruel confiscation?

Was that contribution refused on a pretext of privilege, either on the part of the clergy, or on that of the nobility? No, certainly. As to the clergy, they even ran before the wishes of the third order. Previous to the meeting of the States, they had in all their instructions expressly directed their deputies to renounce every immunity which put them upon a footing distinct from the condition of their fellow-subjects. In this renunciation the clergy were even more explicit than the nobility.

But let us suppose that the deficiency had remained at the fifty-six millions, (or 2,200,000 l. sterling,) as at first stated by M. Necker. Let us allow that all the resources he opposed to that deficiency were impudent and groundless fictions, and that the Assembly (or their lords of articles
%[104]
\footnote{ In the Constitution of Scotland, during the Stuart reigns, a committee sat for preparing bills; and none could pass, but those previously approved by them. This committee was called Lords of Articles.}
 at the Jacobins) were from thence justified in laying the whole burden of that deficiency on the clergy,—yet allowing all this, a necessity of 2,200,000 l. sterling will not support a confiscation to the amount of five millions. The imposition of 2,200,000 l. on the clergy, as partial, would have been oppressive and unjust, but it would not have been altogether ruinous to those on whom it was imposed; and therefore it would not have answered the real purpose of the managers.

Perhaps persons unacquainted with the state of France, on hearing the clergy and the noblesse were privileged in point of taxation, may be led to imagine, that, previous to the Revolution, these bodies had contributed nothing to the state. This is a great mistake. They certainly did not contribute equally with each other, nor either of them equally with the commons. They both, however, contributed largely. Neither nobility nor clergy enjoyed any exemption from the excise on consumable commodities, from duties of custom, or from any of the other numerous indirect impositions, which in France, as well as here, make so very large a proportion of all payments to the public. The noblesse paid the capitation. They paid also a land-tax, called the twentieth penny, to the height sometimes of three, sometimes of four shillings in the pound: both of them direct impositions, of no light nature, and no trivial produce. The clergy of the provinces annexed by conquest to France (which in extent make about an eighth part of the whole, but in wealth a much larger proportion) paid likewise to the capitation and the twentieth penny, at the rate paid by the nobility. The clergy in the old provinces did not pay the capitation; but they had redeemed themselves at the expense of about twenty-four millions, or a little more than a million sterling. They were exempted from the twentieths: but then they made free gifts; they contracted debts for the state; and they were subject to some other charges, the whole computed at about a thirteenth part of their clear income. They ought to have paid annually about forty thousand pounds more, to put them on a par with the contribution of the nobility.

When the terrors of this tremendous proscription hung over the clergy, they made an offer of a contribution, through the Archbishop of Aix, which, for its extravagance, ought not to have been accepted. But it was evidently and obviously more advantageous to the public creditor than anything which could rationally be promised by the confiscation. Why was it not accepted? The reason is plain:—There was no desire that the Church should be brought to serve the State. The service of the State was made a pretext to destroy the Church. In their way to the destruction of the Church they would not scruple to destroy their country: and they have destroyed it. One great end in the project would have been defeated, if the plan of extortion had been adopted in lieu of the scheme of confiscation. The new landed interest connected with the new republic, and connected with it for its very being, could not have been created. This was among the reasons why that extravagant ransom was not accepted.

The madness of the project of confiscation, on the plan that was first pretended, soon became apparent. To bring this unwieldy mass of landed property, enlarged by the confiscation of all the vast landed domain of the crown, at once into market was obviously to defeat the profits proposed by the confiscation, by depreciating the value of those lands, and indeed of all the landed estates throughout France. Such a sudden diversion of all its circulating money from trade to land must be an additional mischief. What step was taken? Did the Assembly, on becoming sensible of the inevitable ill effects of their projected sale, revert to the offers of the clergy? No distress could oblige them to travel in a course which was disgraced by any appearance of justice. Giving over all hopes from a general immediate sale, another project seems to have succeeded. They proposed to take stock in exchange for the Church lands. In that project great difficulties arose in equalizing the objects to be exchanged. Other obstacles also presented themselves, which threw them back again upon some project of sale. The municipalities had taken an alarm. They would not hear of transferring the whole plunder of the kingdom to the stockholders in Paris. Many of those municipalities had been (upon system) reduced to the most deplorable indigence. Money was nowhere to be seen. They were therefore led to the point that was so ardently desired. They panted for a currency of any kind which might revive their perishing industry. The municipalities were, then, to be admitted to a share in the spoil, which evidently rendered the first scheme (if ever it had been seriously entertained) altogether impracticable. Public exigencies pressed upon all sides. The Minister of Finance reiterated his call for supply with, a most urgent, anxious, and boding voice. Thus pressed on all sides, instead of the first plan of converting their bankers into bishops and abbots, instead of paying the old debt, they contracted a new debt, at three per cent, creating a new paper currency, founded on an eventual sale of the Church lands. They issued this paper currency to satisfy in the first instance chiefly the demands made upon them by the bank of discount, the great machine or paper-mill of their fictitious wealth.

The spoil of the Church was now become the only resource of all their operations in finance, the vital principle of all their politics, the sole security for the existence of their power. It was necessary, by all, even the most violent means, to put every individual on the same bottom, and to bind the nation in one guilty interest to uphold this act, and the authority of those by whom it was done. In order to force the most reluctant into a participation of their pillage, they rendered their paper circulation compulsory in all payments. Those who consider the general tendency of their schemes to this one object as a centre, and a centre from which afterwards all their measures radiate, will not think that I dwell too long upon this part of the proceedings of the National Assembly.

To cut off all appearance of connection between the crown and public justice, and to bring the whole under implicit obedience to the dictators in Paris, the old independent judicature of the Parliaments, with all its merits and all its faults, was wholly abolished. Whilst the Parliaments existed, it was evident that the people might some time or other come to resort to them, and rally under the standard of their ancient laws. It became, however, a matter of consideration, that the magistrates and officers in the courts now abolished had purchased their places at a very high rate, for which, as well as for the duty they performed, they received but a very low return of interest. Simple confiscation is a boon only for the clergy: to the lawyers some appearances of equity are to be observed; and they are to receive compensation to an immense amount. Their compensation becomes part of the national debt, for the liquidation of which there is the one exhaustless fund. The lawyers are to obtain their compensation in the new Church paper, which is to march with the new principles of judicature and legislature. The dismissed magistrates are to take their share of martyrdom with the ecclesiastics, or to receive their own property from such a fund and in such a manner as all those who have been seasoned with the ancient principles of jurisprudence, and had been the sworn guardians of property, must look upon with horror. Even the clergy are to receive their miserable allowance out of the depreciated paper, which is stamped with the indelible character of sacrilege, and with the symbols of their own ruin, or they must starve. So violent an outrage upon credit, property, and liberty, as this compulsory paper currency, has seldom been exhibited by the alliance of bankruptcy and tyranny, at any time, or in any nation.

In the course of all these operations, at length comes out the grand arcanum,—that in reality, and in a fair sense, the lands of the Church (so far as anything certain can be gathered from their proceedings) are not to be sold at all. By the late resolutions of the National Assembly, they are, indeed, to be delivered to the highest bidder. But it is to be observed, that a certain portion only of the purchase-money is to be laid down. A period of twelve years is to be given for the payment of the rest. The philosophic purchasers are therefore, on payment of a sort of fine, to be put instantly into possession of the estate. It becomes in some respects a sort of gift to them,—to be held on the feudal tenure of zeal to the new establishment. This project is evidently to let in a body of purchasers without money. The consequence will be, that these purchasers, or rather grantees, will pay, not only from the rents as they accrue, which might as well be received by the state, but from the spoil of the materials of buildings, from waste in woods, and from whatever money, by hands habituated to the gripings of usury, they can wring from the miserable peasant. He is to be delivered over to the mercenary and arbitrary discretion of men who will be stimulated to every species of extortion by the growing demands on the growing profits of an estate held under the precarious settlement of a new political system.

When all the frauds, impostures, violences, rapines, burnings, murders, confiscations, compulsory paper currencies, and every description of tyranny and cruelty employed to bring about and to uphold this Revolution have their natural effect, that is, to shock the moral sentiments of all virtuous and sober minds, the abettors of this philosophic system immediately strain their throats in a declamation against the old monarchical government of France. When they have rendered that deposed power sufficiently black, they then proceed in argument, as if all those who disapprove of their new abuses must of course be partisans of the old,—that those who reprobate their crude and violent schemes of liberty ought to be treated as advocates for servitude. I admit that their necessities do compel them to this base and contemptible fraud. Nothing can reconcile men to their proceedings and projects but the supposition that there is no third option between them and some tyranny as odious as can be furnished by the records of history or by the invention of poets. This prattling of theirs hardly deserves the name of sophistry. It is nothing but plain impudence. Have these gentlemen never heard, in the whole circle of the worlds of theory and practice, of anything between the despotism of the monarch and the despotism of the multitude? Have they never heard of a monarchy directed by laws, controlled and balanced by the great hereditary wealth and hereditary dignity of a nation, and both again controlled by a judicious check from the reason and feeling of the people at large, acting by a suitable and permanent organ? Is it, then, impossible that a man may be found who, without criminal ill intention or pitiable absurdity, shall prefer such a mixed and tempered government to either of the extremes,—and who may repute that nation to be destitute of all wisdom and of all virtue, which, having in its choice to obtain such a government with ease, or rather to confirm it when actually possessed, thought proper to commit a thousand crimes, and to subject their country to a thousand evils, in order to avoid it? Is it, then, a truth so universally acknowledged, that a pure democracy is the only tolerable form into which human society can be thrown, that a man is not permitted to hesitate about its merits, without the suspicion of being a friend to tyranny, that is, of being a foe to mankind?

I do not know under what description to class the present ruling authority in France. It affects to be a pure democracy, though I think it in a direct train of becoming shortly a mischievous and ignoble oligarchy. But for the present I admit it to be a contrivance of the nature and effect of what it pretends to. I reprobate no form of government merely upon abstract principles. There may be situations in which the purely democratic form will become necessary. There may be some (very few, and very particularly circumstanced) where it would be clearly desirable. This I do not take to be the case of France, or of any other great country. Until now, we have seen no examples of considerable democracies. The ancients were better acquainted with them. Not being wholly unread in the authors who had seen the most of those constitutions, and who best understood them, I cannot help concurring with their opinion, that an absolute democracy no more than absolute monarchy is to be reckoned among the legitimate forms of government. They think it rather the corruption and degeneracy than the sound constitution of a republic. If I recollect rightly, Aristotle observes, that a democracy has many striking points of resemblance with a tyranny.
%[105]
\footnote{ When I wrote this I quoted from memory, after many years had elapsed from my reading the passage. A learned friend has found it and it is as follows:—

{\g τὸ ἠ̂θος τὸ αὐτό, καὶ ἄμφω δεσποτικὰ τω̂ν βελτιόνων, καὶ τὰ ψηφίσματα ὥσπερ ἐκει̂ τὰ ἐπιτάγματα, καὶ ὁ δημαγωγὸς καὶ ὁ κόλαξ οἱ αὐτοὶ καὶ ἀνάλογον. καὶ μάλιστα δ' ἑκάτεροι παρ' ἑκατέροις ἰσχύουσιν, οἱ μὲν κόλακες παρὰ τοι̂ς τυράννοις, οἱ δὲ δημαγωγοὶ παρὰ τοι̂ς δήμοις τοι̂ς τοιούτοις.}

"The ethical character is the same: both exercise despotism over the better class of citizens; and decrees are in the one what ordinances and arrêts are in the other: the demagogue, too, and the court favorite, are not unfrequently the same identical men, and always bear a close analogy; and these have the principal power, each in their respective forms of government, favorites with the absolute monarch, and demagogues with a people such as I have described."—Arist. Politic. lib. iv. cap. 4.
}
 Of this I am certain, that in a democracy the majority of the citizens is capable of exercising the most cruel oppressions upon the minority, whenever strong divisions prevail in that kind of polity, as they often must,—and that oppression of the minority will extend to far greater numbers, and will be carried on with much greater fury, than can almost ever be apprehended from the dominion of a single sceptre. In such a popular persecution, individual sufferers are in a much more deplorable condition than in any other. Under a cruel prince they have the balmy compassion of mankind to assuage the smart of their wounds, they have the plaudits of the people to animate their generous constancy under their sufferings: but those who are subjected to wrong under multitudes are deprived of all external consolation; they seem deserted by mankind, overpowered by a conspiracy of their whole species.

But admitting democracy not to have that inevitable tendency to party tyranny which I suppose it to have, and admitting it to possess as much good in it when unmixed as I am sure it possesses when compounded with other forms; does monarchy, on its part, contain nothing at all to recommend it? I do not often quote Bolingbroke, nor have his works in general left any permanent impression on my mind. He is a presumptuous and a superficial writer. But he has one observation which in my opinion is not without depth and solidity. He says that he prefers a monarchy to other governments, because you can better ingraft any description of republic on a monarchy than anything of monarchy upon the republican forms. I think him perfectly in the right. The fact is so historically, and it agrees well with the speculation.

I know how easy a topic it is to dwell on the faults of departed greatness. By a revolution in the state, the fawning sycophant of yesterday is converted into the austere critic of the present hour. But steady, independent minds, when they have an object of so serious a concern to mankind as government under their contemplation, will disdain to assume the part of satirists and declaimers. They will judge of human institutions as they do of human characters. They will sort out the good from the evil, which is mixed in mortal institutions as it is in mortal men.

Your government in France, though usually, and I think justly, reputed the best of the unqualified or ill-qualified monarchies, was still full of abuses. These abuses accumulated in a length of time, as they must accumulate in every monarchy not under the constant inspection of a popular representative. I am no stranger to the faults and defects of the subverted government of France; and I think I am not inclined by nature or policy to make a panegyric upon anything which is a just and natural object of censure. But the question is not now of the vices of that monarchy, but of its existence. Is it, then, true, that the French government was such as to be incapable or undeserving of reform, so that it was of absolute necessity the whole fabric should be at once pulled down, and the area cleared for the erection of a theoretic, experimental edifice in its place? All France was of a different opinion in the beginning of the year 1789. The instructions to the representatives to the States-General, from every district in that kingdom, were filled with projects for the reformation of that government, without the remotest suggestion of a design to destroy it. Had such a design been then even insinuated, I believe there would have been but one voice, and that voice for rejecting it with scorn and horror. Men have been sometimes led by degrees, sometimes hurried, into things of which, if they could have seen the whole together, they never would have permitted the most remote approach. When those instructions were given, there was no question but that abuses existed, and that they demanded a reform: nor is there now. In the interval between the instructions and the Revolution things changed their shape; and in consequence of that change, the true question at present is, whether those who would have reformed or those who have destroyed are in the right.

To hear some men speak of the late monarchy of France, you would imagine that they were talking of Persia bleeding under the ferocious sword of Thamas Kouli Khân,—or at least describing the barbarous anarchic despotism of Turkey, where the finest countries in the most genial climates in the world are wasted by peace more than any countries have been worried by war, where arts are unknown, where manufactures languish, where science is extinguished, where agriculture decays, where the human race itself melts away and perishes under the eye of the observer. Was this the case of France? I have no way of determining the question but by a reference to facts. Facts do not support this resemblance. Along with much evil, there is some good in monarchy itself; and some corrective to its evil from religion, from laws, from manners, from opinions, the French monarchy must have received, which rendered it (though by no means a free, and therefore by no means a good constitution) a despotism rather in appearance than in reality.

Among the standards upon which the effects of government on any country are to be estimated, I must consider the state of its population as not the least certain. No country in which population flourishes, and is in progressive improvement, can be under a very mischievous government. About sixty years ago, the Intendants of the Generalities of France made, with other matters, a report of the population of their several districts. I have not the books, which are very voluminous, by me, nor do I know where to procure them, (I am obliged to speak by memory, and therefore the less positively,) but I think the population of France was by them, even at that period, estimated at twenty-two millions of souls. At the end of the last century it had been generally calculated at eighteen. On either of these estimations, France was not ill-peopled. M. Necker, who is an authority for his own time at least equal to the Intendants for theirs, reckons, and upon apparently sure principles, the people of France, in the year 1780, at twenty-four millions six hundred and seventy thousand. But was this the probable ultimate term under the old establishment? Dr. Price is of opinion that the growth of population in France was by no means at its acme in that year. I certainly defer to Dr. Price's authority a good deal more in these speculations than I do in his general politics. This gentleman, taking ground on M. Necker's data, is very confident that since the period of that minister's calculation the French population has increased rapidly,—so rapidly, that in the year 1789 he will not consent to rate the people of that kingdom at a lower number than thirty millions. After abating much (and much I think ought to be abated) from the sanguine calculation of Dr. Price, I have no doubt that the population of France did increase considerably during this latter period: but supposing that it increased to nothing more than will be sufficient to complete the twenty-four millions six hundred and seventy thousand to twenty-five millions, still a population of twenty-five millions, and that in an increasing progress, on a space of about twenty-seven thousand square leagues, is immense. It is, for instance, a good deal more than the proportionable population of this island, or even than that of England, the best peopled part of the United Kingdom.

It is not universally true that France is a fertile country. Considerable tracts of it are barren, and labor under other natural disadvantages. In the portions of that territory where things are more favorable, as far as I am able to discover, the numbers of the people correspond to the indulgence of Nature.
%[106]
\footnote{ De l'Administration des Finances de la France, par Mons. Necker, Vol. I. p. 288.}
 The Generality of Lisle, (this I admit is the strongest example,) upon an extent of four hundred and four leagues and a half, about ten years ago contained seven hundred and thirty-four thousand six hundred souls, which is one thousand seven hundred and seventy-two inhabitants to each square league. The middle term for the rest of France is about nine hundred inhabitants to the same admeasurement.

I do not attribute this population to the deposed government; because I do not like to compliment the contrivances of men with what is due in a great degree to the bounty of Providence. But that decried government could not have obstructed, most probably it favored, the operation of those causes, (whatever they were,) whether of Nature in the soil, or habits of industry among the people, which has produced so large a number of the species throughout that whole kingdom, and exhibited in some particular places such prodigies of population. I never will suppose that fabric of a state to be the worst of all political institutions which by experience is found to contain a principle favorable (however latent it may be) to the increase of mankind.

The wealth of a country is another, and no contemptible standard, by which we may judge whether, on the whole, a government be protecting or destructive. France far exceeds England in the multitude of her people; but I apprehend that her comparative wealth is much inferior to ours,—that it is not so equal in the distribution, nor so ready in the circulation. I believe the difference in the form of the two governments to be amongst the causes of this advantage on the side of England: I speak of England, not of the whole British dominions,—which, if compared with those of France, will in some degree weaken the comparative rate of wealth upon our side. But that wealth, which will not endure a comparison with the riches of England, may constitute a very respectable degree of opulence. M. Necker's book, published in 1785,
%[107]
\footnote{ De l'Administration des Finances de la France, par M. Necker.}
 contains an accurate and interesting collection of facts relative to public economy and to political arithmetic; and his speculations on the subject are in general wise and liberal. In that work he gives an idea of the state of France, very remote from the portrait of a country whose government was a perfect grievance, an absolute evil, admitting no cure but through the violent and uncertain remedy of a total revolution. He affirms, that from the year 1726 to the year 1784 there was coined at the mint of France, in the species of gold and silver, to the amount of about one hundred millions of pounds sterling.
%[108]
\footnote{ Vol. III. chap. 8 and chap. 9.}


It is impossible that M. Necker should be mistaken in the amount of the bullion which has been coined in the mint. It is a matter of official record. The reasonings of this able financier concerning the quantity of gold and silver which remained for circulation, when he wrote in 1785, that is, about four years before the deposition and imprisonment of the French king, are not of equal certainty; but they are laid on grounds so apparently solid, that it is not easy to refuse a considerable degree of assent to his calculation. He calculates the numéraire, or what we call specie, then actually existing in France, at about eighty-eight millions of the same English money. A great accumulation of wealth for one country, large as that country is! M. Necker was so far from considering this influx of wealth as likely to cease, when he wrote in 1785, that he presumes upon a future annual increase of two per cent upon the money brought into France during the periods from which he computed.

Some adequate cause must have originally introduced all the money coined at its mint into that kingdom; and some cause as operative must have kept at home, or returned into its bosom, such a vast flood of treasure as M. Necker calculates to remain for domestic circulation. Suppose any reasonable deductions from M. Necker's computation, the remainder must still amount to an immense sum. Causes thus powerful to acquire and to retain cannot be found in discouraged industry, insecure property, and a positively destructive government. Indeed, when I consider the face of the kingdom of France, the multitude and opulence of her cities, the useful magnificence of her spacious high-roads and bridges, the opportunity of her artificial canals and navigations opening the conveniences of maritime communication through a solid continent of so immense an extent,—when I turn my eyes to the stupendous works of her ports and harbors, and to her whole naval apparatus, whether for war or trade,—when I bring before my view the number of her fortifications, constructed with so bold and masterly a skill, and made and maintained at so prodigious a charge, presenting an armed front and impenetrable barrier to her enemies upon every side,—when I recollect how very small a part of that extensive region is without cultivation, and to what complete perfection the culture of many of the best productions of the earth have been brought in France,—when I reflect on the excellence of her manufactures and fabrics, second to none but ours, and in some particulars not second,—when I contemplate the grand foundations of charity, public and private,—when I survey the state of all the arts that beautify and polish life,—when I reckon the men she has bled for extending her fame in war, her able statesmen, the multitude of her profound lawyers and theologians, her philosophers, her critics, her historians and antiquaries, her poets and her orators, sacred and profane,—I behold in all this something which awes and commands the imagination, which checks the mind on the brink of precipitate and indiscriminate censure, and which demands that we should very seriously examine what and how great are the latent vices that could authorize us at once to level so spacious a fabric with the ground. I do not recognize in this view of things the despotism of Turkey. Nor do I discern the character of a government that has been on the whole so oppressive, or so corrupt, or so negligent, as to be utterly unfit for all reformation. I must think such a government well deserved to have its excellences heightened, its faults corrected, and its capacities improved into a British Constitution.

Whoever has examined into the proceedings of that deposed government for several years back cannot fail to have observed, amidst the inconstancy and fluctuation natural to courts, an earnest endeavor towards the prosperity and improvement of the country; he must admit that it had long been employed, in some instances wholly to remove, in many considerably to correct, the abusive practices and usages that had prevailed in the state,—and that even the unlimited power of the sovereign over the persons of his subjects, inconsistent, as undoubtedly it was, with law and liberty, had yet been every day growing more mitigated in the exercise. So far from refusing itself to reformation, that government was open, with a censurable degree of facility, to all sorts of projects and projectors on the subject. Rather too much countenance was given to the spirit of innovation, which soon was turned against those who fostered it, and ended in their ruin. It is but cold, and no very flattering justice to that fallen monarchy, to say, that, for many years, it trespassed more by levity and want of judgment in several of its schemes than from any defect in diligence or in public spirit. To compare the government of France for the last fifteen or sixteen years with wise and well-constituted establishments during that, or during any period, is not to act with fairness. But if in point of prodigality in the expenditure of money, or in point of rigor in the exercise of power, it be compared with any of the former reigns, I believe candid judges will give little credit to the good intentions of those who dwell perpetually on the donations to favorites, or on the expenses of the court, or on the horrors of the Bastile, in the reign of Louis the Sixteenth.
%[109]
\footnote{ The world is obliged to M. de Calonne for the pains he has taken to refute the scandalous exaggerations relative to some of the royal expenses, and to detect the fallacious account given of pensions, for the wicked purpose of provoking the populace to all sorts of crimes.}


Whether the system, if it deserves such a name, now built on the ruins of that ancient monarchy, will be able to give a better account of the population and wealth of the country which it has taken under its care, is a matter very doubtful. Instead of improving by the change, I apprehend that a long series of years must be told, before it can recover in any degree the effects of this philosophic Revolution, and before the nation can be replaced on its former footing. If Dr. Price should think fit, a few years hence, to favor us with an estimate of the population of France, he will hardly be able to make up his tale of thirty millions of souls, as computed in 1789, or the Assembly's computation of twenty-six millions of that year, or even M. Necker's twenty-five millions in 1780. I hear that there are considerable emigrations from France,—and that many, quitting that voluptuous climate, and that seductive Circean liberty, have taken refuge in the frozen regions and under the British despotism of Canada.

In the present disappearance of coin, no person could think it the same country in which the present minister of the finances has been able to discover fourscore millions sterling in specie. From its general aspect one would conclude that it had been for some time past under the special direction of the learned academicians of Laputa and Balnibarbi.
%[110]
\footnote{ See Gulliver's Travels for the idea of countries governed by philosophers.}
 Already the population of Paris has so declined, that M. Necker stated to the National Assembly the provision to be made for its subsistence at a fifth less than what had formerly been found requisite.
%[111]
\footnote{ M. de Calonne states the falling off of the population of Paris as far more considerable; and it may be so, since the period of M. Necker's calculation.}
 It is said (and I have never heard it contradicted) that a hundred thousand people are out of employment in that city, though it is become the seat of the imprisoned court and National Assembly. Nothing, I am credibly informed, can exceed the shocking and disgusting spectacle of mendicancy displayed in that capital. Indeed, the votes of the National Assembly leave no doubt of the fact. They have lately appointed a standing committee of mendicancy. They are contriving at once a vigorous police on this subject, and, for the first time, the imposition of a tax to maintain the poor, for whoso present relief great sums appear on the face of the public accounts of the year.
%[112]
\footnote{
\begin{center}
\begin{tabular}{l r r r r}
Travaux de charité pour subvenir au manque & Livres. &	£ &	s. & d. \\
\hspace{0.3cm}de travail à Paris et dans les provinces   & 3,866,920 & 161,121 & 13 & 4 \\
Destruction de vagabondage et de la mendicité & 1,671,417 & 69,642 & 7 & 6 \\
Primes pour l'importation de grains	& 5,671,907 & 235,329 & 9 & 2 \\
Dépenses relatives aux subsistances, déduction  & & & & \\
\hspace{0.3cm}fait des reconvrements qui out en lieu & 39,871,790 & 1,661,324 & 11 & 8 \\
Total	& 51,082,034	& 2,128,418	& 1	& 8
\end{tabular}
\end{center}
When I sent this book to the press, I entertained some doubt concerning the nature and extent of the last article in the above accounts, which is only under a general head, without any detail. Since then I have seen M. de Calonne's work. I must think it a great loss to me that I had not that advantage earlier. M. de Calonne thinks this article to be on account of general subsistence; but as he is not able to comprehend how so great a loss as upwards of 1,661,000l. sterling could be sustained on the difference between the price and the sale of grain, he seems to attribute this enormous head of charge to secret expenses of the Revolution. I cannot say anything positively on that subject. The reader is capable of judging, by the aggregate of these immense charges, on the state and condition of France, and the system of public economy adopted in that nation. These articles of account produced no inquiry or discussion in the National Assembly.
}
 In the mean time the leaders of the legislative clubs and coffee-houses are intoxicated with admiration at their own wisdom and ability. They speak with the most sovereign contempt of the rest of the world. They toll the people, to comfort them in the rags with which they have clothed them, that they are a nation of philosophers; and sometimes, by all the arts of quackish parade, by show, tumult, and bustle, sometimes by the alarms of plots and invasions, they attempt to drown the cries of indigence, and to divert the eyes of the observer from the ruin and wretchedness of the state. A brave people will certainly prefer liberty accompanied with a virtuous poverty to a depraved and wealthy servitude. But before the price of comfort and opulence is paid, one ought to be pretty sure it is real liberty which is purchased, and that she is to be purchased at no other price. I shall always, however, consider that liberty as very equivocal in her appearance, which has not wisdom and justice for her companions, and does not lead prosperity and plenty in her train.

The advocates for this Revolution, not satisfied with exaggerating the vices of their ancient government, strike at the fame of their country itself, by painting almost all that could have attracted the attention of strangers, I mean their nobility and their clergy, as objects of horror. If this were only a libel, there had not been much in it. But it has practical consequences. Had your nobility and gentry, who formed the great body of your landed men and the whole of your military officers, resembled those of Germany, at the period when the Hanse towns were necessitated to confederate against the nobles in defence of their property,—had they been like the Orsini and Vitelli in Italy, who used to sally from their fortified dens to rob the trader and traveller,—had they been such as the Mamelukes in Egypt, or the Nayres on the coast of Malabar,—I do admit that too critical an inquiry might not be advisable into the means of freeing the world from such a nuisance. The statues of Equity and Mercy might be veiled for a moment. The tenderest minds, confounded with the dreadful exigence in which morality submits to the suspension of its own rules in favor of its own principles, might turn aside whilst fraud and violence were accomplishing the destruction of a pretended nobility, which disgraced, whilst it persecuted, human nature. The persons most abhorrent from blood and treason and arbitrary confiscation might remain silent spectators of this civil war between the vices.

But did the privileged nobility who met under the king's precept at Versailles in 1789, or their constituents, deserve to be looked on as the Nayres or Mamelukes of this age, or as the Orsini and Vitelli of ancient times? If I had then asked the question, I should have passed for a madman. What have they since done, that they were to be driven into exile, that their persons should be hunted about, mangled, and tortured, their families dispersed, their houses laid in ashes, and that their order should be abolished, and the memory of it, if possible, extinguished, by ordaining them to change the very names by which they were usually known? Read their instructions to their representatives. They breathe the spirit of liberty as warmly, and they recommend reformation as strongly, as any other order. Their privileges relative to contribution were voluntarily surrendered; as the king, from the beginning, surrendered all pretence to a right of taxation. Upon a free constitution there was but one opinion in France. The absolute monarchy was at an end. It breathed its last without a groan, without struggle, without convulsion. All the struggle, all the dissension, arose afterwards, upon the preference of a despotic democracy to a government of reciprocal control. The triumph of the victorious party was over the principles of a British Constitution.

I have observed the affectation which for many years past has prevailed in Paris, even to a degree perfectly childish, of idolizing the memory of your Henry the Fourth. If anything could put any one out of humor with that ornament to the kingly character, it would be this overdone style of insidious panegyric. The persons who have worked this engine the most busily are those who have ended their panegyrics in dethroning his successor and descendant: a man as good-natured, at the least, as Henry the Fourth; altogether as fond of his people; and who has done infinitely more to correct the ancient vices of the state than that great monarch did, or we are sure he ever meant to do. Well it is for his panegyrists that they have not him to deal with! For Henry of Navarre was a resolute, active, and politic prince. He possessed, indeed, great humanity and mildness, but an humanity and mildness that never stood in the way of his interests. He never sought to be loved without putting himself first in a condition to be feared. He used soft language with determined conduct. He asserted and maintained his authority in the gross, and distributed his acts of concession only in the detail. Ho spent the income of his prerogative nobly, but he took care not to break in upon the capital,—never abandoning for a moment any of the claims which he made under the fundamental laws, nor sparing to shed the blood of those who opposed him, often in the field, sometimes upon the scaffold. Because he knew how to make his virtues respected by the ungrateful, he has merited the praises of those whom, if they had lived in his time, he would have shut up in the Bastile, and brought to punishment along with the regicides whom he hanged after he had famished Paris into a surrender.

If these panegyrists are in earnest in their admiration of Henry the Fourth, they must remember that they cannot think more highly of him than he did of the noblesse of France,—whose virtue, honor, courage, patriotism, and loyalty were his constant theme.

But the nobility of France are degenerated since the days of Henry the Fourth.—This is possible; but it is more than I can believe to be true in any great degree. I do not pretend to know France as correctly as some others; but I have endeavored through my whole life to make myself acquainted with human nature,—otherwise I should be unfit to take even my humble part in the service of mankind. In that study I could not pass by a vast portion of our nature as it appeared modified in a country but twenty-four miles from the shore of this island. On my best observation, compared with my best inquiries, I found your nobility for the greater part composed of men of a high spirit, and of a delicate sense of honor, both with regard to themselves individually, and with regard to their whole corps, over whom they kept, beyond what is common in other countries, a censorial eye. They were tolerably well bred; very officious, humane, and hospitable; in their conversation frank and open; with a good military tone; and reasonably tinctured with literature, particularly of the authors in their own language. Many had pretensions far above this description. I speak of those who were generally met with.

As to their behavior to the inferior classes, they appeared to me to comport themselves towards them with good-nature, and with something more nearly approaching to familiarity than is generally practised with us in the intercourse between the higher and lower ranks of life. To strike any person, even in the most abject condition, was a thing in a manner unknown, and would be highly disgraceful. Instances of other ill-treatment of the humble part of the community were rare; and as to attacks made upon the property or the personal liberty of the commons, I never heard of any whatsoever from them,—nor, whilst the laws were in vigor under the ancient government, would such tyranny in subjects have been permitted. As men of landed estates, I had no fault to find with their conduct, though much to reprehend, and much to wish changed, in many of the old tenures. Where the letting of their land was by rent, I could not discover that their agreements with their farmers were oppressive; nor when they were in partnership with the farmer, as often was the case, have I heard that they had taken the lion's share. The proportions seemed not inequitable. There might be exceptions; but certainly they were exceptions only. I have no reason to believe that in these respects the landed noblesse of France were worse than the landed gentry of this country,—certainly in no respect more vexatious than the landholders, not noble, of their own nation. In cities the nobility had no manner of power; in the country very little. You know, Sir, that much of the civil government, and the police in the most essential parts, was not in the hands of that nobility which presents itself first to our consideration. The revenue, the system and collection of which were the most grievous parts of the French government, was not administered by the men of the sword; nor were they answerable for the vices of its principle, or the vexations, where any such existed, in its management.

Denying, as I am well warranted to do, that the nobility had any considerable share in the oppression of the people, in cases in which real oppression existed, I am ready to admit that they were not without considerable faults and errors. A foolish imitation of the worst part of the manners of England, which impaired their natural character, without substituting in its place what perhaps they meant to copy, has certainly rendered them worse than formerly they were. Habitual dissoluteness of manners, continued beyond the pardonable period of life, was more common amongst them than it is with us; and it reigned with the less hope of remedy, though possibly with something of less mischief, by being covered with more exterior decorum. They countenanced too much that licentious philosophy which has helped to bring on their ruin. There was another error amongst them more fatal. Those of the commons who approached to or exceeded many of the nobility in point of wealth were not fully admitted to the rank and estimation which wealth, in reason and good policy, ought to bestow in every country,—though I think not equally with that of other nobility. The two kinds of aristocracy were too punctiliously kept asunder: less so, however, than in Germany and some other nations.

This separation, as I have already taken the liberty of suggesting to you, I conceive to be one principal cause of the destruction of the old nobility. The military, particularly, was too exclusively reserved for men of family. But, after all, this was an error of opinion, which a conflicting opinion would have rectified. A permanent Assembly, in which the commons had their share of power, would soon abolish whatever was too invidious and insulting in these distinctions; and even the faults in the morals of the nobility would have been probably corrected, by the greater varieties of occupation and pursuit to which a constitution by orders would have given rise.

All this violent cry against the nobility I take to be a mere work of art. To be honored and even privileged by the laws, opinions, and inveterate usages of our country, growing out of the prejudice of ages, has nothing to provoke horror and indignation in any man. Even to be too tenacious of those privileges is not absolutely a crime. The strong struggle in every individual to preserve possession of what he has found to belong to him, and to distinguish him, is one of the securities against injustice and despotism implanted in our nature. It operates as an instinct to secure property, and to preserve communities in a settled state. What is there to shock in this? Nobility is a graceful ornament to the civil order. It is the Corinthian capital of polished society. "Omnes boni nobilitati semper favemus," was the saying of a wise and good man. It is, indeed, one sign of a liberal and benevolent mind to incline to it with some sort of partial propensity. He feels no ennobling principle in his own heart, who wishes to level all the artificial institutions which have been adopted for giving a body to opinion and permanence to fugitive esteem. It is a sour, malignant, envious disposition, without taste for the reality, or for any image or representation of virtue, that sees with joy the unmerited fall of what had long nourished in splendor and in honor. I do not like to see anything destroyed, any void produced in society, any ruin on the face of the land. It was therefore with no disappointment or dissatisfaction that my inquiries and observations did not present to me any incorrigible vices in the noblesse of France, or any abuse which could not be removed by a reform very short of abolition. Your noblesse did not deserve punishment; but to degrade is to punish.

It was with the same satisfaction I found that the result of my inquiry concerning your clergy was not dissimilar. It is no soothing news to my ears, that great bodies of men are incurably corrupt. It is not with much credulity I listen to any, when they speak evil of those whom they are going to plunder. I rather suspect that vices are feigned or exaggerated, when profit is looked for in their punishment. An enemy is a bad witness; a robber is a worse. Vices and abuses there were undoubtedly in that order, and must be. It was an old establishment, and not frequently revised. But I saw no crimes in the individuals that merited confiscation of their substance, nor those cruel insults and degradations, and that unnatural persecution, which have been substituted in the place of meliorating regulation.

If there had been any just cause for this new religions persecution, the atheistic libellers, who act as trumpeters to animate the populace to plunder, do not love anybody so much as not to dwell with complacence on the vices of the existing clergy. This they have not done. They find themselves obliged to rake into the histories of former ages (which they have ransacked with a malignant and profligate industry) for every instance of oppression and persecution which has been made by that body or in its favor, in order to justify, upon very iniquitous because very illogical principles of retaliation, their own persecutions and their own cruelties. After destroying all other genealogies and family distinctions, they invent a sort of pedigree of crimes. It is not very just to chastise men for the offences of their natural ancestors; but to take the fiction of ancestry in a corporate succession, as a ground for punishing men who have no relation to guilty acts, except in names and general descriptions, is a sort of refinement in injustice belonging to the philosophy of this enlightened age. The Assembly punishes men, many, if not most, of whom abhor the violent conduct of ecclesiastics in former times as much as their present persecutors can do, and who would be as loud and as strong in the expression of that sense, if they were not well aware of the purposes for which all this declamation is employed.

Corporate bodies are immortal for the good of the members, but not for their punishment. Nations themselves are such corporations. As well might we in England think of waging inexpiable war upon all Frenchmen for the evils which they have brought upon us in the several periods of our mutual hostilities. You might, on your part, think yourselves justified in falling upon all Englishmen on account of the unparalleled calamities brought upon the people of France by the unjust invasions of our Henrys and our Edwards. Indeed, we should be mutually justified in this exterminatory war upon each other, full as much as you are in the unprovoked persecution of your present countrymen, on account of the conduct of men of the same name in other times.

We do not draw the moral lessons we might from history. On the contrary, without care it may be used to vitiate our minds and to destroy our happiness. In history a great volume is unrolled for our instruction, drawing the materials of future wisdom from the past errors and infirmities of mankind. It may, in the perversion, serve for a magazine, furnishing offensive and defensive weapons for parties in Church and State, and supplying the means of keeping alive or reviving dissensions and animosities, and adding fuel to civil fury. History consists, for the greater part, of the miseries brought upon the world by pride, ambition, avarice, revenge, lust, sedition, hypocrisy, ungoverned zeal, and all the train of disorderly appetites, which shake the public with the same

"troublous storms that toss
The private state, and render life unsweet."
These vices are the causes of those storms. Religion, morals, laws, prerogatives, privileges, liberties, rights of men, are the pretexts. The pretexts are always found in some specious appearance of a real good. You would not secure men from tyranny and sedition by rooting out of the mind the principles to which these fraudulent pretexts apply? If you did, you would root out everything that is valuable in the human breast. As these are the pretexts, so the ordinary actors and instruments in great public evils are kings, priests, magistrates, senates, parliaments, national assemblies, judges, and captains. You would not cure the evil by resolving that there should be no more monarchs, nor ministers of state, nor of the Gospel,—no interpreters of law, no general officers, no public councils. You might change the names: the things in some shape must remain. A certain quantum of power must always exist in the community, in some hands, and under some appellation. Wise men will apply their remedies to vices, not to names,—to the causes of evil, which are permanent, not to the occasional organs by which they act, and the transitory modes in which they appear. Otherwise you will be wise historically, a fool in practice. Seldom have two ages the same fashion in their pretexts, and the same modes of mischief. Wickedness is a little more inventive. Whilst you are discussing fashion, the fashion is gone by. The very same vice assumes a new body. The spirit transmigrates; and, far from losing its principle of life by the change of its appearance, it is renovated in its new organs with the fresh vigor of a juvenile activity. It walks abroad, it continues its ravages, whilst you are gibbeting the carcass or demolishing the tomb. You are terrifying yourselves with ghosts and apparitions, whilst your house is the haunt of robbers. It is thus with all those who, attending only to the shell and husk of history, think they are waging war with intolerance, pride, and cruelty, whilst, under color of abhorring the ill principles of antiquated parties, they are authorizing and feeding the same odious vices in different factions, and perhaps in worse.

Your citizens of Paris formerly had lent themselves as the ready instruments to slaughter the followers of Calvin, at the infamous massacre of St. Bartholomew. What should we say to those who could think of retaliating on the Parisians of this day the abominations and horrors of that time? They are, indeed, brought to abhor that massacre. Ferocious as they are, it is not difficult to make them dislike it, because the politicians and fashionable teachers have no interest in giving their passions exactly the same direction. Still, however, they find it their interest to keep the same savage dispositions alive. It was but the other day that they caused this very massacre to be acted on the stage for the diversion of the descendants of those who committed it. In this tragic farce they produced the Cardinal of Lorraine in his robes of function, ordering general slaughter. Was this spectacle intended to make the Parisians abhor persecution and loathe the effusion of blood? No: it was to teach them to persecute their own pastors; it was to excite them, by raising a disgust and horror of their clergy, to an alacrity in hunting down to destruction an order which, if it ought to exist at all, ought to exist not only in safety, but in reverence. It was to stimulate their cannibal appetites (which one would think had been gorged sufficiently) by variety and seasoning,—and to quicken them to an alertness in new murders and massacres, if it should suit the purpose of the Guises of the day. An Assembly in which sat a multitude of priests and prelates was obliged to suffer this indignity at its door. The author was not sent to the galleys, nor the players to the house of correction. Not long after this exhibition, those players came forward to the Assembly to claim the rites of that very religion which they had dared to expose, and to show their prostituted faces in the senate, whilst the Archbishop of Paris, whose function was known to his people only by his prayers and benedictions, and his wealth only by alms, is forced to abandon his house, and to fly from his flock, (as from ravenous wolves,) because, truly, in the sixteenth century, the Cardinal of Lorraine was a rebel and a murderer.
%[113]
\footnote{ This is on a supposition of the truth of this story; but he was not in France at the time. One name serves as well as another.}


Such is the effect of the perversion of history by those who, for the same nefarious purposes, have perverted every other part of learning. But those who will stand upon that elevation of reason which places centuries under our eye and brings things to the true point of comparison, which obscures little names and effaces the colors of little parties, and to which nothing can ascend but the spirit and moral quality of human actions, will say to the teachers of the Palais Royal,—The Cardinal of Lorraine was the murderer of the sixteenth century; you have the glory of being the murderers in the eighteenth; and this is the only difference between you. But history in the nineteenth century, better understood and better employed, will, I trust, teach a civilized posterity to abhor the misdeeds of both these barbarous ages. It will teach future priests and magistrates not to retaliate upon the speculative and inactive atheists of future times the enormities committed by the present practical zealots and furious fanatics of that wretched error, which, in its quiescent state, is more than punished, whenever it is embraced. It will teach posterity not to make war upon either religion or philosophy for the abuse which the hypocrites of both have made of the two most valuable blessings conferred upon us by the bounty of the universal Patron, who in all things eminently favors and protects the race of man.

If your clergy, or any clergy, should show themselves vicious beyond the fair bounds allowed to human infirmity, and to those professional faults which can hardly be separated from professional virtues, though their vices never can countenance the exercise of oppression, I do admit that they would naturally have the effect of abating very much of our indignation against the tyrants who exceed measure and justice in their punishment. I can allow in clergymen, through all their divisions, some tenaciousness of their own opinion, some overflowings of zeal for its propagation, some predilection to their own state and office, some attachment to the interest of their own corps, some preference to those who Us ten with docility to their doctrines beyond those who scorn and deride them. I allow all this, because I am a man who have to deal with men, and who would not, through a violence of toleration, run into the greatest of all intolerance. I must bear with infirmities, until they fester into crimes.

Undoubtedly, the natural progress of the passions, from frailty to vice, ought to be prevented by a watchful eye and a firm hand. But is it true that the body of your clergy had passed those limits of a just allowance? Prom the general style of your late publications of all sorts, one would be led to believe that your clergy in France were a sort of monsters: an horrible composition of superstition, ignorance, sloth, fraud, avarice, and tyranny. But is this true? Is it true that the lapse of time, the cessation of conflicting interests, the woful experience of the evils resulting from party rage, have had no sort of influence gradually to meliorate their minds? Is it true that they were daily renewing invasions on the civil power, troubling the domestic quiet of their country, and rendering the operations of its government feeble and precarious? Is it true that the clergy of our times have pressed down the laity with an iron hand, and were in all places lighting up the fires of a savage persecution? Did they by every fraud endeavor to increase their estates? Did they use to exceed the due demands on estates that were their own? Or, rigidly screwing up right into wrong, did they convert a legal claim into a vexatious extortion? When not possessed of power, were they filled with the vices of those who envy it? Were they inflamed with a violent, litigious spirit of controversy? Goaded on with the ambition of intellectual sovereignty, were they ready to fly in the face of all magistracy, to fire churches, to massacre the priests of other descriptions, to pull down altars, and to make their way over the ruins of subverted governments to an empire of doctrine, sometimes flattering, sometimes forcing, the consciences of men from the jurisdiction of public institutions into a submission to their personal authority, beginning with a claim of liberty and ending with an abuse of power?

These, or some of these, were the vices objected, and not wholly without foundation, to several of the churchmen of former times, who belonged to the two great parties which then divided and distracted Europe.

If there was in France, as in other countries there visibly is, a great abatement, rather than any increase of these vices, instead of loading the present clergy with the crimes of other men and the odious character of other times, in common equity they ought to be praised, encouraged, and supported, in their departure from a spirit which disgraced their predecessors, and for having assumed a temper of mind and manners more suitable to their sacred function.

When my occasions took me into France, towards the close of the late reign, the clergy, under all their forms, engaged a considerable part of my curiosity. So far from finding (except from one set of men, not then very numerous, though very active) the complaints and discontents against that body which some publications had given me reason to expect, I perceived little or no public or private uneasiness on their account. On further examination, I found the clergy, in general, persons of moderate minds and decorous manners: I include the seculars, and the regulars of both sexes. I had not the good fortune to know a great many of the parochial clergy: but in general I received a perfectly good account of their morals, and of their attention to their duties. With some of the higher clergy I had a personal acquaintance, and of the rest in that class a very good means of information. They were almost all of them persons of noble birth. They resembled others of their own rank; and where there was any difference, it was in their favor. They were more fully educated than the military noblesse,—so as by no means to disgrace their profession by ignorance, or by want of fitness for the exercise of their authority. They seemed to me, beyond the clerical character, liberal and open, with the hearts of gentlemen and men of honor, neither insolent nor servile in their manners and conduct. They seemed to me rather a superior class,—a set of men amongst whom you would not be surprised to find a Fénelon. I saw among the clergy in Paris (many of the description are not to be met with anywhere) men of great learning and candor; and I had reason to believe that this description was not confined to Paris. What I found in other places I know was accidental, and therefore to be presumed a fair sample. I spent a few days in a provincial town, where, in the absence of the bishop, I passed my evenings with three clergymen, his vicars-general, persons who would have done honor to any church. They were all well-informed; two of them of deep, general, and extensive erudition, ancient and modern, Oriental and Western,—particularly in their own profession. They had a more extensive knowledge of our English divines than I expected; and they entered into the genius of those writers with a critical accuracy. One of these gentlemen is since dead: the Abbé Morangis. I pay this tribute without reluctance to the memory of that noble, reverend, learned, and excellent person; and I should do the same with equal cheerfulness to the merits of the others, who I believe are still living, if I did not fear to hurt those whom I am unable to serve.

Some of these ecclesiastics of rank are, by all titles, persons deserving of general respect. They are deserving of gratitude from me, and from many English. If this letter should ever come into their hands, I hope they will believe there are those of our nation who feel for their unmerited fall, and for the cruel confiscation of their fortunes, with no common sensibility. What I say of them is a testimony, as far as one feeble voice can go, which I owe to truth. Whenever the question of this unnatural persecution is concerned, I will pay it. No one shall prevent me from being just and grateful. The time is fitted for the duty; and it is particularly becoming to show our justice and gratitude, when those who have deserved well of us and of mankind are laboring under popular obloquy and the persecutions of oppressive power.

You had before your Revolution about a hundred and twenty bishops. A few of them were men of eminent sanctity, and charity without limit. When we talk of the heroic, of course we talk of rare virtue. I believe the instances of eminent depravity may be as rare amongst them as those of transcendent goodness. Examples of avarice and of licentiousness may be picked out, I do not question it, by those who delight in the investigation which leads to such discoveries. A man as old as I am will not be astonished that several, in every description, do not lead that perfect life of self-denial, with regard to wealth or to pleasure, which is wished for by all, by some expected, but by none exacted with more rigor than by those who are the most attentive to their own interests or the most indulgent to their own passions. When I was in France, I am certain that the number of vicious prelates was not great. Certain individuals among them, not distinguishable for the regularity of their lives, made some amends for their want of the severe virtues in their possession of the liberal, and wore endowed with qualities which made them useful in the Church and State. I am told, that, with few exceptions, Louis the Sixteenth had been more attentive to character, in his promotions to that rank, than his immediate predecessor; and I believe (as some spirit of reform has prevailed through the whole reign) that it may be true. But the present ruling power has shown a disposition only to plunder the Church. It has punished all prelates: which is to favor the vicious, at least in point of reputation. It has made a degrading pensionary establishment, to which no man of liberal ideas or liberal condition will destine his children. It must settle into the lowest classes of the people. As with you the inferior clergy are not numerous enough for their duties, as these duties are beyond measure minute and toilsome, as you have left no middle classes of clergy at their ease, in future nothing of science or erudition can exist in the Gallican Church. To complete the project, without the least attention to the rights of patrons, the Assembly has provided in future an elective clergy: an arrangement which will drive out of the clerical profession all men of sobriety, all who can pretend to independence in their function or their conduct,—and which will throw the whole direction of the public mind into the hands of a set of licentious, bold, crafty, factious, flattering wretches, of such condition and such habits of life as will make their contemptible pensions (in comparison of which the stipend of an exciseman is lucrative and honorable) an object of low and illiberal intrigue. Those officers whom they still call bishops are to be elected to a provision comparatively mean, through the same arts, (that is, electioneering arts,) by men of all religious tenets that are known or can be invented. The new lawgivers have not ascertained anything whatsoever concerning their qualifications, relative either to doctrine or to morals, no more than they have done with regard to the subordinate clergy; nor does it appear but that both the higher and the lower may, at their discretion, practise or preach any mode of religion or irreligion that they please. I do not yet see what the jurisdiction of bishops over their subordinates is to be, or whether they are to have any jurisdiction at all.

In short, Sir, it seems to me that this new ecclesiastical establishment is intended only to be temporary, and preparatory to the utter abolition, under any of its forms, of the Christian religion, whenever the minds of men are prepared for this last stroke against it by the accomplishment of the plan for bringing its ministers into universal contempt. They who will not believe that the philosophical fanatics who guide in these matters have long entertained such a design are utterly ignorant of their character and proceedings. These enthusiasts do not scruple to avow their opinion, that a state can subsist without any religion better than with one, and that they are able to supply the place of any good which may be in it by a project of their own,—namely, by a sort of education they have imagined, founded in a knowledge of the physical wants of men, progressively carried to an enlightened self-interest, which, when well understood, they tell us, will identify with an interest more enlarged and public. The scheme of this education has been long known. Of late they distinguish it (as they have got an entirely new nomenclature of technical terms) by the name of a Civic Education.

I hope their partisans in England (to whom I rather attribute very inconsiderate conduct than the ultimate object in this detestable design) will succeed neither in the pillage of the ecclesiastics nor in the introduction of a principle of popular election to our bishoprics and parochial cures. This, in the present condition of the world, would be the last corruption of the Church, the utter ruin of the clerical character, the most dangerous shock that the state ever received through a misunderstood arrangement of religion. I know well enough that the bishoprics and cures, under kingly and seigniorial patronage, as now they are in England, and as they have been lately in France, are sometimes acquired by unworthy methods; but the other mode of ecclesiastical canvass subjects them infinitely more surely and more generally to all the evil arts of low ambition, which, operating on and through greater numbers, will produce mischief in proportion.

Those of you who have robbed the clergy think that they shall easily reconcile their conduct to all Protestant nations, because the clergy whom they have thus plundered, degraded, and given over to mockery and scorn, are of the Roman Catholic, that is, of their own pretended persuasion. I have no doubt that some miserable bigots will be found here as well as elsewhere, who hate sects and parties different from their own more than they love the substance of religion, and who are more angry with those who differ from them in their particular plans and systems than displeased with those who attack the foundation of our common hope. These men will write and speak on the subject in the manner that is to be expected from their temper and character. Burnet says, that, when he was in France, in the year 1683, "the method which carried over the men of the finest parts to Popery was this: they brought themselves to doubt of the whole Christian religion: when that was once done, it seemed a more indifferent thing of what side or form they continued outwardly." If this was then the ecclesiastic policy of France, it is what they have since but too much reason to repent of. They preferred atheism to a form of religion not agreeable to their ideas. They succeeded in destroying that form; and atheism has succeeded in destroying them. I can readily give credit to Burnet's story; because I have observed too much of a similar spirit (for a little of it is "much too much") amongst ourselves. The humor, however, is not general.

The teachers who reformed our religion in England bore no sort of resemblance to your present reforming doctors in Paris. Perhaps they were (like those whom they opposed) rather more than could be wished under the influence of a party spirit; but they were most sincere believers; men of the most fervent and exalted piety; ready to die (as some of them did die) like true heroes in defence of their particular ideas of Christianity,—as they would with equal fortitude, and more cheerfully, for that stock of general truth for the branches of which they contended with their blood. These men would have disavowed with horror those wretches who claimed a fellowship with them upon no other titles than those of their having pillaged the persons with whom they maintained controversies, and their having despised the common religion, for the purity of which they exerted themselves with a zeal which unequivocally bespoke their highest reverence for the substance of that system which they wished to reform. Many of their descendants have retained the same zeal, but (as less engaged in conflict) with more moderation. They do not forget that justice and mercy are substantial parts of religion. Impious men do not recommend themselves to their communion by iniquity and cruelty towards any description of their fellow-creatures.

We hear these new teachers continually boasting of their spirit of toleration. That those persons should tolerate all opinions, who think none to be of estimation, is a matter of small merit. Equal neglect is not impartial kindness. The species of benevolence which arises from contempt is no true charity. There are in England abundance of men who tolerate in the true spirit of toleration. They think the dogmas of religion, though in different degrees, are all of moment, and that amongst them there is, as amongst all things of value, a just ground of preference. They favor, therefore, and they tolerate. They tolerate, not because they despise opinions, but because they respect justice. They would reverently and affectionately protect all religions, because they love and venerate the great principle upon which they all agree, and the great object to which they are all directed. They begin more and more plainly to discern that we have all a common cause, as against a common enemy. They will not be so misled by the spirit of faction as not to distinguish what is done in favor of their subdivision from those acts of hostility which, through some particular description, are aimed at the whole corps in which they themselves, under another denomination, are included. It is impossible for me to say what may be the character of every description of men amongst us. But I speak for the greater part; and for them, I must tell you, that sacrilege is no part of their doctrine of good works; that, so far from calling you into their fellowship on such title, if your professors are admitted to their communion, they must carefully conceal their doctrine of the lawfulness of the proscription of innocent men, and that they must make restitution of all stolen goods whatsoever. Till then they are none of ours.

You may suppose that we do not approve your confiscation of the revenues of bishops, and deans, and chapters, and parochial clergy possessing independent estates arising from land, because we have the same sort of establishment in England. That objection, you will say, cannot hold as to the confiscation of the goods of monks and nuns, and the abolition of their order. It is true that this particular part of your general confiscation does not affect England, as a precedent in point; but the reason applies, and it goes a great way. The Long Parliament confiscated the lands of deans and chapters in England on the same ideas upon which your Assembly set to sale the lands of the monastic orders. But it is in the principle of injustice that the danger lies, and not in the description of persons on whom it is first exercised. I see, in a country very near us, a course of policy pursued, which sets justice, the common concern of mankind, at defiance. With the National Assembly of France possession is nothing, law and usage are nothing. I see the National Assembly openly reprobate the doctrine of prescription, which one of the greatest of their own lawyers
%[114]
\footnote{ Domat.}
 tells us, with great truth, is a part of the law of Nature. He tells us that the positive ascertainment of its limits, and its security from invasion, were among the causes for which civil society itself has been instituted. If prescription be once shaken, no species of property is secure, when it once becomes an object large enough to tempt the cupidity of indigent power. I see a practice perfectly correspondent to their contempt of this great fundamental part of natural law. I see the confiscators begin with bishops, and chapters, and monasteries; but I do not see them end there. I see the princes of the blood, who, by the oldest usages of that kingdom, held large landed estates, (hardly with the compliment of a debate,) deprived of their possessions, and, in lieu of their stable, independent property, reduced to the hope of some precarious charitable pension at the pleasure of an Assembly, which of course will pay little regard to the rights of pensioners at pleasure, when it despises those of legal proprietors. Flushed with the insolence of their first inglorious victories, and pressed by the distresses caused by their lust of unhallowed lucre, disappointed, but not discouraged, they have at length ventured completely to subvert all property of all descriptions throughout the extent of a great kingdom. They have compelled all men, in all transactions of commerce, in the disposal of lands, in civil dealing, and through the whole communion of life, to accept, as perfect payment and good and lawful tender, the symbols of their speculations on a projected sale of their plunder. What vestiges of liberty or property have they left? The tenant-right of a cabbage-garden, a year's interest in a hovel, the good-will of an ale-house or a baker's shop, the very shadow of a constructive property, are more ceremoniously treated in our Parliament than with you the oldest and most valuable landed possessions, in the hands of the most respectable personages, or than the whole body of the moneyed and commercial interest of your country. We entertain a high opinion of the legislative authority; but we have never dreamt that Parliaments had any right whatever to violate property, to overrule prescription, or to force a currency of their own fiction in the place of that which is real, and recognized by the law of nations. But you, who began with refusing to submit to the most moderate restraints, have ended by establishing an unheard-of despotism. I find the ground upon which your confiscators go is this: that, indeed, their proceedings could not be supported in a court of justice, but that the rules of prescription cannot bind a legislative assembly.
%[115]
\footnote{ Speech of M. Camus, published by order of the National Assembly.}
 So that this legislative assembly of a free nation sits, not for the security, but for the destruction of property,—and not of property only, but of every rule and maxim which can give it stability, and of those instruments which can alone give it circulation.

When the Anabaptists of Munster, in the sixteenth century, had filled Germany with confusion, by their system of levelling, and their wild opinions concerning property, to what country in Europe did not the progress of their fury furnish just cause of alarm? Of all things, wisdom is the most terrified with epidemical fanaticism, because of all enemies it is that against which she is the least able to furnish any kind of resource. We cannot be ignorant of the spirit of atheistical fanaticism, that is inspired by a multitude of writings dispersed with incredible assiduity and expense, and by sermons delivered in all the streets and places of public resort in Paris. These writings and sermons have filled the populace with a black and savage atrocity of mind, which supersedes in them the common feelings of Nature, as well as all sentiments of morality and religion; insomuch that these wretches are induced to bear with a sullen patience the intolerable distresses brought upon them by the violent convulsions and permutations that have been made in property.
%[116]
\footnote{ Whether the following description is strictly true I know not; but it is what the publishers would have pass for true, in order to animate others. In a letter from Toul, given in one of their papers, is the following passage concerning the people of that district:—"Dans la Révolution actuelle, ils ont résisté à toutes les séductions du bigotisme, aux persécutions et aux tracasseries des ennemis de la Révolution. Oubliant leurs plus grands intérêts pour rendre hommage aux vues d'ordre général qui out déterminé l'Assemblée Nationale, ils voient, sans se plaindre, supprimer cette foule d'établissemens ecclésiastiques par lesquels ils subsistoient; et même, en perdant leur siège épiscopal, la seule de toutes ces ressources qui pouvoit, on plutôt qui devoit, en toute équité, leur être conservée, condamnés à la plus effrayante misère sans avoir été ni pu être entendus, ils ne murmurent point, ils restent fidèles aux principes du plus pur patriotisme; ils sont encore prêts à verser leur sang pour le maintien de la constitution, qui va réduire leur ville à la plus déplorable nullité."—These people are not supposed to have endured those sufferings and injustices in a struggle for liberty, for the same account states truly that they have been always free; their patience in beggary and ruin, and their suffering, without remonstrance, the most flagrant and confessed injustice, if strictly true, can be nothing but the effect of this dire fanaticism. A great multitude all over France is in the same condition and the same temper.}
 The spirit of proselytism attends this spirit of fanaticism. They have societies to cabal and correspond at home and abroad for the propagation of their tenets. The republic of Berne, one of the happiest, the most prosperous, and the best-governed countries upon earth, is one of the great objects at the destruction of which they aim. I am told they have in some measure succeeded in sowing there the seeds of discontent. They are busy throughout Germany. Spain and Italy have not been untried. England is not left out of the comprehensive scheme of their malignant charity: and in England we find those who stretch out their arms to them, who recommend their example from more than one pulpit, and who choose, in more than one periodical meeting, publicly to correspond with them, to applaud them, and to hold them up as objects for imitation; who receive from them tokens of confraternity, and standards consecrated amidst their rites and mysteries;
%[117]
\footnote{ See the proceedings of the confederation at Nantes.}
 who suggest to them leagues of perpetual amity, at the very time when the power to which our Constitution has exclusively delegated the federative capacity of this kingdom may find it expedient to make war upon them.

It is not the confiscation of our Church property from this example in France that I dread, though I think this would be no trifling evil. The great source of my solicitude is, lest it should ever be considered in England as the policy of a state to seek a resource in confiscations of any kind, or that any one description of citizens should be brought to regard any of the others as their proper prey.
%[118]
\footnote{ "Si plures sunt ii quibus improbe datum est, quam illi quibus injuste ademptum est, idcirco plus etiam valent? Non enim numero hæc judicantur, sed pondere. Quam autem habet æquitatem, ut agrum multis annis, aut etiam sæculis ante possessum, qui nullum habuit habeat, qui autem habuit amittat? Ac, propter hoc injuriæ genus, Lacedæmonii Lysandrum Ephorum expulerunt; Agin regem (quod nunquam antea apud eos acciderat) necaverunt; exque eo tempore tantæ discordiæ secutæ sunt, ut et tyranni exsisterent, et optimates exterminarentur, et preclarissime constituta respublica dilaberetur. Nec vero solum ipsa cecidit, sed etiam reliquam Græciam evertit contagionibus malorum, quæ a Lacedæmoniis profectæ manarunt latius."—After speaking of the conduct of the model of true patriots, Aratus of Sicyon, which was in a very different spirit, he says,—"Sic par est agere cum civibus; non (ut bis jam vidimus) hastam in foro ponere et bona civium voci subjicere præconis. At ille Græcus (id quod fuit sapientis et præstantis viri) omnibus consulendum esse putavit: eaque est summa ratio et sapientia boni civis, commoda civium non divellere, sed omnes eadem æquitate continere."—Cic. Off. 1. 2.}
 Nations are wading deeper and deeper into an ocean of boundless debt. Public debts, which at first were a security to governments, by interesting many in the public tranquillity, are likely in their excess to become the means of their subversion. If governments provide for these debts by heavy impositions, they perish by becoming odious to the people. If they do not provide for them, they will be undone by the efforts of the most dangerous of all parties: I mean an extensive, discontented moneyed interest, injured and not destroyed. The men who compose this interest look for their security, in the first instance, to the fidelity of government; in the second, to its power. If they find the old governments effete, worn out, and with their springs relaxed, so as not to be of sufficient vigor for their purposes, they may seek new ones that shall be possessed of more energy; and this energy will be derived, not from an acquisition of resources, but from a contempt of justice. Revolutions are favorable to confiscation; and it is impossible to know under what obnoxious names the next confiscations will be authorized. I am sure that the principles predominant in France extend to very many persons, and descriptions of persons, in all countries, who think their innoxious indolence their security. This kind of innocence in proprietors may be argued into inutility; and inutility into an unfitness for their estates. Many parts of Europe are in open disorder. In many others there is a hollow murmuring under ground; a confused movement is felt, that threatens a general earthquake in the political world. Already confederacies and correspondences of the most extraordinary nature are forming in several countries.
%[119]
\footnote{ See two books entitled, "Einige Originalschriften des Illuminatenordens,"—"System und Folgen des Illuminatenordens." München, 1787.}
 In such a state of things we ought to hold ourselves upon our guard. In all mutations (if mutations must be) the circumstance which will serve most to blunt the edge of their mischief, and to promote what good may be in them, is, that they should find us with our minds tenacious of justice and tender of property.

But it will be argued, that this confiscation in France ought not to alarm other nations. They say it is not made from wanton rapacity; that it is a great measure of national policy, adopted to remove an extensive, inveterate, superstitious mischief.—It is with the greatest difficulty that I am able to separate policy from justice. Justice is itself the great standing policy of civil society; and any eminent departure from it, under any circumstances, lies under the suspicion of being no policy at all.

When men are encouraged to go into a certain mode of life by the existing laws, and protected in that mode as in a lawful occupation,—when they have accommodated all their ideas and all their habits to it,—when the law had long made their adherence to its rules a ground of reputation, and their departure from them a ground of disgrace and even of penalty,—I am sure it is unjust in legislature, by an arbitrary act, to offer a sudden violence to their minds and their feelings, forcibly to degrade them from their state and condition, and to stigmatize with shame and infamy that character and those customs which before had been made the measure of their happiness and honor. If to this be added an expulsion from their habitations and a confiscation of all their goods, I am not sagacious enough to discover how this despotic sport made of the feelings, consciences, prejudices, and properties of men can be discriminated from the rankest tyranny.

If the injustice of the course pursued in France be clear, the policy of the measure, that is, the public benefit to be expected from it, ought to be at least as evident, and at least as important. To a man who acts under the influence of no passion, who has nothing in view in his projects but the public good, a great difference will immediately strike him, between what policy would dictate on the original introduction of such institutions, and on a question of their total abolition, where they have cast their roots wide and deep, and where, by long habit, things more valuable than themselves are so adapted to them, and in a manner interwoven with them, that the one cannot be destroyed without notably impairing the other. He might be embarrassed, if the case were really such as sophisters represent it in their paltry style of debating. But in this, as in most questions of state, there is a middle. There is something else than the mere alternative of absolute destruction or unreformed existence. Spartam nactus es; hanc exorna. This is, in my opinion, a rule of profound sense, and ought never to depart from the mind of an honest reformer. I cannot conceive how any man can have brought himself to that pitch of presumption, to consider his country as nothing but carte blanche, upon which he may scribble whatever he pleases. A man full of warm, speculative benevolence may wish his society otherwise constituted than he finds it; but a good patriot, and a true politician, always considers how he shall make the most of the existing materials of his country. A disposition to preserve, and an ability to improve, taken together, would be my standard of a statesman. Everything else is vulgar in the conception, perilous in the execution.

There are moments in the fortune of states, when particular men are called to make improvements by great mental exertion. In those moments, even when they seem to enjoy the confidence of their prince and country, and to be invested with full authority, they have not always apt instruments. A politician, to do great things, looks for a power, what our workmen call a purchase; and if he finds that power, in politics as in mechanics, he cannot be at a loss to apply it. In the monastic institutions, in my opinion, was found a great power for the mechanism of politic benevolence. There were revenues with a public direction; there were men wholly set apart and dedicated to public purposes, without any other than public ties and public principles,—men without the possibility of converting the estate of the community into a private fortune,—men denied to self-interests, whose avarice is for some community,—men to whom personal poverty is honor, and implicit obedience stands in the place of freedom. In vain shall a man look to the possibility of making such things when he wants them. The winds blow as they list. These institutions are the products of enthusiasm; they are the instruments of wisdom. Wisdom cannot create materials; they are the gifts of Nature or of chance; her pride is in the use. The perennial existence of bodies corporate and their fortunes are things particularly suited to a man who has long views,—who meditates designs that require time in fashioning, and which propose duration when they are accomplished. He is not deserving to rank high, or even to be mentioned in the order of great statesmen, who, having obtained the command and direction of such a power as existed in the wealth, the discipline, and the habits of such corporations as those which you have rashly destroyed, cannot find any way of converting it to the great and lasting benefit of his country. On the view of this subject, a thousand uses suggest themselves to a contriving mind. To destroy any power growing wild from the rank productive force of the human mind is almost tantamount, in the moral world, to the destruction of the apparently active properties of bodies in the material. It would be like the attempt to destroy (if it were in our competence to destroy) the expansive force of fixed air in nitre, or the power of steam, or of electricity, or of magnetism. These energies always existed in Nature, and they were always discernible. They seemed, some of them unserviceable, some noxious, some no better than a sport to children,—until contemplative ability, combining with practic skill, tamed their wild nature, subdued them to use, and rendered them at once the most powerful and the most tractable agents, in subservience to the great views and designs of men. Did fifty thousand persons, whose mental and whose bodily labor you might direct, and so many hundred thousand a year of a revenue, which was neither lazy nor superstitious, appear too big for your abilities to wield? Had you no way of using the men, but by converting monks into pensioners? Had you no way of turning the revenue to account, but through the improvident resource of a spendthrift sale? If you were thus destitute of mental funds, the proceeding is in its natural course. Your politicians do not understand their trade; and therefore they sell their tools.

But the institutions savor of superstition in their very principle; and they nourish it by a permanent and standing influence.—This I do not mean to dispute; but this ought not to hinder you from deriving from superstition itself any resources which may thence be furnished for the public advantage. You derive benefits from many dispositions and many passions of the human mind which are of as doubtful a color, in the moral eye, as superstition itself. It was your business to correct and mitigate everything which was noxious in this passion, as in all the passions. But is superstition the greatest of all possible vices? In its possible excess I think it becomes a very great evil. It is, however, a moral subject, and of course admits of all degrees and all modifications. Superstition is the religion of feeble minds; and they must be tolerated in an intermixture of it, in some trifling or some enthusiastic shape or other, else you will deprive weak minds of a resource found necessary to the strongest. The body of all true religion consists, to be sure, in obedience to the will of the Sovereign of the world, in a confidence in His declarations, and in imitation of His perfections. The rest is our own. It may be prejudicial to the great end,—it may be auxiliary. Wise men, who, as such, are not admirers, (not admirers at least of the munera terræ,) are not violently attached to these things, nor do they violently hate them. Wisdom is not the most severe corrector of folly. They are the rival follies which mutually wage so unrelenting a war, and which make so cruel a use of their advantages, as they can happen to engage the immoderate vulgar, on the one side or the other, in their quarrels. Prudence would be neuter; but if, in the contention between fond attachment and fierce antipathy concerning things in their nature not made to produce such heats, a prudent man were obliged to make a choice of what errors and excesses of enthusiasm he would condemn or bear, perhaps he would think the superstition which builds to be more tolerable than that which demolishes,—that which adorns a country, than that which deforms it,—that which endows, than that which plunders,—that which disposes to mistaken beneficence, than that which stimulates to real injustice,—that which leads a man to refuse to himself lawful pleasures, than that which snatches from others the scanty subsistence of their self-denial. Such, I think, is very nearly the state of the question between the ancient founders of monkish superstition and the superstition of the pretended philosophers of the hour.

For the present I postpone all consideration of the supposed public profit of the sale, which, however, I conceive to be perfectly delusive. I shall here only consider it as a transfer of property. On the policy of that transfer I shall trouble you with a few thoughts.

In every prosperous community something more is produced than goes to the immediate support of the producer. This surplus forms the income of the landed capitalist. It will be spent by a proprietor who does not labor. But this idleness is itself the spring of labor, this repose the spur to industry. The only concern for the state is, that the capital taken in rent from the land should be returned again to the industry from whence it came, and that its expenditure should be with the least possible detriment to the morals of those who expend it and to those of the people to whom it is returned.

In all the views of receipt, expenditure, and personal employment, a sober legislator would carefully compare the possessor whom he was recommended to expel with the stranger who was proposed to fill his place. Before the inconveniences are incurred which must attend all violent revolutions in property through extensive confiscation, we ought to have some rational assurance that the purchasers of the confiscated property will be in a considerable degree more laborious, more virtuous, more sober, less disposed to extort an unreasonable proportion of the gains of the laborer, or to consume on themselves a larger share than is fit for the measure of an individual,—or that they should be qualified to dispense the surplus in a more steady and equal mode, so as to answer the purposes of a politic expenditure, than the old possessors, call those possessors bishops, or canons, or commendatory abbots, or monks, or what you please. The monks are lazy. Be it so. Suppose them no otherwise employed than by singing in the choir. They are as usefully employed as those who neither sing nor say,—as usefully even as those who sing upon the stage. They are as usefully employed as if they worked from dawn to dark in the innumerable servile, degrading, unseemly, unmanly, and often most unwholesome and pestiferous occupations to which by the social economy so many wretches are inevitably doomed. If it were not generally pernicious to disturb the natural course of things, and to impede in any degree the great wheel of circulation which is turned by the strangely directed labor of these unhappy people, I should be infinitely more inclined forcibly to rescue them from their miserable industry than violently to disturb the tranquil repose of monastic quietude. Humanity, and perhaps policy, might better justify me in the one than in the other. It is a subject on which I have often reflected, and never reflected without feeling from it. I am sure that no consideration, except the necessity of submitting to the yoke of luxury and the despotism of fancy, who in their own imperious way will distribute the surplus product of the soil, can justify the toleration of such trades and employments in a well-regulated state. But for this purpose of distribution, it seems to me that the idle expenses of monks are quite as well directed as the idle expenses of us lay loiterers.

When the advantages of the possession and of the project are on a par, there is no motive for a change. But in the present case, perhaps, they are not upon a par, and the difference is in favor of the possession. It does not appear to me that the expenses of those whom you are going to expel do in fact take a course so directly and so generally leading to vitiate and degrade and render miserable those through whom they pass as the expenses of those favorites whom you are intruding into their houses. Why should the expenditure of a great landed property, which is a dispersion of the surplus product of the soil, appear intolerable to you or to me, when it takes its course through the accumulation of vast libraries, which are the history of the force and weakness of the human mind,—through great collections of ancient records, medals, and coins, which attest and explain laws and customs,—through paintings and statues, that, by imitating Nature, seem to extend the limits of creation,—through grand monuments of the dead, which continue the regards and connections of life beyond the grave,—through collections of the specimens of Nature, which become a representative assembly of all the classes and families of the world, that by disposition facilitate, and by exciting curiosity open, the avenues to science? If by great permanent establishments all these objects of expense are better secured from the inconstant sport of personal caprice and personal extravagance, are they worse than if the same tastes prevailed in scattered individuals? Does not the sweat of the mason and carpenter, who toil in order to partake the sweat of the peasant, flow as pleasantly and as salubriously in the construction and repair of the majestic edifices of religion as in the painted booths and sordid sties of vice and luxury? as honorably and as profitably in repairing those sacred works which grow hoary with innumerable years as on the momentary receptacles of transient voluptuousness,—in opera-houses, and brothels, and gaming-houses, and club-houses, and obelisks in the Champ de Mars? Is the surplus product of the olive and the vine worse employed in the frugal sustenance of persons whom the fictions of a pious imagination raise to dignity by construing in the service of God than in pampering the innumerable multitude of those who are degraded by being made useless domestics, subservient to the pride of man? Are the decorations of temples an expenditure less worthy a wise man than ribbons, and laces, and national cockades, and petit maisons, and petit soupers, and all the innumerable fopperies and follies in which opulence sports away the burden of its superfluity?

We tolerate even these,—not from love of them, but for fear of worse. We tolerate them, because property and liberty, to a degree, require that toleration. But why proscribe the other, and surely, in every point of view, the more laudable use of estates? Why, through the violation of all property, through an outrage upon every principle of liberty, forcibly carry them from the better to the worse?

This comparison between the new individuals and the old corps is made upon a supposition that no reform could be made in the latter. But, in a question of reformation, I always consider corporate bodies, whether sole or consisting of many, to be much more susceptible of a public direction, by the power of the state, in the use of their property, and in the regulation of modes and habits of life in their members, than private citizens ever can be, or perhaps ought to be; and this seems to me a very material consideration for those who undertake anything which merits the name of a politic enterprise.—So far as to the estates of monasteries.

With regard to the estates possessed by bishops and canons and commendatory abbots, I cannot find out for what reason some landed estates may not be held otherwise than by inheritance. Can any philosophic spoiler undertake to demonstrate the positive or the comparative evil of having a certain, and that, too, a large, portion of landed property passing in succession through persons whose title to it is, always in theory and often in fact, an eminent degree of piety, morals, and learning; a property which by its destination, in their turn, and on the score of merit, gives to the noblest families renovation and support, to the lowest the means of dignity and elevation; a property, the tenure of which is the performance of some duty, (whatever value you may choose to set upon that duty,) and the character of whose proprietors demands at least an exterior decorum and gravity of manners,—who are to exercise a generous, but temperate hospitality,—part of whose income they are to consider as a trust for charity,—and who, even when they fail in their trust, when they slide from their character, and degenerate into a mere common secular nobleman or gentleman, are in no respect worse than those who may succeed them in their forfeited possessions? Is it better that estates should be held by those who have no duty than by those who have one? by those whose character and destination point to virtues than by those who have no rule and direction in the expenditure of their estates but their own will and appetite? Nor are these estates held altogether in the character or with the evils supposed inherent in mortmain. They pass from hand to hand with a more rapid circulation than any other. No excess is good, and therefore too great a proportion of landed property may be held officially for life; but it does not seem to me of material injury to any common wealth that there should exist some estates that have a chance of being acquired by other means than the previous acquisition of money.

This letter is grown to a great length, though it is, indeed, short with regard to the infinite extent of the subject. Various avocations have from time to time called my mind from the subject. I was not sorry to give myself leisure to observe whether in the proceedings of the National Assembly I might not find reasons to change or to qualify some of my first sentiments. Everything has confirmed me more strongly in my first opinions. It was my original purpose to take a view of the principles of the National Assembly with regard to the great and fundamental establishments, and to compare the whole of what you have substituted in the place of what you have destroyed with the several members of our British Constitution. But this plan is of greater extent than at first I computed, and I find that you have little desire to take the advantage of any examples. At present I must content myself with some remarks upon your establishments, reserving for another time what I proposed to say concerning the spirit of our British monarchy, aristocracy, and democracy, as practically they exist.

I have taken a view of what has been done by the governing power in France. I have certainly spoke of it with freedom. Those whose principle it is to despise the ancient, permanent sense of mankind, and to set up a scheme of society on new principles, must naturally expect that such of us who think better of the judgment of the human race than, of theirs should consider both them and their devices as men and schemes upon their trial. They must take it for granted that we attend much to their reason, but not at all to their authority. They have not one of the great influencing prejudices of mankind in their favor. They avow their hostility to opinion. Of course they must expect no support from that influence, which, with every other authority, they have deposed from the seat of its jurisdiction.

I can never consider this Assembly as anything else than a voluntary association of men who have availed themselves of circumstances to seize upon the power of the state. They have not the sanction and authority of the character under which they first met. They have assumed another of a very different nature, and have completely altered and inverted all the relations in which they originally stood. They do not hold the authority they exercise under any constitutional law of the state. They have departed from the instructions of the people by whom they were sent; which instructions, as the Assembly did not act in virtue of any ancient usage or settled law, were the sole source of their authority. The most considerable of their acts have not been done by great majorities; and in this sort of near divisions, which carry only the constructive authority of the whole, strangers will consider reasons as well as resolutions.

If they had set up this new, experimental government as a necessary substitute for an expelled tyranny, mankind would anticipate the time of prescription, which through long usage mellows into legality governments that were violent in their commencement. All those who have affections which lead them to the conservation of civil order would recognize, even in its cradle, the child as legitimate, which has been produced from those principles of cogent expediency to which all just governments owe their birth, and on which they justify their continuance. But they will be late and reluctant in giving any sort of countenance to the operations of a power which has derived its birth from no law and no necessity, but which, on the contrary, has had its origin in those vices and sinister practices by which the social union is often disturbed and sometimes destroyed. This Assembly has hardly a year's prescription. We have their own word for it that they have made a revolution. To make a revolution is a measure which, primâ fronte, requires an apology. To make a revolution is to subvert the ancient state of our country; and no common reasons are called for to justify so violent a proceeding. The sense of mankind authorizes us to examine into the mode of acquiring new power, and to criticize on the use that is made of it, with less awe and reverence than that which is usually conceded to a settled and recognized authority.

In obtaining and securing their power, the Assembly proceeds upon principles the most opposite from those which appear to direct them in the use of it. An observation on this difference will let us into the true spirit of their conduct. Everything which they have done, or continue to do, in order to obtain and keep their power, is by the most common arts. They proceed exactly as their ancestors of ambition have done before them. Trace them through all their artifices, frauds, and violences, you can find nothing at all that is new. They follow precedents and examples with the punctilious exactness of a pleader. They never depart an iota from the authentic formulas of tyranny and usurpation. But in all the regulations relative to the public good the spirit has been the very reverse of this. There they commit the whole to the mercy of untried speculations; they abandon the dearest interests of the public to those loose theories to which none of them would choose to trust the slightest of his private concerns. They make this difference, because in their desire of obtaining and securing power they are thoroughly in earnest; there they travel in the beaten road. The public interests, because about them they have no real solicitude, they abandon wholly to chance: I say to chance, because their schemes have nothing in experience to prove their tendency beneficial.

We must always see with a pity not unmixed with respect the errors of those who are timid and doubtful of themselves with regard to points wherein the happiness of mankind is concerned. But in these gentlemen there is nothing of the tender parental solicitude which fears to cut up the infant for the sake of an experiment. In the vastness of their promises and the confidence of their predictions they far outdo all the boasting of empirics. The arrogance of their pretensions in a manner provokes and challenges us to an inquiry into their foundation.

I am convinced that there are men of considerable parts among the popular leaders in the National Assembly. Some of them display eloquence in their speeches and their writings. This cannot be without powerful and cultivated talents. But eloquence may exist without a proportionable degree of wisdom. When I speak of ability, I am obliged to distinguish. What they have done towards the support of their system bespeaks no ordinary men. In the system itself, taken as the scheme of a republic constructed for procuring the prosperity and security of the citizen, and for promoting the strength and grandeur of the state, I confess myself unable to find out anything which displays, in a single instance, the work of a comprehensive and disposing mind, or even the provisions of a vulgar prudence. Their purpose everywhere seems to have been to evade and slip aside from difficulty. This it has been the glory of the great masters in all the arts to confront, and to overcome,—and when they had overcome the first difficulty, to turn it into an instrument for new conquests over new difficulties: thus to enable them to extend the empire of their science, and even to push forward, beyond the reach of their original thoughts, the landmarks of the human understanding itself. Difficulty is a severe instructor, set over us by the supreme ordinance of a parental Guardian and Legislator, who knows us better than we know ourselves, as He loves us better too. Pater ipse colendi haud facilem esse viam voluit. He that wrestles with us strengthens our nerves and sharpens our skill. Our antagonist is our helper. This amicable conflict with difficulty obliges us to an intimate acquaintance with our object, and compels us to consider it in all its relations. It will not suffer us to be superficial. It is the want of nerves of understanding for such a task, it is the degenerate fondness for tricking short-outs and little fallacious facilities, that has in so many parts of the world created governments with arbitrary powers. They have created the late arbitrary monarchy of France. They have created the arbitrary republic of Paris. With them defects in wisdom are to be supplied by the plenitude of force. They get nothing by it. Commencing their labors on a principle of sloth, they have the common fortune of slothful men. The difficulties, which they rather had eluded than escaped, meet them again in their course; they multiply and thicken on them; they are involved, through a labyrinth of confused detail, in an industry without limit and without direction; and in conclusion, the whole of their work becomes feeble, vicious, and insecure.

It is this inability to wrestle with difficulty which has obliged the arbitrary Assembly of France to commence their schemes of reform with abolition and total destruction.
%[120]
\footnote{ A leading member of the Assembly, M. Rabaut de St. Étienne, has expressed the principle of all their proceedings as clearly as possible; nothing can be more simple:—"Tous les établissemens en France couronnent le malheur du peuple: pour le rendre heureux, il faut le renouveler, changer ses idées, changer ses loix, changer ses mœurs, ... changer les hommes, changer les choses, changer ses mots, ... tout détruire; oui, tout détruire; puisque tout est à récréer."—This gentleman was chosen president in an assembly not sitting at Quinze-Vingt or the Petites Maisons, and composed of persons giving themselves out to be rational beings; but neither his ideas, language, or conduct differ in the smallest degree from the discourses, opinions, and actions of those, within and without the Assembly, who direct the operations of the machine now at work in France.}
 But is it in destroying and pulling down that skill is displayed? Your mob can do this as well at least as your assemblies. The shallowest understanding, the rudest hand, is more than equal to that task. Rage and frenzy will pull down more in half an hour than prudence, deliberation, and foresight can build up in a hundred years. The errors and defects of old establishments are visible and palpable. It calls for little ability to point them out; and where absolute power is given, it requires but a word wholly to abolish the vice and the establishment together. The same lazy, but restless disposition, which loves sloth and hates quiet, directs these politicians, when they come to work for supplying the place of what they have destroyed. To make everything the reverse of what they have seen is quite as easy as to destroy. No difficulties occur in what has never been tried. Criticism is almost baffled in discovering the defects of what has not existed; and eager enthusiasm and cheating hope have all the wide field of imagination, in which they may expatiate with little or no opposition.

At once to preserve and to reform is quite another thing. When the useful parts of an old establishment are kept, and what is superadded is to be fitted to what is retained, a vigorous mind, steady, persevering attention, various powers of comparison and combination, and the resources of an understanding fruitful in expedients are to be exercised; they are to be exercised in a continued conflict with the combined force of opposite vices, with the obstinacy that rejects all improvement, and the levity that is fatigued and disgusted with everything of which it is in possession. But you may object,—"A process of this kind is slow. It is not fit for an Assembly which glories in performing in a few months the work of ages. Such a mode of reforming, possibly, might take up many years." Without question it might; and it ought. It is one of the excellences of a method in which time is amongst the assistants, that its operation is slow, and in some cases almost imperceptible. If circumspection and caution are a part of wisdom, when we work only upon inanimate matter, surely they become a part of duty too, when the subject of our demolition and construction is not brick and timber, but sentient beings, by the sudden alteration of whose state, condition, and habits, multitudes may be rendered miserable. But it seems as if it were the prevalent opinion in Paris, that an unfeeling heart and an undoubting confidence are the sole qualifications for a perfect legislator. Far different are my ideas of that high office. The true lawgiver ought to have a heart full of sensibility. He ought to love and respect his kind, and to fear himself. It may be allowed to his temperament to catch his ultimate object with an intuitive glance; but his movements towards it ought to be deliberate. Political arrangement, as it is a work for social ends, is to be only wrought by social means. There mind must conspire with mind. Time is required to produce that union of minds which alone can produce all the good we aim at. Our patience will achieve more than our force. If I might venture to appeal to what is so much out of fashion in Paris,—I mean to experience,—I should tell you, that in my course I have known, and, according to my measure, have coöperated with great men; and I have never yet seen any plan which has not been mended by the observations of those who were much inferior in understanding to the person who took the lead in the business. By a slow, but well-sustained progress, the effect of each step is watched; the good or ill success of the first gives light to us in the second; and so, from light to light, we are conducted with safety through the whole series. We see that the parts of the system do not clash. The evils latent in the most promising contrivances are provided for as they arise. One advantage is as little as possible sacrificed to another. We compensate, we reconcile, we balance. We are enabled to unite into a consistent whole the various anomalies and contending principles that are found in the minds and affairs of men. From hence arises, not an excellence in simplicity, but one far superior, an excellence in composition. Where the great interests of mankind are concerned through a long succession of generations, that succession ought to be admitted into some share in the councils which are so deeply to affect them. If justice requires this, the work itself requires the aid of more minds than one age can furnish. It is from this view of things that the best legislators have been often satisfied with the establishment of some sure, solid, and ruling principle in government,—a power like that which some of the philosophers have called a plastic Nature; and having fixed the principle, they have left it afterwards to its own operation.

To proceed in this manner, that is, to proceed with a presiding principle and a prolific energy, is with me the criterion of profound wisdom. What your politicians think the marks of a bold, hardy genius are only proofs of a deplorable want of ability. By their violent haste, and their defiance of the process of Nature, they are delivered over blindly to every projector and adventurer, to every alchemist and empiric. They despair of turning to account anything that is common. Diet is nothing in their system of remedy. The worst of it is, that this their despair of curing common distempers by regular methods arises not only from defect of comprehension, but, I fear, from some malignity of disposition. Your legislators seem to have taken their opinions of all professions, ranks, and offices from the declamations and buffooneries of satirists,—who would themselves be astonished, if they were held to the letter of their own descriptions. By listening only to these, your leaders regard all things only on the side of their vices and faults, and view those vices and faults under every color of exaggeration. It is undoubtedly true, though it may seem paradoxical,—but, in general, those who are habitually employed in finding and displaying faults are unqualified for the work of reformation; because their minds are not only unfurnished with patterns of the fair and good, but by habit they come to take no delight in the contemplation of those things. By hating vices too much, they come to love men too little. It is therefore not wonderful that they should be indisposed and unable to serve them. From hence arises the complexional disposition of some of your guides to pull everything in pieces. At this malicious game they display the whole of their quadrimanous activity. As to the rest, the paradoxes of eloquent writers, brought forth purely as a sport of fancy, to try their talents, to rouse attention, and excite surprise, are taken up by these gentlemen, not in the spirit of the original authors, as means of cultivating their taste and improving their style: these paradoxes become with them serious grounds of action, upon which they proceed in regulating the most important concerns of the state. Cicero ludicrously describes Cato as endeavoring to act in the commonwealth upon the school paradoxes which exercised the wits of the junior students in the Stoic philosophy. If this was true of Cato, these gentlemen copy after him in the manner of some persons who lived about his time,—pede nudo Catonem. Mr. Hume told me that he had from Rousseau himself the secret of his principles of composition. That acute, though eccentric observer, had perceived, that, to strike and interest the public, the marvellous must be produced; that the marvellous of the heathen mythology had long since lost its effects; that giants, magicians, fairies, and heroes of romance, which succeeded, had exhausted the portion of credulity which belonged to their age; that now nothing was left to a writer but that species of the marvellous, which might still be produced, and with as great an effect as ever, though in another way,—that is, the marvellous in life, in manners, in characters, and in extraordinary situations, giving rise to new and unlooked-for strokes in politics and morals. I believe, that, were Rousseau alive, and in one of his lucid intervals, he would be shocked at the practical frenzy of his scholars, who in their paradoxes are servile imitators, and even in their incredulity discover an implicit faith.

Men who undertake considerable things, even in a regular way, ought to give us ground to presume ability. But the physician of the state, who, not satisfied with the cure of distempers, undertakes to regenerate constitutions, ought to show uncommon powers. Some very unusual appearances of wisdom ought to display themselves on the face of the designs of those who appeal to no practice and who copy after no model. Has any such been manifested? I shall take a view (it shall for the subject be a very short one) of what the Assembly has done, with regard, first, to the constitution of the legislature; in the next place, to that of the executive power; then to that of the judicature; afterwards to the model of the army; and conclude with the system of finance: to see whether we can discover in any part of their schemes the portentous ability which may justify these bold undertakers in the superiority which they assume over mankind.

It is in the model of the sovereign and presiding part of this new republic that we should expect their grand display. Here they were to prove their title to their proud demands. For the plan itself at large, and for the reasons on which it is grounded, I refer to the journals of the Assembly of the 29th of September, 1789, and to the subsequent proceedings which have made any alterations in the plan. So far as in a matter somewhat confused I can see light, the system remains substantially as it has been originally framed. My few remarks will be such as regard its spirit, its tendency, and its fitness for framing a popular commonwealth, which they profess theirs to be, suited to the ends for which any commonwealth, and particularly such a commonwealth, is made. At the same time I mean to consider its consistency with itself and its own principles.

Old establishments are tried by their effects. If the people are happy, united, wealthy, and powerful, we presume the rest. We conclude that to be good from whence good is derived. In old establishments various correctives have been found for their aberrations from theory. Indeed, they are the results of various necessities and expediences. They are not often constructed after any theory: theories are rather drawn from them. In them we often see the end best obtained, where the means seem not perfectly reconcilable to what we may fancy was the original scheme. The means taught by experience may be better suited to political ends than those contrived in the original project. They again react upon the primitive constitution, and sometimes improve the design itself, from which they seem to have departed. I think all this might be curiously exemplified in the British Constitution. At worst, the errors and deviations of every kind in reckoning are found and computed, and the ship proceeds in her course. This is the case of old establishments; but in a new and merely theoretic system, it is expected that every contrivance shall appear, on the face of it, to answer its ends, especially where the projectors are no way embarrassed with an endeavor to accommodate the new building to an old one, either in the walls or on the foundations.

The French builders, clearing away as mere rubbish whatever they found, and, like their ornamental gardeners, forming everything into an exact level, propose to rest the whole local and general legislature on three bases of three different kinds,—one geometrical, one arithmetical, and the third financial; the first of which they call the basis of territory; the second, the basis of population; and the third, the basis of contribution. For the accomplishment of the first of these purposes, they divide the area of their country into eighty-three pieces, regularly square, of eighteen leagues by eighteen. These large divisions are called Departments. These they portion, proceeding by square measurement, into seventeen hundred and twenty districts, called Communes. These again they subdivide, still proceeding by square measurement, into smaller districts, called Cantons, making in all 6,400.

At first view this geometrical basis of theirs presents not much to admire or to blame. It calls for no great legislative talents. Nothing more than an accurate land-surveyor, with his chain, sight, and theodolite, is requisite for such a plan as this. In the old divisions of the country, various accidents at times, and the ebb and flow of various properties and jurisdictions, settled their bounds. These bounds were not made upon any fixed system, undoubtedly. They were subject to some inconveniences; but they were inconveniences for which use had found remedies, and habit had supplied accommodation and patience. In this new pavement of square within square, and this organization and semi-organization, made on the system of Empedocles and Buffon, and not upon any politic principle, it is impossible that innumerable local inconveniences, to which men are not habituated, must not arise. But these I pass over, because it requires an accurate knowledge of the country, which I do not possess, to specify them.

When these state surveyors came to take a view of their work of measurement, they soon found that in politics the most fallacious of all things was geometrical demonstration. They had then recourse to another basis (or rather buttress) to support the building, which tottered on that false foundation. It was evident that the goodness of the soil, the number of the people, their wealth, and the largeness of their contribution, made such infinite variations between square and square as to render mensuration a ridiculous standard of power in the commonwealth, and equality in geometry the most unequal of all measures in the distribution of men. However, they could not give it up,—but, dividing their political and civil representation into three parts, they allotted one of those parts to the square measurement, without a single fact or calculation to ascertain whether this territorial proportion of representation was fairly assigned, and ought upon any principle really to be a third. Having, however, given to geometry this portion, (of a third for her dower,) out of compliment, I suppose, to that sublime science, they left the other two to be scuffled for between the other parts, population and contribution.

When they came to provide for population, they were not able to proceed quite so smoothly as they had done in the field of their geometry. Here their arithmetic came to bear upon their juridical metaphysics. Had they stuck to their metaphysic principles, the arithmetical process would be simple indeed. Men, with them, are strictly equal, and are entitled to equal rights in their own government. Each head, on this system, would have its vote, and every man would vote directly for the person who was to represent him in the legislature. "But soft,—by regular degrees, not yet." This metaphysic principle, to which law, custom, usage, policy, reason, were to yield, is to yield itself to their pleasure. There must be many degrees, and some stages, before the representative can come in contact with his constituent. Indeed, as we shall soon see, these two persons are to have no sort of communion with each other. First, the voters in the Canton, who compose what they call primary assemblies, are to have a qualification. What! a qualification on the indefeasible rights of men? Yes; but it shall be a very small qualification. Our injustice shall be very little oppressive: only the local valuation of three days' labor paid to the public. Why, this is not much, I readily admit, for anything but the utter subversion of your equalizing principle. As a qualification it might as well be let alone; for it answers no one purpose for which qualifications are established; and, on your ideas, it excludes from a vote the man of all others whose natural equality stands the most in need of protection and defence: I mean the man who has nothing else but his natural equality to guard him. You order him to buy the right which you before told him Nature had given to him gratuitously at his birth, and of which no authority on earth could lawfully deprive him. With regard to the person who cannot come up to your market, a tyrannous aristocracy, as against him, is established at the very outset, by you who pretend to be its sworn foe.

The gradation proceeds. These primary assemblies of the Canton elect deputies to the Commune,—one for every two hundred qualified inhabitants. Here is the first medium put between the primary elector and the representative legislator; and here a new turnpike is fixed for taxing the rights of men with a second qualification: for none can be elected into the Commune who does not pay the amount of ten days' labor. Nor have we yet done. There is still to be another gradation.
%[121]
\footnote{ The Assembly, in executing the plan of their committee, made some alterations. They have struck out one stage in these gradations; this removes a part of the objection; but the main objection, namely, that in their scheme the first constituent voter has no connection with the representative legislator, remains in all its force. There are other alterations, some possibly for the better, some certainly for the worse: but to the author the merit or demerit of these smaller alterations appears to be of no moment, where the scheme itself is fundamentally vicious and absurd.}
 These Communes, chosen by the Canton, choose to the Department; and the deputies of the Department choose their deputies to the National Assembly. Here is a third barrier of a senseless qualification. Every deputy to the National Assembly must pay, in direct contribution, to the value of a mark of silver. Of all these qualifying barriers we must think alike: that they are impotent to secure independence, strong only to destroy the rights of men.

In all this process, which in its fundamental elements affects to consider only population, upon a principle of natural right, there is a manifest attention to property,—which, however just and reasonable on other schemes, is on theirs perfectly unsupportable.

When they come to their third basis, that of Contribution, we find that they have more completely lost sight of the rights of men. This last basis rests entirely on property. A principle totally different from the equality of men, and utterly irreconcilable to it, is thereby admitted: but no sooner is this principle admitted than (as usual) it is subverted; and it is not subverted (as we shall presently see) to approximate the inequality of riches to the level of Nature. The additional share in the third portion of representation (a portion reserved exclusively for the higher contribution) is made to regard the district only, and not the individuals in it who pay. It is easy to perceive, by the course of their reasonings, how much they were embarrassed by their contradictory ideas of the rights of men and the privileges of riches. The Committee of Constitution do as good as admit that they are wholly irreconcilable. "The relation with regard to the contributions is without doubt null, (say they,) when the question is on the balance of the political rights as between individual and individual; without which personal equality would be destroyed, and an aristocracy of the rich would be established. But this inconvenience entirely disappears, when the proportional relation of the contribution is only considered in the great masses, and is solely between province and province; it serves in that case only to form a just reciprocal proportion between the cities, without affecting the personal rights of the citizens."

Here the principle of contribution, as taken between man and man, is reprobated as null, and destructive to equality,—and as pernicious, too, because it leads to the establishment of an aristocracy of the rich. However, it must not be abandoned. And the way of getting rid of the difficulty is to establish the inequality as between department and department, leaving all the individuals in each department upon an exact par. Observe, that this parity between individuals had been before destroyed, when the qualifications within the departments were settled; nor does it seem a matter of great importance whether the equality of men be injured by masses or individually. An individual is not of the same importance in a mass represented by a few as in a mass represented by many. It would be too much to tell a man jealous of his equality, that the elector has the same franchise who votes for three members as he who votes for ten.

Now take it in the other point of view, and let us suppose their principle of representation according to contribution, that is according to riches, to be well imagined, and to be a necessary basis for their republic. In this their third basis they assume that riches ought to be respected, and that justice and policy require that they should entitle men, in some mode or other, to a larger share in the administration of public affairs; it is now to be seen how the Assembly provides for the preëminence, or even for the security of the rich, by conferring, in virtue of their opulence, that larger measure of power to their district which is denied to them personally. I readily admit (indeed, I should lay it down as a fundamental principle) that in a republican government, which has a democratic basis, the rich do require an additional security above what is necessary to them in monarchies. They are subject to envy, and through envy to oppression. On the present scheme it is impossible to divine what advantage they derive from the aristocratic preference upon which the unequal representation of the masses is founded. The rich cannot feel it, either as a support to dignity or as security to fortune: for the aristocratic mass is generated from purely democratic principles; and the prevalence given to it in the general representation has no sort of reference to or connection with the persons upon account of whose property this superiority of the mass is established. If the contrivers of this scheme meant any sort of favor to the rich, in consequence of their contribution, they ought to have conferred the privilege either on the individual rich, or on some class formed of rich persons (as historians represent Servius Tullius to have done in the early constitution of Rome); because the contest between the rich and the poor is not a struggle between corporation and corporation, but a contest between men and men,—a competition, not between districts, but between descriptions. It would answer its purpose better, if the scheme were inverted: that the votes of the masses were rendered equal, and that the votes within each mass were proportioned to property.

Let us suppose one man in a district (it is an easy supposition) to contribute as much as a hundred of his neighbors. Against these he has but one vote. If there were but one representative for the mass, his poor neighbors would outvote him by an hundred to one for that single representative. Bad enough! But amends are to be made him. How? The district, in virtue of his wealth, is to choose, say ten members instead of one: that is to say, by paying a very large contribution he has the happiness of being outvoted, an hundred to one, by the poor, for ten representatives, instead of being outvoted exactly in the same proportion for a single member. In truth, instead of benefiting by this superior quantity of representation, the rich man is subjected to an additional hardship. The increase of representation within his province sets up nine persons more, and as many more than nine as there may be democratic candidates, to cabal and intrigue and to flatter the people at his expense and to his oppression. An interest is by this means held out to multitudes of the inferior sort, in obtaining a salary of eighteen livres a day, (to them a vast object,) besides the pleasure of a residence in Paris, and their share in the government of the kingdom. The more the objects of ambition are multiplied and become democratic, just in that proportion the rich are endangered.

Thus it must fare between the poor and the rich in the province deemed aristocratic, which in its internal relation is the very reverse of that character. In its external relation, that is, in its relation to the other provinces, I cannot see how the unequal representation which is given to masses on account of wealth becomes the means of preserving the equipoise and the tranquillity of the commonwealth. For, if it be one of the objects to secure the weak from being crushed by the strong, (as in all society undoubtedly it is,) how are the smaller and poorer of these masses to be saved from the tyranny of the more wealthy? Is it by adding to the wealthy further and more systematical means of oppressing them? When we come to a balance of representation between corporate bodies, provincial interests, emulations, and jealousies are full as likely to arise among them as among individuals; and their divisions are likely to produce a much hotter spirit of dissension, and something leading much more nearly to a war.

I see that these aristocratic masses are made upon what is called the principle of direct contribution. Nothing can be a more unequal standard than this. The indirect contribution, that which arises from duties on consumption, is in truth a better standard, and follows and discovers wealth more naturally than this of direct contribution. It is difficult, indeed, to fix a standard of local preference on account of the one, or of the other, or of both, because some provinces may pay the more of either or of both on account of causes not intrinsic, but originating from those very districts over whom they have obtained a preference in consequence of their ostensible contribution. If the masses were independent, sovereign bodies, who were to provide for a federative treasury by distinct contingents, and that the revenue had not (as it has) many impositions running through the whole, which affect men individually, and not corporately, and which, by their nature, confound all territorial limits, something might be said for the basis of contribution as founded on masses. But, of all things, this representation, to be measured by contribution, is the most difficult to settle upon principles of equity in a country which considers its districts as members of a whole. For a great city, such as Bordeaux or Paris, appears to pay a vast body of duties, almost out of all assignable proportion to other places, and its mass is considered accordingly. But are these cities the true contributors in that proportion? No. The consumers of the commodities imported into Bordeaux, who are scattered through all France, pay the import duties of Bordeaux. The produce of the vintage in Guienne and Languedoc give to that city the means of its contribution growing out of an export commerce. The landholders who spend their estates in Paris, and are thereby the creators of that city, contribute for Paris from the provinces out of which their revenues arise. Very nearly the same arguments will apply to the representative share given on account of direct contribution: because the direct contribution must be assessed on wealth, real or presumed; and that local wealth will itself arise from causes not local, and which therefore in equity ought not to produce a local preference.

It is very remarkable, that, in this fundamental regulation which settles the representation of the mass upon the direct contribution, they have not yet settled how that direct contribution shall be laid, and how apportioned. Perhaps there is some latent policy towards the continuance of the present Assembly in this strange procedure. However, until they do this, they can have no certain constitution. It must depend at last upon the system of taxation, and must vary with every variation in that system. As they have contrived matters, their taxation does not so much depend on their constitution as their constitution on their taxation. This must introduce great confusion among the masses; as the variable qualification for votes within the district must, if ever real contested elections take place, cause infinite internal controversies.

To compare together the three bases, not on their political reason, but on the ideas on which the Assembly works, and to try its consistency with itself, we cannot avoid observing that the principle which the committee call the basis of population does not begin to operate from the same point with the two other principles, called the bases of territory and of contribution, which are both of an aristocratic nature. The consequence is, that, where all three begin to operate together, there is the most absurd inequality produced by the operation of the former on the two latter principles. Every canton contains four square leagues, and is estimated to contain, on the average, 4,000 inhabitants, or 680 voters in the primary assemblies, which vary in numbers with the population of the canton, and send one deputy to the commune for every 200 voters. Nine cantons make a commune.

Now let us take a canton containing a seaport town of trade, or a great manufacturing town. Let us suppose the population of this canton to be 12,700 inhabitants, or 2,193 voters, forming three primary assemblies, and sending ten deputies to the commune.

Oppose to this one canton two others of the remaining eight in the same commune. These we may suppose to have their fair population, of 4,000 inhabitants, and 680 voters each, or 8,000 inhabitants and 1,360 voters, both together. These will form only two primary assemblies, and send only six deputies to the commune.

When the assembly of the commune comes to vote on the basis of territory, which principle is first admitted to operate in that assembly, the single canton, which has half the territory of the other two, will have ten voices to six in the election of three deputies to the assembly of the department, chosen on the express ground of a representation of territory. This inequality, striking as it is, will be yet highly aggravated, if we suppose, as we fairly may, the several other cantons of the commune to fall proportionally short of the average population, as much as the principal canton exceeds it.

Now as to the basis of contribution, which also is a principle admitted first to operate in the assembly of the commune. Let us again take one canton, such as is stated above. If the whole of the direct contributions paid by a great trading or manufacturing town be divided equally among the inhabitants, each individual will be found to pay much more than an individual living in the country according to the same average. The whole paid by the inhabitants of the former will be more than the whole paid by the inhabitants of the latter,—we may fairly assume one third more. Then the 12,700 inhabitants, or 2,193 voters of the canton, will pay as much as 19,050 inhabitants, or 3,289 voters of the other cantons, which are nearly the estimated proportion of inhabitants and voters of five other cantons. Now the 2,193 voters will, as I before said, send only ten deputies to the assembly; the 3,289 voters will send sixteen. Thus, for an equal share in the contribution of the whole commune, there will be a difference of sixteen voices to ten in voting for deputies to be chosen on the principle of representing the general contribution of the whole commune.

By the same mode of computation, we shall find 15,875 inhabitants, or 2,741 voters of the other cantons, who pay one sixth LESS to the contribution of the whole commune, will have three voices MORE than the 12,700 inhabitants, or 2,193 voters of the one canton.

Such is the fantastical and unjust inequality between mass and mass, in this curious repartition of the rights of representation arising out of territory and contribution. The qualifications which these confer are in truth negative qualifications, that give a right in an inverse proportion to the possession of them.

In this whole contrivance of the three bases, consider it in any light you please, I do not see a variety of objects reconciled in one consistent whole, but several contradictory principles reluctantly and irreconcilably brought and held together by your philosophers, like wild beasts shut up in a cage, to claw and bite each other to their mutual destruction.

I am afraid I have gone too far into their way of considering the formation of a Constitution. They have much, but bad, metaphysics,—much, but bad, geometry,—much, but false, proportionate arithmetic; but if it were all as exact as metaphysics, geometry, and arithmetic ought to be, and if their schemes were perfectly consistent in all their parts, it would make only a more fair and sightly vision. It is remarkable, that, in a great arrangement of mankind, not one reference whatsoever is to be found to anything moral or anything politic,—nothing that relates to the concerns, the actions, the passions, the interests of men. Hominem non sapiunt.

You see I only consider this Constitution as electoral, and leading by steps to the National Assembly. I do not enter into the internal government of the departments, and their genealogy through the communes and cantons. These local governments are, in the original plan, to be as nearly as possible composed in the same manner and on the same principles with the elective assemblies. They are each of them bodies perfectly compact and rounded in themselves.

You cannot but perceive in this scheme, that it has a direct and immediate tendency to sever France into a variety of republics, and to render them totally independent of each other, without any direct constitutional means of coherence, connection, or subordination, except what may be derived from their acquiescence in the determinations of the general congress of the ambassadors from each independent republic. Such in reality is the National Assembly; and such governments, I admit, do exist in the world, though, in forms infinitely more suitable to the local and habitual circumstances of their people. But such associations, rather than bodies politic, have generally been the effect of necessity, not choice; and I believe the present French power is the very first body of citizens who, having obtained full authority to do with their country what they pleased, have chosen to dissever it in this barbarous manner.

It is impossible not to observe, that, in the spirit of this geometrical distribution and arithmetical arrangement, these pretended citizens treat France exactly like a country of conquest. Acting as conquerors, they have imitated the policy of the harshest of that harsh race. The policy of such barbarous victors, who contemn a subdued people, and insult their feelings, has ever been, as much as in them lay, to destroy all vestiges of the ancient country, in religion, in polity, in laws, and in manners; to confound all territorial limits; to produce a general poverty; to put up their properties to auction; to crush their princes, nobles, and pontiffs; to lay low everything which had lifted its head above the level, or which could serve to combine or rally, in their distresses, the disbanded people, under the standard of old opinion. They have made France free in the manner in which those sincere friends to the rights of mankind, the Romans, freed Greece, Macedon, and other nations. They destroyed the bonds of their union, under color of providing for the independence of each of their cities.

When the members who compose these new bodies of cantons, communes, and departments, arrangements purposely produced through the medium of confusion, begin to act, they will find themselves in a great measure strangers to one another. The electors and elected throughout, especially in the rural cantons, will be frequently without any civil habitudes or connections, or any of that natural discipline which is the soul of a true republic. Magistrates and collectors of revenue are now no longer acquainted with their districts, bishops with their dioceses, or curates with their parishes. These new colonies of the rights of men bear a strong resemblance to that sort of military colonies which Tacitus has observed upon in the declining policy of Rome. In better and wiser days (whatever course they took with foreign nations) they were careful to make the elements of a methodical subordination and settlement to be coeval, and even to lay the foundations of discipline in the military.
%[122]
\footnote{ "Non, ut olim, universæ legiones deducebantur, cum tribunis, et centurionibus, et sui cujusque ordinis militibus, ut consensu et caritate rempublicam efficerent; sed ignoti inter se, diversis manipulis, sine rectore, sine affectibus mutuis, quasi ex alio genere mortalium repente in unum collecti, numerus magis quam colonia."—Tac. Annal. lib. 14, sect. 27.—All this will be still more applicable to the unconnected, rotatory, biennial national assemblies, in this absurd and senseless constitution.}
 But when all the good arts had fallen into ruin, they proceeded, as your Assembly does, upon the equality of men, and with as little judgment, and as little care for those things which make a republic tolerable or durable. But in this, as well as almost every instance, your new commonwealth is born and bred and fed in those corruptions which mark degenerated and worn-out republics. Your child comes into the world with the symptoms of death; the facies Hippocratica forms the character of its physiognomy and the prognostic of its fate.

The legislators who framed the ancient republics knew that their business was too arduous to be accomplished with no better apparatus than the metaphysics of an undergraduate and the mathematics and arithmetic of an exciseman. They had to do with men, and they were obliged to study human nature. They had to do with citizens, and they were obliged to study the effects of those habits which are communicated by the circumstances of civil life. They were sensible that the operation of this second nature on the first produced a new combination,—and thence arose many diversities amongst men, according to their birth, their education, their professions, the periods of their lives, their residence in towns or in the country, their several ways of acquiring and of fixing property, and according to the quality of the property itself, all which rendered them, as it were, so many different species of animals. From hence they thought themselves obliged to dispose their citizens into such classes, and to place them in such situations in the state, as their peculiar habits might qualify them to fill, and to allot to them such appropriated privileges as might secure to them what their specific occasions required, and which might furnish to each description such force as might protect it in the conflict caused by the diversity of interests that must exist, and must contend, in all complex society: for the legislator would have been ashamed that the coarse husbandman should well know how to assort and to use his sheep, horses, and oxen, and should have enough of common sense not to abstract and equalize them all into animals, without providing for each kind an appropriate food, care, and employment,—whilst he, the economist, disposer, and shepherd of his own kindred, subliming himself into an airy metaphysician, was resolved to know nothing of his flocks but as men in general. It is for this reason that Montesquieu observed, very justly, that, in their classification of the citizens, the great legislators of antiquity made the greatest display of their powers, and even soared above themselves. It is here that your modern legislators have gone deep into the negative series, and sunk even below their own nothing. As the first sort of legislators attended to the different kinds of citizens, and combined them into one commonwealth, the others, the metaphysical and alchemistical legislators, have taken the directly contrary course. They have attempted to confound all sorts of citizens, as well as they could, into one homogeneous mass; and then they divided this their amalgama into a number of incoherent republics. They reduce men to loose counters, merely for the sake of simple telling, and not to figures, whose power is to arise from their place in the table. The elements of their own metaphysics might have taught them better lessons. The troll of their categorical table might have informed them that there was something else in the intellectual world besides substance and quantity. They might learn from the catechism of metaphysics that there were eight heads more,
%[123]
\footnote{ Qualitas, Relatio, Actio, Passio, Ubi, Quando, Situs, Habitus.}
 in every complex deliberation, which they have never thought of; though these, of all the ten, are the subject on which the skill of man can operate anything at all.

So far from this able disposition of some of the old republican legislators, which follows with a solicitous accuracy the moral conditions and propensities of men, they have levelled and crushed together all the orders which they found, even under the coarse, unartificial arrangement of the monarchy, in which mode of government the classing of the citizens is not of so much importance as in a republic. It is true, however, that every such classification, if properly ordered, is good in all forms of government, and composes a strong barrier against the excesses of despotism, as well as it is the necessary means of giving effect and permanence to a republic. For want of something of this kind, if the present project of a republic should fail, all securities to a moderated freedom fail along with it, all the indirect restraints which mitigate despotism are removed; insomuch that, if monarchy should ever again obtain an entire ascendency in France, under this or any other dynasty, it will probably be, if not voluntarily tempered, at setting out, by the wise and virtuous counsels of the prince, the most completely arbitrary power that has ever appeared on earth. This is to play a most desperate game.

The confusion which attends on all such proceedings they even declare to be one of their objects, and they hope to secure their Constitution by a terror of a return of those evils which attended their making it. "By this," say they, "its destruction will become difficult to authority, which cannot break it up without the entire disorganization of the whole state." They presume, that, if this authority should ever come to the same degree of power that they have acquired, it would make a more moderate and chastised use of it, and would piously tremble entirely to disorganize the state in the savage manner that they have done. They expect from the virtues of returning despotism the security which is to be enjoyed by the offspring of their popular vices.

I wish, Sir, that you and my readers would give an attentive perusal to the work of M. de Calonne on this subject. It is, indeed, not only an eloquent, but an able and instructive performance. I confine myself to what he says relative to the Constitution of the new state, and to the condition of the revenue. As to the disputes of this minister with his rivals, I do not wish to pronounce upon them. As little do I mean to hazard any opinion concerning his ways and means, financial or political, for taking his country out of its present disgraceful and deplorable situation of servitude, anarchy, bankruptcy, and beggary. I cannot speculate quite so sanguinely as he does: but he is a Frenchman, and has a closer duty relative to those objects, and better means of judging of them, than I can have. I wish that the formal avowal which he refers to, made by one of the principal leaders in the Assembly, concerning the tendency of their scheme to bring France not only from a monarchy to a republic, but from a republic to a mere confederacy, may be very particularly attended to. It adds new force to my observations: and, indeed, M. de Calonne's work supplies my deficiencies by many new and striking arguments on most of the subjects of this letter.
%[124]
\footnote{ See l'État de la France, p. 363.}


It is this resolution to break their country into separate republics which has driven them into the greatest number of their difficulties and contradictions. If it were not for this, all the questions of exact equality, and these balances, never to be settled, of individual rights, population, and contribution, would be wholly useless. The representation, though derived from parts, would be a duty which equally regarded the whole. Each deputy to the Assembly would be the representative of France, and of all its descriptions, of the many and of the few, of the rich and of the poor, of the great districts and of the small. All these districts would themselves be subordinate to some standing authority, existing independently of them,—an authority in which their representation, and everything that belongs to it, originated, and to which it was pointed. This standing, unalterable, fundamental government would make, and it is the only thing which could make, that territory truly and properly a whole. With us, when we elect popular representatives, we send them to a council in which each man individually is a subject, and submitted to a government complete in all its ordinary functions. With you the elective Assembly is the sovereign, and the sole sovereign; all the members are therefore integral parts of this sole sovereignty. But with us it is totally different. With us the representative, separated from the other parts, can have no action and no existence. The government is the point of reference of the several members and districts of our representation. This is the centre of our unity. This government of reference is a trustee for the whole, and not for the parts. So is the other branch of our public council: I mean the House of Lords. With us the King and the Lords are several and joint securities for the equality of each district, each province, each city. When did you hear in Great Britain of any province suffering from the inequality of its representation? what district from having no representation at all? Not only our monarchy and our peerage secure the equality on which our unity depends, but it is the spirit of the House of Commons itself. The very inequality of representation, which is so foolishly complained of, is perhaps the very thing which prevents us from thinking or acting as members for districts. Cornwall elects as many members as all Scotland. But is Cornwall better taken care of than Scotland? Few trouble their heads about any of your bases, out of some giddy clubs. Most of those who wish for any change, upon any plausible grounds, desire it on different ideas.

Your new Constitution is the very reverse of ours in its principle; and I am astonished how any persons could dream of holding out anything done in it as an example for Great Britain. With you there is little, or rather no, connection between the last representative and the first constituent. The member who goes to the National Assembly is not chosen by the people, nor accountable to them. There are three elections before he is chosen; two sets of magistracy intervene between him and the primary assembly, so as to render him, as I have said, an ambassador of a state, and not the representative of the people within a state. By this the whole spirit of the election is changed; nor can any corrective your Constitution-mongers have devised render him anything else than what he is. The very attempt to do it would inevitably introduce a confusion, if possible, more horrid than the present. There is no way to make a connection between the original constituent and the representative, but by the circuitous means which may lead the candidate to apply in the first instance to the primary electors, in order that by their authoritative instructions (and something more perhaps) these primary electors may force the two succeeding bodies of electors to make a choice agreeable to their wishes. But this would plainly subvert the whole scheme. It would be to plunge them back into that tumult and confusion of popular election, which, by their interposed gradation of elections, they mean to avoid, and at length to risk the whole fortune of the state with those who have the least knowledge of it and the least interest in it. This is a perpetual dilemma, into which they are thrown by the vicious, weak, and contradictory principles they have chosen. Unless the people break up and level this gradation, it is plain that they do not at all substantially elect to the Assembly; indeed, they elect as little in appearance as reality.

What is it we all seek for in an election? To answer its real purposes, you must first possess the means of knowing the fitness of your man; and then you must retain some hold upon him by personal obligation or dependence. For what end are these primary electors complimented, or rather mocked, with a choice? They can never know anything of the qualities of him that is to serve them, nor has he any obligation whatsoever to them. Of all the powers unfit to be delegated by those who have any real means of judging, that most peculiarly unfit is what relates to a personal choice. In case of abuse, that body of primary electors never can call the representative to an account for his conduct. He is too far removed from them in the chain of representation. If he acts improperly at the end of his two years' lease, it does not concern him for two years more. By the new French Constitution the best and the wisest representatives go equally with the worst into this Limbus Patrum. Their bottoms are supposed foul, and they must go into dock to be refitted. Every man who has served in an Assembly is ineligible for two years after. Just as these magistrates begin to learn their trade, like chimney-sweepers, they are disqualified for exercising it. Superficial, new, petulant acquisition, and interrupted, dronish, broken, ill recollection, is to be the destined character of all your future governors. Your Constitution has too much of jealousy to have much of sense in it. You consider the breach of trust in the representative so principally that you do not at all regard the question of his fitness to execute it.

This purgatory interval is not unfavorable to a faithless representative, who may be as good a canvasser as he was a bad governor. In this time he may cabal himself into a superiority over the wisest and most virtuous. As, in the end, all the members of this elective Constitution are equally fugitive, and exist only for the election, they may be no longer the same persons who had chosen him, to whom he is to be responsible when he solicits for a renewal of his trust. To call all the secondary electors of the commune to account is ridiculous, impracticable, and unjust: they may themselves have been deceived in their choice, as the third set of electors, those of the department, may be in theirs. In your elections responsibility cannot exist.

Finding no sort of principle of coherence with each other in the nature and constitution of the several new republics of France, I considered what cement the legislators had provided for them from any extraneous materials. Their confederations, their spectacles, their civic feasts, and their enthusiasm I take no notice of; they are nothing but mere tricks; but tracing their policy through their actions, I think I can distinguish the arrangements by which they propose to hold these republics together. The first is the confiscation, with the compulsory paper currency annexed to it; the second is the supreme power of the city of Paris; the third is the general army of the state. Of this last I shall reserve what I have to say, until I come to consider the army as an head by itself.

As to the operation of the first (the confiscation and paper currency) merely as a cement, I cannot deny that these, the one depending on the other, may for some time compose some sort of cement, if their madness and folly in the management, and in the tempering of the parts together, does not produce a repulsion in the very outset. But allowing to the scheme some coherence and some duration, it appears to me, that, if, after a while, the confiscation should not be found sufficient to support the paper coinage, (as I am morally certain it will not,) then, instead of cementing, it will add infinitely to the dissociation, distraction, and confusion of these confederate republics, both with relation to each other and to the several parts within themselves. But if the confiscation should so far succeed as to sink the paper currency, the cement is gone with the circulation. In the mean time its binding force will be very uncertain, and it will straiten or relax with every variation in the credit of the paper.

One thing only is certain in this scheme, which is an effect seemingly collateral, but direct, I have no doubt, in the minds of those who conduct this business; that is, its effect in producing an oligarchy in every one of the republics. A paper circulation, not founded on any real money deposited or engaged for, amounting already to four-and-forty millions of English money, and this currency by force substituted in the place of the coin of the kingdom, becoming thereby the substance of its revenue, as well as the medium of all its commercial and civil intercourse, must put the whole of what power, authority, and influence is left, in any form whatsoever it may assume, into the hands of the managers and conductors of this circulation.

In England we feel the influence of the Bank, though it is only the centre of a voluntary dealing. He knows little, indeed, of the influence of money upon mankind, who does not see the force of the management of a moneyed concern which is so much more extensive, and in its nature so much more depending on the managers, than any of ours. But this is not merely a money concern. There is another member in the system inseparably connected with this money management. It consists in the means of drawing out at discretion portions of the confiscated lands for sale, and carrying on a process of continual transmutation of paper into land and land into paper. When we follow this process in its effects, we may conceive something of the intensity of the force with which this system must operate. By this means the spirit of money-jobbing and speculation goes into the mass of land itself, and incorporates with it. By this kind of operation, that species of property becomes, as it were, volatilized; it assumes an unnatural and monstrous activity, and thereby throws into the hands of the several managers, principal and subordinate, Parisian and provincial, all the representative of money, and perhaps a full tenth part of all the land in France, which has now acquired the worst and most pernicious part of the evil of a paper circulation, the greatest possible uncertainty in its value. They have reversed the Latonian kindness to the landed property of Delos. They have sent theirs to be blown about, like the light fragments of a wreck, oras et littora circum.

The new dealers, being all habitually adventurers, and without any fixed habits or local predilections, will purchase to job out again, as the market of paper or of money or of land shall present an advantage. For though a holy bishop thinks that agriculture will derive great advantages from the "enlightened" usurers who are to purchase the Church confiscations, I, who am not a good, but an old farmer, with great humility beg leave to tell his late Lordship that usury is not a tutor of agriculture; and if the word "enlightened" be understood according to the new dictionary, as it always is in your new schools, I cannot conceive how a man's not believing in God can teach him to cultivate the earth with the least of any additional skill or encouragement. "Diis immortalibus sero," said an old Roman, when he held one handle of the plough, whilst Death held the other. Though you were to join in the commission all the directors of the two Academies to the directors of the Caisse d'Escompte, an old experienced peasant is worth them all. I have got more information upon a curious and interesting branch of husbandry, in one short conversation with an old Carthusian monk, than I have derived from all the bank directors that I have ever conversed with. However, there is no cause for apprehension from the meddling of money-dealers with rural economy. These gentlemen are too wise in their generation. At first, perhaps, their tender and susceptible imaginations may be captivated with the innocent and unprofitable delights of a pastoral life; but in a little time they will find that agriculture is a trade much more laborious and much less lucrative than that which they had left. After making its panegyric, they will turn their backs on it, like their great precursor and prototype. They may, like him, begin by singing, "Beatus ille"—but what will be the end?

\begin{verse}
Hæc ubi locutus fœnerator Alphius, \\
Jam jam futurus rusticus, \\
Omnem relegit Idibus pecuniam, \\
Quærit Calendis ponere.
\end{verse}

They will cultivate the Caisse d'Église, under the sacred auspices of this prelate, with much more profit than its vineyards and its corn-fields. They will employ their talents according to their habits and their interests. They will not follow the plough, whilst they can direct treasuries and govern provinces.

Your legislators, in everything new, are the very first who have founded a commonwealth upon gaming, and infused this spirit into it as its vital breath. The great object in these politics is to metamorphose France from a great kingdom into one great play-table,—to turn its inhabitants into a nation of gamesters,—to make speculation as extensive as life,—to mix it with all its concerns,—and to divert the whole of the hopes and fears of the people from their usual channels into the impulses, passions, and superstitions of those who live on chances. They loudly proclaim their opinion, that this their present system of a republic cannot possibly exist without this kind of gaming fund, and that the very thread of its life is spun out of the staple of these speculations. The old gaming in funds was mischievous enough, undoubtedly; but it was so only to individuals. Even when it had its greatest extent, in the Mississippi and South Sea, it affected but few, comparatively; where it extends further, as in lotteries, the spirit has but a single object. But where the law, which in most circumstances forbids, and in none countenances gaming, is itself debauched, so as to reverse its nature and policy, and expressly to force the subject to this destructive table, by bringing the spirit and symbols of gaming into the minutest matters, and engaging everybody in it, and in everything, a more dreadful epidemic distemper of that kind is spread than yet has appeared in the world. With you a man can neither earn nor buy his dinner without a speculation. What he receives in the morning will not have the same value at night. What he is compelled to take as pay for an old debt will not be received as the same, when he comes to pay a debt contracted by himself; nor will it be the same, when by prompt payment he would avoid contracting any debt at all. Industry must wither away. Economy must be driven from your country. Careful provision will have no existence. Who will labor without knowing the amount of his pay? Who will study to increase what none can estimate? Who will accumulate, when he does not know the value of what he saves? If you abstract it from its uses in gaming, to accumulate your paper wealth would be, not the providence of a man, but the distempered instinct of a jackdaw.

The truly melancholy part of the policy of systematically making a nation of gamesters is this,—that, though all are forced to play, few can understand the game, and fewer still are in a condition to avail themselves of that knowledge. The many must be the dupes of the few who conduct the machine of these speculations. What effect it must have on the country-people is visible. The townsman can calculate from day to day; not so the inhabitant of the country. When the peasant first brings his corn to market, the magistrate in the towns obliges him to take the assignat at par; when he goes to the shop with this money, he finds it seven per cent the worse for crossing the way. This market he will not readily resort to again. The towns-people will be inflamed; they will force the country-people to bring their corn. Resistance will begin, and the murders of Paris and St. Denis may be renewed through all France.

What signifies the empty compliment paid to the country, by giving it, perhaps, more than its share in the theory of your representation? Where have you placed the real power over moneyed and landed circulation? Where have you placed the means of raising and falling the value of every man's freehold? Those whose operations can take from or add ten per cent to the possessions of every man in France must be the masters of every man in France. The whole of the power obtained by this Revolution will settle in the towns among the burghers, and the moneyed directors who lead them. The landed gentleman, the yeoman, and the peasant have, none of them, habits or inclinations or experience which can lead them to any share in this the sole source of power and influence now left in France. The very nature of a country life, the very nature of landed property, in all the occupations and all the pleasures they afford, render combination and arrangement (the sole way of procuring and exerting influence) in a manner impossible amongst country-people. Combine them by all the art you can, and all the industry, they are always dissolving into individuality. Anything in the nature of incorporation is almost impracticable amongst them. Hope, fear, alarm, jealousy, the ephemerous tale that does its business and dies in a day, all these things, which are the reins and spurs by which leaders check or urge the minds of followers, are not easily employed, or hardly at all, amongst scattered people. They assemble, they arm, they act, with the utmost difficulty, and at the greatest charge. Their efforts, if ever they can be commenced, cannot be sustained. They cannot proceed systematically. If the country-gentlemen attempt an influence through the mere income of their property, what is it to that of those who have ten times their income to sell, and who can ruin their property by bringing their plunder to meet it at market? If the landed man wishes to mortgage, he falls the value of his land and raises the value of assignats. He augments the power of his enemy by the very means he must take to contend with him. The country-gentleman, therefore, the officer by sea and land, the man of liberal views and habits, attached to no profession, will be as completely excluded from the government of his country as if he were legislatively proscribed. It is obvious, that, in the towns, all the things which conspire against the country-gentleman combine in favor of the money manager and director. In towns combination is natural. The habits of burghers, their occupations, their diversion, their business, their idleness, continually bring them into mutual contact. Their virtues and their vices are sociable; they are always in garrison; and they come embodied and half-disciplined into the hands of those who mean to form them for civil or military action.

All these considerations leave no doubt on my mind, that, if this monster of a Constitution can continue, France will be wholly governed by the agitators in corporations, by societies in the towns, formed of directors in assignats, and trustees for the sale of Church lands, attorneys, agents, money-jobbers, speculators, and adventurers, composing an ignoble oligarchy, founded on the destruction of the crown, the Church, the nobility, and the people. Here end all the deceitful dreams and visions of the equality and rights of men. In "the Serbonian bog" of this base oligarchy they are all absorbed, sunk, and lost forever.

Though human eyes cannot trace them, one would be tempted to think some great offences in France must cry to Heaven, which has thought fit to punish it with a subjection to a vile and inglorious domination, in which no comfort or compensation is to be found in any even of those false splendors which, playing about other tyrannies, prevent mankind from feeling themselves dishonored even whilst they are oppressed. I must confess I am touched with a sorrow mixed with some indignation, at the conduct of a few men, once of great rank, and still of great character, who, deluded with specious names, have engaged in a business too deep for the line of their understanding to fathom,—who have lent their fair reputation and the authority of their high-sounding names to the designs of men with whom they could not be acquainted, and have thereby made their very virtues operate to the ruin of their country.

So far as to the first cementing principle.

The second material of cement for their new republic is the superiority of the city of Paris; and this, I admit, is strongly connected with the other cementing principle of paper circulation and confiscation. It is in this part of the project we must look for the cause of the destruction of all the old bounds of provinces and jurisdictions, ecclesiastical and secular, and the dissolution of all ancient combinations of things, as well as the formation of so many small unconnected republics. The power of the city of Paris is evidently one great spring of all their politics. It is through the power of Paris, now become the centre and focus of jobbing, that the leaders of this faction direct, or rather command, the whole legislative and the whole executive government. Everything, therefore, must be done which can confirm the authority of that city over the other republics. Paris is compact; she has an enormous strength, wholly disproportioned to the force of any of the square republics; and this strength is collected and condensed within a narrow compass. Paris has a natural and easy connection of its parts, which will not be affected by any scheme of a geometrical constitution; nor does it much signify whether its proportion of representation be more or less, since it has the whole draught of fishes in its drag-net. The other divisions of the kingdom, being hackled and torn to pieces, and separated from all their habitual means and even principles of union, cannot, for some time at least, confederate against her. Nothing was to be left in all the subordinate members, but weakness, disconnection, and confusion. To confirm this part of the plan, the Assembly has lately come to a resolution that no two of their republics shall have the same commander-in-chief.

To a person who takes a view of the whole, the strength of Paris, thus formed, will appear a system of general weakness. It is boasted that the geometrical policy has been adopted, that all local ideas should be sunk, and that the people should be no longer Gascons, Picards, Bretons, Normans,—but Frenchmen, with one country, one heart, and one Assembly. But, instead of being all Frenchmen, the greater likelihood is that the inhabitants of that region will shortly have no country. No man ever was attached by a sense of pride, partiality, or real affection, to a description of square measurement. He never will glory in belonging to the chequer No. 71, or to any other badge-ticket. We begin our public affections in our families. No cold relation is a zealous citizen. We pass on to our neighborhoods, and our habitual provincial connections. These are inns and resting-places. Such divisions of our country as have been formed by habit, and not by a sudden jerk of authority, were so many little images of the great country, in which the heart found something which it could fill. The love to the whole is not extinguished by this subordinate partiality. Perhaps it is a sort of elemental training to those higher and more large regards by which alone men come to be affected, as with their own concern, in the prosperity of a kingdom so extensive as that of France. In that general territory itself, as in the old name of Provinces, the citizens are interested from old prejudices and unreasoned habits, and not on account of the geometric properties of its figure. The power and preëminence of Paris does certainly press down and hold these republics together as long as it lasts: but, for the reasons I have already given you, I think it can not last very long.

Passing from the civil creating and the civil cementing principles of this Constitution to the National Assembly, which is to appear and act as sovereign, we see a body in its constitution with every possible power and no possible external control. We see a body without fundamental laws, without established maxims, without respected rules of proceeding, which nothing can keep firm to any system whatsoever. Their idea of their powers is always taken at the utmost stretch of legislative competency, and their examples for common cases from the exceptions of the most urgent necessity. The future is to be in most respects like the present Assembly; but, by the mode of the new elections and the tendency of the new circulations, it will be purged of the small degree of internal control existing in a minority chosen originally from various interests, and preserving something of their spirit. If possible, the next Assembly must be worse than the present. The present, by destroying and altering everything, will leave to their successors apparently nothing popular to do. They will be roused by emulation and example to enterprises the boldest and the most absurd. To suppose such an Assembly sitting in perfect quietude is ridiculous.

Your all-sufficient legislators, in their hurry to do everything at once, have forgot one thing that seems essential, and which, I believe, never has been before, in the theory or the practice, omitted by any projector of a republic. They have forgot to constitute a senate, or something of that nature and character. Never, before this time, was heard of a body politic composed of one legislative and active assembly, and its executive officers, without such a council: without something to which foreign states might connect themselves,—something to which, in the ordinary detail of government, the people could look up,—something which might give a bias and steadiness, and preserve something like consistency in the proceedings of state. Such a body kings generally have as a council. A monarchy may exist without it; but it seems to be in the very essence of a republican government. It holds a sort of middle place between the supreme power exercised by the people, or immediately delegated from them, and the mere executive. Of this there are no traces in your Constitution; and in providing nothing of this kind, your Solons and Numas have, as much as in anything else, discovered a sovereign incapacity.

Let us now turn our eyes to what they have done towards the formation of an executive power. For this they have chosen a degraded king. This their first executive officer is to be a machine, without any sort of deliberative discretion in any one act of his function. At best, he is but a channel to convey to the National Assembly such matter as may import that body to know. If he had been made the exclusive channel, the power would not have been without its importance, though infinitely perilous to those who would choose to exercise it. But public intelligence and statement of facts may pass to the Assembly with equal authenticity through any other conveyance. As to the means, therefore, of giving a direction to measures by the statement of an authorized reporter, this office of intelligence is as nothing.

To consider the French scheme of an executive officer, in its two natural divisions of civil and political.—In the first it must be observed, that, according to the new Constitution, the higher parts of judicature, in either of its lines, are not in the king. The king of France is not the fountain of justice. The judges, neither the original nor the appellate, are of his nomination. He neither proposes the candidates nor has a negative on the choice. He is not even the public prosecutor. He serves only as a notary, to authenticate the choice made of the judges in the several districts. By his officers he is to execute their sentence. When we look into the true nature of his authority, he appears to be nothing more than a chief of bumbailiffs, sergeants-at-mace, catchpoles, jailers, and hangmen. It is impossible to place anything called royalty in a more degrading point of view. A thousand times better it had been for the dignity of this unhappy prince, that he had nothing at all to do with the administration of justice, deprived as he is of all that is venerable and all that is consolatory in that function, without power of originating any process, without a power of suspension, mitigation, or pardon. Everything in justice that is vile and odious is thrown upon him. It was not for nothing that the Assembly has been at such pains to remove the stigma from certain offices, when they were resolved to place the person who had lately been their king in a situation but one degree above the executioner, and in an office nearly of the same quality. It is not in Nature, that, situated as the king of the French now is, he can respect himself or can be respected by others.

View this new executive officer on the side of his political capacity, as he acts under the orders of the National Assembly. To execute laws is a royal office; to execute orders is not to be a king. However, a political executive magistracy, though merely such, is a great trust. It is a trust, indeed, that has much depending upon its faithful and diligent performance, both in the person presiding in it and in all its subordinates. Means of performing this duty ought to be given by regulation; and dispositions towards it ought to be infused by the circumstances attendant on the trust. It ought to be environed with dignity, authority, and consideration, and it ought to lead to glory. The office of execution is an office of exertion. It is not from impotence we are to expect the tasks of power. What sort of person is a king to command executory service, who has no means whatsoever to reward it:—not in a permanent office; not in a grant of land; no, not in a pension of fifty pounds a year; not in the vainest and most trivial title? In France the king is no more the fountain of honor than he is the fountain of justice. All rewards, all distinctions, are in other hands. Those who serve the king can be actuated by no natural motive but fear,—by a fear of everything except their master. His functions of internal coercion are as odious as those which he exercises in the department of justice. If relief is to be given to any municipality, the Assembly gives it. If troops are to be sent to reduce them to obedience to the Assembly, the king is to execute the order; and upon every occasion he is to be spattered over with the blood of his people. He has no negative; yet his name and authority is used to enforce every harsh decree. Nay, he must concur in the butchery of those who shall attempt to free him from his imprisonment, or show the slightest attachment to his person or to his ancient authority.

Executive magistracy ought to be constituted in such a manner that those who compose it should be disposed to love and to venerate those whom they are bound to obey. A purposed neglect, or, what is worse, a literal, but perverse and malignant obedience, must be the ruin of the wisest counsels. In vain will the law attempt to anticipate or to follow such studied neglects and fraudulent attentions. To make them act zealously is not in the competence of law. Kings, even such as are truly kings, may and ought to bear the freedom of subjects that are obnoxious to them. They may, too, without derogating from themselves, bear even the authority of such persons, if it promotes their service. Louis the Thirteenth mortally hated the Cardinal de Richelieu; but his support of that minister against his rivals was the source of all the glory of his reign, and the solid foundation of his throne itself. Louis the Fourteenth, when come to the throne, did not love the Cardinal Mazarin; but for his interests he preserved him in power. When old, he detested Louvois; but for years, whilst he faithfully served his greatness, he endured his person. When George the Second took Mr. Pitt, who certainly was not agreeable to him, into his councils, he did nothing which could humble a wise sovereign. But these ministers, who were chosen by affairs, not by affections, acted in the name of and in trust for kings, and not as their avowed constitutional and ostensible masters. I think it impossible that any king, when he has recovered his first terrors, can cordially infuse vivacity and vigor into measures which he knows to be dictated by those who, he must be persuaded, are in the highest degree ill affected to his person. Will any ministers, who serve such a king (or whatever he may be called) with but a decent appearance of respect, cordially obey the orders of those whom but the other day in his name they had committed to the Bastile? will they obey the orders of those whom, whilst they were exercising despotic justice upon them, they conceived they were treating with lenity, and for whom in a prison they thought they had provided an asylum? If you expect such obedience, amongst your other innovations and regenerations, you ought to make a revolution in Nature, and provide a new constitution, for the human mind: otherwise your supreme government cannot harmonize with its executory system. There are cases in which we cannot take up with names and abstractions. You may call half a dozen leading individuals, whom we have reason to fear and hate, the nation. It makes no other difference than to make us fear and hate them the more. If it had been thought justifiable and expedient to make such a revolution by such means and through such persons as you have made yours, it would have been more wise to have completed the business of the fifth and sixth of October. The new executive officer would then owe his situation to those who are his creators as well as his masters; and he might be bound in interest, in the society of crime, and (if in crimes there could be virtues) in gratitude, to serve those who had promoted him to a place of great lucre and great sensual indulgence,—and of something more: for more he must have received from those who certainly would not have limited an aggrandized creature as they have done a submitting antagonist.

A king circumstanced as the present, if he is totally stupefied by his misfortunes, so as to think it not the necessity, but the premium and privilege of life, to eat and sleep, without any regard to glory, can never be fit for the office. If he feels as men commonly feel, he must he sensible that an office so circumstanced is one in which he can obtain no fame or reputation. He has no generous interest that can excite him to action. At best, his conduct will be passive and defensive. To inferior people such an office might be matter of honor. But to be raised to it and to descend to it are different things, and suggest different sentiments. Does he really name the ministers? They will have a sympathy with him. Are they forced upon him? The whole business between them and the nominal king will be mutual counteraction. In all other countries the office of ministers of state is of the highest dignity. In France it is full of peril, and incapable of glory. Rivals, however, they will have in their nothingness, whilst shallow ambition exists in the world, or the desire of a miserable salary is an incentive to short-sighted avarice. Those competitors of the ministers are enabled by your Constitution to attack them in their vital parts, whilst they have not the means of repelling their charges in any other than the degrading character of culprits. The ministers of state in Prance are the only persons in that country who are incapable of a share in the national councils. What ministers! What councils! What a nation!—But they are responsible. It is a poor service that is to be had from responsibility. The elevation of mind to be derived from fear will never make a nation glorious. Responsibility prevents crimes. It makes all attempts against the laws dangerous. But for a principle of active and zealous service, none but idiots could think of it. Is the conduct of a war to be trusted to a man who may abhor its principle,—who, in every step he may take to render it successful, confirms the power of those by whom he is oppressed? Will foreign states seriously treat with him who has no prerogative of peace or war,—no, not so much as in a single vote by himself or his ministers, or by any one whom he can possibly influence? A state of contempt is not a state for a prince: better get rid of him at once.

I know it will be said that these humors in the court and executive government will continue only through this generation, and that the king has been brought to declare the dauphin shall be educated in a conformity to his situation. If he is made to conform to his situation, he will have no education at all. His training must be worse even than that of an arbitrary monarch. If he reads,—whether he reads or not, some good or evil genius will tell him his ancestors were kings. Thenceforward his object must be to assert himself and to avenge his parents. This you will say is not his duty. That may be; but it is Nature; and whilst you pique Nature against you, you do unwisely to trust to duty. In this futile scheme of polity, the state nurses in its bosom, for the present, a source of weakness, perplexity, counteraction, inefficiency, and decay; and it prepares the means of its final ruin. In short, I see nothing in the executive force (I cannot call it authority) that has even an appearance of vigor, or that has the smallest degree of just correspondence or symmetry or amicable relation with the supreme power, either as it now exists, or as it is planned for the future government.

You have settled, by an economy as perverted as the policy, two
%[125]
\footnote{ In reality three, to reckon the provincial republican establishments.}
 establishments of government,—one real, one fictitious: both maintained at a vast expense; but the fictitious at, I think, the greatest. Such a machine as the latter is not worth the grease of its wheels. The expense is exorbitant; and neither the show nor the use deserve the tenth part of the charge.—Oh! but I don't do justice to the talents of the legislators: I don't allow, as I ought to do, for necessity. Their scheme of executive force was not their choice. This pageant must be kept. The people would not consent to part with it.—Right: I understand you. You do, in spite of your grand theories, to which you would have heaven and earth to bend, you do know how to conform yourselves to the nature and circumstances of things. But when you were obliged to conform thus far to circumstances, you ought to have carried your submission farther, and to have made, what you were obliged to take, a proper instrument, and useful to its end. That was in your power. For instance, among many others, it was in your power to leave to your king the right of peace and war.—What! to leave to the executive magistrate the most dangerous of all prerogatives?—I know none more dangerous; nor any one more necessary to be so trusted. I do not say that this prerogative ought to be trusted to your king, unless he enjoyed other auxiliary trusts along with it, which he does not now hold. But, if he did possess them, hazardous as they are undoubtedly, advantages would arise from such a Constitution, more than compensating the risk. There is no other way of keeping the several potentates of Europe from intriguing distinctly and personally with the members of your Assembly, from intermeddling in all your concerns, and fomenting, in the heart of your country, the most pernicious of all factions,—factions in the interest and under the direction of foreign powers. From that worst of evils, thank God, we are still free. Your skill, if you had any, would be well employed to find out indirect correctives and controls upon this perilous trust. If you did not like those which in England we have chosen, your leaders might have exerted their abilities in contriving better. If it were necessary to exemplify the consequences of such an executive government as yours, in the management of great affairs, I should refer you to the late reports of M. de Montmorin to the National Assembly, and all the other proceedings relative to the differences between Great Britain and Spain. It would be treating your understanding with disrespect to point them out to you.

I hear that the persons who are called ministers have signified an intention of resigning their places. I am rather astonished that they have not resigned long since. For the universe I would not have stood in the situation in which they have been for this last twelvemonth. They wished well, I take it for granted, to the Revolution. Let this fact be as it may, they could not, placed as they were upon an eminence, though an eminence of humiliation, but be the first to see collectively, and to feel each in his own department, the evils which have been produced by that Revolution. In every step which they took, or forbore to take, they must have felt the degraded situation of their country, and their utter incapacity of serving it. They are in a species of subordinate servitude in which no men before them were ever seen. Without confidence from their sovereign on whom they were forced, or from the Assembly who forced them upon him, all the noble functions of their office are executed by committees of the Assembly, without any regard whatsoever to their personal or their official authority. They are to execute, without power; they are to be responsible, without discretion; they are to deliberate, without choice. In their puzzled situation, under two sovereigns, over neither of whom they have any influence, they must act in such a manner as (in effect, whatever they may intend) sometimes to betray the one, sometimes the other, and always to betray themselves. Such has been their situation; such must be the situation of those who succeed them. I have much respect, and many good wishes, for M. Necker. I am obliged to him for attentions. I thought, when his enemies had driven him from Versailles, that his exile was a subject of most serious congratulation. Sed multæ urbes et publica vota vicerunt. He is now sitting on the ruins of the finances and of the monarchy of France.

A great deal more might be observed on the strange constitution of the executory part of the new government; but fatigue must give bounds to the discussion of subjects which in themselves have hardly any limits.

As little genius and talent am I able to perceive in the plan of judicature formed by the National Assembly. According to their invariable course, the framers of your Constitution have begun with the utter abolition of the parliaments. These venerable bodies, like the rest of the old government, stood in need of reform, even though there should be no change made in the monarchy. They required several more alterations to adapt them to the system of a free Constitution. But they had particulars in their constitution, and those not a few, which deserved approbation from the wise. They possessed one fundamental excellence: they were independent. The most doubtful circumstance attendant on their office, that of its being vendible, contributed, however, to this independency of character. They held for life. Indeed, they may be said to have held by inheritance. Appointed by the monarch, they were considered as nearly out of his power. The most determined exertions of that authority against them only showed their radical independence. They composed permanent bodies politic, constituted to resist arbitrary innovation; and from that corporate constitution, and from most of their forms, they were well calculated to afford both certainty and stability to the laws. They had been a safe asylum to secure these laws, in all the revolutions of humor and opinion. They had saved that sacred deposit of the country during the reigns of arbitrary princes and the struggles of arbitrary factions. They kept alive the memory and record of the Constitution. They were the great security to private property; which might be said (when personal liberty had no existence) to be, in fact, as well guarded in France as in any other country. Whatever is supreme in a state ought to have, as much as possible, ifs judicial authority so constituted as not only not to depend upon it, but in some sort to balance it. It ought to give a security to its justice against its power. It ought to make its judicature, as it were, something exterior to the state.

Those parliaments had furnished, not the best certainly, but some considerable corrective to the excesses and vices of the monarchy. Such an independent judicature was ten times more necessary when a democracy became the absolute power of the country. In that Constitution, elective, temporary, local judges, such as you have contrived, exercising their dependent functions in a narrow society, must be the worst of all tribunals. In them it will be vain to look for any appearance of justice towards strangers, towards the obnoxious rich, towards the minority of routed parties, towards all those who in the election have supported unsuccessful candidates. It will be impossible to keep the new tribunals clear of the worst spirit of faction. All contrivances by ballot we know experimentally to be vain and childish to prevent a discovery of inclinations. Where they may the best answer the purposes of concealment, they answer to produce suspicion, and this is a still more mischievous cause of partiality.

If the parliaments had been preserved, instead of being dissolved at so ruinous a change to the nation, they might have served in this new commonwealth, perhaps not precisely the same, (I do not mean an exact parallel,) but near the same purposes as the court and senate of Areopagus did in Athens: that is, as one of the balances and correctives to the evils of a light and unjust democracy. Every one knows that this tribunal was the great stay of that state; every one knows with what care it was upheld, and with what a religious awe it was consecrated. The parliaments were not wholly free from faction, I admit; but this evil was exterior and accidental, and not so much the vice of their constitution itself as it must be in your new contrivance of sexennial elective judicatories. Several English commend the abolition of the old tribunals, as supposing that they determined everything by bribery and corruption. But they have stood the test of monarchic and republican scrutiny. The court was well disposed to prove corruption on those bodies, when they were dissolved in 1771; those who have again dissolved them would have done the same, if they could; but both inquisitions having failed, I conclude that gross pecuniary corruption must have been rather rare amongst them.

It would have been prudent, along with the parliaments, to preserve their ancient power of registering, and of remonstrating at least upon, all the decrees of the National Assembly, as they did upon those which passed in the time of the monarchy. It would be a means of squaring the occasional decrees of a democracy to some principles of general jurisprudence. The vice of the ancient democracies, and one cause of their ruin, was, that they ruled, as you do, by occasional decrees, psephismata. This practice soon broke in upon the tenor and consistency of the laws; it abated the respect of the people towards them, and totally destroyed them in the end.

Your vesting the power of remonstrance, which, in the time of the monarchy, existed in the Parliament of Paris, in your principal executive officer, whom, in spite of common sense, you persevere in calling king, is the height of absurdity. You ought never to suffer remonstrance from him who is to execute. This is to understand neither council nor execution, neither authority nor obedience. The person whom you call king ought not to have this power, or he ought to have more.

Your present arrangement is strictly judicial. Instead of imitating your monarchy, and seating your judges on a bench of independence, your object is to reduce them to the most blind obedience. As you have changed all things, you have invented new principles of order. You first appoint judges, who, I suppose, are to determine according to law, and then you let them know, that, at some time or other, you intend to give them some law by which they are to determine. Any studies which they have made (if any they have made) are to be useless to them. But to supply these studies, they are to be sworn to obey all the rules, orders, and instructions which from time to time they are to receive from the National Assembly. These if they submit to, they leave no ground of law to the subject. They become complete and most dangerous instruments in the hands of the governing power, which, in the midst of a cause, or on the prospect of it, may wholly change the rule of decision. If these orders of the National Assembly come to be contrary to the will of the people who locally choose those judges, such confusion must happen as is terrible to think of. For the judges owe their place to the local authority, and the commands they are sworn to obey come from those who have no share in their appointment. In the mean time they have the example of the court of Châtelet to encourage and guide them in the exercise of their functions. That court is to try criminals sent to it by the National Assembly, or brought before it by other courses of delation. They sit under a guard to save their own lives. They know not by what law they judge, nor under what authority they act, nor by what tenure they hold. It is thought that they are sometimes obliged to condemn at peril of their lives. This is not perhaps certain, nor can it be ascertained; but when they acquit, we know they have seen the persons whom they discharge, with perfect impunity to the actors, hanged at the door of their court.

The Assembly, indeed, promises that they will form a body of law, which shall be short, simple, clear, and so forth. That is, by their short laws, they will leave much to the discretion of the judge, whilst they have exploded the authority of all the learning which could make judicial discretion (a thing perilous at best) deserving the appellation of a sound discretion.

It is curious to observe, that the administrative bodies are carefully exempted from the jurisdiction of these new tribunals. That is, those persons are exempted from the power of the laws who ought to be the most entirely submitted to them. Those who execute public pecuniary trusts ought of all men to be the most strictly held to their duty. One would have thought that it must have been among your earliest cares, if you did not mean that those administrative bodies should be real, sovereign, independent states, to form an awful tribunal, like your late parliaments, or like our King's Bench, where all corporate officers might obtain protection in the legal exercise of their functions, and would find coercion, if they trespassed against their legal duty. But the cause of the exemption is plain. These administrative bodies are the great instruments of the present leaders in their progress through democracy to oligarchy. They must therefore be put above the law. It will be said that the legal tribunals which you have made are unfit to coerce them. They are, undoubtedly. They are unfit for any rational purpose. It will be said, too, that the administrative bodies will be accountable to the general Assembly. This, I fear, is talking without much consideration of the nature of that Assembly or of these corporations. However, to be subject to the pleasure of that Assembly is not to be subject to law, either for protection or for constraint.

This establishment of judges as yet wants something to its completion. It is to be crowned by a new tribunal. This is to be a grand state judicature; and it is to judge of crimes committed against the nation, that is, against the power of the Assembly. It seems as if they had something in their view of the nature of the high court of justice erected in England during the time of the great usurpation. As they have not yet finished this part of the scheme, it is impossible to form a direct judgment upon it. However, if great care is not taken to form it in a spirit very different from that which has guided them in their proceedings relative to state offences, this tribunal, subservient to their inquisition, the Committee of Research, will extinguish the last sparks of liberty in France, and settle the most dreadful and arbitrary tyranny ever known in any nation. If they wish to give to this tribunal any appearance of liberty and justice, they must not evoke from or send to it the causes relative to their own members, at their pleasure. They must also remove the seat of that tribunal out of the republic of Paris.
%[126]
\footnote{ For further elucidations upon the subject of all these judicatures and of the Committee of Research, see M. de Calonne's work.}


Has more wisdom been displayed in the constitution of your army than what is discoverable in your plan of judicature? The able arrangement of this part is the more difficult, and requires the greater skill and attention, not only as a great concern in itself, but as it is the third cementing principle in the new body of republics which you call the French nation. Truly, it is not easy to divine what that army may become at last. You have voted a very large one, and on good appointments, at least fully equal to your apparent means of payment. But what is the principle of its discipline? or whom is it to obey? You have got the wolf by the ears, and I wish you joy of the happy position in which you have chosen to place yourselves, and in which you are well circumstanced for a free deliberation relatively to that army, or to anything else.

The minister and secretary of state for the War Department is M. de La Tour du Pin. This gentleman, like his colleagues in administration, is a most zealous assertor of the Revolution, and a sanguine admirer of the new Constitution which originated in that event. His statement of facts relative to the military of France is important, not only from his official and personal authority, but because it displays very clearly the actual condition of the army in France, and because it throws light on the principles upon which the Assembly proceeds in the administration of this critical object. It may enable us to form some judgment how far it may be expedient in this country to imitate the martial policy of France.

M. de La Tour du Pin, on the fourth of last June, comes to give an account of the state of his department, as it exists under the auspices of the National Assembly. No man knows it so well; no man can express it better. Addressing himself to the National Assembly, he says,—

"His Majesty has this day sent me to apprise you of the multiplied disorders of which every day he receives the most distressing intelligence. The army [le corps militaire] threatens to fall into the most turbulent anarchy. Entire regiments have dared to violate at once the respect due to the laws, to the king, to the order established by your decrees, and to the oaths which they have taken with the most awful solemnity. Compelled by my duty to give you information of these excesses, my heart bleeds, when I consider who they are that have committed them. Those against whom it is not in my power to withhold the most grievous complaints are a part of that very soldiery which to this day have been so full of honor and loyalty, and with whom for fifty years I have lived the comrade and the friend.

"What incomprehensible spirit of delirium and delusion has all at once led them astray? Whilst you are indefatigable in establishing uniformity in the empire and moulding the whole into one coherent and consistent body, whilst the French are taught by you at once the respect which the laws owe to the rights of man and that which the citizens owe to the laws, the administration of the army presents nothing but disturbance and confusion. I see in more than one corps the bonds of discipline relaxed or broken,—the most unheard-of pretensions avowed directly and without any disguise,—the ordinances without force,—the chiefs without authority,—the military chest and the colors carried off,—the authority of the king himself [risum teneatis] proudly defied,—the officers despised, degraded, threatened, driven away, and some of them prisoners in the midst of their corps, dragging on a precarious life in the bosom of disgust and humiliation. To fill up the measure of all these horrors, the commandants of places have had their throats out under the eyes and almost in the arms of their own soldiers.

"These evils are great; but they are not the worst consequences which may be produced by such military insurrections. Sooner or later they may menace the nation itself. The nature of things requires that the army should never act but as an instrument. The moment that, erecting itself into a deliberate body, it shall act according to its own resolutions, the government, be it what it may, will immediately degenerate into a military democracy: a species of political monster which has always ended by devouring those who have produced it.

"After all this, who must not be alarmed at the irregular consultations and turbulent committees formed in some regiments by the common soldiers and non-commissioned officers, without the knowledge, or even in contempt of the authority, of their superiors?—although the presence and concurrence of those superiors could give no authority to such monstrous democratic assemblies [comices]."

It is not necessary to add much to this finished picture,—finished as far as its canvas admits, but, as I apprehend, not taking in the whole of the nature and complexity of the disorders of this military democracy, which, the minister at war truly and wisely observes, wherever it exists, must be the true constitution of the state, by whatever formal appellation it may pass. For, though he informs the Assembly that the more considerable part of the army have not cast off their obedience, but are still attached to their duty, yet those travellers who have seen the corps whose conduct is the best rather observe in them the absence of mutiny than the existence of discipline.

I cannot help pausing here for a moment, to reflect upon the expressions of surprise which this minister has let fall relative to the excesses he relates. To him the departure of the troops from their ancient principles of loyalty and honor seems quite inconceivable. Surely those to whom he addresses himself know the causes of it but too well. They know the doctrines which they have preached, the decrees which they have passed, the practices which they have countenanced. The soldiers remember the sixth of October. They recollect the French guards. They have not forgot the taking of the king's castles in Paris and at Marseilles. That the governors in both places were murdered with impunity is a fact that has not passed out of their minds. They do not abandon the principles, laid down so ostentatiously and laboriously, of the equality of men. They cannot shut their eyes to the degradation of the whole noblesse of France, and the suppression of the very idea of a gentleman. The total abolition of titles and distinctions is not lost upon them. But M. du Pin is astonished at their disloyalty, when the doctors of the Assembly have taught them at the same time the respect due to laws. It is easy to judge which of the two sorts of lessons men with arms in their hands are likely to learn. As to the authority of the king, we may collect from the minister himself (if any argument on that head were not quite superfluous) that it is not of more consideration with these troops than it is with everybody else. "The king," says he, "has over and over again repeated his orders to put a stop to these excesses; but in so terrible a crisis, your [the Assembly's] concurrence is become indispensably necessary to prevent the evils which menace the state. You unite to the force of the legislative power that of opinion, still more important." To be sure, the army can have no opinion of the power or authority of the king. Perhaps the soldier has by this time learned, that the Assembly itself does not enjoy a much greater degree of liberty than that royal figure.

It is now to be seen what has been proposed in this exigency, one of the greatest that can happen in a state. The minister requests the Assembly to array itself in all its terrors, and to call forth all its majesty. He desires that the grave and severe principles announced by them may give vigor to the king's proclamation. After this we should have looked for courts civil and martial, breaking of some corps, decimating of others, and all the terrible means which necessity has employed in such cases to arrest the progress of the most terrible of all evils; particularly, one might expect that a serious inquiry would be made into the murder of commandants in the view of their soldiers. Not one word of all this, or of anything like it. After they had been told that the soldiery trampled upon the decrees of the Assembly promulgated by the king, the Assembly pass new decrees, and they authorize the king to make new proclamations. After the secretary at war had stated that the regiments had paid no regard to oaths, prêtés avec la plus imposante solennité, they propose—what? More oaths. They renew decrees and proclamations as they experience their insufficiency, and they multiply oaths in proportion as they weaken in the minds of men the sanctions of religion. I hope that handy abridgments of the excellent sermons of Voltaire, D'Alembert, Diderot, and Helvétius, on the Immortality of the Soul, on a Particular Superintending Providence, and on a Future State of Rewards and Punishments, are sent down to the soldiers along with their civic oaths. Of this I have no doubt; as I understand that a certain description of reading makes no inconsiderable part of their military exercises, and that they are full as well supplied with the ammunition of pamphlets as of cartridges.

To prevent the mischiefs arising from conspiracies, irregular consultations, seditious committees, and monstrous democratic assemblies [comitia, comices] of the soldiers, and all the disorders arising from idleness, luxury, dissipation, and insubordination, I believe the most astonishing means have been used that ever occurred to men, even in all the inventions of this prolific age. It is no less than this:—The king has promulgated in circular letters to all the regiments his direct authority and encouragement, that the several corps should join themselves with the clubs and confederations in the several municipalities, and mix with them in their feasts and civic entertainments! This jolly discipline, it seems, is to soften the ferocity of their minds, to reconcile them to their bottle companions of other descriptions, and to merge particular conspiracies in more general associations.
%[127]
\footnote{ 'Comme sa Majesté a reconnu, non un système d'associations particulières, mais une réunion de volontés de tous les François pour la liberté et la prospérité communes, ainsi pour le maintien de l'ordre publique, il a pensé qu'il convenoit que chaque régiment prît part à ces fêtes civiques pour multiplier les rapports, et resserrer les liens d'union entre les citoyens et les troupes.'—Lest I should not be credited, I insert the words authorizing the troops to feast with the popular confederacies.}
 That this remedy would be pleasing to the soldiers, as they are described by M. de La Tour du Pin, I can readily believe,—and that, however mutinous otherwise, they will dutifully submit themselves to these royal proclamations. But I should question whether all this civic swearing, clubbing, and feasting would dispose them, more than at present they are disposed, to an obedience to their officers, or teach them better to submit to the austere rules of military discipline. It will make them admirable citizens after the French mode, but not quite so good soldiers after any mode. A doubt might well arise, whether the conversations at these good tables would fit them a great deal the better for the character of mere instruments, which this veteran officer and statesman justly observes the nature of things always requires an army to be.

Concerning the likelihood of this improvement in discipline by the free conversation of the soldiers with the municipal festive societies, which is thus officially encouraged by royal authority and sanction, we may judge by the state of the municipalities themselves, furnished to us by the war minister in this very speech. He conceives good hopes of the success of his endeavors towards restoring order for the present from the good disposition of certain regiments; but he finds something cloudy with regard to the future. As to preventing the return of confusion, "for this the administration" (says he) "cannot be answerable to you, as long as they see the municipalities arrogate to themselves an authority over the troops which your institutions have reserved wholly to the monarch. You have fixed the limits of the military authority and the municipal authority. You have bounded the action which you have permitted to the latter over the former to the right of requisition; but never did the letter or the spirit of your decrees authorize the commons in these municipalities to break the officers, to try them, to give orders to the soldiers, to drive them from the posts committed to their guard, to stop them in their marches ordered by the king, or, in a word, to enslave the troops to the caprice of each of the cities or even market-towns through which they are to pass."

Such is the character and disposition of the municipal society which is to reclaim the soldiery, to bring them back to the true principles of military subordination, and to lender them machines in the hands of the supreme power of the country! Such are the distempers of the French troops! Such is their cure! As the army is, so is the navy. The municipalities supersede the orders of the Assembly, and the seamen in their turn supersede the orders of the municipalities. From my heart I pity the condition of a respectable servant of the public, like this war minister, obliged in his old age to pledge the Assembly in their civic cups, and to enter with a hoary head into all the fantastic vagaries of these juvenile politicians. Such schemes are not like propositions coming from a man of fifty years' wear and tear amongst mankind. They seem rather such as ought to be expected from those grand compounders in politics who shorten the road to their degrees in the state, and have a certain inward fanatical assurance and illumination upon all subjects,—upon the credit of which, one of their doctors has thought fit, with great applause, and greater success, to caution the Assembly not to attend to old men, or to any persons who value themselves upon their experience. I suppose all the ministers of state must qualify, and take this test,—wholly abjuring the errors and heresies of experience and observation. Every man has his own relish; but I think, if I could not attain to the wisdom, I would at least preserve something of the stiff and peremptory dignity of age. These gentlemen deal in regeneration: but at any price I should hardly yield my rigid fibres to be regenerated by them,—nor begin, in my grand climacteric, to squall in their new accents, or to stammer, in my second cradle, the elemental sounds of their barbarous metaphysics.
%[128]
\footnote{ This war minister has since quitted the school and resigned his office.}
 Si isti mihi largiantur ut repuerascam, et in eorum cunis vagiam, valde recusem!

The imbecility of any part of the puerile and pedantic system which they call a Constitution cannot be laid open without discovering the utter insufficiency and mischief of every other part with which it comes in contact, or that bears any the remotest relation to it. You cannot propose a remedy for the incompetence of the crown, without displaying the debility of the Assembly. You cannot deliberate on the confusion of the army of the state, without disclosing the worse disorders of the armed municipalities. The military lays open the civil, and the civil betrays the military anarchy. I wish everybody carefully to peruse the eloquent speech (such it is) of Mons. de La Tour du Pin. He attributes the salvation of the municipalities to the good behavior of some of the troops. These troops are to preserve the well-disposed part of the municipalities, which is confessed to be the weakest, from the pillage of the worst disposed, which is the strongest. But the municipalities affect a sovereignty, and will command those troops which are necessary for their protection. Indeed, they must command them or court them. The municipalities, by the necessity of their situation, and by the republican powers they have obtained, must, with relation to the military, be the masters, or the servants, or the confederates, or each successively, or they must make a jumble of all together, according to circumstances. What government is there to coerce the army but the municipality, or the municipality but the army? To preserve concord where authority is extinguished, at the hazard of all consequences, the Assembly attempts to cure the distempers by the distempers themselves; and they hope to preserve themselves from a purely military democracy by giving it a debauched interest in the municipal.

If the soldiers once come to mix for any time in the municipal clubs, cabals, and confederacies, an elective attraction will draw them to the lowest and most desperate part. With them will be their habits, affections, and sympathies. The military conspiracies which are to be remedied by civic confederacies, the rebellious municipalities which are to be rendered obedient by furnishing them with the means of seducing the very armies of the state that are to keep them in order,—all these chimeras of a monstrous and portentous policy must aggravate the confusion from which they have arisen. There must be blood. The want of common judgment manifested in the construction of all their descriptions of forces, and in all their kinds of civil and judicial authorities, will make it flow. Disorders may be quieted in one time and in one part. They will break out in others; because the evil is radical and intrinsic. All these schemes of mixing mutinous soldiers with seditious citizens must weaken still more and more the military connection of soldiers with their officers, as well as add military and mutinous audacity to turbulent artificers and peasants. To secure a real army, the officer should be first and last in the eye of the soldier,—first and last in his attention, observance, and esteem. Officers, it seems, there are to be, whose chief qualification must be temper and patience. They are to manage their troops by electioneering arts. They must bear themselves as candidates, not as commanders. But as by such means power may be occasionally in their hands, the authority by which they are to be nominated becomes of high importance.

What you may do finally does not appear: nor is it of much moment, whilst the strange and contradictory relation between your army and all the parts of your republic, as well as the puzzled relation of those parts to each other and to the whole, remain as they are. You seem to have given the provisional nomination of the officers, in the first instance, to the king, with a reserve of approbation by the National Assembly. Men who have an interest to pursue are extremely sagacious in discovering the true seat of power. They must soon perceive that those who can negative indefinitely in reality appoint. The officers must therefore look to their intrigues in the Assembly as the sole certain road to promotion. Still, however, by your new Constitution, they must begin their solicitation at court. This double negotiation for military rank seems to me a contrivance, as well adapted as if it were studied for no other end, to promote faction in the Assembly itself relative to this vast military patronage,—and then to poison the corps of officers with factions of a nature still more dangerous to the safety of government, upon any bottom on which it can be placed, and destructive in the end to the efficacy of the army itself. Those officers who lose the promotions intended for them by the crown must become of a faction opposite to that of the Assembly which has rejected their claims, and must nourish discontents in the heart of the army against the ruling powers. Those officers, on the other hand, who, by carrying their point through an interest in the Assembly, feel themselves to be at best only second in the good-will of the crown, though first in that of the Assembly, must slight an authority which would not advance and could not retard their promotion. If, to avoid these evils, you will have no other rule for command or promotion than seniority, you will have an army of formality; at the same time it will become more independent and more of a military republic. Not they, but the king is the machine. A king is not to be deposed by halves. If he is not everything in the command of an army, he is nothing. What is the effect of a power placed nominally at the head of the army, who to that army is no object of gratitude or of fear? Such a cipher is not fit for the administration of an object of all things the most delicate, the supreme command of military men. They must be constrained (and their inclinations lead them to what their necessities require) by a real, vigorous, effective, decided, personal authority. The authority of the Assembly itself suffers by passing through such a debilitating channel as they have chosen. The army will not long look to an Assembly acting through the organ of false show and palpable imposition. They will not seriously yield obedience to a prisoner. They will either despise a pageant, or they will pity a captive king. This relation of your army to the crown will, if I am not greatly mistaken, become a serious dilemma in your politics.

It is besides to be considered, whether an Assembly like yours, even supposing that it was in possession of another sort of organ, through which its orders were to pass, is fit for promoting the obedience and discipline of an army. It is known that armies have hitherto yielded a very precarious and uncertain obedience to any senate or popular authority; and they will least of all yield it to an Assembly which is to have only a continuance of two years. The officers must totally lose the characteristic disposition of military men, if they see with perfect submission and due admiration the dominion of pleaders,—especially when they find that they have a new court to pay to an endless succession of those pleaders, whose military policy, and the genius of whose command, (if they should have any,) must be as uncertain as their duration is transient. In the weakness of one kind of authority, and in the fluctuation of all, the officers of an army will remain for some time mutinous and full of faction, until some popular general, who understands the art of conciliating the soldiery, and who possesses the true spirit of command, shall draw the eyes of all men upon himself. Armies will obey him on his personal account. There is no other way of securing military obedience in this state of things. But the moment in which that event shall happen, the person who really commands the army is your master,—the master (that is little) of your king, the master of your Assembly, the master of your whole republic.

How came the Assembly by their present power over the army? Chiefly, to be sure, by debauching the soldiers from their officers. They have begun by a most terrible operation. They have touched the central point about which the particles that compose armies are at repose. They have destroyed the principle of obedience in the great, essential, critical link between the officer and the soldier, just where the chain of military subordination commences, and on which the whole of that system, depends. The soldier is told he is a citizen, and has the rights of man and citizen. The right of a man, he is told, is, to be his own governor, and to be ruled only by those to whom he delegates that self-government. It is very natural he should think that he ought most of all to have his choice where he is to yield the greatest degree of obedience. He will therefore, in all probability, systematically do what he does at present occasionally: that is, he will exercise at least a negative in the choice of his officers. At present the officers are known at best to be only permissive, and on their good behavior. In fact, there have been many instances in which they have been cashiered by their corps. Here is a second negative on the choice of the king: a negative as effectual, at least, as the other of the Assembly. The soldiers know already that it has been a question, not ill received in the National Assembly, whether they ought not to have the direct choice of their officers, or some proportion of them. When such matters are in deliberation, it is no extravagant supposition that they will incline to the opinion most favorable to their pretensions. They will not bear to be deemed the army of an imprisoned king, whilst another army in the same country, with whom too they are to feast and confederate, is to be considered as the free army of a free Constitution. They will cast their eyes on the other and more permanent army: I mean the municipal. That corps, they well know, does actually elect its own officers. They may not be able to discern the grounds of distinction on which they are not to elect a Marquis de La Fayette (or what is his new name?) of their own. If this election of a commander-in-chief be a part of the rights of men, why not of theirs? They see elective justices of peace, elective judges, elective curates, elective bishops, elective municipalities, and elective commanders of the Parisian army. Why should they alone be excluded? Are the brave troops of France the only men in that nation who are not the fit judges of military merit, and of the qualifications necessary for a commander-in-chief? Are they paid by the state, and do they therefore lose the rights of men? They are a part of that nation themselves, and contribute to that pay. And is not the king, is not the National Assembly, and are not all who elect the National Assembly, likewise paid? Instead of seeing all these forfeit their rights by their receiving a salary, they perceive that in all these cases a salary is given for the exercise of those rights. All your resolutions, all your proceedings, all your debates, all the works of your doctors in religion and politics, have industriously been put into their hands; and you expect that they will apply to their own case just as much of your doctrines and examples as suits your pleasure.

Everything depends upon the army in such a government as yours; for you have industriously destroyed all the opinions and prejudices, and, as far as in you lay, all the instincts which support government. Therefore the moment any difference arises between your National Assembly and any part of the nation, you must have recourse to force. Nothing else is left to you,—or rather, you have left nothing else to yourselves. You see, by the report of your war minister, that the distribution of the army is in a great measure made with a view of internal coercion.
%[129]
\footnote{ Courrier François, 30 July, 1790. Assemblée Nationale, Numero 210.}
 You must rule by an army; and you have infused into that army by which you rule, as well as into the whole body of the nation, principles which after a time must disable you in the use you resolve to make of it. The king is to call out troops to act against his people, when the world has been told, and the assertion is still ringing in our ears, that troops ought not to fire on citizens. The colonies assert to themselves an independent constitution and a free trade. They must be constrained by troops. In what chapter of your code of the rights of men are they able to read that it is a part of the rights of men to have their commerce monopolized and restrained for the benefit of others? As the colonists rise on you, the negroes rise on them. Troops again,—massacre, torture, hanging! These are your rights of men! These are the fruits of metaphysic declarations wantonly made and shamefully retracted! It was but the other day that the farmers of land in one of your provinces refused to pay some sorts of rents to the lord of the soil. In consequence of this, you decree that the country-people shall pay all rents and dues, except those which as grievances you have abolished; and if they refuse, then you order the king to march troops against them. You lay down metaphysic propositions which infer universal consequences, and then you attempt to limit logic by despotism. The leaders of the present system tell them of their rights, as men, to take fortresses, to murder guards, to seize on kings without the least appearance of authority even from the Assembly, whilst, as the sovereign legislative body, that Assembly was sitting in the name of the nation; and yet these leaders presume to order out the troops which have acted in these very disorders, to coerce those who shall judge on the principles and follow the examples which have been guarantied by their own approbation.

The leaders teach the people to abhor and reject all feodality as the barbarism of tyranny; and they tell them afterwards how much of that barbarous tyranny they are to bear with patience. As they are prodigal of light with regard to grievances, so the people find them sparing in the extreme with regard to redress. They know that not only certain quit-rents and personal duties, which you have permitted them to redeem, (but have furnished no money for the redemption,) are as nothing to those burdens for which you have made no provision at all; they know that almost the whole system of landed property in its origin is feudal,—that it is the distribution of the possessions of the original proprietors made by a barbarous conqueror to his barbarous instruments,—and that the most grievous effects of the conquest axe the land-rents of every kind, as without question they are.

The peasants, in all probability, are the descendants of these ancient proprietors, Romans or Gauls. But if they fail, in any degree, in the titles which they make on the principles of antiquaries and lawyers, they retreat into the citadel of the rights of men. There they find that men are equal; and the earth, the kind and equal mother of all, ought not to be monopolized to foster the pride and luxury of any men, who by nature are no better than themselves, and who, if they do not labor for their bread, are worse. They find, that, by the laws of Nature, the occupant and subduer of the soil is the true proprietor,—that there is no prescription against Nature,—and that the agreements (where any there are) which have been made with the landlords during the time of slavery are only the effect of duresse and force,—and that, when the people reëntered into the rights of men, those agreements were made as void as everything else which had been settled under the prevalence of the old feudal and aristocratic tyranny. They will tell you that they see no difference between an idler with a hat and a national cockade and an idler in a cowl or in a rochet. If you ground the title to rents on succession and prescription, they tell you from the speech of M. Camus, published by the National Assembly for their information, that things ill begun cannot avail themselves of prescription,—that the title of those lords was vicious in its origin,—and that force is at least as bad as fraud. As to the title by succession, they will tell you that the succession of those who have cultivated the soil is the true pedigree of property, and not rotten parchments and silly substitutions,—that the lords have enjoyed their usurpation too long,—and that, if they allow to these lay monks any charitable pension, they ought to be thankful to the bounty of the true proprietor, who is so generous towards a false claimant to his goods.

When the peasants give you back that coin of sophistic reason on which you have set your image and superscription, you cry it down as base money, and tell them you will pay for the future with French guards and dragoons and hussars. You hold up, to chastise them, the second-hand authority of a king, who is only the instrument of destroying, without any power of protecting either the people or his own person. Through him, it seems, you will make yourselves obeyed. They answer,—"You have taught us that there are no gentlemen; and which of your principles teach us to bow to kings whom we have not elected? We know, without your teaching, that lands were given for the support of feudal dignities, feudal titles, and feudal offices. When you took down the cause as a grievance, why should the more grievous effect remain? As there are now no hereditary honors and no distinguished families, why are we taxed to maintain what you tell us ought not to exist? You have sent down our old aristocratic landlords in no other character and with no other title but that of exactors under your authority. Have you endeavored to make these your rent-gatherers respectable to us? No. You have sent them to us with their arms reversed, their shields broken, their impresses defaced,—and so displumed, degraded, and metamorphosed, such unfeathered two-legged things, that we no longer know them. They are strangers to us. They do not even go by the names of our ancient lords. Physically they may be the same men,—though we are not quite sure of that, on your new philosophic doctrines of personal identity. In all other respects they are totally changed. We do not see why we have not as good a right to refuse them their rents as you have to abrogate all their honors, titles, and distinctions. This we have never commissioned you to do; and it is one instance among many, indeed, of your assumption of undelegated power. We see the burghers of Paris, through their clubs, their mobs, and their national guards, directing you at their pleasure, and giving that as law to you, which, under your authority, is transmitted as law to us. Through you, these burghers dispose of the lives and fortunes of us all. Why should not you attend as much to the desires of the laborious husbandman with regard to our rent, by which we are affected in the most serious manner, as you do to the demands of these insolent burghers relative to distinctions and titles of honor, by which neither they nor we are affected at all? But we find you pay more regard to their fancies than to our necessities. Is it among the rights of man to pay tribute to his equals? Before this measure of yours we might have thought we were not perfectly equal; we might have entertained some old, habitual, unmeaning prepossession in favor of those landlords; but we cannot conceive with what other view than that of destroying all respect to them you could have made the law that degrades them. You have forbidden us to treat them with any of the old formalities of respect; and now you send troops to sabre and to bayonet us into a submission to fear and force which you did not suffer us to yield to the mild authority of opinion."

The ground of some of these arguments is horrid and ridiculous to all rational ears; but to the politicians of metaphysics, who have opened schools for sophistry, and made establishments for anarchy, it is solid and conclusive. It is obvious, that, on a mere consideration of the right, the leaders in the Assembly would not in the least have scrupled to abrogate the rents along with the titles and family ensigns. It would be only to follow up the principle of their reasonings, and to complete the analogy of their conduct. But they had newly possessed themselves of a great body of landed property by confiscation. They had this commodity at market; and the market would have been wholly destroyed, if they were to permit the husbandmen to riot in the speculations with which they so freely intoxicated themselves. The only security which property enjoys in any one of its descriptions is from the interests of their rapacity with regard to some other. They have left nothing but their own arbitrary pleasure to determine what property is to be protected and what subverted.

Neither have they left any principle by which any of their municipalities can be bound to obedience,—or even conscientiously obliged not to separate from the whole, to become independent, or to connect itself with some other state. The people of Lyons, it seems, have refused lately to pay taxes. Why should they not? What lawful authority is there left to exact them? The king imposed some of them. The old States, methodized by orders, settled the more ancient. They may say to the Assembly,—"Who are you, that are not our kings, nor the States we have elected, nor sit on the principles on which we have elected you? And who are we, that, when we see the gabelles which you have ordered to be paid wholly shaken off, when we see the act of disobedience afterwards ratified by yourselves, who are we, that we are not to judge what taxes we ought or ought not to pay, and are not to avail ourselves of the same powers the validity of which you have approved in others?" To this the answer is, "We will send troops." The last reason of kings is always the first with your Assembly. This military aid may serve for a time, whilst the impression of the increase of pay remains, and the vanity of being umpires in all disputes is flattered. But this weapon will snap short, unfaithful to the hand that employs it. The Assembly keep a school, where, systematically, and with unremitting perseverance, they teach principles and form regulations destructive to all spirit of subordination, civil and military,—and then they expect that they shall hold in obedience an anarchic people by an anarchic army.

The municipal army, which, according to their new policy, is to balance this national army, if considered in itself only, is of a constitution much more simple, and in every respect less exceptionable. It is a mere democratic body, unconnected with the crown or the kingdom, armed and trained and officered at the pleasure of the districts to which the corps severally belong; and the personal service of the individuals who compose, or the fine in lieu of personal service, are directed by the same authority.
%[130]
\footnote{ I see by M. Necker's account, that the national guards of Paris have received, over and above the money levied within their own city, about 145,000l. sterling out of the public treasure. Whether this be an actual payment for the nine months of their existence, or an estimate of their yearly charge, I do not clearly perceive. It is of no great importance, as certainly they may take whatever they please.}
 Nothing is more uniform. If, however, considered in any relation to the crown, to the National Assembly, to the public tribunals, or to the other army, or considered in a view to any coherence or connection between its parts, it seems a monster, and can hardly fail to terminate its perplexed movements in some great national calamity. It is a worse preservative of a general constitution than the systasis of Crete, or the confederation of Poland, or any other ill-devised corrective which has yet been imagined, in the necessities produced by an ill-constructed system of government.

Having concluded my few remarks on the constitution of the supreme power, the executive, the judicature, the military, and on the reciprocal relation of all these establishments, I shall say something of the ability showed by your legislators with regard to the revenue.

In their proceedings relative to this object, if possible, still fewer traces appear of political judgment or financial resource. When the States met, it seemed to be the great object to improve the system of revenue, to enlarge its collection, to cleanse it of oppression and vexation, and to establish it on the most solid footing. Great were the expectations entertained on that head throughout Europe. It was by this grand arrangement that France was to stand or fall; and this became, in my opinion very properly, the test by which the skill and patriotism of those who ruled in that Assembly would be tried. The revenue of the state is the state. In effect, all depends upon it, whether for support or for reformation. The dignity of every occupation wholly depends upon the quantity and the kind of virtue that may be exerted in it. As all great qualities of the mind which operate in public, and are not merely suffering and passive, require force for their display, I had almost said for their unequivocal existence, the revenue, which is the spring of all power, becomes in its administration the sphere of every active virtue. Public virtue, being of a nature magnificent and splendid, instituted for great things, and conversant about great concerns, requires abundant scope and room, and cannot spread and grow under confinement, and in circumstances straitened, narrow, and sordid. Through the revenue alone the body politic can act in its true genius and character; and therefore it will display just as much of its collective virtue, and as much of that virtue which may characterize those who move it, and are, as it were, its life and guiding principle, as it is possessed of a just revenue. For from hence not only magnanimity, and liberality, and beneficence, and fortitude, and providence, and the tutelary protection of all good arts derive their food, and the growth of their organs, but continence, and self-denial, and labor, and vigilance, and frugality, and whatever else there is in which the mind shows itself above the appetite, are nowhere more in their proper element than in the provision and distribution of the public wealth. It is therefore not without reason that the science of speculative and practical finance, which must take to its aid so many auxiliary branches of knowledge, stands high in the estimation not only of the ordinary sort, but of the wisest and best men; and as this science has grown with the progress of its object, the prosperity and improvement of nations has generally increased with the increase of their revenues; and they will both continue to grow and flourish as long as the balance between what is left to strengthen the efforts of individuals and what is collected for the common efforts of the state bear to each other a due reciprocal proportion, and are kept in a close correspondence and communication. And perhaps it may be owing to the greatness of revenues, and to the urgency of state necessities, that old abuses in the constitution of finances are discovered, and their true nature and rational theory comes to be more perfectly understood; insomuch that a smaller revenue might have been more distressing in one period than a far greater is found to be in another, the proportionate wealth even remaining the same. In this state of things, the French Assembly found something in their revenues to preserve, to secure, and wisely to administer, as well as to abrogate and alter. Though their proud assumption might justify the severest tests, yet, in trying their abilities on their financial proceedings, I would only consider what is the plain, obvious duty of a common finance minister, and try them upon that, and not upon models of ideal perfection.

The objects of a financier are, then, to secure an ample revenue; to impose it with judgment and equality; to employ it economically; and when necessity obliges him to make use of credit, to secure its foundations in that instance, and forever, by the clearness and candor of his proceedings, the exactness of his calculations, and the solidity of his funds. On these heads we may take a short and distinct view of the merits and abilities of those in the National Assembly who have taken to themselves the management of this arduous concern.

Far from any increase of revenue in their hands, I find, by a report of M. Vernier, from the Committee of Finances, of the second of August last, that the amount of the national revenue, as compared with its produce before the Revolution, was diminished by the sum of two hundred millions, or eight millions sterling, of the annual income,—considerably more than one third of the whole.

If this be the result of great ability, never surely was ability displayed in a more distinguished manner or with so powerful an effect. No common folly, no vulgar incapacity, no ordinary official negligence, even no official crime, no corruption, no peculation, hardly any direct hostility, which we have seen in the modern world, could in so short a time have made so complete an overthrow of the finances, and, with them, of the strength of a great kingdom.—Cedo quî vestram rempublicam tantam amisistis tam cito?

The sophisters and declaimers, as soon as the Assembly met, began with decrying the ancient constitution of the revenue in many of its most essential branches, such as the public monopoly of salt. They charged it, as truly as unwisely, with being ill-contrived, oppressive, and partial. This representation they were not satisfied to make use of in speeches preliminary to some plan of reform; they declared it in a solemn resolution or public sentence, as it were judicially passed upon it; and this they dispersed throughout the nation. At the time they passed the decree, with the same gravity they ordered the same absurd, oppressive, and partial tax to be paid, until they could find a revenue to replace it. The consequence was inevitable. The provinces which had been always exempted from this salt monopoly, some of whom were charged with other contributions, perhaps equivalent, were totally disinclined to bear any part of the burden, which by an equal distribution was to redeem the others. As to the Assembly, occupied as it was with the declaration and violation of the rights of men, and with their arrangements for general confusion, it had neither leisure nor capacity to contrive, nor authority to enforce, any plan of any kind relative to the replacing the tax, or equalizing it, or compensating the provinces, or for conducting their minds to any scheme of accommodation with the other districts which were to be relieved. The people of the salt provinces, impatient under taxes damned by the authority which had directed their payment, very soon found their patience exhausted. They thought themselves as skilful in demolishing as the Assembly could be. They relieved themselves by throwing off the whole burden. Animated by this example, each district, or part of a district, judging of its own grievance by its own feeling, and of its remedy by its own opinion, did as it pleased with other taxes.

We are next to see how they have conducted themselves in contriving equal impositions, proportioned to the means of the citizens, and the least likely to lean heavy on the active capital employed in the generation of that private wealth from whence the public fortune must be derived. By suffering the several districts, and several of the individuals in each district, to judge of what part of the old revenue they might withhold, instead of better principles of equality, a new inequality was introduced of the most oppressive kind. Payments were regulated by dispositions. The parts of the kingdom which were the most submissive, the most orderly, or the most affectionate to the commonwealth, bore the whole burden of the state. Nothing turns out to be so oppressive and unjust as a feeble government. To fill up all the deficiencies in the old impositions, and the new deficiencies of every kind which were to be expected, what remained to a state without authority? The National Assembly called for a voluntary benevolence,—for a fourth part of the income of all the citizens, to be estimated on the honor of those who were to pay. They obtained something more than could be rationally calculated, but what was far indeed from answerable to their real necessities, and much less to their fond expectations. Rational people could have hoped for little from this their tax in the disguise of a benevolence,—tax weak, ineffective, and unequal,—a tax by which luxury, avarice, and selfishness were screened, and the load thrown upon productive capital, upon integrity, generosity, and public spirit,—a tax of regulation upon virtue. At length the mask is thrown off, and they are now trying means (with little success) of exacting their benevolence by force.

This benevolence, the rickety offspring of weakness, was to be supported by another resource, the twin brother of the same prolific imbecility. The patriotic donations were to make good the failure of the patriotic contribution. John Doe was to become security for Richard Roe. By this scheme they took things of much price from the giver, comparatively of small value to the receiver; they ruined several trades; they pillaged the crown of its ornaments, the churches of their plate, and the people of their personal decorations. The invention of those juvenile pretenders to liberty was in reality nothing more than a servile imitation of one of the poorest resources of doting despotism. They took an old, huge, full-bottomed periwig out of the wardrobe of the antiquated frippery of Louis the Fourteenth, to cover the premature baldness of the National Assembly. They produced this old-fashioned formal folly, though it had been so abundantly exposed in the Memoirs of the Duke de Saint-Simon,—if to reasonable men it had wanted any arguments to display its mischief and insufficiency. A device of the same kind was tried in my memory by Louis the Fifteenth, but it answered at no time. However, the necessities of ruinous wars were some excuse for desperate projects. The deliberations of calamity are rarely wise. But here was a season for disposition and providence. It was in a time of profound peace, then enjoyed for five years, and promising a much longer continuance, that they had recourse to this desperate trifling. They were sure to lose more reputation by sporting, in their serious situation, with these toys and playthings of finance, which have filled half their journals, than could possibly be compensated by the poor temporary supply which they afforded. It seemed as if those who adopted such projects were wholly ignorant of their circumstances, or wholly unequal to their necessities. Whatever virtue may be in these devices, it is obvious that neither the patriotic gifts nor the patriotic contribution can ever be resorted to again. The resources of public folly are soon exhausted. The whole, indeed, of their scheme of revenue is to make, by any artifice, an appearance of a full reservoir for the hour, whilst at the same time they cut off the springs and living fountains of perennial supply. The account not long since furnished by M. Necker was meant, without question, to be favorable. He gives a flattering view of the means of getting through the year; but he expresses, as it is natural he should, some apprehension for that which was to succeed. On this last prognostic, instead of entering into the grounds of this apprehension, in order, by a proper foresight, to prevent the prognosticated evil, M. Necker receives a sort of friendly reprimand from the President of the Assembly.

As to their other schemes of taxation, it is impossible to say anything of them with certainty, because they have not yet had their operation; but nobody is so sanguine as to imagine they will fill up any perceptible part of the wide gaping breach which their incapacity has made in their revenues. At present the state of their treasury sinks every day more and more in cash, and swells more and more in fictitious representation. When so little within or without is now found but paper, the representative not of opulence, but of want, the creature not of credit, but of power, they imagine that our flourishing state in England is owing to that bank-paper, and not the bank-paper to the flourishing condition of our commerce, to the solidity of our credit, and to the total exclusion of all idea of power from any part of the transaction. They forget that in England not one shilling of paper money of any description is received but of choice,—that the whole has had its origin in cash actually deposited,—and that it is convertible at pleasure, in an instant, and without the smallest loss, into cash again. Our paper is of value in commerce, because in law it is of none. It is powerful on 'Change, because in Westminster Hall it is impotent. In payment of a debt of twenty shillings a creditor may refuse all the paper of the Bank of England. Nor is there amongst us a single public security, of any quality or nature whatsoever, that is enforced by authority. In fact, it might be easily shown that our paper wealth, instead of lessening the real coin, has a tendency to increase it,—instead of being a substitute for money, it only facilitates its entry, its exit, and its circulation,—that it is the symbol of prosperity, and not the badge of distress. Never was a scarcity of cash and an exuberance of paper a subject of complaint in this nation.

Well! but a lessening of prodigal expenses, and the economy which has been introduced by the virtuous and sapient Assembly, make amends for the losses sustained in the receipt of revenue. In this at least they have fulfilled the duty of a financier.—Have those who say so looked at the expenses of the National Assembly itself? of the municipalities? of the city of Paris? of the increased pay of the two armies? of the new police? of the new judicatures? Have they even carefully compared the present pension-list with the former? These politicians have been cruel, not economical. Comparing the expenses of the former prodigal government and its relation to the then revenues with the expenses of this new system as opposed to the state of its new treasury, I believe the present will be found beyond all comparison more chargeable.
%[131]
\footnote{ The reader will observe that I have but lightly touched (my plan demanded nothing more) on the condition of the French finances as connected with the demands upon them. If I had intended to do otherwise, the materials in my hands for such a task are not altogether perfect. On this subject I refer the reader to M. de Calonne's work, and the tremendous display that he has made of the havoc and devastation in the public estate, and in all the affairs of France, caused by the presumptuous good intentions of ignorance and incapacity. Such effects those causes will always produce. Looking over that account with a pretty strict eye, and, with perhaps too much rigor, deducting everything which may be placed to the account of a financier out of place, who might be supposed by his enemies desirous of making the most of his cause, I believe it will be found that a more salutary lesson of caution against the daring spirit of innovators than what has been supplied at the expense of France never was at any time furnished to mankind.}


It remains only to consider the proofs of financial ability furnished by the present French managers when they are to raise supplies on credit. Here I am a little at a stand; for credit, properly speaking, they have none. The credit of the ancient government was not, indeed, the best; but they could always, on some terms, command money, not only at home, but from most of the countries of Europe where a surplus capital was accumulated; and the credit of that government was improving daily. The establishment of a system of liberty would of course be supposed to give it new strength: and so it would actually have done, if a system of liberty had been established. What offers has their government of pretended liberty had from Holland, from Hamburg, from Switzerland, from Genoa, from England, for a dealing in their paper? Why should these nations of commerce and economy enter into any pecuniary dealings with a people who attempt to reverse the very nature of things,—amongst whom they see the debtor prescribing at the point of the bayonet the medium of his solvency to the creditor, discharging one of his engagements with another, turning his very penury into his resource, and paying his interest with his rags?

Their fanatical confidence in the omnipotence of Church plunder has induced these philosophers to overlook all care of the public estate, just as the dream of the philosopher's stone induces dupes, under the more plausible delusion of the hermetic art, to neglect all rational means of improving their fortunes. With these philosophic financiers, this universal medicine made of Church mummy is to cure all the evils of the state. These gentlemen perhaps do not believe a great deal in the miracles of piety; but it cannot be questioned that they have an undoubting faith in the prodigies of sacrilege. Is there a debt which presses them? Issue assignats. Are compensations to be made or a maintenance decreed to those whom they have robbed of their freehold in their office or expelled from their profession? Assignats. Is a fleet to be fitted out? Assignats. If sixteen millions sterling of these assignats forced on the people leave the wants of the state as urgent as ever, Issue, says one, thirty millions sterling of assignats,—says another, Issue fourscore millions more of assignats. The only difference among their financial factions is on the greater or the lesser quantity of assignats to be imposed on the public sufferance. They are all professors of assignats. Even those whose natural good sense and knowledge of commerce, not obliterated by philosophy, furnish decisive arguments against this delusion, conclude their arguments by proposing the emission of assignats. I suppose they must talk of assignats, as no other language would be understood. All experience of their inefficacy does not in the least discourage them. Are the old assignats depreciated at market? What is the remedy? Issue new assignats.—Mais si maladia opiniatria non vult se garire, quid illi facere? Assignare; postea assignare; ensuita assignare. The word is a trifle altered. The Latin of your present doctors may be better than that of your old comedy; their wisdom and the variety of their resources are the same. They have not more notes in their song than the cuckoo; though, far from the softness of that harbinger of summer and plenty, their voice is as harsh and as ominous as that of the raven.

Who but the most desperate adventurers in philosophy and finance could at all have thought of destroying the settled revenue of the state, the sole security for the public credit, in the hope of rebuilding it with the materials of confiscated property? If, however, an excessive zeal for the state should have led a pious and venerable prelate (by anticipation a father of the Church
%[132]
\footnote{ La Bruyère of Bossuet.}
) to pillage his own order, and, for the good of the Church and people, to take upon himself the place of grand financier of confiscation and comptroller-general of sacrilege, he and his coadjutors were, in my opinion, bound to show, by their subsequent conduct, that they knew something of the office they assumed. When they had resolved to appropriate to the fisc a certain portion of the landed property of their conquered country, it was their business to render their bank a real fund of credit,—as far as such a bank was capable of becoming so.

To establish a current circulating credit upon any land-bank, under any circumstances whatsoever, has hitherto proved difficult at the very least. The attempt has commonly ended in bankruptcy. But when the Assembly were led, through a contempt of moral, to a defiance of economical principles, it might at least have been expected that nothing would be omitted on their part to lessen this difficulty, to prevent any aggravation of this bankruptcy. It might be expected, that, to render your land-bank tolerable, every means would be adopted that could display openness and candor in the statement of the security, everything which could aid the recovery of the demand. To take things in their most favorable point of view, your condition was that of a man of a large landed estate which he wished to dispose of for the discharge of a debt and the supply of certain services. Not being able instantly to sell, you wished to mortgage. What would a man of fair intentions and a commonly clear understanding do in such circumstances? Ought he not first to ascertain the gross value of the estate, the charges of its management and disposition, the incumbrances perpetual and temporary of all kinds that affect it,—then, striking a net surplus, to calculate the just value of the security? When that surplus (the only security to the creditor) had been clearly ascertained, and properly vested in the hands of trustees, then he would indicate the parcels to be sold, and the time and conditions of sale; after this he would admit the public creditor, if he chose it, to subscribe his stock into this new fund,—or he might receive proposals for an assignat from those who would advance money to purchase this species of security. This would be to proceed like men of business, methodically and rationally, and on the only principles of public and private credit that have an existence. The dealer would then know exactly what he purchased; and the only doubt which could hang upon his mind would be the dread of the resumption of the spoil, which one day might be made (perhaps with an addition of punishment) from the sacrilegious gripe of those execrable wretches who could become purchasers at the auction of their innocent fellow-citizens.

An open, and exact statement of the clear value of the property, and of the time, the circumstances, and the place of sale, were all necessary, to efface as much as possible the stigma that has hitherto been branded on every kind of land-bank. It became necessary on another principle,—that is, on account of a pledge of faith previously given on that subject, that their future fidelity in a slippery concern might be established by their adherence to their first engagement. When they had finally determined on a state resource from Church booty, they came, on the fourteenth of April, 1790, to a solemn resolution on the subject, and pledged themselves to their country, "that, in the statement of the public charges for each year, there should be brought to account a sum sufficient for defraying the expenses of the R.C.A. religion, the support of the ministers at the altars, the relief of the poor, the pensions to the ecclesiastics, secular as well as regular, of the one and of the other sex, in order that the estates and goods which are at the disposal of the nation may be disengaged of all charges, and employed by the representatives, or the legislative body, to the great and most pressing exigencies of the state." They further engaged, on the same day, that the sum necessary for the year 1791 should be forthwith determined.

In this resolution they admit it their duty to show distinctly the expense of the above objects, which, by other resolutions, they had before engaged should be first in the order of provision. They admit that they ought to show the estate clear and disengaged of all charges, and that they should show it immediately. Have they done this immediately, or at any time? Have they ever furnished a rent-roll of the immovable estate, or given in an inventory of the movable effects, which they confiscate to their assignats? In what manner they can fulfil their engagements of holding out to public service "an estate disengaged of all charges," without authenticating the value of the estate or the quantum of the charges, I leave it to their English admirers to explain. Instantly upon this assurance, and previously to any one step towards making it good, they issue, on the credit of so handsome a declaration, sixteen millions sterling of their paper. This was manly. Who, after this masterly stroke, can doubt of their abilities in finance?—But then, before any other emission of these financial indulgences, they took care at least to make good their original promise.—If such estimate, either of the value of the estate or the amount of the incumbrances, has been made, it has escaped me. I never heard of it.

At length they have spoken out, and they have made a full discovery of their abominable fraud in holding out the Church lands as a security for any debts or any service whatsoever. They rob only to enable them to cheat; but in a very short time they defeat the ends both of the robbery and the fraud, by making out accounts for other purposes, which blow up their whole apparatus of force and of deception. I am obliged to M. de Calonne for his reference to the document which proves this extraordinary fact: it had by some means escaped me. Indeed, it was not necessary to make out my assertion as to the breach of faith on the declaration of the fourteenth of April, 1790. By a report of their committee it now appears that the charge of keeping up the reduced ecclesiastical establishments, and other expenses attendant on religion, and maintaining the religious of both sexes, retained or pensioned, and the other concomitant expenses of the same nature, which they have brought upon themselves by this convulsion in property, exceeds the income of the estates acquired by it in the enormous sum of two millions sterling annually,—besides a debt of seven millions and upwards. These are the calculating powers of imposture! This is the finance of philosophy! This is the result of all the delusions held out to engage a miserable people in rebellion, murder, and sacrilege, and to make them prompt and zealous instruments in the ruin of their country! Never did a state, in any case, enrich itself by the confiscations of the citizens. This new experiment has succeeded like all the rest. Every honest mind, every true lover of liberty and humanity, must rejoice to find that injustice is not always good policy, nor rapine the high-road to riches. I subjoin with pleasure, in a note, the able and spirited observations of M. de Calonne on this subject.
%[133]
\footnote{ 'Ce n'est point à l'assemblée entière que je m'adresse ici; je ne parle qu'à ceux qui l'égarent, en lui cachant sous des gazes séduisantes le but où ils l'entraînent. C'est à eux que je dis: Votre objet, vous n'en disconviendrez pas, c'est d'ôter tout espoir au clergé, et de consommer sa ruine; c'est-là, en ne vous soupçonnant d'aucune combinaison de cupidité, d'aucun regard sur le jeu des effets publics, c'est-là ce qu'on doit croire que vous avez en vue dans la terrible opération que vous proposez; c'est ce qui doit en être le fruit. Mais le peuple qui vous y intéressez, quel avantage peut-il y trouver? En vous servant sans cesse de lui, que faites-vous pour lui? Rien, absolument rien; et, au contraire, vous faites ce qui ne conduit qu'à l'accabler de nouvelles charges. Vous avez rejeté, à son préjudice, une offre de 400 millions, dont l'acceptation pouvoit devenir un moyen de soulagement en sa faveur; et à cette ressource, aussi profitable que légitime, vous avez substitué une injustice ruineuse, qui, de votre propre aveu, charge le trésor public, et par consequent le peuple, d'un surcroît de dépense annuelle de 50 millions an moins, et d'un remboursement de 150 millions.

Malheureux peuple! voilà ce que vous vaut en dernier résultat l'expropriation de l'Église, et la dureté des décrets taxateurs du traitement des ministres d'une religion bienfaisante; et désormais ils scront à votre charge: leurs charités soulageoient les pauvres; et vous allez être imposés pour subvenir à leur entretien!'—De l'État de la France, p. 81. See also p. 92, and the following pages.
}


In order to persuade the world of the bottomless resource of ecclesiastical confiscation, the Assembly have proceeded to other confiscations of estates in offices, which could not be done with any common color without being compensated out of this grand confiscation of landed property. They have thrown upon this fund, which was to show a surplus disengaged of all charges, a new charge, namely, the compensation to the whole body of the disbanded judicature, and of all suppressed offices and estates: a charge which I cannot ascertain, but which unquestionably amounts to many French millions. Another of the new charges is an annuity of four hundred and eighty thousand pounds sterling, to be paid (if they choose to keep faith) by daily payments, for the interest of the first assignats. Have they ever given themselves the trouble to state fairly the expense of the management of the Church lands in the hands of the municipalities, to whose care, skill, and diligence, and that of their legion of unknown under-agents, they have chosen to commit the charge of the forfeited estates, and the consequence of which had been so ably pointed out by the Bishop of Nancy?

But it is unnecessary to dwell on these obvious heads of incumbrance. Have they made out any clear state of the grand incumbrance of all, I mean the whole of the general and municipal establishments of all sorts, and compared it with the regular income by revenue? Every deficiency in these becomes a charge on the confiscated estate, before the creditor can plant his cabbages on an acre of Church property. There is no other prop than this confiscation to keep the whole state from tumbling to the ground. In this situation they have purposely covered all, that they ought industriously to have cleared, with a thick fog; and then, blindfold themselves, like bulls that shut their eyes when they push, they drive, by the point of the bayonets, their slaves, blindfolded indeed no worse than their lords, to take their fictions for currencies, and to swallow down paper pills by thirty-four millions sterling at a dose. Then they proudly lay in their claim to a future credit, on failure of all their past engagements, and at a time when (if in such a matter anything can be clear) it is clear that the surplus estates will never answer even the first of their mortgages,—I mean that of the four hundred millions (or sixteen millions sterling) of assignats. In all this procedure I can discern neither the solid sense of plain dealing nor the subtle dexterity of ingenious fraud. The objections within the Assembly to pulling up the flood-gates for this inundation of fraud are unanswered; but they are thoroughly refuted by an hundred thousand financiers in the street. These are the numbers by which the metaphysic arithmeticians compute. These are the grand calculations on which a philosophical public credit is founded in France. They cannot raise supplies; but they can raise mobs. Let them rejoice in the applauses of the club at Dundee for their wisdom and patriotism in having thus applied the plunder of the citizens to the service of the state. I hear of no address upon this subject from the directors of the Bank of England,—though their approbation would be of a little more weight in the scale of credit than that of the club at Dundee. But to do justice to the club, I believe the gentlemen who compose it to be wiser than they appear,—that they will be less liberal of their money than of their addresses, and that they would not give a dog's ear of their most rumpled and ragged Scotch paper for twenty of your fairest assignats.

Early in this year the Assembly issued paper to the amount of sixteen millions sterling. What must have been the state into which the Assembly has brought your affairs, that the relief afforded by so vast a supply has been hardly perceptible? This paper also felt an almost immediate depreciation of five per cent, which in a little time came to about seven. The effect of these assignats on the receipt of the revenue is remarkable. M. Necker found that the collectors of the revenue, who received in coin, paid the treasury in assignats. The collectors made seven per cent by thus receiving in money, and accounting in depreciated paper. It was not very difficult to foresee that this must be inevitable. It was, however, not the less embarrassing. M. Necker was obliged (I believe, for a considerable part, in the market of London) to buy gold and silver for the mint, which amounted to about twelve thousand pounds above the value of the commodity gained. That minister was of opinion, that, whatever their secret nutritive virtue might be, the state could not live upon assignats alone,—that some real silver was necessary, particularly for the satisfaction of those who, having iron in their hands, were not likely to distinguish themselves for patience, when they should perceive, that, whilst an increase of pay was held out to them in real money, it was again to be fraudulently drawn back by depreciated paper. The minister, in this very natural distress, applied to the Assembly, that they should order the collectors to pay in specie what in specie they had received. It could not escape him, that, if the Treasury paid three per cent for the use of a currency which should be returned seven per cent worse than the minister issued it, such a dealing could not very greatly tend to enrich the public. The Assembly took no notice of his recommendation. They were in this dilemma: If they continued to receive the assignats, cash must become an alien to their Treasury; if the Treasury should refuse those paper amulets, or should discountenance them in any degree, they must destroy the credit of their sole resource. They seem, then, to have made their option, and to have given some sort of credit to their paper by taking it themselves; at the same time, in their speeches, they made a sort of swaggering declaration, something, I rather think, above legislative competence,—that is, that there is no difference in value between metallic money and their assignats. This was a good, stout, proof article of faith, pronounced under an anathema by the venerable fathers of this philosophic synod. Credat who will,—certainly not Judæus Apella.

A noble indignation rises in the minds of your popular leaders, on hearing the magic-lantern in their show of finance compared to the fraudulent exhibitions of Mr. Law. They cannot bear to hear the sands of his Mississippi compared with the rock of the Church, on which they build their system. Pray let them suppress this glorious spirit, until they show to the world what piece of solid ground there is for their assignats, which they have not preoccupied by other charges. They do injustice to that great mother fraud, to compare it with their degenerate imitation. It is not true that Law built solely on a speculation concerning the Mississippi. He added the East India trade; he added the African trade; he added the farms of all the farmed revenue of France. All these together unquestionably could not support the structure which the public enthusiasm, not he, chose to build upon these bases. But these were, however, in comparison, generous delusions. They supposed, and they aimed at, an increase of the commerce of France. They opened to it the whole range of the two hemispheres. They did not think of feeding France from its own substance. A grand imagination found in this flight of commerce something to captivate. It was wherewithal to dazzle the eye of an eagle. It was not made to entice the smell of a mole, nuzzling and burying himself in his mother earth, as yours is. Men were not then quite shrunk from their natural dimensions by a degrading and sordid philosophy, and fitted for low and vulgar deceptions. Above all, remember, that, in imposing on the imagination, the then managers of the system made a compliment to the freedom of men. In their fraud there was no mixture of force. This was reserved to our time, to quench the little glimmerings of reason which might break in upon the solid darkness of this enlightened age.

On recollection, I have said nothing of a scheme of finance which may be urged in favor of the abilities of these gentlemen, and which has been introduced with great pomp, though not yet finally adopted in the National Assembly. It comes with something solid in aid of the credit of the paper circulation; and much has been said of its utility and its elegance. I mean the project for coining into money the bells of the suppressed churches. This is their alchemy. There are some follies which baffle argument, which go beyond ridicule, and which excite no feeling in us but disgust; and therefore I say no more upon it.

It is as little worth remarking any farther upon all their drawing and re-drawing, on their circulation for putting off the evil day, on the play between the Treasury and the Caisse d'Escompte, and on all these old, exploded contrivances of mercantile fraud, now exalted into policy of state. The revenue will not be trifled with. The prattling about the rights of men will not be accepted in payment of a biscuit or a pound of gunpowder. Here, then, the metaphysicians descend from their airy speculations, and faithfully follow examples. What examples? The examples of bankrupts. But defeated, baffled, disgraced, when their breath, their strength, their inventions, their fancies desert them, their confidence still maintains its ground. In the manifest failure of their abilities, they take credit for their benevolence. When the revenue disappears in their hands, they have the presumption, in some of their late proceedings, to value themselves on the relief given to the people. They did not relieve the people. If they entertained such intentions, why did they order the obnoxious taxes to be paid? The people relieved themselves, in spite of the Assembly.

But waiving all discussion on the parties who may claim the merit of this fallacious relief, has there been, in effect, any relief to the people in any form? M. Bailly, one of the grand agents of paper circulation, lets you into the nature of this relief. His speech to the National Assembly contained a high and labored panegyric on the inhabitants of Paris, for the constancy and unbroken resolution with which they have borne their distress and misery. A fine picture of public felicity! What! great courage and unconquerable firmness of mind to endure benefits and sustain redress? One would think, from the speech of this learned lord mayor, that the Parisians, for this twelvemonth past, had been suffering the straits of some dreadful blockade,—that Henry the Fourth had been stopping up the avenues to their supply, and Sully thundering with his ordnance at the gates of Paris,—when in reality they are besieged by no other enemies than their own madness and folly, their own credulity and perverseness. But M. Bailly will sooner thaw the eternal ice of his Atlantic regions than restore the central heat to Paris, whilst it remains 'smitten with the cold, dry, petrific mace" of a false and unfeeling philosophy. Some time after this speech, that is, on the thirteenth of last August, the same magistrate, giving an account of his government at the bar of the same Assembly, expresses himself as follows:—"In the month of July, 1789," (the period of everlasting commemoration,) "the finances of the city of Paris were yet in good order; the expenditure was counterbalanced by the receipt, and she had at that time a million [forty thousand pounds sterling] in bank. The expenses which she has been constrained to incur, subsequent to the Revolution, amount to 2,500,000 livres. From these expenses, and the great falling off in the product of the free gifts, not only a momentary, but a total, want of money has taken place." This is the Paris upon whose nourishment, in the course of the last year, such immense sums, drawn from the vitals of all France, have been expended. As long as Paris stands in the place of ancient Rome, so long she will be maintained by the subject provinces. It is an evil inevitably attendant on the dominion of sovereign democratic republics. As it happened in Rome, it may survive that republican domination which gave rise to it. In that case despotism itself must submit to the vices of popularity. Rome, under her emperors, united the evils of both systems; and this unnatural combination was one great cause of her ruin.

To tell the people that they are relieved by the dilapidation of their public estate is a cruel and insolent imposition. Statesmen, before they valued themselves on the relief given to the people by the destruction of their revenue, ought first to have carefully attended to the solution of this problem:—Whether it be more advantageous to the people to pay considerably and to gain in proportion, or to gain little or nothing and to be disburdened of all contribution? My mind is made up to decide in favor of the first proposition. Experience is with me, and, I believe, the best opinions also. To keep a balance between the power of acquisition on the part of the subject and the demands he is to answer on the part of the state is the fundamental part of the skill of a true politician. The means of acquisition are prior in time and in arrangement. Good order is the foundation of all good things. To be enabled to acquire, the people, without being servile, must be tractable and obedient. The magistrate must have his reverence, the laws their authority. The body of the people must not find the principles of natural subordination by art rooted out of their minds. They must respect that property of which they cannot partake. They must labor to obtain what by labor can be obtained; and when they find, as they commonly do, the success disproportioned to the endeavor, they must be taught their consolation in the final proportions of eternal justice. Of this consolation whoever deprives them deadens their industry, and strikes at the root of all acquisition as of all conservation. He that does this is the cruel oppressor, the merciless enemy of the poor and wretched; at the same time that by his wicked speculations he exposes the fruits of successful industry and the accumulations of fortune to the plunder of the negligent, the disappointed, and the unprosperous.

Too many of the financiers by profession are apt to see nothing in revenue but banks, and circulations, and annuities on lives, and tontines, and perpetual rents, and all the small wares of the shop. In a settled order of the state, these things are not to be slighted, nor is the skill in them to be held of trivial estimation. They are good, but then only good when they assume the effects of that settled order, and are built upon it. But when men think that these beggarly contrivances may supply a resource for the evils which result from breaking up the foundations of public order, and from causing or suffering the principles of property to be subverted, they will, in the ruin of their country, leave a melancholy and lasting monument of the effect of preposterous politics, and presumptuous, short-sighted, narrow-minded wisdom.

The effects of the incapacity shown by the popular leaders in all the great members of the commonwealth are to be covered with the "all-atoning name" of Liberty. In some people I see great liberty, indeed; in many, if not in the most, an oppressive, degrading servitude. But what is liberty without wisdom and without virtue? It is the greatest of all possible evils; for it is folly, vice, and madness, without tuition or restraint. Those who know what virtuous liberty is cannot bear to see it disgraced by incapable heads, on account of their having high-sounding words in their mouths. Grand, swelling sentiments of liberty I am sure I do not despise. They warm the heart; they enlarge and liberalize our minds; they animate our courage in a time of conflict. Old as I am, I read the fine raptures of Lucan and Corneille with pleasure. Neither do I wholly condemn the little arts and devices of popularity. They facilitate the carrying of many points of moment; they keep the people together; they refresh the mind in its exertions; and they diffuse occasional gayety over the severe brow of moral freedom. Every politician ought to sacrifice to the Graces, and to join compliance with reason. But in such an undertaking as that in France all these subsidiary sentiments and artifices are of little avail. To make a government requires no great prudence. Settle the seat of power, teach obedience, and the work is done. To give freedom is still more easy. It is not necessary to guide; it only requires to let go the rein. But to form a free government, that is, to temper together these opposite elements of liberty and restraint in one consistent work, requires much thought, deep reflection, a sagacious, powerful, and combining mind. This I do not find in those who take the lead in the National Assembly. Perhaps they are not so miserably deficient as they appear. I rather believe it. It would put them below the common level of human understanding. But when the leaders choose to make themselves bidders at an auction of popularity, their talents, in the construction of the state, will be of no service. They will become flatterers instead of legislators,—the instruments, not the guides of the people. If any of them should happen to propose a scheme of liberty soberly limited, and defined with proper qualifications, he will be immediately outbid by his competitors, who will produce something more splendidly popular. Suspicions will be raised of his fidelity to his cause. Moderation will be stigmatized as the virtue of cowards, and compromise as the prudence of traitors,—until, in hopes of preserving the credit which may enable him to temper and moderate on some occasions, the popular leader is obliged to become active in propagating doctrines and establishing powers that will afterwards defeat any sober purpose at which he ultimately might have aimed.

But am I so unreasonable as to see nothing at all that deserves commendation in the indefatigable labors of this Assembly? I do not deny, that, among an infinite number of acts of violence and folly, some good may have been done. They who destroy everything certainly will remove some grievance. They who make everything new have a chance that they may establish something beneficial. To give them credit for what they have done in virtue of the authority they have usurped, or to excuse them in the crimes by which that authority has been acquired, it must appear that the same things could not have been accomplished without producing such a revolution. Most assuredly they might; because almost every one of the regulations made by them, which is not very equivocal, was either in the cession of the king, voluntarily made at the meeting of the States, or in the concurrent instructions to the orders. Some usages have been abolished on just grounds; but they were such, that, if they had stood as they were to all eternity, they would little detract from the happiness and prosperity of any state. The improvements of the National Assembly are superficial, their errors fundamental.

Whatever they are, I wish my countrymen rather to recommend to our neighbors the example of the British Constitution than to take models from them for the improvement of our own. In the former they have got an invaluable treasure. They are not, I think, without some causes of apprehension and complaint; but these they do not owe to their Constitution, but to their own conduct. I think our happy situation owing to our Constitution,—but owing to the whole of it, and not to any part singly,—owing in a great measure to what we have left standing in our several reviews and reformations, as well as to what we have altered or superadded. Our people will find employment enough for a truly patriotic, free, and independent spirit, in guarding what they possess from violation. I would not exclude alteration neither; but even when I changed, it should be to preserve. I should be led to my remedy by a great grievance. In what I did, I should follow the example of our ancestors. I would make the reparation as nearly as possible in the style of the building. A politic caution, a guarded circumspection, a moral rather than a complexional timidity, were among the ruling principles of our forefathers in their most decided conduct. Not being illuminated with the light of which the gentlemen of France tell us they have got so abundant a, share, they acted under a strong impression of the ignorance and fallibility of mankind. He that had made them thus fallible rewarded them for having in their conduct attended to their nature. Let us imitate their caution, if we wish to deserve their fortune or to retain their bequests. Let us add, if we please, but let us preserve what they have left; and standing on the firm ground of the British Constitution, let us be satisfied to admire, rather than attempt to follow in their desperate flights, the aëronauts of France.

I have told you candidly my sentiments. I think they are not likely to alter yours. I do not know that they ought. You are young; you cannot guide, but must follow, the fortune of your country. But hereafter they may be of some use to you, in some future form which your commonwealth may take. In the present it can hardly remain; but before its final settlement, it may be obliged to pass, as one of our poets says, "through great varieties of untried being," and in all its transmigrations to be purified by fire and blood.

I have little to recommend my opinions but long observation and much impartiality. They come from one who has been no tool of power, no flatterer of greatness, and who in his last acts does not wish to belie the tenor of his life. They come from one almost the whole of whose public exertion has been a struggle for the liberty of others,—from one in whose breast no anger durable or vehement has ever been kindled but by what he considered as tyranny, and who snatches from his share in the endeavors which are used by good men to discredit opulent oppression the hours he has employed on your affairs, and who in so doing persuades himself he has not departed from his usual office. They come from one who desires honors, distinctions, and emoluments but little, and who expects them not at all,—who has no contempt for fame, and no fear of obloquy,—who shuns contention, though he will hazard an opinion; from one who wishes to preserve consistency, but who would preserve consistency by varying his means to secure the unity of his end,—and, when the equipoise of the vessel in which he sails may be endangered by overloading it upon one side, is desirous of carrying the small weight of his reasons to that which may preserve its equipoise.

%FOOTNOTES:
% [77] Ps. cxlix.

% [78] Discourse on the Love of our Country, Nov. 4, 1789, by Dr. Richard Price, 3d edition, p. 17 and 18.

% [79] "Those who dislike that mode of worship which is prescribed by public authority ought, if they can find no worship out of the Church which they approve, to set up a separate worship for themselves; and by doing this, and giving an example of a rational and manly worship, men of weight from their rank and literature may do the greatest service to society and the world."—P. 18, Dr. Price's Sermon.

% [80] P. 34, Discourse on the Love of our Country, by Dr. Price.

% [81] 1st Mary, sess. 3, ch. 1.

% [82] "That King James the Second, having endeavored to subvert the Constitution of the kingdom, by breaking the original contract between king and people, and, by the advice of Jesuits and other wicked persons, having violated the fundamental laws, and having withdrawn himself out of the kingdom, hath abdicated the government, and the throne is thereby vacant."

% [83] P. 23, 23, 24.

% [84] See Blackstone's Magna Charta, printed at Oxford, 1759.

% [85] 1 W. and M.

% [86] Ecclesiasticus, chap, xxxviii. ver. 24, 25. "The wisdom of a learned man cometh by opportunity of leisure: and he that hath little business shall become wise. How can he get wisdom that holdeth the plough, and that glorieth in the goad; that driveth oxen, and is occupied in their labors, and whose talk is of bullocks?"

%Ver. 27. "So every carpenter and workmaster, that laboreth night and day," \&c.

%Ver. 33. "They shall not be sought for in public counsel, nor sit high in the congregation: they shall not sit on the judge's seat, nor understand the sentence of judgment: they cannot declare justice and judgment, and they shall not be found where parables are spoken."

%Ver. 34. "But they will maintain the state of the world."

%I do not determine whether this book be canonical, as the Gallican Church (till lately) has considered it, or apocryphal, as here it is taken. I am sure it contains a great deal of sense and truth.

% [87] Discourse on the Love of our Country, 3rd edit p. 39.

% [88] Another of these reverend gentlemen, who was witness to some of the spectacles which Paris has lately exhibited, expresses himself thus:—"A king dragged in submissive triumph by his conquering subjects is one of those appearances of grandeur which seldom rise in the prospect of human affairs, and which, during the remainder of my life, I shall think of with wonder and gratification." These gentlemen agree marvellously in their feelings.

% [89] State Trials, Vol. II. p. 360, 363.

% [90] 6th of October, 1789.

% [91] "Tous les Évêques à la lanterne!"

% [92] It is proper here to refer to a letter written upon this subject by an eyewitness. That eyewitness was one of the most honest, intelligent, and eloquent members of the National Assembly, one of the most active and zealous reformers of the state. He was obliged to secede from the Assembly; and he afterwards became a voluntary exile, on account of the horrors of this pious triumph, and the dispositions of men, who, profiting of crimes, if not causing them, have taken the lead in public affairs.

%Extract of M. de Lally Tollendal's Second Letter to a Friend.

%"Parlons du parti que j'ai pris; il est bien justifé dans ma conscience.—Ni cette ville coupable, ni cette assemblée plus coupable encore, ne méritoient que je me justifie; mais j'ai à cœur que vous, et les personnes qui pensent comme vous, ne me condamnent pas.—Ma santé, je vous jure, me rendoit mes fonctions impossibles; mais même en les mettant de côté il a été au-dessus de mes forces de supporter plus longtems l'horreur que me causoit ce sang,—ces têtes,—cette reine presque egorgée,—ce roi, amené esclave, entrant à Paris au milieu de ses assassins, et précédé des têtes de ses malheureux gardes,—ces perfides janissaires, ces assassins, ces femmes cannibales,—ce cri de TOUS LES ÉVÊQUES À LA LANTERNE, dans le moment où le roi entre sa capitale avec deux évêques de son conseil dans sa voiture,—un coup de fusil, que j'ai vu tirer dans un des carrosses de la reine,—M. Bailly appellant cela un beau jour,—l'assemblée ayant déclaré froidement le matin, qu'il n'étoit pas de sa dignité d'aller toute entière environner le roi,—M. Mirabeau disant impunément dans cette assemblée, que le vaisseau de l'état, loin d'être arrêté dans sa course, s'élanceroit avec plus de rapidité que jamais vers sa régénération,—M. Barnave, riant avec lui, quand des flots de sang couloient autour de nous,—le vertueux Mounier[A] échappant par miracle à vingt assassins, qui avoient voulu faire de sa tête un trophée de plus: Voilà ce qui me fit jurer de ne plus mettre le pied dans cette caverne d'Antropophages [The National Assembly], où je n'avois plus de force d'élever la voix, où depuis six semaines je l'avois élevée en vain.

%"Moi, Mounier, et tous les honnêtes gens, ont pensé que le dernier effort à faire pour le bien étoit d'en sortir. Aucune idée de crainte ne s'est approchée de moi. Je rougirois de m'en défendre. J'avois encore reçû sur la route de la part de ce peuple, moins coupable que ceux qui l'ont enivré de fureur, des acclamations, et des applaudissements, dont d'autres auroient été flattés, et qui m'ont fait frémir. C'est à l'indignation, c'est à l'horreur, c'est aux convulsions physiques, que le seul aspect du sang me fait éprouver que j'ai cédé. On brave une seule mort; on la brave plusieurs fois, quand elle peut être utile. Mais aucune puissance sous le ciel, mais aucune opinion publique ou privée n'ont le droit de me condamner à souffrir inutilement mille supplices par minute, et à périr de désespoir, de rage, au milieu des triomphes, du crime que je n'ai pu arrêter. Ils me proscriront, ils confisqueront mes biens. Je labourerai la terre, et je ne les verrai plus. Voilà ma justification. Vous pourrez la lire, la montrer, la laisser copier; tant pis pour ceux qui ne la comprendront pas; ce ne sera alors moi qui auroit eu tort de la leur donner."

%This military man had not so good nerves as the peaceable gentlemen of the Old Jewry.—See Mons. Mounier's narrative of these transactions: a man also of honor and virtue and talents, and therefore a fugitive.

%[A] N.B.M. Mounier was then speaker of the National Assembly. He has since been obliged to live in exile, though one of the firmest assertors of liberty.

% [93] See the fate of Bailly and Condorcet, supposed to be here particularly alluded to. Compare the circumstances of the trial and execution of the former with this prediction.

% [94] The English are, I conceive, misrepresented in a letter published in one of the papers, by a gentleman thought to be a Dissenting minister. When writing to Dr. Price of the spirit which prevails at Paris, he says,—"The spirit of the people in this place has abolished all the proud distinctions which the king and nobles had usurped in their minds: whether they talk of the king, the noble, or the priest, their whole language is that of the most enlightened and liberal amongst the English." If this gentleman means to confine the terms enlightened and liberal to one set of men in England, it may be true. It is not generally so.

% [95] Sit igitur hoc ab initio persuasum civibus, dominos esse omnium rerum ac moderatores deos; eaque, quæ gerantur, eorum geri vi, ditione, ac numine; eosdemque optime de genere hominum mereri; et qualis quisque sit, quid agat, quid in se admittat, qua mente, qua pietate colat religiones intueri: piorum et impiorum habere rationem. His enim rebus imbutæ mentes haud sane abhorrebunt ab utili et a vera sententia.—Cic. de Legibus, l. 2.

% [96] Quicquid multis peccatur inultum.

% [97] This (down to the end of the first sentence in the next paragraph) and some other parts, here and there, were inserted, on his reading the manuscript, by my lost son.

% [98] I do not choose to shock the feeling of the moral reader with any quotation of their vulgar, base, and profane language.

% [99] Their connection with Turgot and almost all the people of the finance.

% [100] All have been confiscated in their turn.

% [101] Not his brother, nor any near relation; but this mistake does not affect the argument.

% [102] The rest of the passage is this:—

%\begin{verse}
%"Who, having spent the treasures of his crown, \\
%Condemns their luxury to feed his own. \\
%And yet this act, to varnish o'er the shame \\
%Of sacrilege, must bear Devotion's name. \\
%No crime so bold, but would be understood \\
%A Real, or at least a seeming good. \\
%Who fears not to do ill, yet fears the name, \\
%And free from conscience, is a slave to fame. \\
%Thus he the Church at once protects and spoils: \\
%But princes' swords are sharper than their styles. \\
%And thus to th' ages past he makes amends, \\
%Their charity destroys, their faith defends. \\
%Then did Religion in a lazy cell, \\
%In empty, airy contemplations, dwell; \\
%And like the block, unmovèd lay: but ours, \\
%As much too active, like the stork devours. \\
%Is there no temperate region can be known \\
%Betwixt their frigid and our torrid zone? \\
%Could we not wake from that lethargic dream, \\
%But to be restless in a worse extreme? \\
%And for that lethargy was there no care, \\
%But to be cast into a calenture? \\
%Can knowledge have no bound, but must advance \\
%So far, to make us wish for ignorance, \\
%And rather in the dark to grope our way, \\
%Than, led by a false guide, to err by day? \\
%Who sees these dismal heaps, but would demand \\
%What barbarous invader sack'd the land? \\
%But when he hears no Goth, no Turk did bring \\
%This desolation, but a Christian king, \\
%When nothing but the name of zeal appears \\
%'Twixt our best actions and the worst of theirs, \\
%What does he think our sacrilege would spare, \\
%When such th' effects of our devotions are?"
%\end{verse}

%\hfill Cooper's Hill, by Sir JOHN DENHAM.

% [103] Rapport de Mons. le Directeur-Général des Finances, fait par Ordre du Roi à Versailles. Mai 5, 1789.

% [104] In the Constitution of Scotland, during the Stuart reigns, a committee sat for preparing bills; and none could pass, but those previously approved by them. This committee was called Lords of Articles.

% [105] When I wrote this I quoted from memory, after many years had elapsed from my reading the passage. A learned friend has found it and it is as follows:—

%{\g τὸ ἠ̂θος τὸ αὐτό, καὶ ἄμφω δεσποτικὰ τω̂ν βελτιόνων, καὶ τὰ ψηφίσματα ὥσπερ ἐκει̂ τὰ ἐπιτάγματα, καὶ ὁ δημαγωγὸς καὶ ὁ κόλαξ οἱ αὐτοὶ καὶ ἀνάλογον. καὶ μάλιστα δ' ἑκάτεροι παρ' ἑκατέροις ἰσχύουσιν, οἱ μὲν κόλακες παρὰ τοι̂ς τυράννοις, οἱ δὲ δημαγωγοὶ παρὰ τοι̂ς δήμοις τοι̂ς τοιούτοις.}

%"The ethical character is the same: both exercise despotism over the better class of citizens; and decrees are in the one what ordinances and arrêts are in the other: the demagogue, too, and the court favorite, are not unfrequently the same identical men, and always bear a close analogy; and these have the principal power, each in their respective forms of government, favorites with the absolute monarch, and demagogues with a people such as I have described."—Arist. Politic. lib. iv. cap. 4.

% [106] De l'Administration des Finances de la France, par Mons. Necker, Vol. I. p. 288.

% [107] De l'Administration des Finances de la France, par M. Necker.

% [108] Vol. III. chap. 8 and chap. 9.

% [109] The world is obliged to M. de Calonne for the pains he has taken to refute the scandalous exaggerations relative to some of the royal expenses, and to detect the fallacious account given of pensions, for the wicked purpose of provoking the populace to all sorts of crimes.

% [110] See Gulliver's Travels for the idea of countries governed by philosophers.

% [111] M. de Calonne states the falling off of the population of Paris as far more considerable; and it may be so, since the period of M. Necker's calculation.

% [112]
%\begin{center}
%\begin{tabular}{l r r r r}
%Travaux de charité pour subvenir au manque & Livres. &	£ &	s. & d. \\
%de travail à Paris et dans les provinces   & 3,866,920 & 161,121 & 13 & 4 \\
%Destruction de vagabondage et de la mendicité & 1,671,417 & 69,642 & 7 & 6 \\
%Primes pour l'importation de grains	& 5,671,907 & 235,329 & 9 & 2 \\
%Dépenses relatives aux subsistances, déduction  & & & & \\
%fait des reconvrements qui out en lieu & 39,871,790 & 1,661,324 & 11 & 8 \\
%Total	& 51,082,034	& 2,128,418	& 1	& 8
%\end{tabular}
%\end{center}

%When I sent this book to the press, I entertained some doubt concerning the nature and extent of the last article in the above accounts, which is only under a general head, without any detail. Since then I have seen M. de Calonne's work. I must think it a great loss to me that I had not that advantage earlier. M. de Calonne thinks this article to be on account of general subsistence; but as he is not able to comprehend how so great a loss as upwards of 1,661,000l. sterling could be sustained on the difference between the price and the sale of grain, he seems to attribute this enormous head of charge to secret expenses of the Revolution. I cannot say anything positively on that subject. The reader is capable of judging, by the aggregate of these immense charges, on the state and condition of France, and the system of public economy adopted in that nation. These articles of account produced no inquiry or discussion in the National Assembly.

% [113] This is on a supposition of the truth of this story; but he was not in France at the time. One name serves as well as another.

% [114] Domat.

% [115] Speech of M. Camus, published by order of the National Assembly.

% [116] Whether the following description is strictly true I know not; but it is what the publishers would have pass for true, in order to animate others. In a letter from Toul, given in one of their papers, is the following passage concerning the people of that district:—"Dans la Révolution actuelle, ils ont résisté à toutes les séductions du bigotisme, aux persécutions et aux tracasseries des ennemis de la Révolution. Oubliant leurs plus grands intérêts pour rendre hommage aux vues d'ordre général qui out déterminé l'Assemblée Nationale, ils voient, sans se plaindre, supprimer cette foule d'établissemens ecclésiastiques par lesquels ils subsistoient; et même, en perdant leur siège épiscopal, la seule de toutes ces ressources qui pouvoit, on plutôt qui devoit, en toute équité, leur être conservée, condamnés à la plus effrayante misère sans avoir été ni pu être entendus, ils ne murmurent point, ils restent fidèles aux principes du plus pur patriotisme; ils sont encore prêts à verser leur sang pour le maintien de la constitution, qui va réduire leur ville à la plus déplorable nullité."—These people are not supposed to have endured those sufferings and injustices in a struggle for liberty, for the same account states truly that they have been always free; their patience in beggary and ruin, and their suffering, without remonstrance, the most flagrant and confessed injustice, if strictly true, can be nothing but the effect of this dire fanaticism. A great multitude all over France is in the same condition and the same temper.

% [117] See the proceedings of the confederation at Nantes.

% [118] "Si plures sunt ii quibus improbe datum est, quam illi quibus injuste ademptum est, idcirco plus etiam valent? Non enim numero hæc judicantur, sed pondere. Quam autem habet æquitatem, ut agrum multis annis, aut etiam sæculis ante possessum, qui nullum habuit habeat, qui autem habuit amittat? Ac, propter hoc injuriæ genus, Lacedæmonii Lysandrum Ephorum expulerunt; Agin regem (quod nunquam antea apud eos acciderat) necaverunt; exque eo tempore tantæ discordiæ secutæ sunt, ut et tyranni exsisterent, et optimates exterminarentur, et preclarissime constituta respublica dilaberetur. Nec vero solum ipsa cecidit, sed etiam reliquam Græciam evertit contagionibus malorum, quæ a Lacedæmoniis profectæ manarunt latius."—After speaking of the conduct of the model of true patriots, Aratus of Sicyon, which was in a very different spirit, he says,—"Sic par est agere cum civibus; non (ut bis jam vidimus) hastam in foro ponere et bona civium voci subjicere præconis. At ille Græcus (id quod fuit sapientis et præstantis viri) omnibus consulendum esse putavit: eaque est summa ratio et sapientia boni civis, commoda civium non divellere, sed omnes eadem æquitate continere."—Cic. Off. 1. 2.

% [119] See two books entitled, "Einige Originalschriften des Illuminatenordens,"—"System und Folgen des Illuminatenordens." München, 1787.

% [120] A leading member of the Assembly, M. Rabaut de St. Étienne, has expressed the principle of all their proceedings as clearly as possible; nothing can be more simple:—"Tous les établissemens en France couronnent le malheur du peuple: pour le rendre heureux, il faut le renouveler, changer ses idées, changer ses loix, changer ses mœurs, ... changer les hommes, changer les choses, changer ses mots, ... tout détruire; oui, tout détruire; puisque tout est à récréer."—This gentleman was chosen president in an assembly not sitting at Quinze-Vingt or the Petites Maisons, and composed of persons giving themselves out to be rational beings; but neither his ideas, language, or conduct differ in the smallest degree from the discourses, opinions, and actions of those, within and without the Assembly, who direct the operations of the machine now at work in France.

% [121] The Assembly, in executing the plan of their committee, made some alterations. They have struck out one stage in these gradations; this removes a part of the objection; but the main objection, namely, that in their scheme the first constituent voter has no connection with the representative legislator, remains in all its force. There are other alterations, some possibly for the better, some certainly for the worse: but to the author the merit or demerit of these smaller alterations appears to be of no moment, where the scheme itself is fundamentally vicious and absurd.

% [122] "Non, ut olim, universæ legiones deducebantur, cum tribunis, et centurionibus, et sui cujusque ordinis militibus, ut consensu et caritate rempublicam efficerent; sed ignoti inter se, diversis manipulis, sine rectore, sine affectibus mutuis, quasi ex alio genere mortalium repente in unum collecti, numerus magis quam colonia."—Tac. Annal. lib. 14, sect. 27.—All this will be still more applicable to the unconnected, rotatory, biennial national assemblies, in this absurd and senseless constitution.

% [123] Qualitas, Relatio, Actio, Passio, Ubi, Quando, Situs, Habitus.

% [124] See l'État de la France, p. 363.

% [125] In reality three, to reckon the provincial republican establishments.

% [126] For further elucidations upon the subject of all these judicatures and of the Committee of Research, see M. de Calonne's work.

% [127] "Comme sa Majesté a reconnu, non un système d'associations particulières, mais une réunion de volontés de tous les François pour la liberté et la prospérité communes, ainsi pour le maintien de l'ordre publique, il a pensé qu'il convenoit que chaque régiment prît part à ces fêtes civiques pour multiplier les rapports, et resserrer les liens d'union entre les citoyens et les troupes."—Lest I should not be credited, I insert the words authorizing the troops to feast with the popular confederacies.

% [128] This war minister has since quitted the school and resigned his office.

% [129] Courrier François, 30 July, 1790. Assemblée Nationale, Numero 210.

% [130] I see by M. Necker's account, that the national guards of Paris have received, over and above the money levied within their own city, about 145,000l. sterling out of the public treasure. Whether this be an actual payment for the nine months of their existence, or an estimate of their yearly charge, I do not clearly perceive. It is of no great importance, as certainly they may take whatever they please.

% [131] The reader will observe that I have but lightly touched (my plan demanded nothing more) on the condition of the French finances as connected with the demands upon them. If I had intended to do otherwise, the materials in my hands for such a task are not altogether perfect. On this subject I refer the reader to M. de Calonne's work, and the tremendous display that he has made of the havoc and devastation in the public estate, and in all the affairs of France, caused by the presumptuous good intentions of ignorance and incapacity. Such effects those causes will always produce. Looking over that account with a pretty strict eye, and, with perhaps too much rigor, deducting everything which may be placed to the account of a financier out of place, who might be supposed by his enemies desirous of making the most of his cause, I believe it will be found that a more salutary lesson of caution against the daring spirit of innovators than what has been supplied at the expense of France never was at any time furnished to mankind.

% [132] La Bruyère of Bossuet.

% [133] "Ce n'est point à l'assemblée entière que je m'adresse ici; je ne parle qu'à ceux qui l'égarent, en lui cachant sous des gazes séduisantes le but où ils l'entraînent. C'est à eux que je dis: Votre objet, vous n'en disconviendrez pas, c'est d'ôter tout espoir au clergé, et de consommer sa ruine; c'est-là, en ne vous soupçonnant d'aucune combinaison de cupidité, d'aucun regard sur le jeu des effets publics, c'est-là ce qu'on doit croire que vous avez en vue dans la terrible opération que vous proposez; c'est ce qui doit en être le fruit. Mais le peuple qui vous y intéressez, quel avantage peut-il y trouver? En vous servant sans cesse de lui, que faites-vous pour lui? Rien, absolument rien; et, au contraire, vous faites ce qui ne conduit qu'à l'accabler de nouvelles charges. Vous avez rejeté, à son préjudice, une offre de 400 millions, dont l'acceptation pouvoit devenir un moyen de soulagement en sa faveur; et à cette ressource, aussi profitable que légitime, vous avez substitué une injustice ruineuse, qui, de votre propre aveu, charge le trésor public, et par consequent le peuple, d'un surcroît de dépense annuelle de 50 millions an moins, et d'un remboursement de 150 millions.

%Malheureux peuple! voilà ce que vous vaut en dernier résultat l'expropriation de l'Église, et la dureté des décrets taxateurs du traitement des ministres d'une religion bienfaisante; et désormais ils scront à votre charge: leurs charités soulageoient les pauvres; et vous allez être imposés pour subvenir à leur entretien!"—De l'État de la France, p. 81. See also p. 92, and the following pages.

